%% Generated by Sphinx.
\def\sphinxdocclass{report}
\documentclass[letterpaper,10pt,english]{sphinxmanual}
\ifdefined\pdfpxdimen
   \let\sphinxpxdimen\pdfpxdimen\else\newdimen\sphinxpxdimen
\fi \sphinxpxdimen=.75bp\relax
\ifdefined\pdfimageresolution
    \pdfimageresolution= \numexpr \dimexpr1in\relax/\sphinxpxdimen\relax
\fi
%% let collapsible pdf bookmarks panel have high depth per default
\PassOptionsToPackage{bookmarksdepth=5}{hyperref}

\PassOptionsToPackage{warn}{textcomp}
\usepackage[utf8]{inputenc}
\ifdefined\DeclareUnicodeCharacter
% support both utf8 and utf8x syntaxes
  \ifdefined\DeclareUnicodeCharacterAsOptional
    \def\sphinxDUC#1{\DeclareUnicodeCharacter{"#1}}
  \else
    \let\sphinxDUC\DeclareUnicodeCharacter
  \fi
  \sphinxDUC{00A0}{\nobreakspace}
  \sphinxDUC{2500}{\sphinxunichar{2500}}
  \sphinxDUC{2502}{\sphinxunichar{2502}}
  \sphinxDUC{2514}{\sphinxunichar{2514}}
  \sphinxDUC{251C}{\sphinxunichar{251C}}
  \sphinxDUC{2572}{\textbackslash}
\fi
\usepackage{cmap}
\usepackage[T1]{fontenc}
\usepackage{amsmath,amssymb,amstext}
\usepackage{babel}



\usepackage{tgtermes}
\usepackage{tgheros}
\renewcommand{\ttdefault}{txtt}



\usepackage[Bjarne]{fncychap}
\usepackage{sphinx}

\fvset{fontsize=auto}
\usepackage{geometry}


% Include hyperref last.
\usepackage{hyperref}
% Fix anchor placement for figures with captions.
\usepackage{hypcap}% it must be loaded after hyperref.
% Set up styles of URL: it should be placed after hyperref.
\urlstyle{same}


\usepackage{sphinxmessages}
\setcounter{tocdepth}{1}



\title{tamos}
\date{Nov 09, 2022}
\release{}
\author{BNerot}
\newcommand{\sphinxlogo}{\vbox{}}
\renewcommand{\releasename}{}
\makeindex
\begin{document}

\ifdefined\shorthandoff
  \ifnum\catcode`\=\string=\active\shorthandoff{=}\fi
  \ifnum\catcode`\"=\active\shorthandoff{"}\fi
\fi

\pagestyle{empty}
\sphinxmaketitle
\pagestyle{plain}
\sphinxtableofcontents
\pagestyle{normal}
\phantomsection\label{\detokenize{index::doc}}


\sphinxstepscope


\chapter{Gathering components}
\label{\detokenize{gathering_components:gathering-components}}\label{\detokenize{gathering_components::doc}}

\begin{savenotes}\sphinxattablestart
\centering
\begin{tabulary}{\linewidth}[t]{\X{1}{2}\X{1}{2}}
\hline

\sphinxAtStartPar
{\hyperref[\detokenize{generated/tamos.Hub:tamos.Hub}]{\sphinxcrossref{\sphinxcode{\sphinxupquote{tamos.Hub}}}}}({[}name, components, ...{]})
&
\sphinxAtStartPar

\\
\hline
\sphinxAtStartPar
{\hyperref[\detokenize{generated/tamos.TimeSettings:tamos.TimeSettings}]{\sphinxcrossref{\sphinxcode{\sphinxupquote{tamos.TimeSettings}}}}}(n{[}, step\_value, ...{]})
&
\sphinxAtStartPar

\\
\hline
\sphinxAtStartPar
{\hyperref[\detokenize{generated/tamos.MILPModel:tamos.MILPModel}]{\sphinxcrossref{\sphinxcode{\sphinxupquote{tamos.MILPModel}}}}}(hubs, time\_settings{[}, name{]})
&
\sphinxAtStartPar

\\
\hline
\end{tabulary}
\par
\sphinxattableend\end{savenotes}

\sphinxstepscope


\section{tamos.Hub}
\label{\detokenize{generated/tamos.Hub:tamos-hub}}\label{\detokenize{generated/tamos.Hub::doc}}\index{Hub (class in tamos)@\spxentry{Hub}\spxextra{class in tamos}}

\begin{fulllineitems}
\phantomsection\label{\detokenize{generated/tamos.Hub:tamos.Hub}}
\pysigstartsignatures
\pysiglinewithargsret{\sphinxbfcode{\sphinxupquote{class\DUrole{w}{  }}}\sphinxcode{\sphinxupquote{tamos.}}\sphinxbfcode{\sphinxupquote{Hub}}}{\emph{\DUrole{n}{name}\DUrole{o}{=}\DUrole{default_value}{None}}, \emph{\DUrole{n}{components}\DUrole{o}{=}\DUrole{default_value}{None}}, \emph{\DUrole{n}{components\_assemblies}\DUrole{o}{=}\DUrole{default_value}{None}}, \emph{\DUrole{n}{interface\_masks}\DUrole{o}{=}\DUrole{default_value}{None}}}{}
\pysigstopsignatures\index{\_\_init\_\_() (tamos.Hub method)@\spxentry{\_\_init\_\_()}\spxextra{tamos.Hub method}}

\begin{fulllineitems}
\phantomsection\label{\detokenize{generated/tamos.Hub:tamos.Hub.__init__}}
\pysigstartsignatures
\pysiglinewithargsret{\sphinxbfcode{\sphinxupquote{\_\_init\_\_}}}{\emph{\DUrole{n}{name}\DUrole{o}{=}\DUrole{default_value}{None}}, \emph{\DUrole{n}{components}\DUrole{o}{=}\DUrole{default_value}{None}}, \emph{\DUrole{n}{components\_assemblies}\DUrole{o}{=}\DUrole{default_value}{None}}, \emph{\DUrole{n}{interface\_masks}\DUrole{o}{=}\DUrole{default_value}{None}}}{}
\pysigstopsignatures
\sphinxAtStartPar
Gathers components and allow power exchanges through a dedicated interface.

\sphinxAtStartPar
Enables simple definition of MILP constraints using the \sphinxtitleref{components\_assemblies} and \sphinxtitleref{interface\_masks} attributes.
\begin{quote}\begin{description}
\sphinxlineitem{Parameters}\begin{itemize}
\item {} 
\sphinxAtStartPar
\sphinxstyleliteralstrong{\sphinxupquote{name}} (\sphinxstyleliteralemphasis{\sphinxupquote{str}}\sphinxstyleliteralemphasis{\sphinxupquote{, }}\sphinxstyleliteralemphasis{\sphinxupquote{optional}}) \textendash{} 

\item {} 
\sphinxAtStartPar
\sphinxstyleliteralstrong{\sphinxupquote{components}} (\sphinxstyleliteralemphasis{\sphinxupquote{list of storage}}\sphinxstyleliteralemphasis{\sphinxupquote{, }}\sphinxstyleliteralemphasis{\sphinxupquote{production}}\sphinxstyleliteralemphasis{\sphinxupquote{ or }}\sphinxstyleliteralemphasis{\sphinxupquote{element\_IO instances}}\sphinxstyleliteralemphasis{\sphinxupquote{, }}\sphinxstyleliteralemphasis{\sphinxupquote{optional}}) \textendash{} 

\item {} 
\sphinxAtStartPar
\sphinxstyleliteralstrong{\sphinxupquote{components\_assemblies}} (\sphinxstyleliteralemphasis{\sphinxupquote{list of 3\sphinxhyphen{}tuple objects}}\sphinxstyleliteralemphasis{\sphinxupquote{ (}}\sphinxstyleliteralemphasis{\sphinxupquote{n\_min}}\sphinxstyleliteralemphasis{\sphinxupquote{, }}\sphinxstyleliteralemphasis{\sphinxupquote{n\_max}}\sphinxstyleliteralemphasis{\sphinxupquote{, }}\sphinxstyleliteralemphasis{\sphinxupquote{components}}\sphinxstyleliteralemphasis{\sphinxupquote{)}}\sphinxstyleliteralemphasis{\sphinxupquote{, }}\sphinxstyleliteralemphasis{\sphinxupquote{optional}}) \textendash{} 
\sphinxAtStartPar
Each 3\sphinxhyphen{}tuple is such that:
\begin{itemize}
\item {} 
\sphinxAtStartPar
\sphinxtitleref{n\_min} (\sphinxtitleref{n\_max}) is the minimum (maximum) number of components from \sphinxtitleref{components} that must be installed in the hub.

\item {} \begin{description}
\sphinxlineitem{\sphinxtitleref{components} is a component or list of production, storage or element\_IO components.}
\sphinxAtStartPar
If one component of \sphinxtitleref{components} is not one of hub components,
the 3\sphinxhyphen{}tuple (n\_min, n\_max, components) is ignored during constraints declaration.

\end{description}

\end{itemize}


\item {} 
\sphinxAtStartPar
\sphinxstyleliteralstrong{\sphinxupquote{interface\_masks}} (\sphinxstyleliteralemphasis{\sphinxupquote{list of InterfaceMask instances}}\sphinxstyleliteralemphasis{\sphinxupquote{, }}\sphinxstyleliteralemphasis{\sphinxupquote{optional}}) \textendash{} Instances in \sphinxtitleref{interface\_masks} that do not refer to a component of the hub will be ignored during constraints declaration.

\end{itemize}

\end{description}\end{quote}
\subsubsection*{Examples}

\begin{sphinxVerbatim}[commandchars=\\\{\}]
\PYG{g+gp}{\PYGZgt{}\PYGZgt{}\PYGZgt{} }\PYG{n}{tms}\PYG{o}{.}\PYG{n}{Hub}\PYG{p}{(}\PYG{n}{components}\PYG{o}{=}\PYG{p}{[}\PYG{n}{heat\PYGZus{}load\PYGZus{}1}\PYG{p}{,} \PYG{n}{heat\PYGZus{}load\PYGZus{}2}\PYG{p}{,} \PYG{n}{heat\PYGZus{}load\PYGZus{}3}\PYG{p}{,} \PYG{n}{electric\PYGZus{}heater\PYGZus{}1}\PYG{p}{,} \PYG{n}{electric\PYGZus{}heater\PYGZus{}2}\PYG{p}{,} \PYG{n}{electricity\PYGZus{}grid}\PYG{p}{]}\PYG{p}{,}
\PYG{g+gp}{... }        \PYG{n}{components\PYGZus{}assemblies}\PYG{o}{=}\PYG{p}{[}\PYG{p}{(}\PYG{l+m+mi}{1}\PYG{p}{,} \PYG{l+m+mi}{2}\PYG{p}{,} \PYG{p}{[}\PYG{n}{heat\PYGZus{}load\PYGZus{}1}\PYG{p}{,} \PYG{n}{heat\PYGZus{}load\PYGZus{}2}\PYG{p}{]}\PYG{p}{)}\PYG{p}{,} \PYG{p}{(}\PYG{l+m+mi}{1}\PYG{p}{,} \PYG{l+m+mi}{1}\PYG{p}{,} \PYG{n}{heat\PYGZus{}load\PYGZus{}3}\PYG{p}{)}\PYG{p}{,} \PYG{p}{(}\PYG{l+m+mi}{0}\PYG{p}{,} \PYG{l+m+mi}{1}\PYG{p}{,} \PYG{p}{[}\PYG{n}{electric\PYGZus{}heater\PYGZus{}1}\PYG{p}{,} \PYG{n}{electric\PYGZus{}heater\PYGZus{}2}\PYG{p}{]}\PYG{p}{)}\PYG{p}{]}\PYG{p}{)}
\end{sphinxVerbatim}

\sphinxAtStartPar
At least one of the first and second loads must be used.
The third load must be used. At most one of the two heaters must be used.

\begin{sphinxVerbatim}[commandchars=\\\{\}]
\PYG{g+gp}{\PYGZgt{}\PYGZgt{}\PYGZgt{} }\PYG{n}{tms}\PYG{o}{.}\PYG{n}{Hub}\PYG{p}{(}\PYG{n}{components}\PYG{o}{=}\PYG{p}{[}\PYG{n}{heat\PYGZus{}load}\PYG{p}{,} \PYG{n}{electric\PYGZus{}heater}\PYG{p}{,} \PYG{n}{electricity\PYGZus{}grid}\PYG{p}{,} \PYG{n}{heat\PYGZus{}grid}\PYG{p}{]}\PYG{p}{,}
\PYG{g+gp}{... }        \PYG{n}{components\PYGZus{}assemblies}\PYG{o}{=}\PYG{p}{[}\PYG{p}{(}\PYG{l+m+mi}{0}\PYG{p}{,} \PYG{l+m+mi}{1}\PYG{p}{,} \PYG{p}{[}\PYG{n}{electric\PYGZus{}heater}\PYG{p}{,} \PYG{n}{heat\PYGZus{}grid}\PYG{p}{]}\PYG{p}{)}\PYG{p}{,} \PYG{p}{(}\PYG{l+m+mi}{1}\PYG{p}{,} \PYG{l+m+mi}{1}\PYG{p}{,} \PYG{n}{heat\PYGZus{}load}\PYG{p}{)}\PYG{p}{]}\PYG{p}{)}
\end{sphinxVerbatim}

\sphinxAtStartPar
The load must be used. Either electric\_heater or heat\_grid can be used.

\end{fulllineitems}

\subsubsection*{Methods}


\begin{savenotes}\sphinxattablestart
\centering
\begin{tabulary}{\linewidth}[t]{\X{1}{2}\X{1}{2}}
\hline

\sphinxAtStartPar
{\hyperref[\detokenize{generated/tamos.Hub:tamos.Hub.__init__}]{\sphinxcrossref{\sphinxcode{\sphinxupquote{\_\_init\_\_}}}}}({[}name, components, ...{]})
&
\sphinxAtStartPar
Gathers components and allow power exchanges through a dedicated interface.
\\
\hline
\sphinxAtStartPar
{\hyperref[\detokenize{generated/tamos.Hub:tamos.Hub.change_components}]{\sphinxcrossref{\sphinxcode{\sphinxupquote{change\_components}}}}}(components{[}, add, remove{]})
&
\sphinxAtStartPar
Change the components that can be installed in the hub.
\\
\hline
\sphinxAtStartPar
{\hyperref[\detokenize{generated/tamos.Hub:tamos.Hub.change_interface_masks}]{\sphinxcrossref{\sphinxcode{\sphinxupquote{change\_interface\_masks}}}}}(interface\_masks{[}, ...{]})
&
\sphinxAtStartPar
Change the InterfaceMasks instances related to the hub.
\\
\hline
\sphinxAtStartPar
{\hyperref[\detokenize{generated/tamos.Hub:tamos.Hub.describe}]{\sphinxcrossref{\sphinxcode{\sphinxupquote{describe}}}}}()
&
\sphinxAtStartPar
Show an exhaustive description of the hub.
\\
\hline
\end{tabulary}
\par
\sphinxattableend\end{savenotes}
\subsubsection*{Attributes}


\begin{savenotes}\sphinxattablestart
\centering
\begin{tabulary}{\linewidth}[t]{\X{1}{2}\X{1}{2}}
\hline

\sphinxAtStartPar
{\hyperref[\detokenize{generated/tamos.Hub:tamos.Hub.components}]{\sphinxcrossref{\sphinxcode{\sphinxupquote{components}}}}}
&
\sphinxAtStartPar
production, storage and element\_IO components in the hub.
\\
\hline
\sphinxAtStartPar
{\hyperref[\detokenize{generated/tamos.Hub.components_assemblies:tamos.Hub.components_assemblies}]{\sphinxcrossref{\sphinxcode{\sphinxupquote{components\_assemblies}}}}}
&
\sphinxAtStartPar
Components assemblies of the hub.
\\
\hline
\sphinxAtStartPar
{\hyperref[\detokenize{generated/tamos.Hub:tamos.Hub.element_IOs}]{\sphinxcrossref{\sphinxcode{\sphinxupquote{element\_IOs}}}}}
&
\sphinxAtStartPar
element\_IO components in the hub.
\\
\hline
\sphinxAtStartPar
{\hyperref[\detokenize{generated/tamos.Hub:tamos.Hub.interface_masks}]{\sphinxcrossref{\sphinxcode{\sphinxupquote{interface\_masks}}}}}
&
\sphinxAtStartPar
InterfaceMask instances of the hub.
\\
\hline
\sphinxAtStartPar
{\hyperref[\detokenize{generated/tamos.Hub:tamos.Hub.name}]{\sphinxcrossref{\sphinxcode{\sphinxupquote{name}}}}}
&
\sphinxAtStartPar
str.
\\
\hline
\sphinxAtStartPar
{\hyperref[\detokenize{generated/tamos.Hub:tamos.Hub.possibly_connected_networks}]{\sphinxcrossref{\sphinxcode{\sphinxupquote{possibly\_connected\_networks}}}}}
&
\sphinxAtStartPar
network components that include the hub with at least one edge with a connection status is different from \textquotesingle{}network.no\_connection\textquotesingle{}.
\\
\hline
\sphinxAtStartPar
{\hyperref[\detokenize{generated/tamos.Hub:tamos.Hub.productions}]{\sphinxcrossref{\sphinxcode{\sphinxupquote{productions}}}}}
&
\sphinxAtStartPar
production components in the hub.
\\
\hline
\sphinxAtStartPar
{\hyperref[\detokenize{generated/tamos.Hub:tamos.Hub.storages}]{\sphinxcrossref{\sphinxcode{\sphinxupquote{storages}}}}}
&
\sphinxAtStartPar
storage components in the hub.
\\
\hline
\end{tabulary}
\par
\sphinxattableend\end{savenotes}
\index{change\_components() (tamos.Hub method)@\spxentry{change\_components()}\spxextra{tamos.Hub method}}

\begin{fulllineitems}
\phantomsection\label{\detokenize{generated/tamos.Hub:tamos.Hub.change_components}}
\pysigstartsignatures
\pysiglinewithargsret{\sphinxbfcode{\sphinxupquote{change\_components}}}{\emph{\DUrole{n}{components}}, \emph{\DUrole{n}{add}\DUrole{o}{=}\DUrole{default_value}{False}}, \emph{\DUrole{n}{remove}\DUrole{o}{=}\DUrole{default_value}{False}}}{}
\pysigstopsignatures
\sphinxAtStartPar
Change the components that can be installed in the hub.
\begin{quote}\begin{description}
\sphinxlineitem{Parameters}\begin{itemize}
\item {} 
\sphinxAtStartPar
\sphinxstyleliteralstrong{\sphinxupquote{components}} (\sphinxstyleliteralemphasis{\sphinxupquote{instance}}\sphinxstyleliteralemphasis{\sphinxupquote{ or }}\sphinxstyleliteralemphasis{\sphinxupquote{list of instances of production}}\sphinxstyleliteralemphasis{\sphinxupquote{, }}\sphinxstyleliteralemphasis{\sphinxupquote{storage and element\_IO components.}}) \textendash{} 

\item {} 
\sphinxAtStartPar
\sphinxstyleliteralstrong{\sphinxupquote{add}} (\sphinxstyleliteralemphasis{\sphinxupquote{bool}}\sphinxstyleliteralemphasis{\sphinxupquote{, }}\sphinxstyleliteralemphasis{\sphinxupquote{optional}}) \textendash{} If True, \sphinxtitleref{components} are added to already defined components in hub.

\item {} 
\sphinxAtStartPar
\sphinxstyleliteralstrong{\sphinxupquote{remove}} (\sphinxstyleliteralemphasis{\sphinxupquote{bool}}\sphinxstyleliteralemphasis{\sphinxupquote{, }}\sphinxstyleliteralemphasis{\sphinxupquote{optional}}) \textendash{} If True, all instances from \sphinxtitleref{components} that are components of the hub are removed from the hub.

\end{itemize}

\end{description}\end{quote}
\subsubsection*{Notes}

\sphinxAtStartPar
If \sphinxtitleref{add} and \sphinxtitleref{remove} are False, hub components are replaced by \sphinxtitleref{components}.
\sphinxtitleref{add} and \sphinxtitleref{remove} cannot be both True.

\end{fulllineitems}

\index{change\_interface\_masks() (tamos.Hub method)@\spxentry{change\_interface\_masks()}\spxextra{tamos.Hub method}}

\begin{fulllineitems}
\phantomsection\label{\detokenize{generated/tamos.Hub:tamos.Hub.change_interface_masks}}
\pysigstartsignatures
\pysiglinewithargsret{\sphinxbfcode{\sphinxupquote{change\_interface\_masks}}}{\emph{\DUrole{n}{interface\_masks}}, \emph{\DUrole{n}{add}\DUrole{o}{=}\DUrole{default_value}{False}}, \emph{\DUrole{n}{remove}\DUrole{o}{=}\DUrole{default_value}{False}}}{}
\pysigstopsignatures
\sphinxAtStartPar
Change the InterfaceMasks instances related to the hub.
\begin{quote}\begin{description}
\sphinxlineitem{Parameters}\begin{itemize}
\item {} 
\sphinxAtStartPar
\sphinxstyleliteralstrong{\sphinxupquote{interface\_masks}} (\sphinxstyleliteralemphasis{\sphinxupquote{instance}}\sphinxstyleliteralemphasis{\sphinxupquote{ or }}\sphinxstyleliteralemphasis{\sphinxupquote{list of instances of InterfaceMask.}}) \textendash{} Instances in \sphinxtitleref{interface\_masks} that do not refer to a component of the hub will be ignored during constraints declaration.

\item {} 
\sphinxAtStartPar
\sphinxstyleliteralstrong{\sphinxupquote{add}} (\sphinxstyleliteralemphasis{\sphinxupquote{bool}}\sphinxstyleliteralemphasis{\sphinxupquote{, }}\sphinxstyleliteralemphasis{\sphinxupquote{optional}}) \textendash{} If True, \sphinxtitleref{components} are added to already defined components in hub.

\item {} 
\sphinxAtStartPar
\sphinxstyleliteralstrong{\sphinxupquote{remove}} (\sphinxstyleliteralemphasis{\sphinxupquote{bool}}\sphinxstyleliteralemphasis{\sphinxupquote{, }}\sphinxstyleliteralemphasis{\sphinxupquote{optional}}) \textendash{} If True, all instances from \sphinxtitleref{components} that are components of the hub are removed from the hub.

\end{itemize}

\end{description}\end{quote}
\subsubsection*{Notes}

\sphinxAtStartPar
If \sphinxtitleref{add} and \sphinxtitleref{remove} are False, hub components are replaced by \sphinxtitleref{components}.
\sphinxtitleref{add} and \sphinxtitleref{remove} cannot be both True.

\end{fulllineitems}

\index{components (tamos.Hub property)@\spxentry{components}\spxextra{tamos.Hub property}}

\begin{fulllineitems}
\phantomsection\label{\detokenize{generated/tamos.Hub:tamos.Hub.components}}
\pysigstartsignatures
\pysigline{\sphinxbfcode{\sphinxupquote{property\DUrole{w}{  }}}\sphinxbfcode{\sphinxupquote{components}}}
\pysigstopsignatures
\sphinxAtStartPar
production, storage and element\_IO components in the hub.

\sphinxAtStartPar
These can be modified using the \sphinxtitleref{hub.change\_components} method.

\end{fulllineitems}

\index{components\_assemblies (tamos.Hub property)@\spxentry{components\_assemblies}\spxextra{tamos.Hub property}}

\begin{fulllineitems}
\phantomsection\label{\detokenize{generated/tamos.Hub:tamos.Hub.components_assemblies}}
\pysigstartsignatures
\pysigline{\sphinxbfcode{\sphinxupquote{property\DUrole{w}{  }}}\sphinxbfcode{\sphinxupquote{components\_assemblies}}}
\pysigstopsignatures
\sphinxAtStartPar
Components assemblies of the hub.

\sphinxAtStartPar
Must be provided as a list of 3\sphinxhyphen{}tuple objects (n\_min, n\_max, components) where:
\begin{itemize}
\item {} 
\sphinxAtStartPar
\sphinxtitleref{n\_min} (\sphinxtitleref{n\_max}) is the minimum (maximum) number of components from \sphinxtitleref{components} that must be installed in the hub.

\item {} 
\sphinxAtStartPar
\sphinxtitleref{components} is a component or list of production, storage or element\_IO components. If one component of \sphinxtitleref{components} is not one of hub components,
the 3\sphinxhyphen{}tuple (n\_min, n\_max, components) is ignored during constraints declaration.

\end{itemize}
\subsubsection*{Examples}

\begin{sphinxVerbatim}[commandchars=\\\{\}]
\PYG{g+gp}{\PYGZgt{}\PYGZgt{}\PYGZgt{} }\PYG{n}{hub}\PYG{o}{.}\PYG{n}{components\PYGZus{}assemblies} \PYG{o}{=} \PYG{p}{[}\PYG{p}{(}\PYG{l+m+mi}{1}\PYG{p}{,} \PYG{l+m+mi}{2}\PYG{p}{,} \PYG{p}{[}\PYG{n}{heat\PYGZus{}load\PYGZus{}1}\PYG{p}{,} \PYG{n}{heat\PYGZus{}load\PYGZus{}2}\PYG{p}{]}\PYG{p}{)}\PYG{p}{,} \PYG{p}{(}\PYG{l+m+mi}{1}\PYG{p}{,} \PYG{l+m+mi}{1}\PYG{p}{,} \PYG{n}{heat\PYGZus{}load\PYGZus{}3}\PYG{p}{)}\PYG{p}{,} \PYG{p}{(}\PYG{l+m+mi}{0}\PYG{p}{,} \PYG{l+m+mi}{1}\PYG{p}{,} \PYG{p}{[}\PYG{n}{electric\PYGZus{}heater\PYGZus{}1}\PYG{p}{,} \PYG{n}{electric\PYGZus{}heater\PYGZus{}2}\PYG{p}{]}\PYG{p}{)}\PYG{p}{]}\PYG{p}{)}
\end{sphinxVerbatim}

\sphinxAtStartPar
At least one of the first and second loads must be used.
The third load must be used. At most one of the two heaters must be used.

\begin{sphinxVerbatim}[commandchars=\\\{\}]
\PYG{g+gp}{\PYGZgt{}\PYGZgt{}\PYGZgt{} }\PYG{n}{hub}\PYG{o}{.}\PYG{n}{components\PYGZus{}assemblies} \PYG{o}{=} \PYG{p}{[}\PYG{p}{(}\PYG{l+m+mi}{0}\PYG{p}{,} \PYG{l+m+mi}{1}\PYG{p}{,} \PYG{p}{[}\PYG{n}{electric\PYGZus{}heater}\PYG{p}{,} \PYG{n}{heat\PYGZus{}grid}\PYG{p}{]}\PYG{p}{)}\PYG{p}{,} \PYG{p}{(}\PYG{l+m+mi}{1}\PYG{p}{,} \PYG{l+m+mi}{1}\PYG{p}{,} \PYG{n}{heat\PYGZus{}load}\PYG{p}{)}\PYG{p}{]}\PYG{p}{)}
\end{sphinxVerbatim}

\sphinxAtStartPar
The load must be used. Either electric\_heater or heat\_grid can be used.

\end{fulllineitems}

\index{describe() (tamos.Hub method)@\spxentry{describe()}\spxextra{tamos.Hub method}}

\begin{fulllineitems}
\phantomsection\label{\detokenize{generated/tamos.Hub:tamos.Hub.describe}}
\pysigstartsignatures
\pysiglinewithargsret{\sphinxbfcode{\sphinxupquote{describe}}}{}{}
\pysigstopsignatures
\sphinxAtStartPar
Show an exhaustive description of the hub.

\end{fulllineitems}

\index{element\_IOs (tamos.Hub property)@\spxentry{element\_IOs}\spxextra{tamos.Hub property}}

\begin{fulllineitems}
\phantomsection\label{\detokenize{generated/tamos.Hub:tamos.Hub.element_IOs}}
\pysigstartsignatures
\pysigline{\sphinxbfcode{\sphinxupquote{property\DUrole{w}{  }}}\sphinxbfcode{\sphinxupquote{element\_IOs}}}
\pysigstopsignatures
\sphinxAtStartPar
element\_IO components in the hub.
These can be modified using the \sphinxtitleref{hub.change\_components} method.

\end{fulllineitems}

\index{interface\_masks (tamos.Hub property)@\spxentry{interface\_masks}\spxextra{tamos.Hub property}}

\begin{fulllineitems}
\phantomsection\label{\detokenize{generated/tamos.Hub:tamos.Hub.interface_masks}}
\pysigstartsignatures
\pysigline{\sphinxbfcode{\sphinxupquote{property\DUrole{w}{  }}}\sphinxbfcode{\sphinxupquote{interface\_masks}}}
\pysigstopsignatures
\sphinxAtStartPar
InterfaceMask instances of the hub.

\sphinxAtStartPar
These can be modified using the \sphinxtitleref{hub.change\_interface\_masks} method.

\end{fulllineitems}

\index{name (tamos.Hub property)@\spxentry{name}\spxextra{tamos.Hub property}}

\begin{fulllineitems}
\phantomsection\label{\detokenize{generated/tamos.Hub:tamos.Hub.name}}
\pysigstartsignatures
\pysigline{\sphinxbfcode{\sphinxupquote{property\DUrole{w}{  }}}\sphinxbfcode{\sphinxupquote{name}}}
\pysigstopsignatures
\sphinxAtStartPar
str.
This name is used in MILP model description.
names must not exceed 255 characters,
all of which must be alphanumeric (a\sphinxhyphen{}z, A\sphinxhyphen{}Z, 0\sphinxhyphen{}9) or one of these symbols:
! ” \# \$ \% \& , . ; ? @ \_ ‘ ’ \{ \} \textasciitilde{}.
\begin{quote}\begin{description}
\sphinxlineitem{Type}
\sphinxAtStartPar
Name of the instance

\end{description}\end{quote}

\end{fulllineitems}

\index{possibly\_connected\_networks (tamos.Hub property)@\spxentry{possibly\_connected\_networks}\spxextra{tamos.Hub property}}

\begin{fulllineitems}
\phantomsection\label{\detokenize{generated/tamos.Hub:tamos.Hub.possibly_connected_networks}}
\pysigstartsignatures
\pysigline{\sphinxbfcode{\sphinxupquote{property\DUrole{w}{  }}}\sphinxbfcode{\sphinxupquote{possibly\_connected\_networks}}}
\pysigstopsignatures
\sphinxAtStartPar
network components that include the hub with at least one edge with a connection status is different from ‘network.no\_connection’.

\end{fulllineitems}

\index{productions (tamos.Hub property)@\spxentry{productions}\spxextra{tamos.Hub property}}

\begin{fulllineitems}
\phantomsection\label{\detokenize{generated/tamos.Hub:tamos.Hub.productions}}
\pysigstartsignatures
\pysigline{\sphinxbfcode{\sphinxupquote{property\DUrole{w}{  }}}\sphinxbfcode{\sphinxupquote{productions}}}
\pysigstopsignatures
\sphinxAtStartPar
production components in the hub.

\sphinxAtStartPar
These can be modified using the \sphinxtitleref{hub.change\_components} method.

\end{fulllineitems}

\index{storages (tamos.Hub property)@\spxentry{storages}\spxextra{tamos.Hub property}}

\begin{fulllineitems}
\phantomsection\label{\detokenize{generated/tamos.Hub:tamos.Hub.storages}}
\pysigstartsignatures
\pysigline{\sphinxbfcode{\sphinxupquote{property\DUrole{w}{  }}}\sphinxbfcode{\sphinxupquote{storages}}}
\pysigstopsignatures
\sphinxAtStartPar
storage components in the hub.

\sphinxAtStartPar
These can be modified using the \sphinxtitleref{hub.change\_components} method.

\end{fulllineitems}


\end{fulllineitems}


\sphinxstepscope


\section{tamos.TimeSettings}
\label{\detokenize{generated/tamos.TimeSettings:tamos-timesettings}}\label{\detokenize{generated/tamos.TimeSettings::doc}}\index{TimeSettings (class in tamos)@\spxentry{TimeSettings}\spxextra{class in tamos}}

\begin{fulllineitems}
\phantomsection\label{\detokenize{generated/tamos.TimeSettings:tamos.TimeSettings}}
\pysigstartsignatures
\pysiglinewithargsret{\sphinxbfcode{\sphinxupquote{class\DUrole{w}{  }}}\sphinxcode{\sphinxupquote{tamos.}}\sphinxbfcode{\sphinxupquote{TimeSettings}}}{\emph{\DUrole{n}{n}}, \emph{\DUrole{n}{step\_value}\DUrole{o}{=}\DUrole{default_value}{1}}, \emph{\DUrole{n}{system\_lifetime}\DUrole{o}{=}\DUrole{default_value}{40}}}{}
\pysigstopsignatures\index{\_\_init\_\_() (tamos.TimeSettings method)@\spxentry{\_\_init\_\_()}\spxextra{tamos.TimeSettings method}}

\begin{fulllineitems}
\phantomsection\label{\detokenize{generated/tamos.TimeSettings:tamos.TimeSettings.__init__}}
\pysigstartsignatures
\pysiglinewithargsret{\sphinxbfcode{\sphinxupquote{\_\_init\_\_}}}{\emph{\DUrole{n}{n}}, \emph{\DUrole{n}{step\_value}\DUrole{o}{=}\DUrole{default_value}{1}}, \emph{\DUrole{n}{system\_lifetime}\DUrole{o}{=}\DUrole{default_value}{40}}}{}
\pysigstopsignatures
\sphinxAtStartPar
Defines temporal parameters used for the optimization of energy systems.
Provide helper function to implement a reduced\sphinxhyphen{}complexity temporal approach regarding system operation.
See examples section for in\sphinxhyphen{}depth understanding.
\begin{quote}\begin{description}
\sphinxlineitem{Parameters}\begin{itemize}
\item {} 
\sphinxAtStartPar
\sphinxstyleliteralstrong{\sphinxupquote{n}} (\sphinxstyleliteralemphasis{\sphinxupquote{int}}) \textendash{} The length of the operation period.

\item {} 
\sphinxAtStartPar
\sphinxstyleliteralstrong{\sphinxupquote{step\_value}} (\sphinxstyleliteralemphasis{\sphinxupquote{int}}\sphinxstyleliteralemphasis{\sphinxupquote{, }}\sphinxstyleliteralemphasis{\sphinxupquote{optional}}\sphinxstyleliteralemphasis{\sphinxupquote{, }}\sphinxstyleliteralemphasis{\sphinxupquote{default 1}}) \textendash{} Length of each time step of the operation period, in hours.
Required to be consistent with some component physical properties.

\item {} 
\sphinxAtStartPar
\sphinxstyleliteralstrong{\sphinxupquote{system\_lifetime}} (\sphinxstyleliteralemphasis{\sphinxupquote{int}}\sphinxstyleliteralemphasis{\sphinxupquote{, }}\sphinxstyleliteralemphasis{\sphinxupquote{optional}}\sphinxstyleliteralemphasis{\sphinxupquote{, }}\sphinxstyleliteralemphasis{\sphinxupquote{default 40}}) \textendash{} Number of periods considered for economic amortization of components.
Is related to the property “Discount rate (\%)” of components. See method \sphinxtitleref{component.compute\_actualized\_cost}.
The default value \sphinxtitleref{40} is a typical value relevant for an annual operation, i.e. such that \sphinxtitleref{n} * \sphinxtitleref{step\_value} = 8760

\end{itemize}

\end{description}\end{quote}
\subsubsection*{Examples}

\begin{sphinxVerbatim}[commandchars=\\\{\}]
\PYG{g+gp}{\PYGZgt{}\PYGZgt{}\PYGZgt{} }\PYG{n}{time\PYGZus{}settings} \PYG{o}{=} \PYG{n}{TimeSettings}\PYG{p}{(}\PYG{n}{n}\PYG{o}{=}\PYG{l+m+mi}{8760}\PYG{p}{,} \PYG{n}{step\PYGZus{}value}\PYG{o}{=}\PYG{l+m+mi}{1}\PYG{p}{,} \PYG{n}{system\PYGZus{}lifetime}\PYG{o}{=}\PYG{l+m+mi}{30}\PYG{p}{)}
\end{sphinxVerbatim}

\sphinxAtStartPar
The operation period has a length of \sphinxtitleref{n} = 8760.
Each time serie parameter of components must be of length \sphinxtitleref{n}. Their values last all 1 hour (\sphinxtitleref{step\_value} = 1).
8760 values of 1 hour is one year.
Economic amortization is calculated for 30 periods, i.e. 30 years.
Up to now, \sphinxtitleref{time\_settings} does not define any relevant time\_step for operation. They must be added using the methods:
\begin{itemize}
\item {} 
\sphinxAtStartPar
add\_regular

\item {} 
\sphinxAtStartPar
add\_extreme\_values

\item {} 
\sphinxAtStartPar
add\_large\_diff

\end{itemize}

\begin{sphinxVerbatim}[commandchars=\\\{\}]
\PYG{g+gp}{\PYGZgt{}\PYGZgt{}\PYGZgt{} }\PYG{n}{time\PYGZus{}settings}\PYG{o}{.}\PYG{n}{add\PYGZus{}regular}\PYG{p}{(}\PYG{l+m+mi}{5}\PYG{p}{)}
\end{sphinxVerbatim}

\sphinxAtStartPar
Every index multiple of 5 is selected for operation.
At this point, the three first time steps (for instance) are defined by the following sets of indexes:
\begin{itemize}
\item {} 
\sphinxAtStartPar
\{0, 1, 2, 3, 4\}

\item {} 
\sphinxAtStartPar
\{5, 6, 7, 8, 9\}

\item {} 
\sphinxAtStartPar
\{10, 11, 12, 13, 14\}

\item {} 
\sphinxAtStartPar
…

\end{itemize}

\sphinxAtStartPar
This leads to consider approximately 8760/5 = 1752 time steps for operation.
Decision variables are indexed on the first element of these index sets.
Each parameter of the model given as a numpy.ndarray instance of length 8760 is averaged according to these time steps,
defining a new array \sphinxtitleref{new\_array} of length 1752. For instance:
\begin{itemize}
\item {} 
\sphinxAtStartPar
new\_array{[}0{]} = array{[}0:5{]}.mean()

\item {} 
\sphinxAtStartPar
new\_array{[}1{]} = array{[}5:10{]}.mean()

\item {} 
\sphinxAtStartPar
new\_array{[}2{]} = array{[}10:15{]}.mean()

\item {} 
\sphinxAtStartPar
…

\end{itemize}

\begin{sphinxVerbatim}[commandchars=\\\{\}]
\PYG{g+gp}{\PYGZgt{}\PYGZgt{}\PYGZgt{} }\PYG{n}{time\PYGZus{}settings}\PYG{o}{.}\PYG{n}{add\PYGZus{}extreme\PYGZus{}values}\PYG{p}{(}\PYG{n}{array}\PYG{o}{=}\PYG{n}{load\PYGZus{}values}\PYG{p}{,} \PYG{n}{number\PYGZus{}max}\PYG{o}{=}\PYG{l+m+mi}{3}\PYG{p}{)}
\end{sphinxVerbatim}

\sphinxAtStartPar
The indexes of the 3 largest values of \sphinxtitleref{load\_values} are added to the operation temporal vector, splitting existing time steps.
For instance, if 2 is one of these 3 indexes, the sets of indexes defining time steps become:
\begin{itemize}
\item {} 
\sphinxAtStartPar
\{0, 1\}

\item {} 
\sphinxAtStartPar
\{2\}

\item {} 
\sphinxAtStartPar
\{3, 4\}

\item {} 
\sphinxAtStartPar
\{5, 6, 7, 8, 9\}

\item {} 
\sphinxAtStartPar
\{10, 11, 12, 13, 14\}

\item {} 
\sphinxAtStartPar
…

\end{itemize}

\sphinxAtStartPar
Now, any array will be average according to:
\begin{itemize}
\item {} 
\sphinxAtStartPar
new\_array{[}0{]} = array{[}0:2{]}.mean()

\item {} 
\sphinxAtStartPar
new\_array{[}1{]} = array{[}2{]}

\item {} 
\sphinxAtStartPar
new\_array{[}2{]} = array{[}3:5{]}.mean()

\item {} 
\sphinxAtStartPar
new\_array{[}3{]} = array{[}5:10{]}.mean()

\item {} 
\sphinxAtStartPar
…

\end{itemize}

\begin{sphinxVerbatim}[commandchars=\\\{\}]
\PYG{g+gp}{\PYGZgt{}\PYGZgt{}\PYGZgt{} }\PYG{n}{time\PYGZus{}settings}\PYG{o}{.}\PYG{n}{add\PYGZus{}extreme\PYGZus{}values}\PYG{p}{(}\PYG{n}{array}\PYG{o}{=}\PYG{n}{temperature\PYGZus{}values}\PYG{p}{,} \PYG{n}{number\PYGZus{}min}\PYG{o}{=}\PYG{l+m+mi}{2}\PYG{p}{,} \PYG{n}{number\PYGZus{}max}\PYG{o}{=}\PYG{l+m+mi}{3}\PYG{p}{)}
\end{sphinxVerbatim}

\sphinxAtStartPar
The indexes of the two smallest and three largest values of \sphinxtitleref{temperature\_values} are added to the operation temporal vector.

\end{fulllineitems}

\subsubsection*{Methods}


\begin{savenotes}\sphinxattablestart
\centering
\begin{tabulary}{\linewidth}[t]{\X{1}{2}\X{1}{2}}
\hline

\sphinxAtStartPar
{\hyperref[\detokenize{generated/tamos.TimeSettings:tamos.TimeSettings.__init__}]{\sphinxcrossref{\sphinxcode{\sphinxupquote{\_\_init\_\_}}}}}(n{[}, step\_value, system\_lifetime{]})
&
\sphinxAtStartPar
Defines temporal parameters used for the optimization of energy systems.
\\
\hline
\sphinxAtStartPar
{\hyperref[\detokenize{generated/tamos.TimeSettings:tamos.TimeSettings.add}]{\sphinxcrossref{\sphinxcode{\sphinxupquote{add}}}}}(indexes)
&
\sphinxAtStartPar
Adds specific indexes to the operation temporal vector.
\\
\hline
\sphinxAtStartPar
{\hyperref[\detokenize{generated/tamos.TimeSettings:tamos.TimeSettings.add_extreme_values}]{\sphinxcrossref{\sphinxcode{\sphinxupquote{add\_extreme\_values}}}}}(array{[}, number\_min, ...{]})
&
\sphinxAtStartPar
Add the time steps corresponding to minimum and maximum values of an array to the operation temporal vector.
\\
\hline
\sphinxAtStartPar
{\hyperref[\detokenize{generated/tamos.TimeSettings:tamos.TimeSettings.add_large_diff}]{\sphinxcrossref{\sphinxcode{\sphinxupquote{add\_large\_diff}}}}}(array, number\_max)
&
\sphinxAtStartPar
Apply \textquotesingle{}add\_extreme\_values\textquotesingle{} to the absolute value of the temporal derivative of \textquotesingle{}array\textquotesingle{}.
\\
\hline
\sphinxAtStartPar
{\hyperref[\detokenize{generated/tamos.TimeSettings:tamos.TimeSettings.add_regular}]{\sphinxcrossref{\sphinxcode{\sphinxupquote{add\_regular}}}}}(step)
&
\sphinxAtStartPar
Add every multiple of \sphinxtitleref{step} to the operation temporal vector.
\\
\hline
\sphinxAtStartPar
{\hyperref[\detokenize{generated/tamos.TimeSettings:tamos.TimeSettings.plot_array_aggregation}]{\sphinxcrossref{\sphinxcode{\sphinxupquote{plot\_array\_aggregation}}}}}(array)
&
\sphinxAtStartPar
Plot the aggregated version of an array, aggregation being performed  according to the \sphinxtitleref{time\_steps} attribute.
\\
\hline
\sphinxAtStartPar
{\hyperref[\detokenize{generated/tamos.TimeSettings:tamos.TimeSettings.prepare_array}]{\sphinxcrossref{\sphinxcode{\sphinxupquote{prepare\_array}}}}}(array{[}, raise\_error{]})
&
\sphinxAtStartPar
Checks that array length complies with the length of the operation period.
\\
\hline
\end{tabulary}
\par
\sphinxattableend\end{savenotes}
\subsubsection*{Attributes}


\begin{savenotes}\sphinxattablestart
\centering
\begin{tabulary}{\linewidth}[t]{\X{1}{2}\X{1}{2}}
\hline

\sphinxAtStartPar
{\hyperref[\detokenize{generated/tamos.TimeSettings:tamos.TimeSettings.time_steps}]{\sphinxcrossref{\sphinxcode{\sphinxupquote{time\_steps}}}}}
&
\sphinxAtStartPar
Presents each time step of the temporal operation vector defined by its set of indexes.
\\
\hline
\end{tabulary}
\par
\sphinxattableend\end{savenotes}
\index{add() (tamos.TimeSettings method)@\spxentry{add()}\spxextra{tamos.TimeSettings method}}

\begin{fulllineitems}
\phantomsection\label{\detokenize{generated/tamos.TimeSettings:tamos.TimeSettings.add}}
\pysigstartsignatures
\pysiglinewithargsret{\sphinxbfcode{\sphinxupquote{add}}}{\emph{\DUrole{n}{indexes}}}{}
\pysigstopsignatures
\sphinxAtStartPar
Adds specific indexes to the operation temporal vector.
\begin{quote}\begin{description}
\sphinxlineitem{Parameters}
\sphinxAtStartPar
\sphinxstyleliteralstrong{\sphinxupquote{indexes}} (\sphinxstyleliteralemphasis{\sphinxupquote{int}}\sphinxstyleliteralemphasis{\sphinxupquote{ or }}\sphinxstyleliteralemphasis{\sphinxupquote{list of int}}) \textendash{} All values greater than \sphinxtitleref{n} will be ignored.
Each value of \sphinxtitleref{indexes} defines a new time step,
i.e. time series parameters of components are not averaged on these indexes.

\end{description}\end{quote}

\end{fulllineitems}

\index{add\_extreme\_values() (tamos.TimeSettings method)@\spxentry{add\_extreme\_values()}\spxextra{tamos.TimeSettings method}}

\begin{fulllineitems}
\phantomsection\label{\detokenize{generated/tamos.TimeSettings:tamos.TimeSettings.add_extreme_values}}
\pysigstartsignatures
\pysiglinewithargsret{\sphinxbfcode{\sphinxupquote{add\_extreme\_values}}}{\emph{\DUrole{n}{array}}, \emph{\DUrole{n}{number\_min}\DUrole{o}{=}\DUrole{default_value}{None}}, \emph{\DUrole{n}{number\_max}\DUrole{o}{=}\DUrole{default_value}{None}}}{}
\pysigstopsignatures
\sphinxAtStartPar
Add the time steps corresponding to minimum and maximum values of an array to the operation temporal vector.
\begin{quote}\begin{description}
\sphinxlineitem{Parameters}\begin{itemize}
\item {} 
\sphinxAtStartPar
\sphinxstyleliteralstrong{\sphinxupquote{array}} (\sphinxstyleliteralemphasis{\sphinxupquote{numpy.ndarray}}) \textendash{} 

\item {} 
\sphinxAtStartPar
\sphinxstyleliteralstrong{\sphinxupquote{number\_min}} (\sphinxstyleliteralemphasis{\sphinxupquote{int}}\sphinxstyleliteralemphasis{\sphinxupquote{, }}\sphinxstyleliteralemphasis{\sphinxupquote{optional}}) \textendash{} The indexes of the \sphinxtitleref{number\_min} (\sphinxtitleref{numer\_max}) smallest (largest) values of \sphinxtitleref{array}
will be added to the operation temporal vector.

\item {} 
\sphinxAtStartPar
\sphinxstyleliteralstrong{\sphinxupquote{number\_max}} (\sphinxstyleliteralemphasis{\sphinxupquote{int}}\sphinxstyleliteralemphasis{\sphinxupquote{, }}\sphinxstyleliteralemphasis{\sphinxupquote{optional}}) \textendash{} The indexes of the \sphinxtitleref{number\_min} (\sphinxtitleref{numer\_max}) smallest (largest) values of \sphinxtitleref{array}
will be added to the operation temporal vector.

\end{itemize}

\end{description}\end{quote}

\end{fulllineitems}

\index{add\_large\_diff() (tamos.TimeSettings method)@\spxentry{add\_large\_diff()}\spxextra{tamos.TimeSettings method}}

\begin{fulllineitems}
\phantomsection\label{\detokenize{generated/tamos.TimeSettings:tamos.TimeSettings.add_large_diff}}
\pysigstartsignatures
\pysiglinewithargsret{\sphinxbfcode{\sphinxupquote{add\_large\_diff}}}{\emph{\DUrole{n}{array}}, \emph{\DUrole{n}{number\_max}}}{}
\pysigstopsignatures
\sphinxAtStartPar
Apply ‘add\_extreme\_values’ to the absolute value of the temporal derivative of ‘array’.
\begin{quote}\begin{description}
\sphinxlineitem{Parameters}\begin{itemize}
\item {} 
\sphinxAtStartPar
\sphinxstyleliteralstrong{\sphinxupquote{array}} (\sphinxstyleliteralemphasis{\sphinxupquote{numpy.ndarray}}) \textendash{} 

\item {} 
\sphinxAtStartPar
\sphinxstyleliteralstrong{\sphinxupquote{number\_max}} (\sphinxstyleliteralemphasis{\sphinxupquote{int}}) \textendash{} The indexes of the \sphinxtitleref{number\_max} largest values to be added to the operation temporal vector.

\end{itemize}

\end{description}\end{quote}
\subsubsection*{Notes}
\begin{description}
\sphinxlineitem{Temporal derivative of \sphinxtitleref{array} is d\_array defined as:}
\sphinxAtStartPar
d\_array(t) = abs(array(t+1)\sphinxhyphen{}array(t))

\end{description}

\end{fulllineitems}

\index{add\_regular() (tamos.TimeSettings method)@\spxentry{add\_regular()}\spxextra{tamos.TimeSettings method}}

\begin{fulllineitems}
\phantomsection\label{\detokenize{generated/tamos.TimeSettings:tamos.TimeSettings.add_regular}}
\pysigstartsignatures
\pysiglinewithargsret{\sphinxbfcode{\sphinxupquote{add\_regular}}}{\emph{\DUrole{n}{step}}}{}
\pysigstopsignatures
\sphinxAtStartPar
Add every multiple of \sphinxtitleref{step} to the operation temporal vector.
\begin{quote}\begin{description}
\sphinxlineitem{Parameters}
\sphinxAtStartPar
\sphinxstyleliteralstrong{\sphinxupquote{step}} (\sphinxstyleliteralemphasis{\sphinxupquote{int}}) \textendash{} 

\end{description}\end{quote}

\end{fulllineitems}

\index{plot\_array\_aggregation() (tamos.TimeSettings method)@\spxentry{plot\_array\_aggregation()}\spxextra{tamos.TimeSettings method}}

\begin{fulllineitems}
\phantomsection\label{\detokenize{generated/tamos.TimeSettings:tamos.TimeSettings.plot_array_aggregation}}
\pysigstartsignatures
\pysiglinewithargsret{\sphinxbfcode{\sphinxupquote{plot\_array\_aggregation}}}{\emph{\DUrole{n}{array}}}{}
\pysigstopsignatures
\sphinxAtStartPar
Plot the aggregated version of an array, aggregation being performed  according to the \sphinxtitleref{time\_steps} attribute.

\sphinxAtStartPar
This aggregation is how arrays are dealt with in \sphinxtitleref{MILPModel} instances.
\begin{quote}\begin{description}
\sphinxlineitem{Parameters}
\sphinxAtStartPar
\sphinxstyleliteralstrong{\sphinxupquote{array}} (\sphinxstyleliteralemphasis{\sphinxupquote{numpy.ndarray}}) \textendash{} 

\end{description}\end{quote}

\end{fulllineitems}

\index{prepare\_array() (tamos.TimeSettings method)@\spxentry{prepare\_array()}\spxextra{tamos.TimeSettings method}}

\begin{fulllineitems}
\phantomsection\label{\detokenize{generated/tamos.TimeSettings:tamos.TimeSettings.prepare_array}}
\pysigstartsignatures
\pysiglinewithargsret{\sphinxbfcode{\sphinxupquote{prepare\_array}}}{\emph{\DUrole{n}{array}}, \emph{\DUrole{n}{raise\_error}\DUrole{o}{=}\DUrole{default_value}{False}}}{}
\pysigstopsignatures
\sphinxAtStartPar
Checks that array length complies with the length of the operation period.
If array is too long, this method acts depending on \sphinxtitleref{raise\_error}.
If array is too short, an attributeError is raised.
\begin{quote}\begin{description}
\sphinxlineitem{Parameters}\begin{itemize}
\item {} 
\sphinxAtStartPar
\sphinxstyleliteralstrong{\sphinxupquote{array}} (\sphinxstyleliteralemphasis{\sphinxupquote{numpy.ndarray}}) \textendash{} The array to format.

\item {} 
\sphinxAtStartPar
\sphinxstyleliteralstrong{\sphinxupquote{raise\_error}} (\sphinxstyleliteralemphasis{\sphinxupquote{bool}}\sphinxstyleliteralemphasis{\sphinxupquote{, }}\sphinxstyleliteralemphasis{\sphinxupquote{optional}}\sphinxstyleliteralemphasis{\sphinxupquote{, }}\sphinxstyleliteralemphasis{\sphinxupquote{default False}}) \textendash{} 
\sphinxAtStartPar
Describes how \sphinxtitleref{array} is processed when its length exceeds the length of the operation period:
\begin{itemize}
\item {} 
\sphinxAtStartPar
If True, an AttributeError is raised.

\item {} 
\sphinxAtStartPar
If False, \sphinxtitleref{array} is sliced so that its first n elements are returned.

\end{itemize}


\end{itemize}

\sphinxlineitem{Returns}
\sphinxAtStartPar
\begin{itemize}
\item {} 
\sphinxAtStartPar
* AttributeError if \sphinxtitleref{raise\_error} is True and array is too long, or array is too short.

\item {} 
\sphinxAtStartPar
* The first n values of \sphinxtitleref{array} otherwise.

\end{itemize}


\end{description}\end{quote}

\end{fulllineitems}

\index{time\_steps (tamos.TimeSettings property)@\spxentry{time\_steps}\spxextra{tamos.TimeSettings property}}

\begin{fulllineitems}
\phantomsection\label{\detokenize{generated/tamos.TimeSettings:tamos.TimeSettings.time_steps}}
\pysigstartsignatures
\pysigline{\sphinxbfcode{\sphinxupquote{property\DUrole{w}{  }}}\sphinxbfcode{\sphinxupquote{time\_steps}}}
\pysigstopsignatures
\sphinxAtStartPar
Presents each time step of the temporal operation vector defined by its set of indexes.

\end{fulllineitems}


\end{fulllineitems}


\sphinxstepscope


\section{tamos.MILPModel}
\label{\detokenize{generated/tamos.MILPModel:tamos-milpmodel}}\label{\detokenize{generated/tamos.MILPModel::doc}}\index{MILPModel (class in tamos)@\spxentry{MILPModel}\spxextra{class in tamos}}

\begin{fulllineitems}
\phantomsection\label{\detokenize{generated/tamos.MILPModel:tamos.MILPModel}}
\pysigstartsignatures
\pysiglinewithargsret{\sphinxbfcode{\sphinxupquote{class\DUrole{w}{  }}}\sphinxcode{\sphinxupquote{tamos.}}\sphinxbfcode{\sphinxupquote{MILPModel}}}{\emph{\DUrole{n}{hubs}\DUrole{p}{:}\DUrole{w}{  }\DUrole{n}{{\hyperref[\detokenize{generated/tamos.Hub:tamos.Hub}]{\sphinxcrossref{Hub}}}}}, \emph{\DUrole{n}{time\_settings}}, \emph{\DUrole{n}{name}\DUrole{o}{=}\DUrole{default_value}{None}}}{}
\pysigstopsignatures\index{\_\_init\_\_() (tamos.MILPModel method)@\spxentry{\_\_init\_\_()}\spxextra{tamos.MILPModel method}}

\begin{fulllineitems}
\phantomsection\label{\detokenize{generated/tamos.MILPModel:tamos.MILPModel.__init__}}
\pysigstartsignatures
\pysiglinewithargsret{\sphinxbfcode{\sphinxupquote{\_\_init\_\_}}}{\emph{\DUrole{n}{hubs}\DUrole{p}{:}\DUrole{w}{  }\DUrole{n}{{\hyperref[\detokenize{generated/tamos.Hub:tamos.Hub}]{\sphinxcrossref{Hub}}}}}, \emph{\DUrole{n}{time\_settings}}, \emph{\DUrole{n}{name}\DUrole{o}{=}\DUrole{default_value}{None}}}{}
\pysigstopsignatures
\sphinxAtStartPar
Manages the declaration and solving processes of the MILP problem associated to the components of hubs \sphinxtitleref{hubs}.
\begin{quote}\begin{description}
\sphinxlineitem{Parameters}\begin{itemize}
\item {} 
\sphinxAtStartPar
\sphinxstyleliteralstrong{\sphinxupquote{hubs}} ({\hyperref[\detokenize{generated/tamos.Hub:tamos.Hub}]{\sphinxcrossref{\sphinxstyleliteralemphasis{\sphinxupquote{Hub}}}}}\sphinxstyleliteralemphasis{\sphinxupquote{ or }}\sphinxstyleliteralemphasis{\sphinxupquote{list of Hub}}) \textendash{} production, storage, element\_IO and network components related to \sphinxtitleref{hubs}
are sized and operated using \sphinxtitleref{time\_settings} parameters.

\item {} 
\sphinxAtStartPar
\sphinxstyleliteralstrong{\sphinxupquote{time\_settings}} (\sphinxstyleliteralemphasis{\sphinxupquote{TimeSettings instance}}) \textendash{} 

\item {} 
\sphinxAtStartPar
\sphinxstyleliteralstrong{\sphinxupquote{name}} (\sphinxstyleliteralemphasis{\sphinxupquote{str}}\sphinxstyleliteralemphasis{\sphinxupquote{, }}\sphinxstyleliteralemphasis{\sphinxupquote{optional}}) \textendash{} 

\end{itemize}

\end{description}\end{quote}

\end{fulllineitems}

\subsubsection*{Methods}


\begin{savenotes}\sphinxattablestart
\centering
\begin{tabulary}{\linewidth}[t]{\X{1}{2}\X{1}{2}}
\hline

\sphinxAtStartPar
{\hyperref[\detokenize{generated/tamos.MILPModel:tamos.MILPModel.__init__}]{\sphinxcrossref{\sphinxcode{\sphinxupquote{\_\_init\_\_}}}}}(hubs, time\_settings{[}, name{]})
&
\sphinxAtStartPar
Manages the declaration and solving processes of the MILP problem associated to the components of hubs \sphinxtitleref{hubs}.
\\
\hline
\sphinxAtStartPar
{\hyperref[\detokenize{generated/tamos.MILPModel:tamos.MILPModel.declare_constraints_and_KPIs}]{\sphinxcrossref{\sphinxcode{\sphinxupquote{declare\_constraints\_and\_KPIs}}}}}()
&
\sphinxAtStartPar
Declare constraints and KPIs associated with components related to \sphinxtitleref{hubs}.
\\
\hline
\sphinxAtStartPar
{\hyperref[\detokenize{generated/tamos.MILPModel:tamos.MILPModel.declare_max_KPI_constraint}]{\sphinxcrossref{\sphinxcode{\sphinxupquote{declare\_max\_KPI\_constraint}}}}}(kind, max\_value)
&
\sphinxAtStartPar
Constrains one of the objective KPI.
\\
\hline
\sphinxAtStartPar
{\hyperref[\detokenize{generated/tamos.MILPModel:tamos.MILPModel.declare_variables}]{\sphinxcrossref{\sphinxcode{\sphinxupquote{declare\_variables}}}}}()
&
\sphinxAtStartPar
Declare all the decision variables associated with components related to \sphinxtitleref{hubs}.
\\
\hline
\sphinxAtStartPar
{\hyperref[\detokenize{generated/tamos.MILPModel:tamos.MILPModel.remove_constraints_and_KPIs}]{\sphinxcrossref{\sphinxcode{\sphinxupquote{remove\_constraints\_and\_KPIs}}}}}({[}component{]})
&
\sphinxAtStartPar
{[}experimental, unstable{]} Removes from the MILP model some constraints and KPIs.
\\
\hline
\sphinxAtStartPar
{\hyperref[\detokenize{generated/tamos.MILPModel:tamos.MILPModel.remove_max_KPI_constraint}]{\sphinxcrossref{\sphinxcode{\sphinxupquote{remove\_max\_KPI\_constraint}}}}}(kind)
&
\sphinxAtStartPar
Removes the limit on the objective KPI of kind \sphinxtitleref{kind}, if it exists.
\\
\hline
\sphinxAtStartPar
{\hyperref[\detokenize{generated/tamos.MILPModel:tamos.MILPModel.solve}]{\sphinxcrossref{\sphinxcode{\sphinxupquote{solve}}}}}(kind{[}, MIP\_gap, threads, timelimit{]})
&
\sphinxAtStartPar
Solves the MILP model minimizing the objective function of kind \sphinxtitleref{kind}.
\\
\hline
\end{tabulary}
\par
\sphinxattableend\end{savenotes}
\subsubsection*{Attributes}


\begin{savenotes}\sphinxattablestart
\centering
\begin{tabulary}{\linewidth}[t]{\X{1}{2}\X{1}{2}}
\hline

\sphinxAtStartPar
{\hyperref[\detokenize{generated/tamos.MILPModel.components_assemblies:tamos.MILPModel.components_assemblies}]{\sphinxcrossref{\sphinxcode{\sphinxupquote{components\_assemblies}}}}}
&
\sphinxAtStartPar
Components assemblies of the model.
\\
\hline
\sphinxAtStartPar
{\hyperref[\detokenize{generated/tamos.MILPModel:tamos.MILPModel.description}]{\sphinxcrossref{\sphinxcode{\sphinxupquote{description}}}}}
&
\sphinxAtStartPar
Stores a description of the model.
\\
\hline
\sphinxAtStartPar
{\hyperref[\detokenize{generated/tamos.MILPModel:tamos.MILPModel.hubs}]{\sphinxcrossref{\sphinxcode{\sphinxupquote{hubs}}}}}
&
\sphinxAtStartPar
production, storage, element\_IO and network components related to \sphinxtitleref{hubs} are sized and operated using \sphinxtitleref{time\_settings} parameters.
\\
\hline
\sphinxAtStartPar
{\hyperref[\detokenize{generated/tamos.MILPModel:tamos.MILPModel.name}]{\sphinxcrossref{\sphinxcode{\sphinxupquote{name}}}}}
&
\sphinxAtStartPar
Name of this MILPModel instance.
\\
\hline
\end{tabulary}
\par
\sphinxattableend\end{savenotes}
\index{components\_assemblies (tamos.MILPModel property)@\spxentry{components\_assemblies}\spxextra{tamos.MILPModel property}}

\begin{fulllineitems}
\phantomsection\label{\detokenize{generated/tamos.MILPModel:tamos.MILPModel.components_assemblies}}
\pysigstartsignatures
\pysigline{\sphinxbfcode{\sphinxupquote{property\DUrole{w}{  }}}\sphinxbfcode{\sphinxupquote{components\_assemblies}}}
\pysigstopsignatures
\sphinxAtStartPar
Components assemblies of the model.

\sphinxAtStartPar
Must be provided as a list of 3\sphinxhyphen{}tuple objects (n\_min, n\_max, components) where:
\begin{itemize}
\item {} 
\sphinxAtStartPar
\sphinxtitleref{n\_min} (\sphinxtitleref{n\_max}) is the minimum (maximum) number of components from \sphinxtitleref{components} that must be installed, all hubs of \sphinxtitleref{hubs} included.

\item {} 
\sphinxAtStartPar
\sphinxtitleref{components} is a component or list of production, storage or element\_IO components.
For any component of \sphinxtitleref{components}, if this component is not in at least one hub of this MILPModel instance,
the 3\sphinxhyphen{}tuple (n\_min, n\_max, components) is ignored during constraints declaration.

\end{itemize}
\subsubsection*{Examples}

\begin{sphinxVerbatim}[commandchars=\\\{\}]
\PYG{g+gp}{\PYGZgt{}\PYGZgt{}\PYGZgt{} }\PYG{n}{hub\PYGZus{}1} \PYG{o}{=} \PYG{n}{Hub}\PYG{p}{(}\PYG{n}{components}\PYG{o}{=}\PYG{p}{[}\PYG{n}{heat\PYGZus{}load\PYGZus{}1}\PYG{p}{,} \PYG{n}{heat\PYGZus{}load\PYGZus{}2}\PYG{p}{]}\PYG{p}{)}
\PYG{g+gp}{\PYGZgt{}\PYGZgt{}\PYGZgt{} }\PYG{n}{hub\PYGZus{}2} \PYG{o}{=} \PYG{n}{Hub}\PYG{p}{(}\PYG{n}{components}\PYG{o}{=}\PYG{p}{[}\PYG{n}{heat\PYGZus{}load\PYGZus{}1}\PYG{p}{]}\PYG{p}{)}
\PYG{g+gp}{\PYGZgt{}\PYGZgt{}\PYGZgt{} }\PYG{n}{hub\PYGZus{}3} \PYG{o}{=} \PYG{n}{Hub}\PYG{p}{(}\PYG{n}{components}\PYG{o}{=}\PYG{p}{[}\PYG{n}{heat\PYGZus{}load\PYGZus{}2}\PYG{p}{]}\PYG{p}{)}
\PYG{g+gp}{\PYGZgt{}\PYGZgt{}\PYGZgt{} }\PYG{n}{MILPModel} \PYG{o}{=} \PYG{n}{MILPModel}\PYG{p}{(}\PYG{n}{hubs}\PYG{o}{=}\PYG{p}{[}\PYG{n}{hub\PYGZus{}1}\PYG{p}{,} \PYG{n}{hub\PYGZus{}2}\PYG{p}{,} \PYG{n}{hub\PYGZus{}3}\PYG{p}{]}\PYG{p}{)}
\PYG{g+gp}{\PYGZgt{}\PYGZgt{}\PYGZgt{} }\PYG{n}{MILPModel}\PYG{o}{.}\PYG{n}{components\PYGZus{}assemblies} \PYG{o}{=} \PYG{p}{[}\PYG{p}{(}\PYG{l+m+mi}{0}\PYG{p}{,} \PYG{l+m+mi}{1}\PYG{p}{,} \PYG{n}{heat\PYGZus{}load\PYGZus{}1}\PYG{p}{)}\PYG{p}{,} \PYG{p}{(}\PYG{l+m+mi}{2}\PYG{p}{,} \PYG{l+m+mi}{3}\PYG{p}{,} \PYG{p}{[}\PYG{n}{heat\PYGZus{}load\PYGZus{}1}\PYG{p}{,} \PYG{n}{heat\PYGZus{}load\PYGZus{}2}\PYG{p}{]}\PYG{p}{)}\PYG{p}{]}
\end{sphinxVerbatim}

\sphinxAtStartPar
heat\_load\_1 might be used at most one time, all hubs {[}hub\_1, hub\_2, hub\_3{]} included.
the number of times heat\_load\_1 or heat\_load\_2 are used in all hubs {[}hub\_1, hub\_2, hub\_3{]}
is greater than 2 but smaller than 3.

\end{fulllineitems}

\index{declare\_constraints\_and\_KPIs() (tamos.MILPModel method)@\spxentry{declare\_constraints\_and\_KPIs()}\spxextra{tamos.MILPModel method}}

\begin{fulllineitems}
\phantomsection\label{\detokenize{generated/tamos.MILPModel:tamos.MILPModel.declare_constraints_and_KPIs}}
\pysigstartsignatures
\pysiglinewithargsret{\sphinxbfcode{\sphinxupquote{declare\_constraints\_and\_KPIs}}}{}{}
\pysigstopsignatures
\sphinxAtStartPar
Declare constraints and KPIs associated with components related to \sphinxtitleref{hubs}.
All KPIs are defined, i.e. covering each kind of objective function (Eco, Exergy, CO2).

\end{fulllineitems}

\index{declare\_max\_KPI\_constraint() (tamos.MILPModel method)@\spxentry{declare\_max\_KPI\_constraint()}\spxextra{tamos.MILPModel method}}

\begin{fulllineitems}
\phantomsection\label{\detokenize{generated/tamos.MILPModel:tamos.MILPModel.declare_max_KPI_constraint}}
\pysigstartsignatures
\pysiglinewithargsret{\sphinxbfcode{\sphinxupquote{declare\_max\_KPI\_constraint}}}{\emph{\DUrole{n}{kind}}, \emph{\DUrole{n}{max\_value}}}{}
\pysigstopsignatures
\sphinxAtStartPar
Constrains one of the objective KPI.
The values of these objective KPI are described in the \sphinxtitleref{solution\_summary} attribute of ResultsExport instances.
This constraint can be removed by a call to \sphinxtitleref{remove\_max\_KPI\_constraint(kind)}.
\begin{quote}\begin{description}
\sphinxlineitem{Parameters}\begin{itemize}
\item {} 
\sphinxAtStartPar
\sphinxstyleliteralstrong{\sphinxupquote{kind}} (\sphinxstyleliteralemphasis{\sphinxupquote{\{\textquotesingle{}Eco\textquotesingle{}}}\sphinxstyleliteralemphasis{\sphinxupquote{, }}\sphinxstyleliteralemphasis{\sphinxupquote{\textquotesingle{}CO2\textquotesingle{}}}\sphinxstyleliteralemphasis{\sphinxupquote{, }}\sphinxstyleliteralemphasis{\sphinxupquote{\textquotesingle{}Exergy\textquotesingle{}\}}}) \textendash{} 

\item {} 
\sphinxAtStartPar
\sphinxstyleliteralstrong{\sphinxupquote{max\_value}} (\sphinxstyleliteralemphasis{\sphinxupquote{float}}) \textendash{} 

\end{itemize}

\end{description}\end{quote}
\subsubsection*{Notes}
\begin{enumerate}
\sphinxsetlistlabels{\arabic}{enumi}{enumii}{}{.}%
\item {} 
\sphinxAtStartPar
This method calls \sphinxtitleref{remove\_max\_KPI\_constraint(kind)} if necessary to remove any already defined constraint
regarding kind \sphinxtitleref{kind}.

\item {} 
\sphinxAtStartPar
A constraint on kind \sphinxtitleref{kind} can be defined no matter the kind of objective function.

\item {} 
\sphinxAtStartPar
More than one kind can be constrained at a time.

\end{enumerate}
\subsubsection*{Examples}

\begin{sphinxVerbatim}[commandchars=\\\{\}]
\PYG{g+gp}{\PYGZgt{}\PYGZgt{}\PYGZgt{} }\PYG{n}{MILPModel}\PYG{o}{.}\PYG{n}{declare\PYGZus{}max\PYGZus{}KPI\PYGZus{}constraint}\PYG{p}{(}\PYG{l+s+s2}{\PYGZdq{}}\PYG{l+s+s2}{Eco}\PYG{l+s+s2}{\PYGZdq{}}\PYG{p}{,} \PYG{l+m+mf}{1e6}\PYG{p}{)}
\end{sphinxVerbatim}

\sphinxAtStartPar
The total cost of the system over its lifetime cannot exceed 1 million euros.
\textgreater{}\textgreater{}\textgreater{} MILPModel.declare\_max\_KPI\_constraint(“CO2”, 2e5)

\sphinxAtStartPar
The net CO2 emissions of the system over the operation period (net = entering \sphinxhyphen{} exiting) cannot exceed 200 tEqCO2
\textgreater{}\textgreater{}\textgreater{} MILPModel.declare\_max\_KPI\_constraint(“Exergy”, 1e5)

\sphinxAtStartPar
The net exergetical potential consumed by the system over the operation period (net = entering \sphinxhyphen{} exiting) cannot exceed 1e5 kWh.
\textgreater{}\textgreater{}\textgreater{} MILPModel.declare\_max\_KPI\_constraint(“Eco”, \sphinxhyphen{}1e5)

\sphinxAtStartPar
The system must be economically profitable and generate at least 100 000 euros over its lifetime.

\end{fulllineitems}

\index{declare\_variables() (tamos.MILPModel method)@\spxentry{declare\_variables()}\spxextra{tamos.MILPModel method}}

\begin{fulllineitems}
\phantomsection\label{\detokenize{generated/tamos.MILPModel:tamos.MILPModel.declare_variables}}
\pysigstartsignatures
\pysiglinewithargsret{\sphinxbfcode{\sphinxupquote{declare\_variables}}}{}{}
\pysigstopsignatures
\sphinxAtStartPar
Declare all the decision variables associated with components related to \sphinxtitleref{hubs}.

\end{fulllineitems}

\index{description (tamos.MILPModel property)@\spxentry{description}\spxextra{tamos.MILPModel property}}

\begin{fulllineitems}
\phantomsection\label{\detokenize{generated/tamos.MILPModel:tamos.MILPModel.description}}
\pysigstartsignatures
\pysigline{\sphinxbfcode{\sphinxupquote{property\DUrole{w}{  }}}\sphinxbfcode{\sphinxupquote{description}}}
\pysigstopsignatures
\sphinxAtStartPar
Stores a description of the model.
These properties are used to gather ResultsExport objects written on disk and create ResultsBatch objects.
They define the \sphinxtitleref{relevant\_descriptors} attribute of ResultsBatch objects.

\sphinxAtStartPar
Some of these properties are automatically declared by tamos. The other are user\sphinxhyphen{}defined.
description: dict
\begin{itemize}
\item {} 
\sphinxAtStartPar
keys are str

\item {} 
\sphinxAtStartPar
values are int, float or str

\end{itemize}

\end{fulllineitems}

\index{hubs (tamos.MILPModel property)@\spxentry{hubs}\spxextra{tamos.MILPModel property}}

\begin{fulllineitems}
\phantomsection\label{\detokenize{generated/tamos.MILPModel:tamos.MILPModel.hubs}}
\pysigstartsignatures
\pysigline{\sphinxbfcode{\sphinxupquote{property\DUrole{w}{  }}}\sphinxbfcode{\sphinxupquote{hubs}}}
\pysigstopsignatures
\sphinxAtStartPar
production, storage, element\_IO and network components related to \sphinxtitleref{hubs}
are sized and operated using \sphinxtitleref{time\_settings} parameters.

\end{fulllineitems}

\index{name (tamos.MILPModel property)@\spxentry{name}\spxextra{tamos.MILPModel property}}

\begin{fulllineitems}
\phantomsection\label{\detokenize{generated/tamos.MILPModel:tamos.MILPModel.name}}
\pysigstartsignatures
\pysigline{\sphinxbfcode{\sphinxupquote{property\DUrole{w}{  }}}\sphinxbfcode{\sphinxupquote{name}}}
\pysigstopsignatures
\sphinxAtStartPar
Name of this MILPModel instance. If None, name is the current time.

\end{fulllineitems}

\index{remove\_constraints\_and\_KPIs() (tamos.MILPModel method)@\spxentry{remove\_constraints\_and\_KPIs()}\spxextra{tamos.MILPModel method}}

\begin{fulllineitems}
\phantomsection\label{\detokenize{generated/tamos.MILPModel:tamos.MILPModel.remove_constraints_and_KPIs}}
\pysigstartsignatures
\pysiglinewithargsret{\sphinxbfcode{\sphinxupquote{remove\_constraints\_and\_KPIs}}}{\emph{\DUrole{n}{component}\DUrole{o}{=}\DUrole{default_value}{\textquotesingle{}all\textquotesingle{}}}}{}
\pysigstopsignatures
\sphinxAtStartPar
{[}experimental, unstable{]}
Removes from the MILP model some constraints and KPIs. The next call to \sphinxtitleref{declare\_constraints\_and\_KPIs}
will only redefine the missing constraints and KPIs, saving model declaration time.
Relevant after a component modification.
\begin{quote}\begin{description}
\sphinxlineitem{Parameters}
\sphinxAtStartPar
\sphinxstyleliteralstrong{\sphinxupquote{component}} (\sphinxstyleliteralemphasis{\sphinxupquote{Four possible kinds of arg are accepted:}}) \textendash{} \begin{itemize}
\item {} 
\sphinxAtStartPar
‘all’: all constraints and KPIs of the MILP model are removed (default)

\item {} 
\sphinxAtStartPar
None: assembly\_constraints of this MILPModel instance are removed.
To remove and define constraints on Eco, Exergy and CO2 KPIs, please use the dedicated methods.

\item {} 
\sphinxAtStartPar
production, storage, element\_IO, or network component: constraints and KPIs of the corresponding component
are removed, no matter the hub(s) they belong or are related to.

\item {} 
\sphinxAtStartPar
hub: constraints and KPIs related to the hub are removed. This includes:
\begin{itemize}
\item {} 
\sphinxAtStartPar
hub assemblies constraints

\item {} 
\sphinxAtStartPar
constraints and KPIs of all production, storage and element\_IO components of hub

\item {} 
\sphinxAtStartPar
hub interface constraints (the ones that bind every element exchanges between components)

\end{itemize}

\end{itemize}


\end{description}\end{quote}
\subsubsection*{Notes}
\begin{enumerate}
\sphinxsetlistlabels{\arabic}{enumi}{enumii}{}{.}%
\item {} 
\sphinxAtStartPar
In case component is a production, storage, element\_IO, or network component.
Constraints and KPIs removal of component does not affect the contribution of the component to the hub interface.
Unconstrained energy flows will occur if constraints and KPIs are not declared again after call to this method.

\item {} 
\sphinxAtStartPar
Some model modifications require a complete model redeclaration (variables, constraints, KPIs). These cases are:
\begin{itemize}
\item {} 
\sphinxAtStartPar
Some components were added or removed from hubs (production, storage, element\_IO)

\item {} 
\sphinxAtStartPar
New networks are used

\item {} 
\sphinxAtStartPar
\sphinxtitleref{time\_steps} attribute of \sphinxtitleref{time\_settings} was modified

\end{itemize}

\sphinxAtStartPar
In these cases, variable redeclaration is automatically performed.

\item {} 
\sphinxAtStartPar
Regarding constraints and KPIs, \sphinxtitleref{Cost} instances are bound to the element\_IO component they describe.

\item {} 
\sphinxAtStartPar
Regarding constraints and KPIs, \sphinxtitleref{InterfaceMask} instances are bound to the component they describe.

\end{enumerate}

\end{fulllineitems}

\index{remove\_max\_KPI\_constraint() (tamos.MILPModel method)@\spxentry{remove\_max\_KPI\_constraint()}\spxextra{tamos.MILPModel method}}

\begin{fulllineitems}
\phantomsection\label{\detokenize{generated/tamos.MILPModel:tamos.MILPModel.remove_max_KPI_constraint}}
\pysigstartsignatures
\pysiglinewithargsret{\sphinxbfcode{\sphinxupquote{remove\_max\_KPI\_constraint}}}{\emph{\DUrole{n}{kind}}}{}
\pysigstopsignatures
\sphinxAtStartPar
Removes the limit on the objective KPI of kind \sphinxtitleref{kind}, if it exists.
\begin{quote}\begin{description}
\sphinxlineitem{Parameters}
\sphinxAtStartPar
\sphinxstyleliteralstrong{\sphinxupquote{kind}} (\sphinxstyleliteralemphasis{\sphinxupquote{\{\textquotesingle{}Eco\textquotesingle{}}}\sphinxstyleliteralemphasis{\sphinxupquote{, }}\sphinxstyleliteralemphasis{\sphinxupquote{\textquotesingle{}CO2\textquotesingle{}}}\sphinxstyleliteralemphasis{\sphinxupquote{, }}\sphinxstyleliteralemphasis{\sphinxupquote{\textquotesingle{}Exergy\textquotesingle{}\}}}) \textendash{} 

\end{description}\end{quote}

\end{fulllineitems}

\index{solve() (tamos.MILPModel method)@\spxentry{solve()}\spxextra{tamos.MILPModel method}}

\begin{fulllineitems}
\phantomsection\label{\detokenize{generated/tamos.MILPModel:tamos.MILPModel.solve}}
\pysigstartsignatures
\pysiglinewithargsret{\sphinxbfcode{\sphinxupquote{solve}}}{\emph{\DUrole{n}{kind}}, \emph{\DUrole{n}{MIP\_gap}\DUrole{o}{=}\DUrole{default_value}{0.0001}}, \emph{\DUrole{n}{threads}\DUrole{o}{=}\DUrole{default_value}{0}}, \emph{\DUrole{n}{timelimit}\DUrole{o}{=}\DUrole{default_value}{43200}}}{}
\pysigstopsignatures
\sphinxAtStartPar
Solves the MILP model minimizing the objective function of kind \sphinxtitleref{kind}.
Model is solved using the Cplex solver. To use another MILP solver, please export the model in LP or MPS format
using a ResultsExport instance.
\begin{quote}\begin{description}
\sphinxlineitem{Parameters}\begin{itemize}
\item {} 
\sphinxAtStartPar
\sphinxstyleliteralstrong{\sphinxupquote{kind}} (\sphinxstyleliteralemphasis{\sphinxupquote{\{\textquotesingle{}Eco\textquotesingle{}}}\sphinxstyleliteralemphasis{\sphinxupquote{, }}\sphinxstyleliteralemphasis{\sphinxupquote{\textquotesingle{}CO2\textquotesingle{}}}\sphinxstyleliteralemphasis{\sphinxupquote{, }}\sphinxstyleliteralemphasis{\sphinxupquote{\textquotesingle{}Exergy\textquotesingle{}\}}}) \textendash{} 

\item {} 
\sphinxAtStartPar
\sphinxstyleliteralstrong{\sphinxupquote{MIP\_gap}} (\sphinxstyleliteralemphasis{\sphinxupquote{float}}\sphinxstyleliteralemphasis{\sphinxupquote{, }}\sphinxstyleliteralemphasis{\sphinxupquote{optional}}\sphinxstyleliteralemphasis{\sphinxupquote{, }}\sphinxstyleliteralemphasis{\sphinxupquote{default 1e\sphinxhyphen{}4}}) \textendash{} 0 \textless{}= MIG\_gap \textless{}= 1
According to Cplex documentation, gap between the best integer objective and the objective of the best node remaining.

\item {} 
\sphinxAtStartPar
\sphinxstyleliteralstrong{\sphinxupquote{threads}} (\sphinxstyleliteralemphasis{\sphinxupquote{int}}\sphinxstyleliteralemphasis{\sphinxupquote{, }}\sphinxstyleliteralemphasis{\sphinxupquote{optional}}\sphinxstyleliteralemphasis{\sphinxupquote{, }}\sphinxstyleliteralemphasis{\sphinxupquote{default 0}}) \textendash{} 
\sphinxAtStartPar
0 \textless{}= threads
According to Cplex documentation, maximal number of parallel threads that will be invoked by any CPLEX parallel optimizer.
\begin{itemize}
\item {} 
\sphinxAtStartPar
If 0, let Cplex decide

\item {} 
\sphinxAtStartPar
Else, uses up to \sphinxtitleref{threads} threads

\end{itemize}


\item {} 
\sphinxAtStartPar
\sphinxstyleliteralstrong{\sphinxupquote{timelimit}} (\sphinxstyleliteralemphasis{\sphinxupquote{int}}\sphinxstyleliteralemphasis{\sphinxupquote{, }}\sphinxstyleliteralemphasis{\sphinxupquote{optional}}\sphinxstyleliteralemphasis{\sphinxupquote{, }}\sphinxstyleliteralemphasis{\sphinxupquote{default 43200}}) \textendash{} 0 \textless{} timelimit
According to Cplex documentation, maximum time, in seconds, for a call to an optimizer.

\end{itemize}

\end{description}\end{quote}
\subsubsection*{Notes}

\sphinxAtStartPar
Other Cplex parameters can be set by accessing the attribute \sphinxtitleref{\_model\_data.mdl.parameters} of this \sphinxtitleref{MILPModel} instance.
For instance:
\textgreater{}\textgreater{}\textgreater{} MILPModel.\_model\_data.mdl.parameters.mip.cuts.flowcovers.set(2)

\end{fulllineitems}


\end{fulllineitems}


\sphinxstepscope


\chapter{Energy vectors}
\label{\detokenize{element:energy-vectors}}\label{\detokenize{element::doc}}

\section{Elements}
\label{\detokenize{element:elements}}

\begin{savenotes}\sphinxattablestart
\centering
\begin{tabulary}{\linewidth}[t]{\X{1}{2}\X{1}{2}}
\hline

\sphinxAtStartPar
{\hyperref[\detokenize{generated/tamos.element.ElectricityVector:tamos.element.ElectricityVector}]{\sphinxcrossref{\sphinxcode{\sphinxupquote{tamos.element.ElectricityVector}}}}}({[}name{]})
&
\sphinxAtStartPar

\\
\hline
\sphinxAtStartPar
{\hyperref[\detokenize{generated/tamos.element.FuelVector:tamos.element.FuelVector}]{\sphinxcrossref{\sphinxcode{\sphinxupquote{tamos.element.FuelVector}}}}}(exergy\_factor{[}, name{]})
&
\sphinxAtStartPar

\\
\hline
\sphinxAtStartPar
{\hyperref[\detokenize{generated/tamos.element.ThermalVector:tamos.element.ThermalVector}]{\sphinxcrossref{\sphinxcode{\sphinxupquote{tamos.element.ThermalVector}}}}}(temperature{[}, name{]})
&
\sphinxAtStartPar

\\
\hline
\end{tabulary}
\par
\sphinxattableend\end{savenotes}

\sphinxstepscope


\subsection{tamos.element.ElectricityVector}
\label{\detokenize{generated/tamos.element.ElectricityVector:tamos-element-electricityvector}}\label{\detokenize{generated/tamos.element.ElectricityVector::doc}}\index{ElectricityVector (class in tamos.element)@\spxentry{ElectricityVector}\spxextra{class in tamos.element}}

\begin{fulllineitems}
\phantomsection\label{\detokenize{generated/tamos.element.ElectricityVector:tamos.element.ElectricityVector}}
\pysigstartsignatures
\pysiglinewithargsret{\sphinxbfcode{\sphinxupquote{class\DUrole{w}{  }}}\sphinxcode{\sphinxupquote{tamos.element.}}\sphinxbfcode{\sphinxupquote{ElectricityVector}}}{\emph{\DUrole{n}{name}\DUrole{o}{=}\DUrole{default_value}{None}}}{}
\pysigstopsignatures\index{\_\_init\_\_() (tamos.element.ElectricityVector method)@\spxentry{\_\_init\_\_()}\spxextra{tamos.element.ElectricityVector method}}

\begin{fulllineitems}
\phantomsection\label{\detokenize{generated/tamos.element.ElectricityVector:tamos.element.ElectricityVector.__init__}}
\pysigstartsignatures
\pysiglinewithargsret{\sphinxbfcode{\sphinxupquote{\_\_init\_\_}}}{\emph{\DUrole{n}{name}\DUrole{o}{=}\DUrole{default_value}{None}}}{}
\pysigstopsignatures
\sphinxAtStartPar
ElectricityVector instances are only defined by an exergy factor (default to 1).
Power is exchanged in kW.
\begin{quote}\begin{description}
\sphinxlineitem{Parameters}
\sphinxAtStartPar
\sphinxstyleliteralstrong{\sphinxupquote{name}} (\sphinxstyleliteralemphasis{\sphinxupquote{str}}\sphinxstyleliteralemphasis{\sphinxupquote{, }}\sphinxstyleliteralemphasis{\sphinxupquote{optional}}) \textendash{} 

\end{description}\end{quote}
\subsubsection*{Notes}

\sphinxAtStartPar
The exergy factor can be redefined using the \sphinxtitleref{exergy\_factor} attribute.

\end{fulllineitems}

\subsubsection*{Methods}


\begin{savenotes}\sphinxattablestart
\centering
\begin{tabulary}{\linewidth}[t]{\X{1}{2}\X{1}{2}}
\hline

\sphinxAtStartPar
{\hyperref[\detokenize{generated/tamos.element.ElectricityVector:tamos.element.ElectricityVector.__init__}]{\sphinxcrossref{\sphinxcode{\sphinxupquote{\_\_init\_\_}}}}}({[}name{]})
&
\sphinxAtStartPar
ElectricityVector instances are only defined by an exergy factor (default to 1).
\\
\hline
\sphinxAtStartPar
{\hyperref[\detokenize{generated/tamos.element.ElectricityVector:tamos.element.ElectricityVector.get_vectors}]{\sphinxcrossref{\sphinxcode{\sphinxupquote{get\_vectors}}}}}()
&
\sphinxAtStartPar
Returns this instance.
\\
\hline
\end{tabulary}
\par
\sphinxattableend\end{savenotes}
\subsubsection*{Attributes}


\begin{savenotes}\sphinxattablestart
\centering
\begin{tabulary}{\linewidth}[t]{\X{1}{2}\X{1}{2}}
\hline

\sphinxAtStartPar
{\hyperref[\detokenize{generated/tamos.element.ElectricityVector:tamos.element.ElectricityVector.exergy_factor}]{\sphinxcrossref{\sphinxcode{\sphinxupquote{exergy\_factor}}}}}
&
\sphinxAtStartPar
Quantity of exergy associated with a 1 kW power flow of this element.
\\
\hline
\sphinxAtStartPar
{\hyperref[\detokenize{generated/tamos.element.ElectricityVector:tamos.element.ElectricityVector.name}]{\sphinxcrossref{\sphinxcode{\sphinxupquote{name}}}}}
&
\sphinxAtStartPar
str.
\\
\hline
\end{tabulary}
\par
\sphinxattableend\end{savenotes}
\index{exergy\_factor (tamos.element.ElectricityVector property)@\spxentry{exergy\_factor}\spxextra{tamos.element.ElectricityVector property}}

\begin{fulllineitems}
\phantomsection\label{\detokenize{generated/tamos.element.ElectricityVector:tamos.element.ElectricityVector.exergy_factor}}
\pysigstartsignatures
\pysigline{\sphinxbfcode{\sphinxupquote{property\DUrole{w}{  }}}\sphinxbfcode{\sphinxupquote{exergy\_factor}}}
\pysigstopsignatures
\sphinxAtStartPar
Quantity of exergy associated with a 1 kW power flow of this element.
Theoretically positive, usually smaller than 1. In kW/kW.

\sphinxAtStartPar
int, float or numpy.ndarray
0 \textless{}= exergy\_factor
\subsubsection*{Notes}

\sphinxAtStartPar
FuelVector instances are exchanged between components considering their LHV value.

\end{fulllineitems}

\index{get\_vectors() (tamos.element.ElectricityVector method)@\spxentry{get\_vectors()}\spxextra{tamos.element.ElectricityVector method}}

\begin{fulllineitems}
\phantomsection\label{\detokenize{generated/tamos.element.ElectricityVector:tamos.element.ElectricityVector.get_vectors}}
\pysigstartsignatures
\pysiglinewithargsret{\sphinxbfcode{\sphinxupquote{get\_vectors}}}{}{}
\pysigstopsignatures
\sphinxAtStartPar
Returns this instance.

\end{fulllineitems}

\index{name (tamos.element.ElectricityVector property)@\spxentry{name}\spxextra{tamos.element.ElectricityVector property}}

\begin{fulllineitems}
\phantomsection\label{\detokenize{generated/tamos.element.ElectricityVector:tamos.element.ElectricityVector.name}}
\pysigstartsignatures
\pysigline{\sphinxbfcode{\sphinxupquote{property\DUrole{w}{  }}}\sphinxbfcode{\sphinxupquote{name}}}
\pysigstopsignatures
\sphinxAtStartPar
str.
This name is used in MILP model description.
names must not exceed 255 characters,
all of which must be alphanumeric (a\sphinxhyphen{}z, A\sphinxhyphen{}Z, 0\sphinxhyphen{}9) or one of these symbols:
! ” \# \$ \% \& , . ; ? @ \_ ‘ ’ \{ \} \textasciitilde{}.
\begin{quote}\begin{description}
\sphinxlineitem{Type}
\sphinxAtStartPar
Name of the instance

\end{description}\end{quote}

\end{fulllineitems}


\end{fulllineitems}


\sphinxstepscope


\subsection{tamos.element.FuelVector}
\label{\detokenize{generated/tamos.element.FuelVector:tamos-element-fuelvector}}\label{\detokenize{generated/tamos.element.FuelVector::doc}}\index{FuelVector (class in tamos.element)@\spxentry{FuelVector}\spxextra{class in tamos.element}}

\begin{fulllineitems}
\phantomsection\label{\detokenize{generated/tamos.element.FuelVector:tamos.element.FuelVector}}
\pysigstartsignatures
\pysiglinewithargsret{\sphinxbfcode{\sphinxupquote{class\DUrole{w}{  }}}\sphinxcode{\sphinxupquote{tamos.element.}}\sphinxbfcode{\sphinxupquote{FuelVector}}}{\emph{\DUrole{n}{exergy\_factor}}, \emph{\DUrole{n}{name}\DUrole{o}{=}\DUrole{default_value}{None}}}{}
\pysigstopsignatures\index{\_\_init\_\_() (tamos.element.FuelVector method)@\spxentry{\_\_init\_\_()}\spxextra{tamos.element.FuelVector method}}

\begin{fulllineitems}
\phantomsection\label{\detokenize{generated/tamos.element.FuelVector:tamos.element.FuelVector.__init__}}
\pysigstartsignatures
\pysiglinewithargsret{\sphinxbfcode{\sphinxupquote{\_\_init\_\_}}}{\emph{\DUrole{n}{exergy\_factor}}, \emph{\DUrole{n}{name}\DUrole{o}{=}\DUrole{default_value}{None}}}{}
\pysigstopsignatures
\sphinxAtStartPar
A FuelVector instance may describe every energy vector whose exergy factor does not depend on a temperature
(i.e. ThermalVector, ThermalVectorPair) or is not unity (i.e. ElectricityVector).
Power is exchanged in kW.

\sphinxAtStartPar
These vectors are usually associated with a combustion flame temperature which can be used by the user
to define the exergy factor according to:
exergy\_factor = 1 \sphinxhyphen{} dead\_state\_temperature / flame temperature.
\begin{quote}\begin{description}
\sphinxlineitem{Parameters}\begin{itemize}
\item {} 
\sphinxAtStartPar
\sphinxstyleliteralstrong{\sphinxupquote{exergy\_factor}} (\sphinxstyleliteralemphasis{\sphinxupquote{float}}) \textendash{} 0 \textless{}= exergy\_factor

\item {} 
\sphinxAtStartPar
\sphinxstyleliteralstrong{\sphinxupquote{name}} (\sphinxstyleliteralemphasis{\sphinxupquote{str}}\sphinxstyleliteralemphasis{\sphinxupquote{, }}\sphinxstyleliteralemphasis{\sphinxupquote{optional}}) \textendash{} 

\end{itemize}

\end{description}\end{quote}
\subsubsection*{Notes}

\sphinxAtStartPar
1. The dead state temperature is the reference temperature for calculation of exergy factor of ThermalVectorPair and ThermalVector instances.
It can be modified using the \sphinxtitleref{set\_dead\_state\_temperature} function of tamos.element.
2. Some typical flame temperatures are stored in the class attribute \sphinxtitleref{FuelVector.typical\_flame\_temperature} (natural gas, biomass).
3. Power flows related to a FuelVector take for reference the lower heating value of the fuel. This impacts the efficiency
models in production components.
4. The exergy factor can be redefined using the \sphinxtitleref{exergy\_factor} attribute.

\end{fulllineitems}

\subsubsection*{Methods}


\begin{savenotes}\sphinxattablestart
\centering
\begin{tabulary}{\linewidth}[t]{\X{1}{2}\X{1}{2}}
\hline

\sphinxAtStartPar
{\hyperref[\detokenize{generated/tamos.element.FuelVector:tamos.element.FuelVector.__init__}]{\sphinxcrossref{\sphinxcode{\sphinxupquote{\_\_init\_\_}}}}}(exergy\_factor{[}, name{]})
&
\sphinxAtStartPar
A FuelVector instance may describe every energy vector whose exergy factor does not depend on a temperature (i.e.
\\
\hline
\sphinxAtStartPar
{\hyperref[\detokenize{generated/tamos.element.FuelVector:tamos.element.FuelVector.get_vectors}]{\sphinxcrossref{\sphinxcode{\sphinxupquote{get\_vectors}}}}}()
&
\sphinxAtStartPar
Returns this instance.
\\
\hline
\end{tabulary}
\par
\sphinxattableend\end{savenotes}
\subsubsection*{Attributes}


\begin{savenotes}\sphinxattablestart
\centering
\begin{tabulary}{\linewidth}[t]{\X{1}{2}\X{1}{2}}
\hline

\sphinxAtStartPar
{\hyperref[\detokenize{generated/tamos.element.FuelVector:tamos.element.FuelVector.exergy_factor}]{\sphinxcrossref{\sphinxcode{\sphinxupquote{exergy\_factor}}}}}
&
\sphinxAtStartPar
Quantity of exergy associated with a 1 kW power flow of this element.
\\
\hline
\sphinxAtStartPar
{\hyperref[\detokenize{generated/tamos.element.FuelVector:tamos.element.FuelVector.name}]{\sphinxcrossref{\sphinxcode{\sphinxupquote{name}}}}}
&
\sphinxAtStartPar
str.
\\
\hline
\sphinxAtStartPar
\sphinxcode{\sphinxupquote{typical\_flame\_temperature}}
&
\sphinxAtStartPar

\\
\hline
\end{tabulary}
\par
\sphinxattableend\end{savenotes}
\index{exergy\_factor (tamos.element.FuelVector property)@\spxentry{exergy\_factor}\spxextra{tamos.element.FuelVector property}}

\begin{fulllineitems}
\phantomsection\label{\detokenize{generated/tamos.element.FuelVector:tamos.element.FuelVector.exergy_factor}}
\pysigstartsignatures
\pysigline{\sphinxbfcode{\sphinxupquote{property\DUrole{w}{  }}}\sphinxbfcode{\sphinxupquote{exergy\_factor}}}
\pysigstopsignatures
\sphinxAtStartPar
Quantity of exergy associated with a 1 kW power flow of this element.
Theoretically positive, usually smaller than 1. In kW/kW.

\sphinxAtStartPar
int, float or numpy.ndarray
0 \textless{}= exergy\_factor
\subsubsection*{Notes}

\sphinxAtStartPar
FuelVector instances are exchanged between components considering their LHV value.

\end{fulllineitems}

\index{get\_vectors() (tamos.element.FuelVector method)@\spxentry{get\_vectors()}\spxextra{tamos.element.FuelVector method}}

\begin{fulllineitems}
\phantomsection\label{\detokenize{generated/tamos.element.FuelVector:tamos.element.FuelVector.get_vectors}}
\pysigstartsignatures
\pysiglinewithargsret{\sphinxbfcode{\sphinxupquote{get\_vectors}}}{}{}
\pysigstopsignatures
\sphinxAtStartPar
Returns this instance.

\end{fulllineitems}

\index{name (tamos.element.FuelVector property)@\spxentry{name}\spxextra{tamos.element.FuelVector property}}

\begin{fulllineitems}
\phantomsection\label{\detokenize{generated/tamos.element.FuelVector:tamos.element.FuelVector.name}}
\pysigstartsignatures
\pysigline{\sphinxbfcode{\sphinxupquote{property\DUrole{w}{  }}}\sphinxbfcode{\sphinxupquote{name}}}
\pysigstopsignatures
\sphinxAtStartPar
str.
This name is used in MILP model description.
names must not exceed 255 characters,
all of which must be alphanumeric (a\sphinxhyphen{}z, A\sphinxhyphen{}Z, 0\sphinxhyphen{}9) or one of these symbols:
! ” \# \$ \% \& , . ; ? @ \_ ‘ ’ \{ \} \textasciitilde{}.
\begin{quote}\begin{description}
\sphinxlineitem{Type}
\sphinxAtStartPar
Name of the instance

\end{description}\end{quote}

\end{fulllineitems}


\end{fulllineitems}


\sphinxstepscope


\subsection{tamos.element.ThermalVector}
\label{\detokenize{generated/tamos.element.ThermalVector:tamos-element-thermalvector}}\label{\detokenize{generated/tamos.element.ThermalVector::doc}}\index{ThermalVector (class in tamos.element)@\spxentry{ThermalVector}\spxextra{class in tamos.element}}

\begin{fulllineitems}
\phantomsection\label{\detokenize{generated/tamos.element.ThermalVector:tamos.element.ThermalVector}}
\pysigstartsignatures
\pysiglinewithargsret{\sphinxbfcode{\sphinxupquote{class\DUrole{w}{  }}}\sphinxcode{\sphinxupquote{tamos.element.}}\sphinxbfcode{\sphinxupquote{ThermalVector}}}{\emph{\DUrole{n}{temperature}}, \emph{\DUrole{n}{name}\DUrole{o}{=}\DUrole{default_value}{None}}}{}
\pysigstopsignatures\index{\_\_init\_\_() (tamos.element.ThermalVector method)@\spxentry{\_\_init\_\_()}\spxextra{tamos.element.ThermalVector method}}

\begin{fulllineitems}
\phantomsection\label{\detokenize{generated/tamos.element.ThermalVector:tamos.element.ThermalVector.__init__}}
\pysigstartsignatures
\pysiglinewithargsret{\sphinxbfcode{\sphinxupquote{\_\_init\_\_}}}{\emph{\DUrole{n}{temperature}}, \emph{\DUrole{n}{name}\DUrole{o}{=}\DUrole{default_value}{None}}}{}
\pysigstopsignatures
\sphinxAtStartPar
ThermalVector instances may:
\begin{itemize}
\item {} 
\sphinxAtStartPar
be used in a ThermalVectorPair instance.

\item {} 
\sphinxAtStartPar
describe an infinite thermal reservoir.
It is used to model a thermal source or sink that would not be affected by the energy taken from or given to it.

\begin{sphinxVerbatim}[commandchars=\\\{\}]
\PYG{g+gp}{\PYGZgt{}\PYGZgt{}\PYGZgt{} }\PYG{n}{ThermalVector}\PYG{p}{(}\PYG{n}{temperature}\PYG{o}{=}\PYG{n}{temperature\PYGZus{}profile}\PYG{p}{)}
\end{sphinxVerbatim}

\sphinxAtStartPar
For example, it is relevant to model the ambiant air using a ThermalVector. Such a vector, associated with
a Grid instance (i.e. an element\_IO component), would describe the heat rejected by heat pumps on their condenser side.

\begin{sphinxVerbatim}[commandchars=\\\{\}]
\PYG{g+gp}{\PYGZgt{}\PYGZgt{}\PYGZgt{} }\PYG{n}{Grid}\PYG{p}{(}\PYG{n}{element}\PYG{o}{=}\PYG{n}{air}\PYG{p}{,} \PYG{n}{exergy\PYGZus{}count}\PYG{o}{=}\PYG{k+kc}{False}\PYG{p}{)}
\end{sphinxVerbatim}

\sphinxAtStartPar
In that case, exergy count migh not be relevant thus it can be disabled using the \sphinxtitleref{exergy\_count} argument of the Grid
class.

\end{itemize}

\sphinxAtStartPar
Power is exchanged in kW.
\begin{quote}\begin{description}
\sphinxlineitem{Parameters}\begin{itemize}
\item {} 
\sphinxAtStartPar
\sphinxstyleliteralstrong{\sphinxupquote{temperature}} (\sphinxstyleliteralemphasis{\sphinxupquote{int}}\sphinxstyleliteralemphasis{\sphinxupquote{, }}\sphinxstyleliteralemphasis{\sphinxupquote{float}}\sphinxstyleliteralemphasis{\sphinxupquote{ or }}\sphinxstyleliteralemphasis{\sphinxupquote{numpy.ndarray}}) \textendash{} 
\sphinxAtStartPar
In Kelvins (K).
Used in:
\begin{itemize}
\item {} 
\sphinxAtStartPar
the efficiency definition of some production components (e.g. heat pumps)

\item {} 
\sphinxAtStartPar
the \sphinxtitleref{exergy\_factor} attribute

\item {} 
\sphinxAtStartPar
power and mass balance if this instance is used in a THermalVectorPair

\end{itemize}


\item {} 
\sphinxAtStartPar
\sphinxstyleliteralstrong{\sphinxupquote{name}} (\sphinxstyleliteralemphasis{\sphinxupquote{str}}\sphinxstyleliteralemphasis{\sphinxupquote{, }}\sphinxstyleliteralemphasis{\sphinxupquote{optional}}) \textendash{} 

\end{itemize}

\end{description}\end{quote}
\subsubsection*{Notes}
\begin{enumerate}
\sphinxsetlistlabels{\arabic}{enumi}{enumii}{}{.}%
\item {} 
\sphinxAtStartPar
Two ThermalVector instances having the same temperature are still considered different.

\item {} 
\sphinxAtStartPar
The dead state temperature is the reference temperature for calculation of exergy factor of ThermalVectorPair and ThermalVector instances.
It can be modified using the \sphinxtitleref{set\_dead\_state\_temperature} function of tamos.element.
The exergy factor calculation is done as follow:
exergy\_factor = 1 \sphinxhyphen{} dead\_state\_temperature / temperature

\item {} 
\sphinxAtStartPar
The exergy factor can be user\sphinxhyphen{}defined using the \sphinxtitleref{exergy\_factor} attribute.

\end{enumerate}

\end{fulllineitems}

\subsubsection*{Methods}


\begin{savenotes}\sphinxattablestart
\centering
\begin{tabulary}{\linewidth}[t]{\X{1}{2}\X{1}{2}}
\hline

\sphinxAtStartPar
{\hyperref[\detokenize{generated/tamos.element.ThermalVector:tamos.element.ThermalVector.__init__}]{\sphinxcrossref{\sphinxcode{\sphinxupquote{\_\_init\_\_}}}}}(temperature{[}, name{]})
&
\sphinxAtStartPar
ThermalVector instances may:
\\
\hline
\sphinxAtStartPar
{\hyperref[\detokenize{generated/tamos.element.ThermalVector:tamos.element.ThermalVector.get_vectors}]{\sphinxcrossref{\sphinxcode{\sphinxupquote{get\_vectors}}}}}()
&
\sphinxAtStartPar
Returns this instance.
\\
\hline
\end{tabulary}
\par
\sphinxattableend\end{savenotes}
\subsubsection*{Attributes}


\begin{savenotes}\sphinxattablestart
\centering
\begin{tabulary}{\linewidth}[t]{\X{1}{2}\X{1}{2}}
\hline

\sphinxAtStartPar
{\hyperref[\detokenize{generated/tamos.element.ThermalVector:tamos.element.ThermalVector.exergy_factor}]{\sphinxcrossref{\sphinxcode{\sphinxupquote{exergy\_factor}}}}}
&
\sphinxAtStartPar
Quantity of exergy associated with a 1 kW power flow of this element.
\\
\hline
\sphinxAtStartPar
{\hyperref[\detokenize{generated/tamos.element.ThermalVector:tamos.element.ThermalVector.name}]{\sphinxcrossref{\sphinxcode{\sphinxupquote{name}}}}}
&
\sphinxAtStartPar
str.
\\
\hline
\sphinxAtStartPar
{\hyperref[\detokenize{generated/tamos.element.ThermalVector:tamos.element.ThermalVector.temperature}]{\sphinxcrossref{\sphinxcode{\sphinxupquote{temperature}}}}}
&
\sphinxAtStartPar
int, float or numpy.ndarray In Kelvins (K)
\\
\hline
\end{tabulary}
\par
\sphinxattableend\end{savenotes}
\index{exergy\_factor (tamos.element.ThermalVector property)@\spxentry{exergy\_factor}\spxextra{tamos.element.ThermalVector property}}

\begin{fulllineitems}
\phantomsection\label{\detokenize{generated/tamos.element.ThermalVector:tamos.element.ThermalVector.exergy_factor}}
\pysigstartsignatures
\pysigline{\sphinxbfcode{\sphinxupquote{property\DUrole{w}{  }}}\sphinxbfcode{\sphinxupquote{exergy\_factor}}}
\pysigstopsignatures
\sphinxAtStartPar
Quantity of exergy associated with a 1 kW power flow of this element.
Theoretically positive, usually smaller than 1. In kW/kW.

\sphinxAtStartPar
int, float or numpy.ndarray
0 \textless{}= exergy\_factor
\subsubsection*{Notes}

\sphinxAtStartPar
FuelVector instances are exchanged between components considering their LHV value.

\end{fulllineitems}

\index{get\_vectors() (tamos.element.ThermalVector method)@\spxentry{get\_vectors()}\spxextra{tamos.element.ThermalVector method}}

\begin{fulllineitems}
\phantomsection\label{\detokenize{generated/tamos.element.ThermalVector:tamos.element.ThermalVector.get_vectors}}
\pysigstartsignatures
\pysiglinewithargsret{\sphinxbfcode{\sphinxupquote{get\_vectors}}}{}{}
\pysigstopsignatures
\sphinxAtStartPar
Returns this instance.

\end{fulllineitems}

\index{name (tamos.element.ThermalVector property)@\spxentry{name}\spxextra{tamos.element.ThermalVector property}}

\begin{fulllineitems}
\phantomsection\label{\detokenize{generated/tamos.element.ThermalVector:tamos.element.ThermalVector.name}}
\pysigstartsignatures
\pysigline{\sphinxbfcode{\sphinxupquote{property\DUrole{w}{  }}}\sphinxbfcode{\sphinxupquote{name}}}
\pysigstopsignatures
\sphinxAtStartPar
str.
This name is used in MILP model description.
names must not exceed 255 characters,
all of which must be alphanumeric (a\sphinxhyphen{}z, A\sphinxhyphen{}Z, 0\sphinxhyphen{}9) or one of these symbols:
! ” \# \$ \% \& , . ; ? @ \_ ‘ ’ \{ \} \textasciitilde{}.
\begin{quote}\begin{description}
\sphinxlineitem{Type}
\sphinxAtStartPar
Name of the instance

\end{description}\end{quote}

\end{fulllineitems}

\index{temperature (tamos.element.ThermalVector property)@\spxentry{temperature}\spxextra{tamos.element.ThermalVector property}}

\begin{fulllineitems}
\phantomsection\label{\detokenize{generated/tamos.element.ThermalVector:tamos.element.ThermalVector.temperature}}
\pysigstartsignatures
\pysigline{\sphinxbfcode{\sphinxupquote{property\DUrole{w}{  }}}\sphinxbfcode{\sphinxupquote{temperature}}}
\pysigstopsignatures
\sphinxAtStartPar
int, float or numpy.ndarray
In Kelvins (K)

\end{fulllineitems}


\end{fulllineitems}



\section{Utilities}
\label{\detokenize{element:utilities}}

\begin{savenotes}\sphinxattablestart
\centering
\begin{tabulary}{\linewidth}[t]{\X{1}{2}\X{1}{2}}
\hline

\sphinxAtStartPar
{\hyperref[\detokenize{generated/tamos.element.fetch_TVP:tamos.element.fetch_TVP}]{\sphinxcrossref{\sphinxcode{\sphinxupquote{tamos.element.fetch\_TVP}}}}}(in\_TV, out\_TV{[}, Cp, ...{]})
&
\sphinxAtStartPar
This method defines a ThermalVectorPair instance by its incoming and outcoming ThermalVector instances.
\\
\hline
\sphinxAtStartPar
{\hyperref[\detokenize{generated/tamos.element.get_existing_TVPs:tamos.element.get_existing_TVPs}]{\sphinxcrossref{\sphinxcode{\sphinxupquote{tamos.element.get\_existing\_TVPs}}}}}()
&
\sphinxAtStartPar

\\
\hline
\sphinxAtStartPar
{\hyperref[\detokenize{generated/tamos.element.get_dead_state_temperature:tamos.element.get_dead_state_temperature}]{\sphinxcrossref{\sphinxcode{\sphinxupquote{tamos.element.get\_dead\_state\_temperature}}}}}()
&
\sphinxAtStartPar
The dead state temperature is the reference temperature automatically used in calculation of attribute \sphinxtitleref{exergy\_factor} of ThermalVector and ThermalVectorPair instances.
\\
\hline
\sphinxAtStartPar
{\hyperref[\detokenize{generated/tamos.element.set_dead_state_temperature:tamos.element.set_dead_state_temperature}]{\sphinxcrossref{\sphinxcode{\sphinxupquote{tamos.element.set\_dead\_state\_temperature}}}}}(...)
&
\sphinxAtStartPar
Assign a new value to the dead state temperature, which is common to all elements.
\\
\hline
\end{tabulary}
\par
\sphinxattableend\end{savenotes}

\sphinxstepscope


\subsection{tamos.element.fetch\_TVP}
\label{\detokenize{generated/tamos.element.fetch_TVP:tamos-element-fetch-tvp}}\label{\detokenize{generated/tamos.element.fetch_TVP::doc}}\index{fetch\_TVP() (in module tamos.element)@\spxentry{fetch\_TVP()}\spxextra{in module tamos.element}}

\begin{fulllineitems}
\phantomsection\label{\detokenize{generated/tamos.element.fetch_TVP:tamos.element.fetch_TVP}}
\pysigstartsignatures
\pysiglinewithargsret{\sphinxcode{\sphinxupquote{tamos.element.}}\sphinxbfcode{\sphinxupquote{fetch\_TVP}}}{\emph{\DUrole{n}{in\_TV}}, \emph{\DUrole{n}{out\_TV}}, \emph{\DUrole{n}{Cp}\DUrole{o}{=}\DUrole{default_value}{None}}, \emph{\DUrole{n}{name}\DUrole{o}{=}\DUrole{default_value}{None}}}{}
\pysigstopsignatures
\sphinxAtStartPar
This method defines a ThermalVectorPair instance by its incoming and outcoming ThermalVector instances.

\sphinxAtStartPar
A ThermalVectorPair instance aims to represent a fluid (typically water) receiving or giving thermal energy
to a subsystem by going through a sensible heat exchange. The fluid entering the subsystem is described by ThermalVector
\sphinxtitleref{in\_TV} with temperature in\_TV.temperature. The fluid exiting the subsystem is \sphinxtitleref{out\_TV}, with temperature out\_TV.temperature.
Power is exchanged in kW.
\begin{quote}\begin{description}
\sphinxlineitem{Parameters}\begin{itemize}
\item {} 
\sphinxAtStartPar
\sphinxstyleliteralstrong{\sphinxupquote{in\_TV}} ({\hyperref[\detokenize{generated/tamos.element.ThermalVector:tamos.element.ThermalVector}]{\sphinxcrossref{\sphinxstyleliteralemphasis{\sphinxupquote{ThermalVector}}}}}) \textendash{} 

\item {} 
\sphinxAtStartPar
\sphinxstyleliteralstrong{\sphinxupquote{out\_TV}} ({\hyperref[\detokenize{generated/tamos.element.ThermalVector:tamos.element.ThermalVector}]{\sphinxcrossref{\sphinxstyleliteralemphasis{\sphinxupquote{ThermalVector}}}}}) \textendash{} 

\item {} 
\sphinxAtStartPar
\sphinxstyleliteralstrong{\sphinxupquote{Cp}} (\sphinxstyleliteralemphasis{\sphinxupquote{int}}\sphinxstyleliteralemphasis{\sphinxupquote{, }}\sphinxstyleliteralemphasis{\sphinxupquote{float}}\sphinxstyleliteralemphasis{\sphinxupquote{ or }}\sphinxstyleliteralemphasis{\sphinxupquote{numpy.ndarray}}\sphinxstyleliteralemphasis{\sphinxupquote{, }}\sphinxstyleliteralemphasis{\sphinxupquote{optional}}) \textendash{} 

\item {} 
\sphinxAtStartPar
\sphinxstyleliteralstrong{\sphinxupquote{name}} (\sphinxstyleliteralemphasis{\sphinxupquote{str}}\sphinxstyleliteralemphasis{\sphinxupquote{, }}\sphinxstyleliteralemphasis{\sphinxupquote{optional}}) \textendash{} 

\end{itemize}

\sphinxlineitem{Returns}
\sphinxAtStartPar
\begin{itemize}
\item {} 
\sphinxAtStartPar
\sphinxstyleemphasis{* If a ThermalVectorPair instance defined by (in\_TV, out\_TV) already exists, returns this instance} \textendash{}
\begin{itemize}
\item {} 
\sphinxAtStartPar
If \sphinxtitleref{name} is None, current name of the instance is kept.

\item {} 
\sphinxAtStartPar
Else, the instance is renamed.

\end{itemize}

\item {} 
\sphinxAtStartPar
\sphinxstyleemphasis{* Else, returns a new ThermalVectorPair instance.}

\end{itemize}


\end{description}\end{quote}
\subsubsection*{Notes}
\begin{enumerate}
\sphinxsetlistlabels{\arabic}{enumi}{enumii}{}{.}%
\item {} 
\sphinxAtStartPar
It is convenient to speak about ThermalVectorPair as ‘TVP’.

\item {} 
\sphinxAtStartPar
A positive power flow associated with a TVP (in\_TV, out\_TV) in a component describes
a positive mass flow rate of ThermalVector in\_TV (mass enters the system)
and the negated same mass flow rate of ThermalVector out\_TV (mass leaves the system).

\item {} 
\sphinxAtStartPar
If (in\_TV, out\_TV) defines a ThermalVectorPair instance \sphinxtitleref{TVP1}, the ThermalVectorPair (out\_TV, in\_TV) is called
‘the invert of \sphinxtitleref{TVP1}’, for example \sphinxtitleref{TVP2}, and can be accessed following two ways:

\begin{sphinxVerbatim}[commandchars=\\\{\}]
\PYG{g+gp}{\PYGZgt{}\PYGZgt{}\PYGZgt{} }\PYG{n}{TVP2\PYGZus{}first} \PYG{o}{=} \PYG{n}{fetch\PYGZus{}TVP}\PYG{p}{(}\PYG{n}{in\PYGZus{}TV}\PYG{o}{=}\PYG{n}{TVP1}\PYG{o}{.}\PYG{n}{out\PYGZus{}TV}\PYG{p}{,} \PYG{n}{out\PYGZus{}TV}\PYG{o}{=}\PYG{n}{TVP1}\PYG{o}{.}\PYG{n}{in\PYGZus{}TV}\PYG{p}{)}
\PYG{g+gp}{\PYGZgt{}\PYGZgt{}\PYGZgt{} }\PYG{n}{TVP2\PYGZus{}second} \PYG{o}{=} \PYG{o}{\PYGZti{}}\PYG{n}{TVP1}
\PYG{g+gp}{\PYGZgt{}\PYGZgt{}\PYGZgt{} }\PYG{n}{TVP2\PYGZus{}first} \PYG{o+ow}{is} \PYG{n}{TVP2\PYGZus{}second}
\PYG{g+go}{True}
\end{sphinxVerbatim}

\item {} 
\sphinxAtStartPar
Though balances are based on power flows in kW at the component scale, \sphinxtitleref{Cp} is required to perform mass balances at hub interface scale.

\item {} 
\sphinxAtStartPar
Cp should depend on the temperature of \sphinxtitleref{in\_TV} and \sphinxtitleref{out\_TV}. In practice, Cp variations are small on the temperature ranges {[}0°C, 110°C{]}.
e.g.
\begin{itemize}
\item {} 
\sphinxAtStartPar
max: Cp(T=110°C)=4228.3 J/(kg.K)

\item {} 
\sphinxAtStartPar
min: Cp(T=40°C)=4179.6 J/(kg.K)

\end{itemize}

\item {} 
\sphinxAtStartPar
The \sphinxtitleref{exergy\_factor} attribute is calculated as follow:
exergy\_factor = 1 \sphinxhyphen{} dead\_state\_temperature * log(out\_TV.temperature/in\_TV.temperature) / (out\_TV.temperature\sphinxhyphen{}in\_TV.temperature)

\end{enumerate}

\end{fulllineitems}


\sphinxstepscope


\subsection{tamos.element.get\_existing\_TVPs}
\label{\detokenize{generated/tamos.element.get_existing_TVPs:tamos-element-get-existing-tvps}}\label{\detokenize{generated/tamos.element.get_existing_TVPs::doc}}\index{get\_existing\_TVPs() (in module tamos.element)@\spxentry{get\_existing\_TVPs()}\spxextra{in module tamos.element}}

\begin{fulllineitems}
\phantomsection\label{\detokenize{generated/tamos.element.get_existing_TVPs:tamos.element.get_existing_TVPs}}
\pysigstartsignatures
\pysiglinewithargsret{\sphinxcode{\sphinxupquote{tamos.element.}}\sphinxbfcode{\sphinxupquote{get\_existing\_TVPs}}}{}{}
\pysigstopsignatures
\end{fulllineitems}


\sphinxstepscope


\subsection{tamos.element.get\_dead\_state\_temperature}
\label{\detokenize{generated/tamos.element.get_dead_state_temperature:tamos-element-get-dead-state-temperature}}\label{\detokenize{generated/tamos.element.get_dead_state_temperature::doc}}\index{get\_dead\_state\_temperature() (in module tamos.element)@\spxentry{get\_dead\_state\_temperature()}\spxextra{in module tamos.element}}

\begin{fulllineitems}
\phantomsection\label{\detokenize{generated/tamos.element.get_dead_state_temperature:tamos.element.get_dead_state_temperature}}
\pysigstartsignatures
\pysiglinewithargsret{\sphinxcode{\sphinxupquote{tamos.element.}}\sphinxbfcode{\sphinxupquote{get\_dead\_state\_temperature}}}{}{}
\pysigstopsignatures
\sphinxAtStartPar
The dead state temperature is the reference temperature automatically used in calculation of attribute \sphinxtitleref{exergy\_factor} of
ThermalVector and ThermalVectorPair instances.
It can be modified using the \sphinxtitleref{set\_dead\_state\_temperature} function of tamos.element.
\begin{quote}\begin{description}
\sphinxlineitem{Return type}
\sphinxAtStartPar
The dead state temperature, in Kelvins (K).

\end{description}\end{quote}

\end{fulllineitems}


\sphinxstepscope


\subsection{tamos.element.set\_dead\_state\_temperature}
\label{\detokenize{generated/tamos.element.set_dead_state_temperature:tamos-element-set-dead-state-temperature}}\label{\detokenize{generated/tamos.element.set_dead_state_temperature::doc}}\index{set\_dead\_state\_temperature() (in module tamos.element)@\spxentry{set\_dead\_state\_temperature()}\spxextra{in module tamos.element}}

\begin{fulllineitems}
\phantomsection\label{\detokenize{generated/tamos.element.set_dead_state_temperature:tamos.element.set_dead_state_temperature}}
\pysigstartsignatures
\pysiglinewithargsret{\sphinxcode{\sphinxupquote{tamos.element.}}\sphinxbfcode{\sphinxupquote{set\_dead\_state\_temperature}}}{\emph{\DUrole{n}{dead\_state\_temperature}}}{}
\pysigstopsignatures
\sphinxAtStartPar
Assign a new value to the dead state temperature, which is common to all elements.
Already defined elements are not affected by this change.
\begin{quote}\begin{description}
\sphinxlineitem{Parameters}
\sphinxAtStartPar
\sphinxstyleliteralstrong{\sphinxupquote{dead\_state\_temperature}} (\sphinxstyleliteralemphasis{\sphinxupquote{int}}\sphinxstyleliteralemphasis{\sphinxupquote{, }}\sphinxstyleliteralemphasis{\sphinxupquote{float}}\sphinxstyleliteralemphasis{\sphinxupquote{ or }}\sphinxstyleliteralemphasis{\sphinxupquote{numpy.ndarray}}) \textendash{} 

\end{description}\end{quote}

\end{fulllineitems}


\sphinxstepscope


\chapter{Production components}
\label{\detokenize{production_components:production-components}}\label{\detokenize{production_components::doc}}

\begin{savenotes}\sphinxattablestart
\centering
\begin{tabulary}{\linewidth}[t]{\X{1}{2}\X{1}{2}}
\hline

\sphinxAtStartPar
{\hyperref[\detokenize{generated/tamos.production.AbsHP:tamos.production.AbsHP}]{\sphinxcrossref{\sphinxcode{\sphinxupquote{tamos.production.AbsHP}}}}}(energy\_drive, ...{[}, ...{]})
&
\sphinxAtStartPar

\\
\hline
\sphinxAtStartPar
{\hyperref[\detokenize{generated/tamos.production.BiomassBoiler:tamos.production.BiomassBoiler}]{\sphinxcrossref{\sphinxcode{\sphinxupquote{tamos.production.BiomassBoiler}}}}}(...{[}, ...{]})
&
\sphinxAtStartPar

\\
\hline
\sphinxAtStartPar
{\hyperref[\detokenize{generated/tamos.production.CHP:tamos.production.CHP}]{\sphinxcrossref{\sphinxcode{\sphinxupquote{tamos.production.CHP}}}}}(energy\_source, ...{[}, ...{]})
&
\sphinxAtStartPar

\\
\hline
\sphinxAtStartPar
{\hyperref[\detokenize{generated/tamos.production.COPModels:tamos.production.COPModels}]{\sphinxcrossref{\sphinxcode{\sphinxupquote{tamos.production.COPModels}}}}}()
&
\sphinxAtStartPar

\\
\hline
\sphinxAtStartPar
{\hyperref[\detokenize{generated/tamos.production.CompHP:tamos.production.CompHP}]{\sphinxcrossref{\sphinxcode{\sphinxupquote{tamos.production.CompHP}}}}}(energy\_drive, ...{[}, ...{]})
&
\sphinxAtStartPar

\\
\hline
\sphinxAtStartPar
{\hyperref[\detokenize{generated/tamos.production.DryCooler:tamos.production.DryCooler}]{\sphinxcrossref{\sphinxcode{\sphinxupquote{tamos.production.DryCooler}}}}}(energy\_drive, ...)
&
\sphinxAtStartPar

\\
\hline
\sphinxAtStartPar
{\hyperref[\detokenize{generated/tamos.production.ElectricHeater:tamos.production.ElectricHeater}]{\sphinxcrossref{\sphinxcode{\sphinxupquote{tamos.production.ElectricHeater}}}}}(...{[}, ...{]})
&
\sphinxAtStartPar

\\
\hline
\sphinxAtStartPar
{\hyperref[\detokenize{generated/tamos.production.ElementConverter:tamos.production.ElementConverter}]{\sphinxcrossref{\sphinxcode{\sphinxupquote{tamos.production.ElementConverter}}}}}(element\_1, ...)
&
\sphinxAtStartPar

\\
\hline
\sphinxAtStartPar
{\hyperref[\detokenize{generated/tamos.production.FPSolar:tamos.production.FPSolar}]{\sphinxcrossref{\sphinxcode{\sphinxupquote{tamos.production.FPSolar}}}}}(energy\_sink, ...{[}, ...{]})
&
\sphinxAtStartPar

\\
\hline
\sphinxAtStartPar
{\hyperref[\detokenize{generated/tamos.production.GasBoiler:tamos.production.GasBoiler}]{\sphinxcrossref{\sphinxcode{\sphinxupquote{tamos.production.GasBoiler}}}}}(energy\_source, ...)
&
\sphinxAtStartPar

\\
\hline
\sphinxAtStartPar
{\hyperref[\detokenize{generated/tamos.production.HeatExchanger:tamos.production.HeatExchanger}]{\sphinxcrossref{\sphinxcode{\sphinxupquote{tamos.production.HeatExchanger}}}}}(...{[}, ...{]})
&
\sphinxAtStartPar

\\
\hline
\sphinxAtStartPar
{\hyperref[\detokenize{generated/tamos.production.Pump:tamos.production.Pump}]{\sphinxcrossref{\sphinxcode{\sphinxupquote{tamos.production.Pump}}}}}(element\_1, element\_2, ...)
&
\sphinxAtStartPar

\\
\hline
\end{tabulary}
\par
\sphinxattableend\end{savenotes}

\sphinxstepscope


\section{tamos.production.AbsHP}
\label{\detokenize{generated/tamos.production.AbsHP:tamos-production-abshp}}\label{\detokenize{generated/tamos.production.AbsHP::doc}}\index{AbsHP (class in tamos.production)@\spxentry{AbsHP}\spxextra{class in tamos.production}}

\begin{fulllineitems}
\phantomsection\label{\detokenize{generated/tamos.production.AbsHP:tamos.production.AbsHP}}
\pysigstartsignatures
\pysiglinewithargsret{\sphinxbfcode{\sphinxupquote{class\DUrole{w}{  }}}\sphinxcode{\sphinxupquote{tamos.production.}}\sphinxbfcode{\sphinxupquote{AbsHP}}}{\emph{\DUrole{n}{energy\_drive}}, \emph{\DUrole{n}{energy\_source}}, \emph{\DUrole{n}{energy\_sink}}, \emph{\DUrole{n}{electricity\_aux}}, \emph{\DUrole{n}{properties}}, \emph{\DUrole{n}{given\_sizing}\DUrole{o}{=}\DUrole{default_value}{None}}, \emph{\DUrole{n}{pump\_consumption}\DUrole{o}{=}\DUrole{default_value}{0.026}}, \emph{\DUrole{n}{name}\DUrole{o}{=}\DUrole{default_value}{None}}, \emph{\DUrole{n}{units\_number\_ub}\DUrole{o}{=}\DUrole{default_value}{1}}, \emph{\DUrole{n}{units\_number\_lb}\DUrole{o}{=}\DUrole{default_value}{1}}, \emph{\DUrole{n}{eco\_count}\DUrole{o}{=}\DUrole{default_value}{True}}}{}
\pysigstopsignatures\index{\_\_init\_\_() (tamos.production.AbsHP method)@\spxentry{\_\_init\_\_()}\spxextra{tamos.production.AbsHP method}}

\begin{fulllineitems}
\phantomsection\label{\detokenize{generated/tamos.production.AbsHP:tamos.production.AbsHP.__init__}}
\pysigstartsignatures
\pysiglinewithargsret{\sphinxbfcode{\sphinxupquote{\_\_init\_\_}}}{\emph{\DUrole{n}{energy\_drive}}, \emph{\DUrole{n}{energy\_source}}, \emph{\DUrole{n}{energy\_sink}}, \emph{\DUrole{n}{electricity\_aux}}, \emph{\DUrole{n}{properties}}, \emph{\DUrole{n}{given\_sizing}\DUrole{o}{=}\DUrole{default_value}{None}}, \emph{\DUrole{n}{pump\_consumption}\DUrole{o}{=}\DUrole{default_value}{0.026}}, \emph{\DUrole{n}{name}\DUrole{o}{=}\DUrole{default_value}{None}}, \emph{\DUrole{n}{units\_number\_ub}\DUrole{o}{=}\DUrole{default_value}{1}}, \emph{\DUrole{n}{units\_number\_lb}\DUrole{o}{=}\DUrole{default_value}{1}}, \emph{\DUrole{n}{eco\_count}\DUrole{o}{=}\DUrole{default_value}{True}}}{}
\pysigstopsignatures
\sphinxAtStartPar
AbsHP components describe absorption heat pumps.
The COP model is an adaptation of (Boudéhenn et al., 2016) %
\begin{footnote}[2]\sphinxAtStartFootnote
Boudéhenn F, Bonnot S, Demasles H, Lefrançois F, Perier\sphinxhyphen{}Muzet M, Triché D.
Development and Performances Overview of Ammonia\sphinxhyphen{}water Absorption Chillers with Cooling Capacities from 5 to 100 kW.
Energy Procedia 2016;91:707\textendash{}16. \sphinxurl{https://doi.org/10.1016/j.egypro.2016.06.234}.
%
\end{footnote}.

\sphinxAtStartPar
This component declares the following exported decision variables:
\begin{itemize}
\item {} 
\sphinxAtStartPar
X\_P, binary.
Whether the component is used by the hub.

\item {} 
\sphinxAtStartPar
SP\_P, continuous, in kW.
The maximum capacity of the component. Defines the investment costs.

\item {} 
\sphinxAtStartPar
For all t, for all element e, F\_P(e, t), continuous, in kW.
The power related to element e entering the component (i.e. leaving the hub interface).

\item {} 
\sphinxAtStartPar
For all t, Q\_P(t), continuous, in kW.
The reference power related to the component. Defines the variable cost.
This power is a lower bound of SP\_P.
There exists one element e such that Q\_P(t) = F\_P(e, t) or Q\_P(t) = \sphinxhyphen{} F\_P(e, t).
For this component, e is either \sphinxtitleref{energy\_sink} or \sphinxtitleref{energy\_source} depending on the \sphinxtitleref{ref\_production} attribute.

\end{itemize}

\sphinxAtStartPar
This component declares the following KPIs:
\begin{itemize}
\item {} 
\sphinxAtStartPar
\sphinxtitleref{COST\_production}
In euros.
Contributes to the “Eco” objective function.

\end{itemize}
\begin{quote}\begin{description}
\sphinxlineitem{Parameters}\begin{itemize}
\item {} 
\sphinxAtStartPar
\sphinxstyleliteralstrong{\sphinxupquote{energy\_drive}} (\sphinxstyleliteralemphasis{\sphinxupquote{ThermalVectorPair}}) \textendash{} Thermal flow that is cooled down for the desorption process.

\item {} 
\sphinxAtStartPar
\sphinxstyleliteralstrong{\sphinxupquote{energy\_source}} (\sphinxstyleliteralemphasis{\sphinxupquote{ThermalVectorPair}}\sphinxstyleliteralemphasis{\sphinxupquote{, }}{\hyperref[\detokenize{generated/tamos.element.ThermalVector:tamos.element.ThermalVector}]{\sphinxcrossref{\sphinxstyleliteralemphasis{\sphinxupquote{ThermalVector}}}}}) \textendash{} Element that gives thermal energy.
Must be cooled down if ThermalVectorPair.

\item {} 
\sphinxAtStartPar
\sphinxstyleliteralstrong{\sphinxupquote{energy\_sink}} (\sphinxstyleliteralemphasis{\sphinxupquote{ThermalVectorPair}}\sphinxstyleliteralemphasis{\sphinxupquote{, }}{\hyperref[\detokenize{generated/tamos.element.ThermalVector:tamos.element.ThermalVector}]{\sphinxcrossref{\sphinxstyleliteralemphasis{\sphinxupquote{ThermalVector}}}}}) \textendash{} Element that receives thermal energy.
Must be warmed up if ThermalVectorPair.

\item {} 
\sphinxAtStartPar
\sphinxstyleliteralstrong{\sphinxupquote{electricity\_aux}} ({\hyperref[\detokenize{generated/tamos.element.ElectricityVector:tamos.element.ElectricityVector}]{\sphinxcrossref{\sphinxstyleliteralemphasis{\sphinxupquote{ElectricityVector}}}}}) \textendash{} Electricity consumed by the machine.

\item {} 
\sphinxAtStartPar
\sphinxstyleliteralstrong{\sphinxupquote{properties}} (\sphinxstyleliteralemphasis{\sphinxupquote{dict \{str: int}}\sphinxstyleliteralemphasis{\sphinxupquote{ | }}\sphinxstyleliteralemphasis{\sphinxupquote{float\}}}) \textendash{} 
\sphinxAtStartPar
Techno\sphinxhyphen{}economic properties of the component.
The \sphinxtitleref{properties} attribute must include the following keys:
\begin{itemize}
\item {} 
\sphinxAtStartPar
”LB max output power (kW)”

\item {} 
\sphinxAtStartPar
”UB max output power (kW)”

\item {} 
\sphinxAtStartPar
”CAPEX (EUR/kW)”

\item {} 
\sphinxAtStartPar
”OPEX (\%CAPEX)”

\item {} 
\sphinxAtStartPar
”Variable OPEX (EUR/MWh)”

\end{itemize}


\item {} 
\sphinxAtStartPar
\sphinxstyleliteralstrong{\sphinxupquote{given\_sizing}} (\sphinxstyleliteralemphasis{\sphinxupquote{int}}\sphinxstyleliteralemphasis{\sphinxupquote{ or }}\sphinxstyleliteralemphasis{\sphinxupquote{float}}\sphinxstyleliteralemphasis{\sphinxupquote{, }}\sphinxstyleliteralemphasis{\sphinxupquote{optional}}) \textendash{} The maximum capacity of the component, in kW.
Relates to decision variable ‘SP\_P’.
If specified, only the operation of this component is performed by the MILP solver.
If let unknown, both sizing and operation are performed.

\item {} 
\sphinxAtStartPar
\sphinxstyleliteralstrong{\sphinxupquote{pump\_consumption}} (\sphinxstyleliteralemphasis{\sphinxupquote{float}}\sphinxstyleliteralemphasis{\sphinxupquote{, }}\sphinxstyleliteralemphasis{\sphinxupquote{optional}}\sphinxstyleliteralemphasis{\sphinxupquote{, }}\sphinxstyleliteralemphasis{\sphinxupquote{default 0.026}}) \textendash{} Electricity consumption of the heat pump, in proportion of the power extracted from \sphinxtitleref{energy\_source}.

\item {} 
\sphinxAtStartPar
\sphinxstyleliteralstrong{\sphinxupquote{name}} (\sphinxstyleliteralemphasis{\sphinxupquote{str}}\sphinxstyleliteralemphasis{\sphinxupquote{, }}\sphinxstyleliteralemphasis{\sphinxupquote{optional}}) \textendash{} 

\item {} 
\sphinxAtStartPar
\sphinxstyleliteralstrong{\sphinxupquote{eco\_count}} (\sphinxstyleliteralemphasis{\sphinxupquote{bool}}\sphinxstyleliteralemphasis{\sphinxupquote{, }}\sphinxstyleliteralemphasis{\sphinxupquote{optional}}\sphinxstyleliteralemphasis{\sphinxupquote{, }}\sphinxstyleliteralemphasis{\sphinxupquote{default True}}) \textendash{} Whether this instance contributes to the system “Eco” KPI.

\item {} 
\sphinxAtStartPar
\sphinxstyleliteralstrong{\sphinxupquote{units\_number\_lb}} (\sphinxstyleliteralemphasis{\sphinxupquote{int}}\sphinxstyleliteralemphasis{\sphinxupquote{, }}\sphinxstyleliteralemphasis{\sphinxupquote{optional}}\sphinxstyleliteralemphasis{\sphinxupquote{, }}\sphinxstyleliteralemphasis{\sphinxupquote{default 1}}) \textendash{} The lower bound (upper bound) of the number of real components that this instance aims to stand for.
Setting \sphinxtitleref{units\_number\_lb} (\sphinxtitleref{units\_number\_ub}) has a meaning if “LB max output power (kW)” property is
different from 0.

\item {} 
\sphinxAtStartPar
\sphinxstyleliteralstrong{\sphinxupquote{units\_number\_ub}} (\sphinxstyleliteralemphasis{\sphinxupquote{int}}\sphinxstyleliteralemphasis{\sphinxupquote{, }}\sphinxstyleliteralemphasis{\sphinxupquote{optional}}\sphinxstyleliteralemphasis{\sphinxupquote{, }}\sphinxstyleliteralemphasis{\sphinxupquote{default 1}}) \textendash{} The lower bound (upper bound) of the number of real components that this instance aims to stand for.
Setting \sphinxtitleref{units\_number\_lb} (\sphinxtitleref{units\_number\_ub}) has a meaning if “LB max output power (kW)” property is
different from 0.

\end{itemize}

\end{description}\end{quote}
\subsubsection*{References}

\end{fulllineitems}

\subsubsection*{Methods}


\begin{savenotes}\sphinxattablestart
\centering
\begin{tabulary}{\linewidth}[t]{\X{1}{2}\X{1}{2}}
\hline

\sphinxAtStartPar
{\hyperref[\detokenize{generated/tamos.production.AbsHP:tamos.production.AbsHP.__init__}]{\sphinxcrossref{\sphinxcode{\sphinxupquote{\_\_init\_\_}}}}}(energy\_drive, energy\_source, ...{[}, ...{]})
&
\sphinxAtStartPar
AbsHP components describe absorption heat pumps.
\\
\hline
\sphinxAtStartPar
{\hyperref[\detokenize{generated/tamos.production.AbsHP:tamos.production.AbsHP.compute_actualized_cost}]{\sphinxcrossref{\sphinxcode{\sphinxupquote{compute\_actualized\_cost}}}}}(CAPEX, OPEX, ...{[}, ...{]})
&
\sphinxAtStartPar
Computes the cost of a component using its \textquotesingle{}Lifetime\textquotesingle{} and \textquotesingle{}Discount rate (\%)\textquotesingle{} properties.
\\
\hline
\sphinxAtStartPar
{\hyperref[\detokenize{generated/tamos.production.AbsHP:tamos.production.AbsHP.set_efficiency_model}]{\sphinxcrossref{\sphinxcode{\sphinxupquote{set\_efficiency\_model}}}}}(source\_pinch, sink\_pinch)
&
\sphinxAtStartPar
Defines the efficiency of the heat pump, i.e. the coefficient of performance (COP).
\\
\hline
\end{tabulary}
\par
\sphinxattableend\end{savenotes}
\subsubsection*{Attributes}


\begin{savenotes}\sphinxattablestart
\centering
\begin{tabulary}{\linewidth}[t]{\X{1}{2}\X{1}{2}}
\hline

\sphinxAtStartPar
{\hyperref[\detokenize{generated/tamos.production.AbsHP:tamos.production.AbsHP.eco_count}]{\sphinxcrossref{\sphinxcode{\sphinxupquote{eco\_count}}}}}
&
\sphinxAtStartPar
Whether this instance contributes to the system "Eco" KPI.
\\
\hline
\sphinxAtStartPar
{\hyperref[\detokenize{generated/tamos.production.AbsHP:tamos.production.AbsHP.efficiency}]{\sphinxcrossref{\sphinxcode{\sphinxupquote{efficiency}}}}}
&
\sphinxAtStartPar
Defines explicitly the efficiency of the heat pump, i.e. the coefficient of performance (COP).
\\
\hline
\sphinxAtStartPar
{\hyperref[\detokenize{generated/tamos.production.AbsHP:tamos.production.AbsHP.energy_drive}]{\sphinxcrossref{\sphinxcode{\sphinxupquote{energy\_drive}}}}}
&
\sphinxAtStartPar
Element enabling the pressurization of the heat pump refrigerant.
\\
\hline
\sphinxAtStartPar
{\hyperref[\detokenize{generated/tamos.production.AbsHP:tamos.production.AbsHP.energy_sink}]{\sphinxcrossref{\sphinxcode{\sphinxupquote{energy\_sink}}}}}
&
\sphinxAtStartPar
Element that receives thermal energy.
\\
\hline
\sphinxAtStartPar
{\hyperref[\detokenize{generated/tamos.production.AbsHP:tamos.production.AbsHP.energy_source}]{\sphinxcrossref{\sphinxcode{\sphinxupquote{energy\_source}}}}}
&
\sphinxAtStartPar
Element that gives thermal energy.
\\
\hline
\sphinxAtStartPar
{\hyperref[\detokenize{generated/tamos.production.AbsHP:tamos.production.AbsHP.given_sizing}]{\sphinxcrossref{\sphinxcode{\sphinxupquote{given\_sizing}}}}}
&
\sphinxAtStartPar
The maximum capacity of the component, in kW.
\\
\hline
\sphinxAtStartPar
{\hyperref[\detokenize{generated/tamos.production.AbsHP:tamos.production.AbsHP.name}]{\sphinxcrossref{\sphinxcode{\sphinxupquote{name}}}}}
&
\sphinxAtStartPar
str.
\\
\hline
\sphinxAtStartPar
\sphinxcode{\sphinxupquote{pump\_consumption}}
&
\sphinxAtStartPar

\\
\hline
\sphinxAtStartPar
{\hyperref[\detokenize{generated/tamos.production.AbsHP:tamos.production.AbsHP.ref_production}]{\sphinxcrossref{\sphinxcode{\sphinxupquote{ref\_production}}}}}
&
\sphinxAtStartPar
Defines whether \sphinxtitleref{energy\_sink} or \sphinxtitleref{energy\_source} is the reference production.
\\
\hline
\sphinxAtStartPar
{\hyperref[\detokenize{generated/tamos.production.AbsHP:tamos.production.AbsHP.units_number_lb}]{\sphinxcrossref{\sphinxcode{\sphinxupquote{units\_number\_lb}}}}}
&
\sphinxAtStartPar
The lower bound of the number of real components that this instance aims to stand for.
\\
\hline
\sphinxAtStartPar
{\hyperref[\detokenize{generated/tamos.production.AbsHP:tamos.production.AbsHP.units_number_ub}]{\sphinxcrossref{\sphinxcode{\sphinxupquote{units\_number\_ub}}}}}
&
\sphinxAtStartPar
The upper bound of the number of real components that this instance aims to stand for.
\\
\hline
\sphinxAtStartPar
{\hyperref[\detokenize{generated/tamos.production.AbsHP:tamos.production.AbsHP.used_elements}]{\sphinxcrossref{\sphinxcode{\sphinxupquote{used\_elements}}}}}
&
\sphinxAtStartPar
Elements used by the component.
\\
\hline
\end{tabulary}
\par
\sphinxattableend\end{savenotes}
\index{compute\_actualized\_cost() (tamos.production.AbsHP method)@\spxentry{compute\_actualized\_cost()}\spxextra{tamos.production.AbsHP method}}

\begin{fulllineitems}
\phantomsection\label{\detokenize{generated/tamos.production.AbsHP:tamos.production.AbsHP.compute_actualized_cost}}
\pysigstartsignatures
\pysiglinewithargsret{\sphinxbfcode{\sphinxupquote{compute\_actualized\_cost}}}{\emph{\DUrole{n}{CAPEX}}, \emph{\DUrole{n}{OPEX}}, \emph{\DUrole{n}{system\_lifetime}}, \emph{\DUrole{n}{lifetime}\DUrole{o}{=}\DUrole{default_value}{None}}, \emph{\DUrole{n}{discount\_rate}\DUrole{o}{=}\DUrole{default_value}{None}}}{}
\pysigstopsignatures
\sphinxAtStartPar
Computes the cost of a component using its ‘Lifetime’ and ‘Discount rate (\%)’ properties.
\begin{quote}\begin{description}
\sphinxlineitem{Parameters}\begin{itemize}
\item {} 
\sphinxAtStartPar
\sphinxstyleliteralstrong{\sphinxupquote{CAPEX}} (\sphinxstyleliteralemphasis{\sphinxupquote{float}}) \textendash{} Capital Expenditure. Cost in euros paid every \sphinxtitleref{technical\_lifetime} periods.

\item {} 
\sphinxAtStartPar
\sphinxstyleliteralstrong{\sphinxupquote{OPEX}} (\sphinxstyleliteralemphasis{\sphinxupquote{float}}) \textendash{} Operational Expenditure. Cost in euros paid each period.

\item {} 
\sphinxAtStartPar
\sphinxstyleliteralstrong{\sphinxupquote{system\_lifetime}} (\sphinxstyleliteralemphasis{\sphinxupquote{int}}) \textendash{} Number of periods defining the existence of the energy system.

\item {} 
\sphinxAtStartPar
\sphinxstyleliteralstrong{\sphinxupquote{lifetime}} (\sphinxstyleliteralemphasis{\sphinxupquote{int}}\sphinxstyleliteralemphasis{\sphinxupquote{, }}\sphinxstyleliteralemphasis{\sphinxupquote{optional}}) \textendash{} Number of periods defining the existence of the component.
If specified, overwrite the “Lifetime” property.

\item {} 
\sphinxAtStartPar
\sphinxstyleliteralstrong{\sphinxupquote{discount\_rate}} (\sphinxstyleliteralemphasis{\sphinxupquote{float}}) \textendash{} In percent (\%). Describes the importance of the economic amortization process, per period.
If specified, overwrite the “Discount rate (\%)” property.

\end{itemize}

\sphinxlineitem{Returns}
\sphinxAtStartPar
\begin{itemize}
\item {} 
\sphinxAtStartPar
\sphinxstyleemphasis{A 3\sphinxhyphen{}tuple (total\_cost, CAPEX\_share, OPEX\_share) where}

\item {} 
\sphinxAtStartPar
* CAPEX\_share is the share of total cost related to \sphinxtitleref{CAPEX}

\item {} 
\sphinxAtStartPar
* OPEX\_share is the share of total cost related to \sphinxtitleref{OPEX}

\item {} 
\sphinxAtStartPar
\sphinxstyleemphasis{* total\_cost = CAPEX\_share + OPEX\_share}

\end{itemize}


\end{description}\end{quote}
\subsubsection*{Notes}

\sphinxAtStartPar
Takes into account residual value of component in the case \sphinxtitleref{system\_lifetime} is not a multiple of \sphinxtitleref{lifetime}.
In this case, the last replacement occuring at period replacement\_period is paid in proportion of ‘CAPEX’
depending linearly on the number of periods left:
CAPEX * (system\_lifetime \sphinxhyphen{} replacement\_period) / lifetime

\end{fulllineitems}

\index{eco\_count (tamos.production.AbsHP property)@\spxentry{eco\_count}\spxextra{tamos.production.AbsHP property}}

\begin{fulllineitems}
\phantomsection\label{\detokenize{generated/tamos.production.AbsHP:tamos.production.AbsHP.eco_count}}
\pysigstartsignatures
\pysigline{\sphinxbfcode{\sphinxupquote{property\DUrole{w}{  }}}\sphinxbfcode{\sphinxupquote{eco\_count}}}
\pysigstopsignatures
\sphinxAtStartPar
Whether this instance contributes to the system “Eco” KPI.
bool

\end{fulllineitems}

\index{efficiency (tamos.production.AbsHP property)@\spxentry{efficiency}\spxextra{tamos.production.AbsHP property}}

\begin{fulllineitems}
\phantomsection\label{\detokenize{generated/tamos.production.AbsHP:tamos.production.AbsHP.efficiency}}
\pysigstartsignatures
\pysigline{\sphinxbfcode{\sphinxupquote{property\DUrole{w}{  }}}\sphinxbfcode{\sphinxupquote{efficiency}}}
\pysigstopsignatures
\sphinxAtStartPar
Defines explicitly the efficiency of the heat pump, i.e. the coefficient of performance (COP).

\sphinxAtStartPar
The COP is a heating COP, i.e. it is used according to the two following constraints:
1. energy\_from\_source(t) + energy\_from\_drive(t) = energy\_to\_sink(t)
2. energy\_from\_sink(t) = energy\_from\_drive(t) * COP(t)

\sphinxAtStartPar
If called, replaces the definition of the efficiency using \sphinxtitleref{set\_efficiency\_model} (default).
int, float or numpy.ndarray

\end{fulllineitems}

\index{energy\_drive (tamos.production.AbsHP property)@\spxentry{energy\_drive}\spxextra{tamos.production.AbsHP property}}

\begin{fulllineitems}
\phantomsection\label{\detokenize{generated/tamos.production.AbsHP:tamos.production.AbsHP.energy_drive}}
\pysigstartsignatures
\pysigline{\sphinxbfcode{\sphinxupquote{property\DUrole{w}{  }}}\sphinxbfcode{\sphinxupquote{energy\_drive}}}
\pysigstopsignatures
\sphinxAtStartPar
Element enabling the pressurization of the heat pump refrigerant.

\end{fulllineitems}

\index{energy\_sink (tamos.production.AbsHP property)@\spxentry{energy\_sink}\spxextra{tamos.production.AbsHP property}}

\begin{fulllineitems}
\phantomsection\label{\detokenize{generated/tamos.production.AbsHP:tamos.production.AbsHP.energy_sink}}
\pysigstartsignatures
\pysigline{\sphinxbfcode{\sphinxupquote{property\DUrole{w}{  }}}\sphinxbfcode{\sphinxupquote{energy\_sink}}}
\pysigstopsignatures
\sphinxAtStartPar
Element that receives thermal energy.
Must be warmed up if ThermalVectorPair.                ——\sphinxhyphen{}

\end{fulllineitems}

\index{energy\_source (tamos.production.AbsHP property)@\spxentry{energy\_source}\spxextra{tamos.production.AbsHP property}}

\begin{fulllineitems}
\phantomsection\label{\detokenize{generated/tamos.production.AbsHP:tamos.production.AbsHP.energy_source}}
\pysigstartsignatures
\pysigline{\sphinxbfcode{\sphinxupquote{property\DUrole{w}{  }}}\sphinxbfcode{\sphinxupquote{energy\_source}}}
\pysigstopsignatures
\sphinxAtStartPar
Element that gives thermal energy.
Must be cooled down if ThermalVectorPair.

\end{fulllineitems}

\index{given\_sizing (tamos.production.AbsHP property)@\spxentry{given\_sizing}\spxextra{tamos.production.AbsHP property}}

\begin{fulllineitems}
\phantomsection\label{\detokenize{generated/tamos.production.AbsHP:tamos.production.AbsHP.given_sizing}}
\pysigstartsignatures
\pysigline{\sphinxbfcode{\sphinxupquote{property\DUrole{w}{  }}}\sphinxbfcode{\sphinxupquote{given\_sizing}}}
\pysigstopsignatures
\sphinxAtStartPar
The maximum capacity of the component, in kW.
Relates to decision variable ‘SP\_P’.
int or float

\end{fulllineitems}

\index{name (tamos.production.AbsHP property)@\spxentry{name}\spxextra{tamos.production.AbsHP property}}

\begin{fulllineitems}
\phantomsection\label{\detokenize{generated/tamos.production.AbsHP:tamos.production.AbsHP.name}}
\pysigstartsignatures
\pysigline{\sphinxbfcode{\sphinxupquote{property\DUrole{w}{  }}}\sphinxbfcode{\sphinxupquote{name}}}
\pysigstopsignatures
\sphinxAtStartPar
str.
This name is used in MILP model description.
names must not exceed 255 characters,
all of which must be alphanumeric (a\sphinxhyphen{}z, A\sphinxhyphen{}Z, 0\sphinxhyphen{}9) or one of these symbols:
! ” \# \$ \% \& , . ; ? @ \_ ‘ ’ \{ \} \textasciitilde{}.
\begin{quote}\begin{description}
\sphinxlineitem{Type}
\sphinxAtStartPar
Name of the instance

\end{description}\end{quote}

\end{fulllineitems}

\index{ref\_production (tamos.production.AbsHP property)@\spxentry{ref\_production}\spxextra{tamos.production.AbsHP property}}

\begin{fulllineitems}
\phantomsection\label{\detokenize{generated/tamos.production.AbsHP:tamos.production.AbsHP.ref_production}}
\pysigstartsignatures
\pysigline{\sphinxbfcode{\sphinxupquote{property\DUrole{w}{  }}}\sphinxbfcode{\sphinxupquote{ref\_production}}}
\pysigstopsignatures
\sphinxAtStartPar
Defines whether \sphinxtitleref{energy\_sink} or \sphinxtitleref{energy\_source} is the reference production.
Either \sphinxtitleref{energy\_sink} or \sphinxtitleref{energy\_source}.

\sphinxAtStartPar
The reference production defines decision variable ‘Q\_P’ and thus ‘SP\_P’,
which defines the cost of the component.
If costs properties are typical values for a heat pump use, set \sphinxtitleref{HP.ref\_production=HP.energy\_sink}.
If the costs properties are given for a chiller use, set \sphinxtitleref{HP.ref\_production=HP.energy\_source}.

\end{fulllineitems}

\index{set\_efficiency\_model() (tamos.production.AbsHP method)@\spxentry{set\_efficiency\_model()}\spxextra{tamos.production.AbsHP method}}

\begin{fulllineitems}
\phantomsection\label{\detokenize{generated/tamos.production.AbsHP:tamos.production.AbsHP.set_efficiency_model}}
\pysigstartsignatures
\pysiglinewithargsret{\sphinxbfcode{\sphinxupquote{set\_efficiency\_model}}}{\emph{\DUrole{n}{source\_pinch}}, \emph{\DUrole{n}{sink\_pinch}}, \emph{\DUrole{n}{as\_HEx}\DUrole{o}{=}\DUrole{default_value}{True}}, \emph{\DUrole{o}{**}\DUrole{n}{model\_kwargs}}}{}
\pysigstopsignatures
\sphinxAtStartPar
Defines the efficiency of the heat pump, i.e. the coefficient of performance (COP).

\sphinxAtStartPar
The COP is a heating COP, i.e. it is used according to the two following constraints:
1. energy\_from\_source(t) + energy\_from\_drive(t) = energy\_to\_sink(t)
2. energy\_from\_sink(t) = energy\_from\_drive(t) * COP(t)
\begin{quote}\begin{description}
\sphinxlineitem{Parameters}\begin{itemize}
\item {} 
\sphinxAtStartPar
\sphinxstyleliteralstrong{\sphinxupquote{source\_pinch}} (\sphinxstyleliteralemphasis{\sphinxupquote{int}}\sphinxstyleliteralemphasis{\sphinxupquote{, }}\sphinxstyleliteralemphasis{\sphinxupquote{float}}\sphinxstyleliteralemphasis{\sphinxupquote{ or }}\sphinxstyleliteralemphasis{\sphinxupquote{numpy.ndarray}}) \textendash{} Temperature difference between \sphinxtitleref{energy\_source} and the refrigerant fluid of the heat pump (evaporator side).
If \sphinxtitleref{energy\_source} is a ThermalVectorPair, the relevant temperature is the one of the cold vector (outcoming).

\item {} 
\sphinxAtStartPar
\sphinxstyleliteralstrong{\sphinxupquote{sink\_pinch}} (\sphinxstyleliteralemphasis{\sphinxupquote{int}}\sphinxstyleliteralemphasis{\sphinxupquote{, }}\sphinxstyleliteralemphasis{\sphinxupquote{float}}\sphinxstyleliteralemphasis{\sphinxupquote{ or }}\sphinxstyleliteralemphasis{\sphinxupquote{numpy.ndarray}}) \textendash{} Temperature difference between \sphinxtitleref{energy\_sink} and the refrigerant fluid of the heat pump (condenser side).
If \sphinxtitleref{energy\_sink} is a ThermalVectorPair, the relevant temperature is the one of the warm vector (outcoming).

\item {} 
\sphinxAtStartPar
\sphinxstyleliteralstrong{\sphinxupquote{as\_HEx}} (\sphinxstyleliteralemphasis{\sphinxupquote{bool}}\sphinxstyleliteralemphasis{\sphinxupquote{, }}\sphinxstyleliteralemphasis{\sphinxupquote{optional}}\sphinxstyleliteralemphasis{\sphinxupquote{, }}\sphinxstyleliteralemphasis{\sphinxupquote{default True}}) \textendash{} 
\sphinxAtStartPar
Describes the behavior of the component when the temperature of the heat source exceeds the one of the heat sink.
\begin{itemize}
\item {} 
\sphinxAtStartPar
If True, the heat pump behaves similarly as a heat exchanger since its COP is very high.
This high COP is calculated by defining equal temperatures for energy source and sink:
new\_source\_temperature(t) = new\_sink\_temperature(t) = (source\_temperature(t)+sink\_temperature(t))/2

\item {} 
\sphinxAtStartPar
If False, COP model is applied as of, which could lead to non physical results.

\end{itemize}


\item {} 
\sphinxAtStartPar
\sphinxstyleliteralstrong{\sphinxupquote{model\_kwargs}} (\sphinxstyleliteralemphasis{\sphinxupquote{keyword arguments passed to the COP calculation function.}}) \textendash{} \begin{itemize}
\item {} 
\sphinxAtStartPar
For CompHP components, these can be:
\begin{itemize}
\item {} 
\sphinxAtStartPar
’fluid’: \{‘ammonia’, ‘isobutane’\}, default ‘ammonia’
Refrigerant fluid of the heat pump.

\item {} 
\sphinxAtStartPar
’eta\_is’, float, default 0.75
Isentropic efficiency of the compression.

\item {} 
\sphinxAtStartPar
’f\_Q’: float, default 0.2
Compressor heat loss ratio.

\end{itemize}

\item {} 
\sphinxAtStartPar
For AbsHP components, these can be:
\begin{itemize}
\item {} 
\sphinxAtStartPar
’couple’: \{‘NH3/H2O’, ‘H2O/LiBr’\}, default ‘H2O/LiBr’
Refrigerant and absorbent fluids.

\end{itemize}

\end{itemize}


\end{itemize}

\end{description}\end{quote}
\subsubsection*{Notes}

\sphinxAtStartPar
See class \sphinxtitleref{COPModels} from tamos.production.

\end{fulllineitems}

\index{units\_number\_lb (tamos.production.AbsHP property)@\spxentry{units\_number\_lb}\spxextra{tamos.production.AbsHP property}}

\begin{fulllineitems}
\phantomsection\label{\detokenize{generated/tamos.production.AbsHP:tamos.production.AbsHP.units_number_lb}}
\pysigstartsignatures
\pysigline{\sphinxbfcode{\sphinxupquote{property\DUrole{w}{  }}}\sphinxbfcode{\sphinxupquote{units\_number\_lb}}}
\pysigstopsignatures
\sphinxAtStartPar
The lower bound of the number of real components that this instance aims to stand for.
Setting \sphinxtitleref{units\_number\_lb} has a meaning if “LB max output power (kW)” property is different from 0.
int

\end{fulllineitems}

\index{units\_number\_ub (tamos.production.AbsHP property)@\spxentry{units\_number\_ub}\spxextra{tamos.production.AbsHP property}}

\begin{fulllineitems}
\phantomsection\label{\detokenize{generated/tamos.production.AbsHP:tamos.production.AbsHP.units_number_ub}}
\pysigstartsignatures
\pysigline{\sphinxbfcode{\sphinxupquote{property\DUrole{w}{  }}}\sphinxbfcode{\sphinxupquote{units\_number\_ub}}}
\pysigstopsignatures
\sphinxAtStartPar
The upper bound of the number of real components that this instance aims to stand for.
Setting \sphinxtitleref{units\_number\_ub} has a meaning if “LB max output power (kW)” property is different from 0.
int

\end{fulllineitems}

\index{used\_elements (tamos.production.AbsHP property)@\spxentry{used\_elements}\spxextra{tamos.production.AbsHP property}}

\begin{fulllineitems}
\phantomsection\label{\detokenize{generated/tamos.production.AbsHP:tamos.production.AbsHP.used_elements}}
\pysigstartsignatures
\pysigline{\sphinxbfcode{\sphinxupquote{property\DUrole{w}{  }}}\sphinxbfcode{\sphinxupquote{used\_elements}}}
\pysigstopsignatures
\sphinxAtStartPar
Elements used by the component.

\end{fulllineitems}


\end{fulllineitems}


\sphinxstepscope


\section{tamos.production.BiomassBoiler}
\label{\detokenize{generated/tamos.production.BiomassBoiler:tamos-production-biomassboiler}}\label{\detokenize{generated/tamos.production.BiomassBoiler::doc}}\index{BiomassBoiler (class in tamos.production)@\spxentry{BiomassBoiler}\spxextra{class in tamos.production}}

\begin{fulllineitems}
\phantomsection\label{\detokenize{generated/tamos.production.BiomassBoiler:tamos.production.BiomassBoiler}}
\pysigstartsignatures
\pysiglinewithargsret{\sphinxbfcode{\sphinxupquote{class\DUrole{w}{  }}}\sphinxcode{\sphinxupquote{tamos.production.}}\sphinxbfcode{\sphinxupquote{BiomassBoiler}}}{\emph{\DUrole{n}{energy\_source}}, \emph{\DUrole{n}{energy\_sink}}, \emph{\DUrole{n}{properties}}, \emph{\DUrole{n}{given\_sizing}\DUrole{o}{=}\DUrole{default_value}{None}}, \emph{\DUrole{n}{name}\DUrole{o}{=}\DUrole{default_value}{None}}, \emph{\DUrole{n}{units\_number\_ub}\DUrole{o}{=}\DUrole{default_value}{1}}, \emph{\DUrole{n}{units\_number\_lb}\DUrole{o}{=}\DUrole{default_value}{1}}, \emph{\DUrole{n}{eco\_count}\DUrole{o}{=}\DUrole{default_value}{True}}}{}
\pysigstopsignatures\index{\_\_init\_\_() (tamos.production.BiomassBoiler method)@\spxentry{\_\_init\_\_()}\spxextra{tamos.production.BiomassBoiler method}}

\begin{fulllineitems}
\phantomsection\label{\detokenize{generated/tamos.production.BiomassBoiler:tamos.production.BiomassBoiler.__init__}}
\pysigstartsignatures
\pysiglinewithargsret{\sphinxbfcode{\sphinxupquote{\_\_init\_\_}}}{\emph{\DUrole{n}{energy\_source}}, \emph{\DUrole{n}{energy\_sink}}, \emph{\DUrole{n}{properties}}, \emph{\DUrole{n}{given\_sizing}\DUrole{o}{=}\DUrole{default_value}{None}}, \emph{\DUrole{n}{name}\DUrole{o}{=}\DUrole{default_value}{None}}, \emph{\DUrole{n}{units\_number\_ub}\DUrole{o}{=}\DUrole{default_value}{1}}, \emph{\DUrole{n}{units\_number\_lb}\DUrole{o}{=}\DUrole{default_value}{1}}, \emph{\DUrole{n}{eco\_count}\DUrole{o}{=}\DUrole{default_value}{True}}}{}
\pysigstopsignatures
\sphinxAtStartPar
BiomassBoiler components produce heat given an energy efficiency of typical biomass boilers.
This efficiency takes into account the condensing property of flue gases.

\sphinxAtStartPar
This component declares the following exported decision variables:
\begin{itemize}
\item {} 
\sphinxAtStartPar
X\_P, binary.
Whether the component is used by the hub.

\item {} 
\sphinxAtStartPar
SP\_P, continuous, in kW.
The maximum capacity of the component. Defines the investment costs.

\item {} 
\sphinxAtStartPar
For all t, for all element e, F\_P(e, t), continuous, in kW.
The power related to element e entering the component (i.e. leaving the hub interface).

\item {} 
\sphinxAtStartPar
For all t, Q\_P(t), continuous, in kW.
The reference power related to the component. Defines the variable cost.
This power is a lower bound of SP\_P.
There exists one element e such that Q\_P(t) = F\_P(e, t) or Q\_P(t) = \sphinxhyphen{} F\_P(e, t).
For this component, e is \sphinxtitleref{energy\_sink}.

\end{itemize}

\sphinxAtStartPar
This component declares the following KPIs:
\begin{itemize}
\item {} 
\sphinxAtStartPar
\sphinxtitleref{COST\_production}
In euros.
Contributes to the “Eco” objective function.

\end{itemize}
\begin{quote}\begin{description}
\sphinxlineitem{Parameters}\begin{itemize}
\item {} 
\sphinxAtStartPar
\sphinxstyleliteralstrong{\sphinxupquote{energy\_source}} ({\hyperref[\detokenize{generated/tamos.element.FuelVector:tamos.element.FuelVector}]{\sphinxcrossref{\sphinxstyleliteralemphasis{\sphinxupquote{FuelVector}}}}}) \textendash{} Biomass that is consumed.

\item {} 
\sphinxAtStartPar
\sphinxstyleliteralstrong{\sphinxupquote{energy\_sink}} (\sphinxstyleliteralemphasis{\sphinxupquote{ThermalVectorPair}}) \textendash{} Thermal flow that is warmed up by the boiler.

\item {} 
\sphinxAtStartPar
\sphinxstyleliteralstrong{\sphinxupquote{properties}} (\sphinxstyleliteralemphasis{\sphinxupquote{dict \{str: int}}\sphinxstyleliteralemphasis{\sphinxupquote{ | }}\sphinxstyleliteralemphasis{\sphinxupquote{float\}}}) \textendash{} 
\sphinxAtStartPar
Techno\sphinxhyphen{}economic properties of the component.
The \sphinxtitleref{properties} attribute must include the following keys:
\begin{itemize}
\item {} 
\sphinxAtStartPar
”LB max output power (kW)”

\item {} 
\sphinxAtStartPar
”UB max output power (kW)”

\item {} 
\sphinxAtStartPar
”CAPEX (EUR/kW)”

\item {} 
\sphinxAtStartPar
”OPEX (\%CAPEX)”

\item {} 
\sphinxAtStartPar
”Variable OPEX (EUR/MWh)”

\end{itemize}


\item {} 
\sphinxAtStartPar
\sphinxstyleliteralstrong{\sphinxupquote{given\_sizing}} (\sphinxstyleliteralemphasis{\sphinxupquote{int}}\sphinxstyleliteralemphasis{\sphinxupquote{ or }}\sphinxstyleliteralemphasis{\sphinxupquote{float}}\sphinxstyleliteralemphasis{\sphinxupquote{, }}\sphinxstyleliteralemphasis{\sphinxupquote{optional}}) \textendash{} The maximum capacity of the component, in kW.
Relates to decision variable ‘SP\_P’.
If specified, only the operation of this component is performed by the MILP solver.
If let unknown, both sizing and operation are performed.

\item {} 
\sphinxAtStartPar
\sphinxstyleliteralstrong{\sphinxupquote{name}} (\sphinxstyleliteralemphasis{\sphinxupquote{str}}\sphinxstyleliteralemphasis{\sphinxupquote{, }}\sphinxstyleliteralemphasis{\sphinxupquote{optional}}) \textendash{} 

\item {} 
\sphinxAtStartPar
\sphinxstyleliteralstrong{\sphinxupquote{units\_number\_lb}} (\sphinxstyleliteralemphasis{\sphinxupquote{int}}\sphinxstyleliteralemphasis{\sphinxupquote{, }}\sphinxstyleliteralemphasis{\sphinxupquote{optional}}\sphinxstyleliteralemphasis{\sphinxupquote{, }}\sphinxstyleliteralemphasis{\sphinxupquote{default 1}}) \textendash{} The lower bound (upper bound) of the number of real components that this instance aims to stand for.
Setting \sphinxtitleref{units\_number\_lb} (\sphinxtitleref{units\_number\_ub}) has a meaning if “LB max output power (kW)” property is
different from 0.

\item {} 
\sphinxAtStartPar
\sphinxstyleliteralstrong{\sphinxupquote{units\_number\_ub}} (\sphinxstyleliteralemphasis{\sphinxupquote{int}}\sphinxstyleliteralemphasis{\sphinxupquote{, }}\sphinxstyleliteralemphasis{\sphinxupquote{optional}}\sphinxstyleliteralemphasis{\sphinxupquote{, }}\sphinxstyleliteralemphasis{\sphinxupquote{default 1}}) \textendash{} The lower bound (upper bound) of the number of real components that this instance aims to stand for.
Setting \sphinxtitleref{units\_number\_lb} (\sphinxtitleref{units\_number\_ub}) has a meaning if “LB max output power (kW)” property is
different from 0.

\item {} 
\sphinxAtStartPar
\sphinxstyleliteralstrong{\sphinxupquote{eco\_count}} (\sphinxstyleliteralemphasis{\sphinxupquote{bool}}\sphinxstyleliteralemphasis{\sphinxupquote{, }}\sphinxstyleliteralemphasis{\sphinxupquote{optional}}\sphinxstyleliteralemphasis{\sphinxupquote{, }}\sphinxstyleliteralemphasis{\sphinxupquote{default True}}) \textendash{} Whether this instance contributes to the system “Eco” KPI.

\end{itemize}

\end{description}\end{quote}

\end{fulllineitems}

\subsubsection*{Methods}


\begin{savenotes}\sphinxattablestart
\centering
\begin{tabulary}{\linewidth}[t]{\X{1}{2}\X{1}{2}}
\hline

\sphinxAtStartPar
{\hyperref[\detokenize{generated/tamos.production.BiomassBoiler:tamos.production.BiomassBoiler.__init__}]{\sphinxcrossref{\sphinxcode{\sphinxupquote{\_\_init\_\_}}}}}(energy\_source, energy\_sink, properties)
&
\sphinxAtStartPar
BiomassBoiler components produce heat given an energy efficiency of typical biomass boilers.
\\
\hline
\sphinxAtStartPar
{\hyperref[\detokenize{generated/tamos.production.BiomassBoiler:tamos.production.BiomassBoiler.compute_actualized_cost}]{\sphinxcrossref{\sphinxcode{\sphinxupquote{compute\_actualized\_cost}}}}}(CAPEX, OPEX, ...{[}, ...{]})
&
\sphinxAtStartPar
Computes the cost of a component using its \textquotesingle{}Lifetime\textquotesingle{} and \textquotesingle{}Discount rate (\%)\textquotesingle{} properties.
\\
\hline
\sphinxAtStartPar
\sphinxcode{\sphinxupquote{default\_efficiency}}(T)
&
\sphinxAtStartPar

\\
\hline
\sphinxAtStartPar
{\hyperref[\detokenize{generated/tamos.production.BiomassBoiler:tamos.production.BiomassBoiler.set_efficiency_model}]{\sphinxcrossref{\sphinxcode{\sphinxupquote{set\_efficiency\_model}}}}}(efficiency\_function, pinch)
&
\sphinxAtStartPar
Defines the efficiency of the conversion of \sphinxtitleref{energy\_source} to \sphinxtitleref{energy\_sink} using a function of the cold temperature of \sphinxtitleref{energy\_sink}.
\\
\hline
\end{tabulary}
\par
\sphinxattableend\end{savenotes}
\subsubsection*{Attributes}


\begin{savenotes}\sphinxattablestart
\centering
\begin{tabulary}{\linewidth}[t]{\X{1}{2}\X{1}{2}}
\hline

\sphinxAtStartPar
{\hyperref[\detokenize{generated/tamos.production.BiomassBoiler:tamos.production.BiomassBoiler.eco_count}]{\sphinxcrossref{\sphinxcode{\sphinxupquote{eco\_count}}}}}
&
\sphinxAtStartPar
Whether this instance contributes to the system "Eco" KPI.
\\
\hline
\sphinxAtStartPar
{\hyperref[\detokenize{generated/tamos.production.BiomassBoiler:tamos.production.BiomassBoiler.efficiency}]{\sphinxcrossref{\sphinxcode{\sphinxupquote{efficiency}}}}}
&
\sphinxAtStartPar
Defines explicitly the efficiency of the conversion of \sphinxtitleref{energy\_source} to \sphinxtitleref{energy\_sink}.
\\
\hline
\sphinxAtStartPar
{\hyperref[\detokenize{generated/tamos.production.BiomassBoiler:tamos.production.BiomassBoiler.energy_sink}]{\sphinxcrossref{\sphinxcode{\sphinxupquote{energy\_sink}}}}}
&
\sphinxAtStartPar
Thermal flow that is warmed up by the boiler.
\\
\hline
\sphinxAtStartPar
\sphinxcode{\sphinxupquote{energy\_source}}
&
\sphinxAtStartPar

\\
\hline
\sphinxAtStartPar
{\hyperref[\detokenize{generated/tamos.production.BiomassBoiler:tamos.production.BiomassBoiler.given_sizing}]{\sphinxcrossref{\sphinxcode{\sphinxupquote{given\_sizing}}}}}
&
\sphinxAtStartPar
The maximum capacity of the component, in kW.
\\
\hline
\sphinxAtStartPar
{\hyperref[\detokenize{generated/tamos.production.BiomassBoiler:tamos.production.BiomassBoiler.name}]{\sphinxcrossref{\sphinxcode{\sphinxupquote{name}}}}}
&
\sphinxAtStartPar
str.
\\
\hline
\sphinxAtStartPar
{\hyperref[\detokenize{generated/tamos.production.BiomassBoiler:tamos.production.BiomassBoiler.units_number_lb}]{\sphinxcrossref{\sphinxcode{\sphinxupquote{units\_number\_lb}}}}}
&
\sphinxAtStartPar
The lower bound of the number of real components that this instance aims to stand for.
\\
\hline
\sphinxAtStartPar
{\hyperref[\detokenize{generated/tamos.production.BiomassBoiler:tamos.production.BiomassBoiler.units_number_ub}]{\sphinxcrossref{\sphinxcode{\sphinxupquote{units\_number\_ub}}}}}
&
\sphinxAtStartPar
The upper bound of the number of real components that this instance aims to stand for.
\\
\hline
\sphinxAtStartPar
{\hyperref[\detokenize{generated/tamos.production.BiomassBoiler:tamos.production.BiomassBoiler.used_elements}]{\sphinxcrossref{\sphinxcode{\sphinxupquote{used\_elements}}}}}
&
\sphinxAtStartPar
Elements used by the component.
\\
\hline
\end{tabulary}
\par
\sphinxattableend\end{savenotes}
\index{compute\_actualized\_cost() (tamos.production.BiomassBoiler method)@\spxentry{compute\_actualized\_cost()}\spxextra{tamos.production.BiomassBoiler method}}

\begin{fulllineitems}
\phantomsection\label{\detokenize{generated/tamos.production.BiomassBoiler:tamos.production.BiomassBoiler.compute_actualized_cost}}
\pysigstartsignatures
\pysiglinewithargsret{\sphinxbfcode{\sphinxupquote{compute\_actualized\_cost}}}{\emph{\DUrole{n}{CAPEX}}, \emph{\DUrole{n}{OPEX}}, \emph{\DUrole{n}{system\_lifetime}}, \emph{\DUrole{n}{lifetime}\DUrole{o}{=}\DUrole{default_value}{None}}, \emph{\DUrole{n}{discount\_rate}\DUrole{o}{=}\DUrole{default_value}{None}}}{}
\pysigstopsignatures
\sphinxAtStartPar
Computes the cost of a component using its ‘Lifetime’ and ‘Discount rate (\%)’ properties.
\begin{quote}\begin{description}
\sphinxlineitem{Parameters}\begin{itemize}
\item {} 
\sphinxAtStartPar
\sphinxstyleliteralstrong{\sphinxupquote{CAPEX}} (\sphinxstyleliteralemphasis{\sphinxupquote{float}}) \textendash{} Capital Expenditure. Cost in euros paid every \sphinxtitleref{technical\_lifetime} periods.

\item {} 
\sphinxAtStartPar
\sphinxstyleliteralstrong{\sphinxupquote{OPEX}} (\sphinxstyleliteralemphasis{\sphinxupquote{float}}) \textendash{} Operational Expenditure. Cost in euros paid each period.

\item {} 
\sphinxAtStartPar
\sphinxstyleliteralstrong{\sphinxupquote{system\_lifetime}} (\sphinxstyleliteralemphasis{\sphinxupquote{int}}) \textendash{} Number of periods defining the existence of the energy system.

\item {} 
\sphinxAtStartPar
\sphinxstyleliteralstrong{\sphinxupquote{lifetime}} (\sphinxstyleliteralemphasis{\sphinxupquote{int}}\sphinxstyleliteralemphasis{\sphinxupquote{, }}\sphinxstyleliteralemphasis{\sphinxupquote{optional}}) \textendash{} Number of periods defining the existence of the component.
If specified, overwrite the “Lifetime” property.

\item {} 
\sphinxAtStartPar
\sphinxstyleliteralstrong{\sphinxupquote{discount\_rate}} (\sphinxstyleliteralemphasis{\sphinxupquote{float}}) \textendash{} In percent (\%). Describes the importance of the economic amortization process, per period.
If specified, overwrite the “Discount rate (\%)” property.

\end{itemize}

\sphinxlineitem{Returns}
\sphinxAtStartPar
\begin{itemize}
\item {} 
\sphinxAtStartPar
\sphinxstyleemphasis{A 3\sphinxhyphen{}tuple (total\_cost, CAPEX\_share, OPEX\_share) where}

\item {} 
\sphinxAtStartPar
* CAPEX\_share is the share of total cost related to \sphinxtitleref{CAPEX}

\item {} 
\sphinxAtStartPar
* OPEX\_share is the share of total cost related to \sphinxtitleref{OPEX}

\item {} 
\sphinxAtStartPar
\sphinxstyleemphasis{* total\_cost = CAPEX\_share + OPEX\_share}

\end{itemize}


\end{description}\end{quote}
\subsubsection*{Notes}

\sphinxAtStartPar
Takes into account residual value of component in the case \sphinxtitleref{system\_lifetime} is not a multiple of \sphinxtitleref{lifetime}.
In this case, the last replacement occuring at period replacement\_period is paid in proportion of ‘CAPEX’
depending linearly on the number of periods left:
CAPEX * (system\_lifetime \sphinxhyphen{} replacement\_period) / lifetime

\end{fulllineitems}

\index{eco\_count (tamos.production.BiomassBoiler property)@\spxentry{eco\_count}\spxextra{tamos.production.BiomassBoiler property}}

\begin{fulllineitems}
\phantomsection\label{\detokenize{generated/tamos.production.BiomassBoiler:tamos.production.BiomassBoiler.eco_count}}
\pysigstartsignatures
\pysigline{\sphinxbfcode{\sphinxupquote{property\DUrole{w}{  }}}\sphinxbfcode{\sphinxupquote{eco\_count}}}
\pysigstopsignatures
\sphinxAtStartPar
Whether this instance contributes to the system “Eco” KPI.
bool

\end{fulllineitems}

\index{efficiency (tamos.production.BiomassBoiler property)@\spxentry{efficiency}\spxextra{tamos.production.BiomassBoiler property}}

\begin{fulllineitems}
\phantomsection\label{\detokenize{generated/tamos.production.BiomassBoiler:tamos.production.BiomassBoiler.efficiency}}
\pysigstartsignatures
\pysigline{\sphinxbfcode{\sphinxupquote{property\DUrole{w}{  }}}\sphinxbfcode{\sphinxupquote{efficiency}}}
\pysigstopsignatures
\sphinxAtStartPar
Defines explicitly the efficiency of the conversion of \sphinxtitleref{energy\_source} to \sphinxtitleref{energy\_sink}.
If called, replaces the definition of the efficiency using \sphinxtitleref{set\_efficiency\_model} (default).
int, float or numpy.ndarray

\end{fulllineitems}

\index{energy\_sink (tamos.production.BiomassBoiler property)@\spxentry{energy\_sink}\spxextra{tamos.production.BiomassBoiler property}}

\begin{fulllineitems}
\phantomsection\label{\detokenize{generated/tamos.production.BiomassBoiler:tamos.production.BiomassBoiler.energy_sink}}
\pysigstartsignatures
\pysigline{\sphinxbfcode{\sphinxupquote{property\DUrole{w}{  }}}\sphinxbfcode{\sphinxupquote{energy\_sink}}}
\pysigstopsignatures
\sphinxAtStartPar
Thermal flow that is warmed up by the boiler.

\end{fulllineitems}

\index{given\_sizing (tamos.production.BiomassBoiler property)@\spxentry{given\_sizing}\spxextra{tamos.production.BiomassBoiler property}}

\begin{fulllineitems}
\phantomsection\label{\detokenize{generated/tamos.production.BiomassBoiler:tamos.production.BiomassBoiler.given_sizing}}
\pysigstartsignatures
\pysigline{\sphinxbfcode{\sphinxupquote{property\DUrole{w}{  }}}\sphinxbfcode{\sphinxupquote{given\_sizing}}}
\pysigstopsignatures
\sphinxAtStartPar
The maximum capacity of the component, in kW.
Relates to decision variable ‘SP\_P’.
int or float

\end{fulllineitems}

\index{name (tamos.production.BiomassBoiler property)@\spxentry{name}\spxextra{tamos.production.BiomassBoiler property}}

\begin{fulllineitems}
\phantomsection\label{\detokenize{generated/tamos.production.BiomassBoiler:tamos.production.BiomassBoiler.name}}
\pysigstartsignatures
\pysigline{\sphinxbfcode{\sphinxupquote{property\DUrole{w}{  }}}\sphinxbfcode{\sphinxupquote{name}}}
\pysigstopsignatures
\sphinxAtStartPar
str.
This name is used in MILP model description.
names must not exceed 255 characters,
all of which must be alphanumeric (a\sphinxhyphen{}z, A\sphinxhyphen{}Z, 0\sphinxhyphen{}9) or one of these symbols:
! ” \# \$ \% \& , . ; ? @ \_ ‘ ’ \{ \} \textasciitilde{}.
\begin{quote}\begin{description}
\sphinxlineitem{Type}
\sphinxAtStartPar
Name of the instance

\end{description}\end{quote}

\end{fulllineitems}

\index{set\_efficiency\_model() (tamos.production.BiomassBoiler method)@\spxentry{set\_efficiency\_model()}\spxextra{tamos.production.BiomassBoiler method}}

\begin{fulllineitems}
\phantomsection\label{\detokenize{generated/tamos.production.BiomassBoiler:tamos.production.BiomassBoiler.set_efficiency_model}}
\pysigstartsignatures
\pysiglinewithargsret{\sphinxbfcode{\sphinxupquote{set\_efficiency\_model}}}{\emph{\DUrole{n}{efficiency\_function}}, \emph{\DUrole{n}{pinch}}}{}
\pysigstopsignatures
\sphinxAtStartPar
Defines the efficiency of the conversion of \sphinxtitleref{energy\_source} to \sphinxtitleref{energy\_sink} using
a function of the cold temperature of \sphinxtitleref{energy\_sink}.
\begin{quote}\begin{description}
\sphinxlineitem{Parameters}\begin{itemize}
\item {} 
\sphinxAtStartPar
\sphinxstyleliteralstrong{\sphinxupquote{efficiency\_function}} (\sphinxstyleliteralemphasis{\sphinxupquote{callable f}}\sphinxstyleliteralemphasis{\sphinxupquote{(}}\sphinxstyleliteralemphasis{\sphinxupquote{T}}\sphinxstyleliteralemphasis{\sphinxupquote{)}}) \textendash{} T is the temperature of the cold vector of \sphinxtitleref{energy\_sink}, in Kelvins (K).

\item {} 
\sphinxAtStartPar
\sphinxstyleliteralstrong{\sphinxupquote{pinch}} (\sphinxstyleliteralemphasis{\sphinxupquote{int}}\sphinxstyleliteralemphasis{\sphinxupquote{, }}\sphinxstyleliteralemphasis{\sphinxupquote{float}}\sphinxstyleliteralemphasis{\sphinxupquote{ or }}\sphinxstyleliteralemphasis{\sphinxupquote{numpy.ndarray}}) \textendash{} Temperature difference between the flue gases of the boiler and the cold vector of \sphinxtitleref{energy\_sink}, in Kelvins (K).

\end{itemize}

\end{description}\end{quote}
\subsubsection*{Notes}

\sphinxAtStartPar
By default, set\_efficiency\_model is called with the \sphinxtitleref{default\_efficiency} attribute of this instance and pinch = 2.

\end{fulllineitems}

\index{units\_number\_lb (tamos.production.BiomassBoiler property)@\spxentry{units\_number\_lb}\spxextra{tamos.production.BiomassBoiler property}}

\begin{fulllineitems}
\phantomsection\label{\detokenize{generated/tamos.production.BiomassBoiler:tamos.production.BiomassBoiler.units_number_lb}}
\pysigstartsignatures
\pysigline{\sphinxbfcode{\sphinxupquote{property\DUrole{w}{  }}}\sphinxbfcode{\sphinxupquote{units\_number\_lb}}}
\pysigstopsignatures
\sphinxAtStartPar
The lower bound of the number of real components that this instance aims to stand for.
Setting \sphinxtitleref{units\_number\_lb} has a meaning if “LB max output power (kW)” property is different from 0.
int

\end{fulllineitems}

\index{units\_number\_ub (tamos.production.BiomassBoiler property)@\spxentry{units\_number\_ub}\spxextra{tamos.production.BiomassBoiler property}}

\begin{fulllineitems}
\phantomsection\label{\detokenize{generated/tamos.production.BiomassBoiler:tamos.production.BiomassBoiler.units_number_ub}}
\pysigstartsignatures
\pysigline{\sphinxbfcode{\sphinxupquote{property\DUrole{w}{  }}}\sphinxbfcode{\sphinxupquote{units\_number\_ub}}}
\pysigstopsignatures
\sphinxAtStartPar
The upper bound of the number of real components that this instance aims to stand for.
Setting \sphinxtitleref{units\_number\_ub} has a meaning if “LB max output power (kW)” property is different from 0.
int

\end{fulllineitems}

\index{used\_elements (tamos.production.BiomassBoiler property)@\spxentry{used\_elements}\spxextra{tamos.production.BiomassBoiler property}}

\begin{fulllineitems}
\phantomsection\label{\detokenize{generated/tamos.production.BiomassBoiler:tamos.production.BiomassBoiler.used_elements}}
\pysigstartsignatures
\pysigline{\sphinxbfcode{\sphinxupquote{property\DUrole{w}{  }}}\sphinxbfcode{\sphinxupquote{used\_elements}}}
\pysigstopsignatures
\sphinxAtStartPar
Elements used by the component.

\end{fulllineitems}


\end{fulllineitems}


\sphinxstepscope


\section{tamos.production.CHP}
\label{\detokenize{generated/tamos.production.CHP:tamos-production-chp}}\label{\detokenize{generated/tamos.production.CHP::doc}}\index{CHP (class in tamos.production)@\spxentry{CHP}\spxextra{class in tamos.production}}

\begin{fulllineitems}
\phantomsection\label{\detokenize{generated/tamos.production.CHP:tamos.production.CHP}}
\pysigstartsignatures
\pysiglinewithargsret{\sphinxbfcode{\sphinxupquote{class\DUrole{w}{  }}}\sphinxcode{\sphinxupquote{tamos.production.}}\sphinxbfcode{\sphinxupquote{CHP}}}{\emph{\DUrole{n}{energy\_source}}, \emph{\DUrole{n}{energy\_sink\_1}}, \emph{\DUrole{n}{energy\_sink\_2}}, \emph{\DUrole{n}{mode}}, \emph{\DUrole{n}{properties}}, \emph{\DUrole{n}{heat\_efficiency\_function}}, \emph{\DUrole{n}{given\_sizing}\DUrole{o}{=}\DUrole{default_value}{None}}, \emph{\DUrole{n}{name}\DUrole{o}{=}\DUrole{default_value}{None}}, \emph{\DUrole{n}{units\_number\_ub}\DUrole{o}{=}\DUrole{default_value}{1}}, \emph{\DUrole{n}{units\_number\_lb}\DUrole{o}{=}\DUrole{default_value}{1}}, \emph{\DUrole{n}{eco\_count}\DUrole{o}{=}\DUrole{default_value}{True}}}{}
\pysigstopsignatures\index{\_\_init\_\_() (tamos.production.CHP method)@\spxentry{\_\_init\_\_()}\spxextra{tamos.production.CHP method}}

\begin{fulllineitems}
\phantomsection\label{\detokenize{generated/tamos.production.CHP:tamos.production.CHP.__init__}}
\pysigstartsignatures
\pysiglinewithargsret{\sphinxbfcode{\sphinxupquote{\_\_init\_\_}}}{\emph{\DUrole{n}{energy\_source}}, \emph{\DUrole{n}{energy\_sink\_1}}, \emph{\DUrole{n}{energy\_sink\_2}}, \emph{\DUrole{n}{mode}}, \emph{\DUrole{n}{properties}}, \emph{\DUrole{n}{heat\_efficiency\_function}}, \emph{\DUrole{n}{given\_sizing}\DUrole{o}{=}\DUrole{default_value}{None}}, \emph{\DUrole{n}{name}\DUrole{o}{=}\DUrole{default_value}{None}}, \emph{\DUrole{n}{units\_number\_ub}\DUrole{o}{=}\DUrole{default_value}{1}}, \emph{\DUrole{n}{units\_number\_lb}\DUrole{o}{=}\DUrole{default_value}{1}}, \emph{\DUrole{n}{eco\_count}\DUrole{o}{=}\DUrole{default_value}{True}}}{}
\pysigstopsignatures
\sphinxAtStartPar
CHP components transform a FuelVector element into heat and electricity.

\sphinxAtStartPar
This model is an adaptation of %
\begin{footnote}[1]\sphinxAtStartFootnote
DAHL, Magnus, BRUN, Adam et ANDRESEN, Gorm B., 2019.
Cost sensitivity of optimal sector\sphinxhyphen{}coupled district heating production systems.
Energy. 1 janvier 2019. Vol.166, pp.624‑636. DOI 10.1016/j.energy.2018.10.044.
%
\end{footnote}.

\sphinxAtStartPar
This component declares the following exported decision variables:
\begin{itemize}
\item {} 
\sphinxAtStartPar
X\_P, binary.
Whether the component is used by the hub.

\item {} 
\sphinxAtStartPar
SP\_P, continuous, in kW.
The maximum capacity of the component. Defines the investment costs.

\item {} 
\sphinxAtStartPar
For all t, for all element e, F\_P(e, t), continuous, in kW.
The power related to element e entering the component (i.e. leaving the hub interface).

\item {} 
\sphinxAtStartPar
For all t, Q\_P(t), continuous, in kW.
The reference power related to the component. Defines the variable cost.
This power is a lower bound of SP\_P.
There exists one element e such that Q\_P(t) = F\_P(e, t) or Q\_P(t) = * F\_P(e, t).
For this component, e is \sphinxtitleref{energy\_sink\_1}.

\end{itemize}

\sphinxAtStartPar
This component declares the following KPIs:
\begin{itemize}
\item {} 
\sphinxAtStartPar
\sphinxtitleref{COST\_production}
In euros.
Contributes to the “Eco” objective function.

\end{itemize}
\begin{quote}\begin{description}
\sphinxlineitem{Parameters}\begin{itemize}
\item {} 
\sphinxAtStartPar
\sphinxstyleliteralstrong{\sphinxupquote{energy\_source}} ({\hyperref[\detokenize{generated/tamos.element.FuelVector:tamos.element.FuelVector}]{\sphinxcrossref{\sphinxstyleliteralemphasis{\sphinxupquote{FuelVector}}}}}) \textendash{} Vector that is consumed.

\item {} 
\sphinxAtStartPar
\sphinxstyleliteralstrong{\sphinxupquote{energy\_sink\_1}} ({\hyperref[\detokenize{generated/tamos.element.ElectricityVector:tamos.element.ElectricityVector}]{\sphinxcrossref{\sphinxstyleliteralemphasis{\sphinxupquote{ElectricityVector}}}}}) \textendash{} 

\item {} 
\sphinxAtStartPar
\sphinxstyleliteralstrong{\sphinxupquote{energy\_sink\_2}} (\sphinxstyleliteralemphasis{\sphinxupquote{ThermalVectorPair}}) \textendash{} Thermal flow that is warmed up by the CHP.

\item {} 
\sphinxAtStartPar
\sphinxstyleliteralstrong{\sphinxupquote{mode}} (\sphinxstyleliteralemphasis{\sphinxupquote{\{\textquotesingle{}By\sphinxhyphen{}pass\textquotesingle{}}}\sphinxstyleliteralemphasis{\sphinxupquote{, }}\sphinxstyleliteralemphasis{\sphinxupquote{\textquotesingle{}Extraction\sphinxhyphen{}condensation\textquotesingle{}}}\sphinxstyleliteralemphasis{\sphinxupquote{, }}\sphinxstyleliteralemphasis{\sphinxupquote{\textquotesingle{}Back\sphinxhyphen{}pressure\textquotesingle{}\}}}) \textendash{} 
\sphinxAtStartPar
Operating mode of the CHP:
\begin{itemize}
\item {} 
\sphinxAtStartPar
’Back\sphinxhyphen{}pressure’: heat production is proportional to electricity production

\item {} 
\sphinxAtStartPar
’Extraction\sphinxhyphen{}condensing’: heat and electricity productions are decoupled, with electricity production being favored

\item {} 
\sphinxAtStartPar
’By\sphinxhyphen{}pass’: heat and electricity productions are decoupled, with heat production being favored

\end{itemize}


\item {} 
\sphinxAtStartPar
\sphinxstyleliteralstrong{\sphinxupquote{properties}} (\sphinxstyleliteralemphasis{\sphinxupquote{dict \{str: int}}\sphinxstyleliteralemphasis{\sphinxupquote{ | }}\sphinxstyleliteralemphasis{\sphinxupquote{float\}}}) \textendash{} 
\sphinxAtStartPar
Techno\sphinxhyphen{}economic properties of the component.
The \sphinxtitleref{properties} attribute must include the following keys:
\begin{itemize}
\item {} 
\sphinxAtStartPar
”alpha”: power\sphinxhyphen{}to\sphinxhyphen{}heat ratio
(back\sphinxhyphen{}pressure line)

\item {} 
\sphinxAtStartPar
”beta”: marginal power ratio (increase in power when heat decreases)
(by\sphinxhyphen{}pass mode)

\item {} 
\sphinxAtStartPar
”ksi”: marginal power ratio (increase in power when heat decreases)
(extraction\sphinxhyphen{}condensation mode)

\item {} 
\sphinxAtStartPar
”eta”: electrical efficiency

\item {} 
\sphinxAtStartPar
”LB max output power (kW)”

\item {} 
\sphinxAtStartPar
”UB max output power (kW)”

\item {} 
\sphinxAtStartPar
”CAPEX (EUR/kW)”

\item {} 
\sphinxAtStartPar
”OPEX (\%CAPEX)”

\item {} 
\sphinxAtStartPar
”Variable OPEX (EUR/MWh)”

\end{itemize}


\item {} 
\sphinxAtStartPar
\sphinxstyleliteralstrong{\sphinxupquote{heat\_efficiency\_function}} (\sphinxstyleliteralemphasis{\sphinxupquote{callable f}}\sphinxstyleliteralemphasis{\sphinxupquote{(}}\sphinxstyleliteralemphasis{\sphinxupquote{T}}\sphinxstyleliteralemphasis{\sphinxupquote{)}}) \textendash{} {[}experimental{]}
Only for modes ‘Back\sphinxhyphen{}pressure’ and ‘Extraction\sphinxhyphen{}condensation’.
T is the temperature of the cold vector of \sphinxtitleref{energy\_sink\_2}, in Kelvins (K).
See method \sphinxtitleref{set\_efficiency\_model} of \sphinxtitleref{GasBoiler} and \sphinxtitleref{BiomassBoiler} classes.
Setting \sphinxtitleref{heat\_efficiency=1} after instanciation makes this attribute unused.

\item {} 
\sphinxAtStartPar
\sphinxstyleliteralstrong{\sphinxupquote{given\_sizing}} (\sphinxstyleliteralemphasis{\sphinxupquote{int}}\sphinxstyleliteralemphasis{\sphinxupquote{ or }}\sphinxstyleliteralemphasis{\sphinxupquote{float}}\sphinxstyleliteralemphasis{\sphinxupquote{, }}\sphinxstyleliteralemphasis{\sphinxupquote{optional}}) \textendash{} The maximum capacity of the component, in kW.
Relates to decision variable ‘SP\_P’.
If specified, only the operation of this component is performed by the MILP solver.
If let unknown, both sizing and operation are performed.

\item {} 
\sphinxAtStartPar
\sphinxstyleliteralstrong{\sphinxupquote{name}} (\sphinxstyleliteralemphasis{\sphinxupquote{str}}\sphinxstyleliteralemphasis{\sphinxupquote{, }}\sphinxstyleliteralemphasis{\sphinxupquote{optional}}) \textendash{} 

\item {} 
\sphinxAtStartPar
\sphinxstyleliteralstrong{\sphinxupquote{units\_number\_lb}} (\sphinxstyleliteralemphasis{\sphinxupquote{int}}\sphinxstyleliteralemphasis{\sphinxupquote{, }}\sphinxstyleliteralemphasis{\sphinxupquote{optional}}\sphinxstyleliteralemphasis{\sphinxupquote{, }}\sphinxstyleliteralemphasis{\sphinxupquote{default 1}}) \textendash{} The lower bound (upper bound) of the number of real components that this instance aims to stand for.
Setting \sphinxtitleref{units\_number\_lb} (\sphinxtitleref{units\_number\_ub}) has a meaning if “LB max output power (kW)” property is
different from 0.

\item {} 
\sphinxAtStartPar
\sphinxstyleliteralstrong{\sphinxupquote{units\_number\_ub}} (\sphinxstyleliteralemphasis{\sphinxupquote{int}}\sphinxstyleliteralemphasis{\sphinxupquote{, }}\sphinxstyleliteralemphasis{\sphinxupquote{optional}}\sphinxstyleliteralemphasis{\sphinxupquote{, }}\sphinxstyleliteralemphasis{\sphinxupquote{default 1}}) \textendash{} The lower bound (upper bound) of the number of real components that this instance aims to stand for.
Setting \sphinxtitleref{units\_number\_lb} (\sphinxtitleref{units\_number\_ub}) has a meaning if “LB max output power (kW)” property is
different from 0.

\item {} 
\sphinxAtStartPar
\sphinxstyleliteralstrong{\sphinxupquote{eco\_count}} (\sphinxstyleliteralemphasis{\sphinxupquote{bool}}\sphinxstyleliteralemphasis{\sphinxupquote{, }}\sphinxstyleliteralemphasis{\sphinxupquote{optional}}\sphinxstyleliteralemphasis{\sphinxupquote{, }}\sphinxstyleliteralemphasis{\sphinxupquote{default True}}) \textendash{} Whether this instance contributes to the system “Eco” KPI.

\end{itemize}

\end{description}\end{quote}
\subsubsection*{References}

\end{fulllineitems}

\subsubsection*{Methods}


\begin{savenotes}\sphinxattablestart
\centering
\begin{tabulary}{\linewidth}[t]{\X{1}{2}\X{1}{2}}
\hline

\sphinxAtStartPar
{\hyperref[\detokenize{generated/tamos.production.CHP:tamos.production.CHP.__init__}]{\sphinxcrossref{\sphinxcode{\sphinxupquote{\_\_init\_\_}}}}}(energy\_source, energy\_sink\_1, ...)
&
\sphinxAtStartPar
CHP components transform a FuelVector element into heat and electricity.
\\
\hline
\sphinxAtStartPar
{\hyperref[\detokenize{generated/tamos.production.CHP:tamos.production.CHP.compute_actualized_cost}]{\sphinxcrossref{\sphinxcode{\sphinxupquote{compute\_actualized\_cost}}}}}(CAPEX, OPEX, ...{[}, ...{]})
&
\sphinxAtStartPar
Computes the cost of a component using its \textquotesingle{}Lifetime\textquotesingle{} and \textquotesingle{}Discount rate (\%)\textquotesingle{} properties.
\\
\hline
\sphinxAtStartPar
{\hyperref[\detokenize{generated/tamos.production.CHP:tamos.production.CHP.set_heat_efficiency_model}]{\sphinxcrossref{\sphinxcode{\sphinxupquote{set\_heat\_efficiency\_model}}}}}(...)
&
\sphinxAtStartPar
{[}experimental{]} Defines the heat efficiency of the CHP.
\\
\hline
\end{tabulary}
\par
\sphinxattableend\end{savenotes}
\subsubsection*{Attributes}


\begin{savenotes}\sphinxattablestart
\centering
\begin{tabulary}{\linewidth}[t]{\X{1}{2}\X{1}{2}}
\hline

\sphinxAtStartPar
{\hyperref[\detokenize{generated/tamos.production.CHP:tamos.production.CHP.eco_count}]{\sphinxcrossref{\sphinxcode{\sphinxupquote{eco\_count}}}}}
&
\sphinxAtStartPar
Whether this instance contributes to the system "Eco" KPI.
\\
\hline
\sphinxAtStartPar
\sphinxcode{\sphinxupquote{energy\_sink\_1}}
&
\sphinxAtStartPar

\\
\hline
\sphinxAtStartPar
\sphinxcode{\sphinxupquote{energy\_sink\_2}}
&
\sphinxAtStartPar

\\
\hline
\sphinxAtStartPar
{\hyperref[\detokenize{generated/tamos.production.CHP:tamos.production.CHP.energy_source}]{\sphinxcrossref{\sphinxcode{\sphinxupquote{energy\_source}}}}}
&
\sphinxAtStartPar
Vector that is consumed.
\\
\hline
\sphinxAtStartPar
{\hyperref[\detokenize{generated/tamos.production.CHP:tamos.production.CHP.given_sizing}]{\sphinxcrossref{\sphinxcode{\sphinxupquote{given\_sizing}}}}}
&
\sphinxAtStartPar
The maximum capacity of the component, in kW.
\\
\hline
\sphinxAtStartPar
{\hyperref[\detokenize{generated/tamos.production.CHP:tamos.production.CHP.heat_efficiency}]{\sphinxcrossref{\sphinxcode{\sphinxupquote{heat\_efficiency}}}}}
&
\sphinxAtStartPar
{[}experimental{]} Defines explicitly the heat efficiency of the CHP.
\\
\hline
\sphinxAtStartPar
{\hyperref[\detokenize{generated/tamos.production.CHP:tamos.production.CHP.mode}]{\sphinxcrossref{\sphinxcode{\sphinxupquote{mode}}}}}
&
\sphinxAtStartPar
Operating mode of the CHP:
\\
\hline
\sphinxAtStartPar
{\hyperref[\detokenize{generated/tamos.production.CHP:tamos.production.CHP.name}]{\sphinxcrossref{\sphinxcode{\sphinxupquote{name}}}}}
&
\sphinxAtStartPar
str.
\\
\hline
\sphinxAtStartPar
{\hyperref[\detokenize{generated/tamos.production.CHP:tamos.production.CHP.units_number_lb}]{\sphinxcrossref{\sphinxcode{\sphinxupquote{units\_number\_lb}}}}}
&
\sphinxAtStartPar
The lower bound of the number of real components that this instance aims to stand for.
\\
\hline
\sphinxAtStartPar
{\hyperref[\detokenize{generated/tamos.production.CHP:tamos.production.CHP.units_number_ub}]{\sphinxcrossref{\sphinxcode{\sphinxupquote{units\_number\_ub}}}}}
&
\sphinxAtStartPar
The upper bound of the number of real components that this instance aims to stand for.
\\
\hline
\sphinxAtStartPar
{\hyperref[\detokenize{generated/tamos.production.CHP:tamos.production.CHP.used_elements}]{\sphinxcrossref{\sphinxcode{\sphinxupquote{used\_elements}}}}}
&
\sphinxAtStartPar
Elements used by the component.
\\
\hline
\end{tabulary}
\par
\sphinxattableend\end{savenotes}
\index{compute\_actualized\_cost() (tamos.production.CHP method)@\spxentry{compute\_actualized\_cost()}\spxextra{tamos.production.CHP method}}

\begin{fulllineitems}
\phantomsection\label{\detokenize{generated/tamos.production.CHP:tamos.production.CHP.compute_actualized_cost}}
\pysigstartsignatures
\pysiglinewithargsret{\sphinxbfcode{\sphinxupquote{compute\_actualized\_cost}}}{\emph{\DUrole{n}{CAPEX}}, \emph{\DUrole{n}{OPEX}}, \emph{\DUrole{n}{system\_lifetime}}, \emph{\DUrole{n}{lifetime}\DUrole{o}{=}\DUrole{default_value}{None}}, \emph{\DUrole{n}{discount\_rate}\DUrole{o}{=}\DUrole{default_value}{None}}}{}
\pysigstopsignatures
\sphinxAtStartPar
Computes the cost of a component using its ‘Lifetime’ and ‘Discount rate (\%)’ properties.
\begin{quote}\begin{description}
\sphinxlineitem{Parameters}\begin{itemize}
\item {} 
\sphinxAtStartPar
\sphinxstyleliteralstrong{\sphinxupquote{CAPEX}} (\sphinxstyleliteralemphasis{\sphinxupquote{float}}) \textendash{} Capital Expenditure. Cost in euros paid every \sphinxtitleref{technical\_lifetime} periods.

\item {} 
\sphinxAtStartPar
\sphinxstyleliteralstrong{\sphinxupquote{OPEX}} (\sphinxstyleliteralemphasis{\sphinxupquote{float}}) \textendash{} Operational Expenditure. Cost in euros paid each period.

\item {} 
\sphinxAtStartPar
\sphinxstyleliteralstrong{\sphinxupquote{system\_lifetime}} (\sphinxstyleliteralemphasis{\sphinxupquote{int}}) \textendash{} Number of periods defining the existence of the energy system.

\item {} 
\sphinxAtStartPar
\sphinxstyleliteralstrong{\sphinxupquote{lifetime}} (\sphinxstyleliteralemphasis{\sphinxupquote{int}}\sphinxstyleliteralemphasis{\sphinxupquote{, }}\sphinxstyleliteralemphasis{\sphinxupquote{optional}}) \textendash{} Number of periods defining the existence of the component.
If specified, overwrite the “Lifetime” property.

\item {} 
\sphinxAtStartPar
\sphinxstyleliteralstrong{\sphinxupquote{discount\_rate}} (\sphinxstyleliteralemphasis{\sphinxupquote{float}}) \textendash{} In percent (\%). Describes the importance of the economic amortization process, per period.
If specified, overwrite the “Discount rate (\%)” property.

\end{itemize}

\sphinxlineitem{Returns}
\sphinxAtStartPar
\begin{itemize}
\item {} 
\sphinxAtStartPar
\sphinxstyleemphasis{A 3\sphinxhyphen{}tuple (total\_cost, CAPEX\_share, OPEX\_share) where}

\item {} 
\sphinxAtStartPar
* CAPEX\_share is the share of total cost related to \sphinxtitleref{CAPEX}

\item {} 
\sphinxAtStartPar
* OPEX\_share is the share of total cost related to \sphinxtitleref{OPEX}

\item {} 
\sphinxAtStartPar
\sphinxstyleemphasis{* total\_cost = CAPEX\_share + OPEX\_share}

\end{itemize}


\end{description}\end{quote}
\subsubsection*{Notes}

\sphinxAtStartPar
Takes into account residual value of component in the case \sphinxtitleref{system\_lifetime} is not a multiple of \sphinxtitleref{lifetime}.
In this case, the last replacement occuring at period replacement\_period is paid in proportion of ‘CAPEX’
depending linearly on the number of periods left:
CAPEX * (system\_lifetime \sphinxhyphen{} replacement\_period) / lifetime

\end{fulllineitems}

\index{eco\_count (tamos.production.CHP property)@\spxentry{eco\_count}\spxextra{tamos.production.CHP property}}

\begin{fulllineitems}
\phantomsection\label{\detokenize{generated/tamos.production.CHP:tamos.production.CHP.eco_count}}
\pysigstartsignatures
\pysigline{\sphinxbfcode{\sphinxupquote{property\DUrole{w}{  }}}\sphinxbfcode{\sphinxupquote{eco\_count}}}
\pysigstopsignatures
\sphinxAtStartPar
Whether this instance contributes to the system “Eco” KPI.
bool

\end{fulllineitems}

\index{energy\_source (tamos.production.CHP property)@\spxentry{energy\_source}\spxextra{tamos.production.CHP property}}

\begin{fulllineitems}
\phantomsection\label{\detokenize{generated/tamos.production.CHP:tamos.production.CHP.energy_source}}
\pysigstartsignatures
\pysigline{\sphinxbfcode{\sphinxupquote{property\DUrole{w}{  }}}\sphinxbfcode{\sphinxupquote{energy\_source}}}
\pysigstopsignatures
\sphinxAtStartPar
Vector that is consumed.

\end{fulllineitems}

\index{given\_sizing (tamos.production.CHP property)@\spxentry{given\_sizing}\spxextra{tamos.production.CHP property}}

\begin{fulllineitems}
\phantomsection\label{\detokenize{generated/tamos.production.CHP:tamos.production.CHP.given_sizing}}
\pysigstartsignatures
\pysigline{\sphinxbfcode{\sphinxupquote{property\DUrole{w}{  }}}\sphinxbfcode{\sphinxupquote{given\_sizing}}}
\pysigstopsignatures
\sphinxAtStartPar
The maximum capacity of the component, in kW.
Relates to decision variable ‘SP\_P’.
int or float

\end{fulllineitems}

\index{heat\_efficiency (tamos.production.CHP property)@\spxentry{heat\_efficiency}\spxextra{tamos.production.CHP property}}

\begin{fulllineitems}
\phantomsection\label{\detokenize{generated/tamos.production.CHP:tamos.production.CHP.heat_efficiency}}
\pysigstartsignatures
\pysigline{\sphinxbfcode{\sphinxupquote{property\DUrole{w}{  }}}\sphinxbfcode{\sphinxupquote{heat\_efficiency}}}
\pysigstopsignatures
\sphinxAtStartPar
{[}experimental{]}
Defines explicitly the heat efficiency of the CHP.
Only for modes ‘Back\sphinxhyphen{}pressure’ and ‘Extraction\sphinxhyphen{}condensation’.
If called, replaces the definition of the efficiency using \sphinxtitleref{set\_heat\_efficiency\_model} (default).
Setting \sphinxtitleref{heat\_efficiency=1} is a safe value.
int, float or numpy.ndarray

\end{fulllineitems}

\index{mode (tamos.production.CHP property)@\spxentry{mode}\spxextra{tamos.production.CHP property}}

\begin{fulllineitems}
\phantomsection\label{\detokenize{generated/tamos.production.CHP:tamos.production.CHP.mode}}
\pysigstartsignatures
\pysigline{\sphinxbfcode{\sphinxupquote{property\DUrole{w}{  }}}\sphinxbfcode{\sphinxupquote{mode}}}
\pysigstopsignatures
\sphinxAtStartPar
Operating mode of the CHP:
\begin{itemize}
\item {} 
\sphinxAtStartPar
‘Back\sphinxhyphen{}pressure’: heat production is proportional to electricity production

\item {} 
\sphinxAtStartPar
‘Extraction\sphinxhyphen{}condensing’: heat and electricity productions are decoupled, with electricity production being favored

\item {} 
\sphinxAtStartPar
‘By\sphinxhyphen{}pass’: heat and electricity productions are decoupled, with heat production being favored

\end{itemize}

\end{fulllineitems}

\index{name (tamos.production.CHP property)@\spxentry{name}\spxextra{tamos.production.CHP property}}

\begin{fulllineitems}
\phantomsection\label{\detokenize{generated/tamos.production.CHP:tamos.production.CHP.name}}
\pysigstartsignatures
\pysigline{\sphinxbfcode{\sphinxupquote{property\DUrole{w}{  }}}\sphinxbfcode{\sphinxupquote{name}}}
\pysigstopsignatures
\sphinxAtStartPar
str.
This name is used in MILP model description.
names must not exceed 255 characters,
all of which must be alphanumeric (a\sphinxhyphen{}z, A\sphinxhyphen{}Z, 0\sphinxhyphen{}9) or one of these symbols:
! ” \# \$ \% \& , . ; ? @ \_ ‘ ’ \{ \} \textasciitilde{}.
\begin{quote}\begin{description}
\sphinxlineitem{Type}
\sphinxAtStartPar
Name of the instance

\end{description}\end{quote}

\end{fulllineitems}

\index{set\_heat\_efficiency\_model() (tamos.production.CHP method)@\spxentry{set\_heat\_efficiency\_model()}\spxextra{tamos.production.CHP method}}

\begin{fulllineitems}
\phantomsection\label{\detokenize{generated/tamos.production.CHP:tamos.production.CHP.set_heat_efficiency_model}}
\pysigstartsignatures
\pysiglinewithargsret{\sphinxbfcode{\sphinxupquote{set\_heat\_efficiency\_model}}}{\emph{\DUrole{n}{heat\_efficiency\_function}}, \emph{\DUrole{n}{pinch}}}{}
\pysigstopsignatures
\sphinxAtStartPar
{[}experimental{]}
Defines the heat efficiency of the CHP.
\begin{quote}\begin{description}
\sphinxlineitem{Parameters}\begin{itemize}
\item {} 
\sphinxAtStartPar
\sphinxstyleliteralstrong{\sphinxupquote{heat\_efficiency\_function}} (\sphinxstyleliteralemphasis{\sphinxupquote{callable f}}\sphinxstyleliteralemphasis{\sphinxupquote{(}}\sphinxstyleliteralemphasis{\sphinxupquote{T}}\sphinxstyleliteralemphasis{\sphinxupquote{)}}) \textendash{} T is the temperature of the cold vector of \sphinxtitleref{energy\_sink}, in Kelvins (K).

\item {} 
\sphinxAtStartPar
\sphinxstyleliteralstrong{\sphinxupquote{pinch}} (\sphinxstyleliteralemphasis{\sphinxupquote{int}}\sphinxstyleliteralemphasis{\sphinxupquote{, }}\sphinxstyleliteralemphasis{\sphinxupquote{float}}\sphinxstyleliteralemphasis{\sphinxupquote{ or }}\sphinxstyleliteralemphasis{\sphinxupquote{numpy.ndarray}}) \textendash{} Temperature difference between the flue gases of the boiler and the cold vector of \sphinxtitleref{energy\_sink}, in Kelvins (K).

\item {} 
\sphinxAtStartPar
\sphinxstyleliteralstrong{\sphinxupquote{classes.}} (\sphinxstyleliteralemphasis{\sphinxupquote{See method set\_efficiency\_model of GasBoiler and BiomassBoiler}}) \textendash{} 

\end{itemize}

\end{description}\end{quote}

\end{fulllineitems}

\index{units\_number\_lb (tamos.production.CHP property)@\spxentry{units\_number\_lb}\spxextra{tamos.production.CHP property}}

\begin{fulllineitems}
\phantomsection\label{\detokenize{generated/tamos.production.CHP:tamos.production.CHP.units_number_lb}}
\pysigstartsignatures
\pysigline{\sphinxbfcode{\sphinxupquote{property\DUrole{w}{  }}}\sphinxbfcode{\sphinxupquote{units\_number\_lb}}}
\pysigstopsignatures
\sphinxAtStartPar
The lower bound of the number of real components that this instance aims to stand for.
Setting \sphinxtitleref{units\_number\_lb} has a meaning if “LB max output power (kW)” property is different from 0.
int

\end{fulllineitems}

\index{units\_number\_ub (tamos.production.CHP property)@\spxentry{units\_number\_ub}\spxextra{tamos.production.CHP property}}

\begin{fulllineitems}
\phantomsection\label{\detokenize{generated/tamos.production.CHP:tamos.production.CHP.units_number_ub}}
\pysigstartsignatures
\pysigline{\sphinxbfcode{\sphinxupquote{property\DUrole{w}{  }}}\sphinxbfcode{\sphinxupquote{units\_number\_ub}}}
\pysigstopsignatures
\sphinxAtStartPar
The upper bound of the number of real components that this instance aims to stand for.
Setting \sphinxtitleref{units\_number\_ub} has a meaning if “LB max output power (kW)” property is different from 0.
int

\end{fulllineitems}

\index{used\_elements (tamos.production.CHP property)@\spxentry{used\_elements}\spxextra{tamos.production.CHP property}}

\begin{fulllineitems}
\phantomsection\label{\detokenize{generated/tamos.production.CHP:tamos.production.CHP.used_elements}}
\pysigstartsignatures
\pysigline{\sphinxbfcode{\sphinxupquote{property\DUrole{w}{  }}}\sphinxbfcode{\sphinxupquote{used\_elements}}}
\pysigstopsignatures
\sphinxAtStartPar
Elements used by the component.

\end{fulllineitems}


\end{fulllineitems}


\sphinxstepscope


\section{tamos.production.COPModels}
\label{\detokenize{generated/tamos.production.COPModels:tamos-production-copmodels}}\label{\detokenize{generated/tamos.production.COPModels::doc}}\index{COPModels (class in tamos.production)@\spxentry{COPModels}\spxextra{class in tamos.production}}

\begin{fulllineitems}
\phantomsection\label{\detokenize{generated/tamos.production.COPModels:tamos.production.COPModels}}
\pysigstartsignatures
\pysigline{\sphinxbfcode{\sphinxupquote{class\DUrole{w}{  }}}\sphinxcode{\sphinxupquote{tamos.production.}}\sphinxbfcode{\sphinxupquote{COPModels}}}
\pysigstopsignatures\index{\_\_init\_\_() (tamos.production.COPModels method)@\spxentry{\_\_init\_\_()}\spxextra{tamos.production.COPModels method}}

\begin{fulllineitems}
\phantomsection\label{\detokenize{generated/tamos.production.COPModels:tamos.production.COPModels.__init__}}
\pysigstartsignatures
\pysiglinewithargsret{\sphinxbfcode{\sphinxupquote{\_\_init\_\_}}}{}{}
\pysigstopsignatures
\end{fulllineitems}

\subsubsection*{Methods}


\begin{savenotes}\sphinxattablestart
\centering
\begin{tabulary}{\linewidth}[t]{\X{1}{2}\X{1}{2}}
\hline

\sphinxAtStartPar
{\hyperref[\detokenize{generated/tamos.production.COPModels:tamos.production.COPModels.__init__}]{\sphinxcrossref{\sphinxcode{\sphinxupquote{\_\_init\_\_}}}}}()
&
\sphinxAtStartPar

\\
\hline
\sphinxAtStartPar
{\hyperref[\detokenize{generated/tamos.production.COPModels:tamos.production.COPModels.absorption}]{\sphinxcrossref{\sphinxcode{\sphinxupquote{absorption}}}}}({[}couple{]})
&
\sphinxAtStartPar
Computes the expression of the COP model for absorption heat pumps.
\\
\hline
\sphinxAtStartPar
{\hyperref[\detokenize{generated/tamos.production.COPModels:tamos.production.COPModels.compression}]{\sphinxcrossref{\sphinxcode{\sphinxupquote{compression}}}}}({[}fluid, eta\_is, f\_Q{]})
&
\sphinxAtStartPar
Computes the expression of the COP model for compression heat pumps.
\\
\hline
\sphinxAtStartPar
{\hyperref[\detokenize{generated/tamos.production.COPModels:tamos.production.COPModels.plot}]{\sphinxcrossref{\sphinxcode{\sphinxupquote{plot}}}}}({[}range\_T\_H\_O, range\_T\_C\_O, ...{]})
&
\sphinxAtStartPar
Plots typical values returned by the absorption and compression COP models.
\\
\hline
\end{tabulary}
\par
\sphinxattableend\end{savenotes}
\index{absorption() (tamos.production.COPModels static method)@\spxentry{absorption()}\spxextra{tamos.production.COPModels static method}}

\begin{fulllineitems}
\phantomsection\label{\detokenize{generated/tamos.production.COPModels:tamos.production.COPModels.absorption}}
\pysigstartsignatures
\pysiglinewithargsret{\sphinxbfcode{\sphinxupquote{static\DUrole{w}{  }}}\sphinxbfcode{\sphinxupquote{absorption}}}{\emph{\DUrole{n}{couple}\DUrole{o}{=}\DUrole{default_value}{\textquotesingle{}H2O/LiBr\textquotesingle{}}}}{}
\pysigstopsignatures
\sphinxAtStartPar
Computes the expression of the COP model for absorption heat pumps.

\sphinxAtStartPar
This model is an adaptation of (Boudéhenn et al., 2016) %
\begin{footnote}[2]\sphinxAtStartFootnote
Boudéhenn F, Bonnot S, Demasles H, Lefrançois F, Perier\sphinxhyphen{}Muzet M, Triché D.
Development and Performances Overview of Ammonia\sphinxhyphen{}water Absorption Chillers with Cooling Capacities from 5 to 100 kW.
Energy Procedia 2016;91:707\textendash{}16. \sphinxurl{https://doi.org/10.1016/j.egypro.2016.06.234}.
%
\end{footnote}.
\begin{quote}\begin{description}
\sphinxlineitem{Parameters}
\sphinxAtStartPar
\sphinxstyleliteralstrong{\sphinxupquote{couple}} (\sphinxstyleliteralemphasis{\sphinxupquote{\{\textquotesingle{}NH3/H2O\textquotesingle{}}}\sphinxstyleliteralemphasis{\sphinxupquote{, }}\sphinxstyleliteralemphasis{\sphinxupquote{\textquotesingle{}H2O/LiBr\textquotesingle{}\}}}\sphinxstyleliteralemphasis{\sphinxupquote{, }}\sphinxstyleliteralemphasis{\sphinxupquote{default \textquotesingle{}H2O/LiBr\textquotesingle{}}}) \textendash{} Refrigerant and absorbent fluids.

\sphinxlineitem{Returns}
\sphinxAtStartPar
\begin{itemize}
\item {} 
\sphinxAtStartPar
T\_H\_O: temperature of the outcoming flow of energy\_sink, in Kelvins (K)

\item {} 
\sphinxAtStartPar
DT\_H: difference of temperature between incoming and outcoming flows of energy\_sink, in Kelvins (K)

\item {} 
\sphinxAtStartPar
T\_C\_O: temperature of the outcoming flow of energy\_source, in Kelvins (K)

\item {} 
\sphinxAtStartPar
DT\_C: difference of temperature between incoming and outcoming flows of energy\_source, in Kelvins (K)

\item {} 
\sphinxAtStartPar
DT\_P\_H: difference of temperature between the refrigerant fluid and energy\_sink, in Kelvins (K)

\item {} 
\sphinxAtStartPar
DT\_P\_C: difference of temperature between the refrigerant fluid and energy\_source, in Kelvins (K)

\item {} 
\sphinxAtStartPar
T\_G\_I: temperature of the incoming flow of energy\_drive, in Kelvins (K)

\end{itemize}


\sphinxlineitem{Return type}
\sphinxAtStartPar
A sympy callable returning the COP. Its arguments are

\end{description}\end{quote}
\subsubsection*{Notes}

\sphinxAtStartPar
The COP is a heating COP, i.e. it is used according to the two following constraints:
1. energy\_from\_source(t) + energy\_from\_drive(t) = energy\_to\_sink(t)
2. energy\_from\_sink(t) = energy\_from\_drive(t) * COP(t)
\subsubsection*{References}

\end{fulllineitems}

\index{compression() (tamos.production.COPModels static method)@\spxentry{compression()}\spxextra{tamos.production.COPModels static method}}

\begin{fulllineitems}
\phantomsection\label{\detokenize{generated/tamos.production.COPModels:tamos.production.COPModels.compression}}
\pysigstartsignatures
\pysiglinewithargsret{\sphinxbfcode{\sphinxupquote{static\DUrole{w}{  }}}\sphinxbfcode{\sphinxupquote{compression}}}{\emph{\DUrole{n}{fluid}\DUrole{o}{=}\DUrole{default_value}{\textquotesingle{}ammonia\textquotesingle{}}}, \emph{\DUrole{n}{eta\_is}\DUrole{o}{=}\DUrole{default_value}{0.75}}, \emph{\DUrole{n}{f\_Q}\DUrole{o}{=}\DUrole{default_value}{0.2}}}{}
\pysigstopsignatures
\sphinxAtStartPar
Computes the expression of the COP model for compression heat pumps.

\sphinxAtStartPar
This model is an adaptation of (Jensen et al., 2018) %
\begin{footnote}[1]\sphinxAtStartFootnote
JENSEN J. K, OMMEN T, REINHOLDT L, Et Al. Heat pump COP, part 2: generalized COP estimation of heat pump processes.
2018. \sphinxurl{https://doi.org/10.18462/IIR.GL.2018.1386}.
%
\end{footnote}.
\begin{quote}\begin{description}
\sphinxlineitem{Parameters}\begin{itemize}
\item {} 
\sphinxAtStartPar
\sphinxstyleliteralstrong{\sphinxupquote{fluid}} (\sphinxstyleliteralemphasis{\sphinxupquote{\{\textquotesingle{}ammonia\textquotesingle{}}}\sphinxstyleliteralemphasis{\sphinxupquote{, }}\sphinxstyleliteralemphasis{\sphinxupquote{\textquotesingle{}isobutane\textquotesingle{}\}}}\sphinxstyleliteralemphasis{\sphinxupquote{, }}\sphinxstyleliteralemphasis{\sphinxupquote{optional}}\sphinxstyleliteralemphasis{\sphinxupquote{, }}\sphinxstyleliteralemphasis{\sphinxupquote{default \textquotesingle{}ammonia\textquotesingle{}}}) \textendash{} Refrigerant fluid of the heat pump.

\item {} 
\sphinxAtStartPar
\sphinxstyleliteralstrong{\sphinxupquote{eta\_is}} (\sphinxstyleliteralemphasis{\sphinxupquote{float}}\sphinxstyleliteralemphasis{\sphinxupquote{, }}\sphinxstyleliteralemphasis{\sphinxupquote{default 0.75}}) \textendash{} Isentropic efficiency of the compression.

\item {} 
\sphinxAtStartPar
\sphinxstyleliteralstrong{\sphinxupquote{f\_Q}} (\sphinxstyleliteralemphasis{\sphinxupquote{float}}\sphinxstyleliteralemphasis{\sphinxupquote{, }}\sphinxstyleliteralemphasis{\sphinxupquote{default 0.2}}) \textendash{} Compressor heat loss ratio.

\end{itemize}

\sphinxlineitem{Returns}
\sphinxAtStartPar
\begin{itemize}
\item {} 
\sphinxAtStartPar
T\_H\_O: temperature of the outcoming flow of energy\_sink, in Kelvins (K)

\item {} 
\sphinxAtStartPar
DT\_H: difference of temperature between incoming and outcoming flows of energy\_sink, in Kelvins (K)

\item {} 
\sphinxAtStartPar
T\_C\_O: temperature of the outcoming flow of energy\_source, in Kelvins (K)

\item {} 
\sphinxAtStartPar
DT\_C: difference of temperature between incoming and outcoming flows of energy\_source, in Kelvins (K)

\item {} 
\sphinxAtStartPar
DT\_P\_H: difference of temperature between the refrigerant fluid and energy\_sink, in Kelvins (K)

\item {} 
\sphinxAtStartPar
DT\_P\_C: difference of temperature between the refrigerant fluid and energy\_source, in Kelvins (K)

\end{itemize}


\sphinxlineitem{Return type}
\sphinxAtStartPar
A sympy callable returning the COP. Its arguments are

\end{description}\end{quote}
\subsubsection*{Notes}

\sphinxAtStartPar
The COP is a heating COP, i.e. it is used according to the two following constraints:
1. energy\_from\_source(t) + energy\_from\_drive(t) = energy\_to\_sink(t)
2. energy\_from\_sink(t) = energy\_from\_drive(t) * COP(t)
\subsubsection*{References}

\end{fulllineitems}

\index{plot() (tamos.production.COPModels static method)@\spxentry{plot()}\spxextra{tamos.production.COPModels static method}}

\begin{fulllineitems}
\phantomsection\label{\detokenize{generated/tamos.production.COPModels:tamos.production.COPModels.plot}}
\pysigstartsignatures
\pysiglinewithargsret{\sphinxbfcode{\sphinxupquote{static\DUrole{w}{  }}}\sphinxbfcode{\sphinxupquote{plot}}}{\emph{\DUrole{n}{range\_T\_H\_O}\DUrole{o}{=}\DUrole{default_value}{range(278, 344)}}, \emph{\DUrole{n}{range\_T\_C\_O}\DUrole{o}{=}\DUrole{default_value}{range(273, 299, 5)}}, \emph{\DUrole{n}{range\_T\_G\_I}\DUrole{o}{=}\DUrole{default_value}{range(363, 403, 10)}}, \emph{\DUrole{n}{celsius}\DUrole{o}{=}\DUrole{default_value}{False}}}{}
\pysigstopsignatures
\sphinxAtStartPar
Plots typical values returned by the absorption and compression COP models.
\begin{quote}\begin{description}
\sphinxlineitem{Parameters}\begin{itemize}
\item {} 
\sphinxAtStartPar
\sphinxstyleliteralstrong{\sphinxupquote{range\_T\_H\_O}} (\sphinxstyleliteralemphasis{\sphinxupquote{list\sphinxhyphen{}like}}\sphinxstyleliteralemphasis{\sphinxupquote{, }}\sphinxstyleliteralemphasis{\sphinxupquote{optional}}\sphinxstyleliteralemphasis{\sphinxupquote{, }}\sphinxstyleliteralemphasis{\sphinxupquote{default range}}\sphinxstyleliteralemphasis{\sphinxupquote{(}}\sphinxstyleliteralemphasis{\sphinxupquote{5+273}}\sphinxstyleliteralemphasis{\sphinxupquote{, }}\sphinxstyleliteralemphasis{\sphinxupquote{71+273}}\sphinxstyleliteralemphasis{\sphinxupquote{, }}\sphinxstyleliteralemphasis{\sphinxupquote{1}}\sphinxstyleliteralemphasis{\sphinxupquote{)}}) \textendash{} Temperatures of the outcoming flow of energy\_sink, in Kelvins (K).

\item {} 
\sphinxAtStartPar
\sphinxstyleliteralstrong{\sphinxupquote{range\_T\_C\_O}} (\sphinxstyleliteralemphasis{\sphinxupquote{list\sphinxhyphen{}like}}\sphinxstyleliteralemphasis{\sphinxupquote{, }}\sphinxstyleliteralemphasis{\sphinxupquote{optional}}\sphinxstyleliteralemphasis{\sphinxupquote{, }}\sphinxstyleliteralemphasis{\sphinxupquote{default range}}\sphinxstyleliteralemphasis{\sphinxupquote{(}}\sphinxstyleliteralemphasis{\sphinxupquote{0+273}}\sphinxstyleliteralemphasis{\sphinxupquote{, }}\sphinxstyleliteralemphasis{\sphinxupquote{26+273}}\sphinxstyleliteralemphasis{\sphinxupquote{, }}\sphinxstyleliteralemphasis{\sphinxupquote{5}}\sphinxstyleliteralemphasis{\sphinxupquote{)}}) \textendash{} Temperatures of the outcoming flow of energy\_source, in Kelvins (K).

\item {} 
\sphinxAtStartPar
\sphinxstyleliteralstrong{\sphinxupquote{range\_T\_G\_I}} (\sphinxstyleliteralemphasis{\sphinxupquote{list\sphinxhyphen{}like}}\sphinxstyleliteralemphasis{\sphinxupquote{, }}\sphinxstyleliteralemphasis{\sphinxupquote{optional}}\sphinxstyleliteralemphasis{\sphinxupquote{, }}\sphinxstyleliteralemphasis{\sphinxupquote{default range}}\sphinxstyleliteralemphasis{\sphinxupquote{(}}\sphinxstyleliteralemphasis{\sphinxupquote{90+273}}\sphinxstyleliteralemphasis{\sphinxupquote{, }}\sphinxstyleliteralemphasis{\sphinxupquote{130+273}}\sphinxstyleliteralemphasis{\sphinxupquote{, }}\sphinxstyleliteralemphasis{\sphinxupquote{10}}\sphinxstyleliteralemphasis{\sphinxupquote{)}}) \textendash{} Temperatures of the incoming flow of energy\_drive, in Kelvins (K).
Relevant for absorption COP model.

\item {} 
\sphinxAtStartPar
\sphinxstyleliteralstrong{\sphinxupquote{celsius}} (\sphinxstyleliteralemphasis{\sphinxupquote{bool}}\sphinxstyleliteralemphasis{\sphinxupquote{, }}\sphinxstyleliteralemphasis{\sphinxupquote{optional}}\sphinxstyleliteralemphasis{\sphinxupquote{, }}\sphinxstyleliteralemphasis{\sphinxupquote{default False}}) \textendash{} If True, arguments \sphinxtitleref{range\_T\_H\_O}, \sphinxtitleref{range\_T\_C\_O} and \sphinxtitleref{range\_T\_G\_I} must describe temperatures in Celsius (°C).

\end{itemize}

\sphinxlineitem{Returns}
\sphinxAtStartPar
\sphinxstylestrong{figs}

\sphinxlineitem{Return type}
\sphinxAtStartPar
list of plotly.graph\_objs.\_figure.Figure

\end{description}\end{quote}
\subsubsection*{Notes}

\sphinxAtStartPar
Figures are displayed on the local host browser using the plotly package.

\end{fulllineitems}


\end{fulllineitems}


\sphinxstepscope


\section{tamos.production.CompHP}
\label{\detokenize{generated/tamos.production.CompHP:tamos-production-comphp}}\label{\detokenize{generated/tamos.production.CompHP::doc}}\index{CompHP (class in tamos.production)@\spxentry{CompHP}\spxextra{class in tamos.production}}

\begin{fulllineitems}
\phantomsection\label{\detokenize{generated/tamos.production.CompHP:tamos.production.CompHP}}
\pysigstartsignatures
\pysiglinewithargsret{\sphinxbfcode{\sphinxupquote{class\DUrole{w}{  }}}\sphinxcode{\sphinxupquote{tamos.production.}}\sphinxbfcode{\sphinxupquote{CompHP}}}{\emph{\DUrole{n}{energy\_drive}}, \emph{\DUrole{n}{energy\_source}}, \emph{\DUrole{n}{energy\_sink}}, \emph{\DUrole{n}{properties}}, \emph{\DUrole{n}{given\_sizing}\DUrole{o}{=}\DUrole{default_value}{None}}, \emph{\DUrole{n}{name}\DUrole{o}{=}\DUrole{default_value}{None}}, \emph{\DUrole{n}{eco\_count}\DUrole{o}{=}\DUrole{default_value}{True}}, \emph{\DUrole{n}{units\_number\_ub}\DUrole{o}{=}\DUrole{default_value}{1}}, \emph{\DUrole{n}{units\_number\_lb}\DUrole{o}{=}\DUrole{default_value}{1}}}{}
\pysigstopsignatures\index{\_\_init\_\_() (tamos.production.CompHP method)@\spxentry{\_\_init\_\_()}\spxextra{tamos.production.CompHP method}}

\begin{fulllineitems}
\phantomsection\label{\detokenize{generated/tamos.production.CompHP:tamos.production.CompHP.__init__}}
\pysigstartsignatures
\pysiglinewithargsret{\sphinxbfcode{\sphinxupquote{\_\_init\_\_}}}{\emph{\DUrole{n}{energy\_drive}}, \emph{\DUrole{n}{energy\_source}}, \emph{\DUrole{n}{energy\_sink}}, \emph{\DUrole{n}{properties}}, \emph{\DUrole{n}{given\_sizing}\DUrole{o}{=}\DUrole{default_value}{None}}, \emph{\DUrole{n}{name}\DUrole{o}{=}\DUrole{default_value}{None}}, \emph{\DUrole{n}{eco\_count}\DUrole{o}{=}\DUrole{default_value}{True}}, \emph{\DUrole{n}{units\_number\_ub}\DUrole{o}{=}\DUrole{default_value}{1}}, \emph{\DUrole{n}{units\_number\_lb}\DUrole{o}{=}\DUrole{default_value}{1}}}{}
\pysigstopsignatures
\sphinxAtStartPar
CompHP components describe vapor\sphinxhyphen{}compression heat pumps.
The COP model is an adaptation of (Jensen et al., 2018) %
\begin{footnote}[1]\sphinxAtStartFootnote
JENSEN J. K, OMMEN T, REINHOLDT L, Et Al. Heat pump COP, part 2: generalized COP estimation of heat pump processes.
2018. \sphinxurl{https://doi.org/10.18462/IIR.GL.2018.1386}.
%
\end{footnote}.

\sphinxAtStartPar
This component declares the following exported decision variables:
\begin{itemize}
\item {} 
\sphinxAtStartPar
X\_P, binary.
Whether the component is used by the hub.

\item {} 
\sphinxAtStartPar
SP\_P, continuous, in kW.
The maximum capacity of the component. Defines the investment costs.

\item {} 
\sphinxAtStartPar
For all t, for all element e, F\_P(e, t), continuous, in kW.
The power related to element e entering the component (i.e. leaving the hub interface).

\item {} 
\sphinxAtStartPar
For all t, Q\_P(t), continuous, in kW.
The reference power related to the component. Defines the variable cost.
This power is a lower bound of SP\_P.
There exists one element e such that Q\_P(t) = F\_P(e, t) or Q\_P(t) = \sphinxhyphen{} F\_P(e, t).
For this component, e is either \sphinxtitleref{energy\_sink} or \sphinxtitleref{energy\_source} depending on the \sphinxtitleref{ref\_production} attribute.

\end{itemize}

\sphinxAtStartPar
This component declares the following KPIs:
\begin{itemize}
\item {} 
\sphinxAtStartPar
\sphinxtitleref{COST\_production}
In euros.
Contributes to the “Eco” objective function.

\end{itemize}
\begin{quote}\begin{description}
\sphinxlineitem{Parameters}\begin{itemize}
\item {} 
\sphinxAtStartPar
\sphinxstyleliteralstrong{\sphinxupquote{energy\_drive}} ({\hyperref[\detokenize{generated/tamos.element.ElectricityVector:tamos.element.ElectricityVector}]{\sphinxcrossref{\sphinxstyleliteralemphasis{\sphinxupquote{ElectricityVector}}}}}) \textendash{} Electricity consumed by the machine.

\item {} 
\sphinxAtStartPar
\sphinxstyleliteralstrong{\sphinxupquote{energy\_source}} (\sphinxstyleliteralemphasis{\sphinxupquote{ThermalVectorPair}}\sphinxstyleliteralemphasis{\sphinxupquote{, }}{\hyperref[\detokenize{generated/tamos.element.ThermalVector:tamos.element.ThermalVector}]{\sphinxcrossref{\sphinxstyleliteralemphasis{\sphinxupquote{ThermalVector}}}}}) \textendash{} Element that gives thermal energy.
Must be cooled down if ThermalVectorPair.

\item {} 
\sphinxAtStartPar
\sphinxstyleliteralstrong{\sphinxupquote{energy\_sink}} (\sphinxstyleliteralemphasis{\sphinxupquote{ThermalVectorPair}}\sphinxstyleliteralemphasis{\sphinxupquote{, }}{\hyperref[\detokenize{generated/tamos.element.ThermalVector:tamos.element.ThermalVector}]{\sphinxcrossref{\sphinxstyleliteralemphasis{\sphinxupquote{ThermalVector}}}}}) \textendash{} Element that receives thermal energy.
Must be warmed up if ThermalVectorPair.

\item {} 
\sphinxAtStartPar
\sphinxstyleliteralstrong{\sphinxupquote{properties}} (\sphinxstyleliteralemphasis{\sphinxupquote{dict \{str: int}}\sphinxstyleliteralemphasis{\sphinxupquote{ | }}\sphinxstyleliteralemphasis{\sphinxupquote{float\}}}) \textendash{} 
\sphinxAtStartPar
Techno\sphinxhyphen{}economic properties of the component.
The \sphinxtitleref{properties} attribute must include the following keys:
\begin{itemize}
\item {} 
\sphinxAtStartPar
”LB max output power (kW)”

\item {} 
\sphinxAtStartPar
”UB max output power (kW)”

\item {} 
\sphinxAtStartPar
”CAPEX (EUR/kW)”

\item {} 
\sphinxAtStartPar
”OPEX (\%CAPEX)”

\item {} 
\sphinxAtStartPar
”Variable OPEX (EUR/MWh)”

\end{itemize}


\item {} 
\sphinxAtStartPar
\sphinxstyleliteralstrong{\sphinxupquote{given\_sizing}} (\sphinxstyleliteralemphasis{\sphinxupquote{int}}\sphinxstyleliteralemphasis{\sphinxupquote{ or }}\sphinxstyleliteralemphasis{\sphinxupquote{float}}\sphinxstyleliteralemphasis{\sphinxupquote{, }}\sphinxstyleliteralemphasis{\sphinxupquote{optional}}) \textendash{} The maximum capacity of the component, in kW.
Relates to decision variable ‘SP\_P’.
If specified, only the operation of this component is performed by the MILP solver.
If let unknown, both sizing and operation are performed.

\item {} 
\sphinxAtStartPar
\sphinxstyleliteralstrong{\sphinxupquote{name}} (\sphinxstyleliteralemphasis{\sphinxupquote{str}}\sphinxstyleliteralemphasis{\sphinxupquote{, }}\sphinxstyleliteralemphasis{\sphinxupquote{optional}}) \textendash{} 

\item {} 
\sphinxAtStartPar
\sphinxstyleliteralstrong{\sphinxupquote{eco\_count}} (\sphinxstyleliteralemphasis{\sphinxupquote{bool}}\sphinxstyleliteralemphasis{\sphinxupquote{, }}\sphinxstyleliteralemphasis{\sphinxupquote{optional}}\sphinxstyleliteralemphasis{\sphinxupquote{, }}\sphinxstyleliteralemphasis{\sphinxupquote{default True}}) \textendash{} Whether this instance contributes to the system “Eco” KPI.

\item {} 
\sphinxAtStartPar
\sphinxstyleliteralstrong{\sphinxupquote{units\_number\_lb}} (\sphinxstyleliteralemphasis{\sphinxupquote{int}}\sphinxstyleliteralemphasis{\sphinxupquote{, }}\sphinxstyleliteralemphasis{\sphinxupquote{optional}}\sphinxstyleliteralemphasis{\sphinxupquote{, }}\sphinxstyleliteralemphasis{\sphinxupquote{default 1}}) \textendash{} The lower bound (upper bound) of the number of real components that this instance aims to stand for.
Setting \sphinxtitleref{units\_number\_lb} (\sphinxtitleref{units\_number\_ub}) has a meaning if “LB max output power (kW)” property is
different from 0.

\item {} 
\sphinxAtStartPar
\sphinxstyleliteralstrong{\sphinxupquote{units\_number\_ub}} (\sphinxstyleliteralemphasis{\sphinxupquote{int}}\sphinxstyleliteralemphasis{\sphinxupquote{, }}\sphinxstyleliteralemphasis{\sphinxupquote{optional}}\sphinxstyleliteralemphasis{\sphinxupquote{, }}\sphinxstyleliteralemphasis{\sphinxupquote{default 1}}) \textendash{} The lower bound (upper bound) of the number of real components that this instance aims to stand for.
Setting \sphinxtitleref{units\_number\_lb} (\sphinxtitleref{units\_number\_ub}) has a meaning if “LB max output power (kW)” property is
different from 0.

\end{itemize}

\end{description}\end{quote}
\subsubsection*{References}

\end{fulllineitems}

\subsubsection*{Methods}


\begin{savenotes}\sphinxattablestart
\centering
\begin{tabulary}{\linewidth}[t]{\X{1}{2}\X{1}{2}}
\hline

\sphinxAtStartPar
{\hyperref[\detokenize{generated/tamos.production.CompHP:tamos.production.CompHP.__init__}]{\sphinxcrossref{\sphinxcode{\sphinxupquote{\_\_init\_\_}}}}}(energy\_drive, energy\_source, ...{[}, ...{]})
&
\sphinxAtStartPar
CompHP components describe vapor\sphinxhyphen{}compression heat pumps.
\\
\hline
\sphinxAtStartPar
{\hyperref[\detokenize{generated/tamos.production.CompHP:tamos.production.CompHP.compute_actualized_cost}]{\sphinxcrossref{\sphinxcode{\sphinxupquote{compute\_actualized\_cost}}}}}(CAPEX, OPEX, ...{[}, ...{]})
&
\sphinxAtStartPar
Computes the cost of a component using its \textquotesingle{}Lifetime\textquotesingle{} and \textquotesingle{}Discount rate (\%)\textquotesingle{} properties.
\\
\hline
\sphinxAtStartPar
{\hyperref[\detokenize{generated/tamos.production.CompHP:tamos.production.CompHP.set_efficiency_model}]{\sphinxcrossref{\sphinxcode{\sphinxupquote{set\_efficiency\_model}}}}}(source\_pinch, sink\_pinch)
&
\sphinxAtStartPar
Defines the efficiency of the heat pump, i.e. the coefficient of performance (COP).
\\
\hline
\end{tabulary}
\par
\sphinxattableend\end{savenotes}
\subsubsection*{Attributes}


\begin{savenotes}\sphinxattablestart
\centering
\begin{tabulary}{\linewidth}[t]{\X{1}{2}\X{1}{2}}
\hline

\sphinxAtStartPar
{\hyperref[\detokenize{generated/tamos.production.CompHP:tamos.production.CompHP.eco_count}]{\sphinxcrossref{\sphinxcode{\sphinxupquote{eco\_count}}}}}
&
\sphinxAtStartPar
Whether this instance contributes to the system "Eco" KPI.
\\
\hline
\sphinxAtStartPar
{\hyperref[\detokenize{generated/tamos.production.CompHP:tamos.production.CompHP.efficiency}]{\sphinxcrossref{\sphinxcode{\sphinxupquote{efficiency}}}}}
&
\sphinxAtStartPar
Defines explicitly the efficiency of the heat pump, i.e. the coefficient of performance (COP).
\\
\hline
\sphinxAtStartPar
{\hyperref[\detokenize{generated/tamos.production.CompHP:tamos.production.CompHP.energy_drive}]{\sphinxcrossref{\sphinxcode{\sphinxupquote{energy\_drive}}}}}
&
\sphinxAtStartPar
Element enabling the pressurization of the heat pump refrigerant.
\\
\hline
\sphinxAtStartPar
{\hyperref[\detokenize{generated/tamos.production.CompHP:tamos.production.CompHP.energy_sink}]{\sphinxcrossref{\sphinxcode{\sphinxupquote{energy\_sink}}}}}
&
\sphinxAtStartPar
Element that receives thermal energy.
\\
\hline
\sphinxAtStartPar
{\hyperref[\detokenize{generated/tamos.production.CompHP:tamos.production.CompHP.energy_source}]{\sphinxcrossref{\sphinxcode{\sphinxupquote{energy\_source}}}}}
&
\sphinxAtStartPar
Element that gives thermal energy.
\\
\hline
\sphinxAtStartPar
{\hyperref[\detokenize{generated/tamos.production.CompHP:tamos.production.CompHP.given_sizing}]{\sphinxcrossref{\sphinxcode{\sphinxupquote{given\_sizing}}}}}
&
\sphinxAtStartPar
The maximum capacity of the component, in kW.
\\
\hline
\sphinxAtStartPar
{\hyperref[\detokenize{generated/tamos.production.CompHP:tamos.production.CompHP.name}]{\sphinxcrossref{\sphinxcode{\sphinxupquote{name}}}}}
&
\sphinxAtStartPar
str.
\\
\hline
\sphinxAtStartPar
{\hyperref[\detokenize{generated/tamos.production.CompHP:tamos.production.CompHP.ref_production}]{\sphinxcrossref{\sphinxcode{\sphinxupquote{ref\_production}}}}}
&
\sphinxAtStartPar
Defines whether \sphinxtitleref{energy\_sink} or \sphinxtitleref{energy\_source} is the reference production.
\\
\hline
\sphinxAtStartPar
{\hyperref[\detokenize{generated/tamos.production.CompHP:tamos.production.CompHP.units_number_lb}]{\sphinxcrossref{\sphinxcode{\sphinxupquote{units\_number\_lb}}}}}
&
\sphinxAtStartPar
The lower bound of the number of real components that this instance aims to stand for.
\\
\hline
\sphinxAtStartPar
{\hyperref[\detokenize{generated/tamos.production.CompHP:tamos.production.CompHP.units_number_ub}]{\sphinxcrossref{\sphinxcode{\sphinxupquote{units\_number\_ub}}}}}
&
\sphinxAtStartPar
The upper bound of the number of real components that this instance aims to stand for.
\\
\hline
\sphinxAtStartPar
{\hyperref[\detokenize{generated/tamos.production.CompHP:tamos.production.CompHP.used_elements}]{\sphinxcrossref{\sphinxcode{\sphinxupquote{used\_elements}}}}}
&
\sphinxAtStartPar
Elements used by the component.
\\
\hline
\end{tabulary}
\par
\sphinxattableend\end{savenotes}
\index{compute\_actualized\_cost() (tamos.production.CompHP method)@\spxentry{compute\_actualized\_cost()}\spxextra{tamos.production.CompHP method}}

\begin{fulllineitems}
\phantomsection\label{\detokenize{generated/tamos.production.CompHP:tamos.production.CompHP.compute_actualized_cost}}
\pysigstartsignatures
\pysiglinewithargsret{\sphinxbfcode{\sphinxupquote{compute\_actualized\_cost}}}{\emph{\DUrole{n}{CAPEX}}, \emph{\DUrole{n}{OPEX}}, \emph{\DUrole{n}{system\_lifetime}}, \emph{\DUrole{n}{lifetime}\DUrole{o}{=}\DUrole{default_value}{None}}, \emph{\DUrole{n}{discount\_rate}\DUrole{o}{=}\DUrole{default_value}{None}}}{}
\pysigstopsignatures
\sphinxAtStartPar
Computes the cost of a component using its ‘Lifetime’ and ‘Discount rate (\%)’ properties.
\begin{quote}\begin{description}
\sphinxlineitem{Parameters}\begin{itemize}
\item {} 
\sphinxAtStartPar
\sphinxstyleliteralstrong{\sphinxupquote{CAPEX}} (\sphinxstyleliteralemphasis{\sphinxupquote{float}}) \textendash{} Capital Expenditure. Cost in euros paid every \sphinxtitleref{technical\_lifetime} periods.

\item {} 
\sphinxAtStartPar
\sphinxstyleliteralstrong{\sphinxupquote{OPEX}} (\sphinxstyleliteralemphasis{\sphinxupquote{float}}) \textendash{} Operational Expenditure. Cost in euros paid each period.

\item {} 
\sphinxAtStartPar
\sphinxstyleliteralstrong{\sphinxupquote{system\_lifetime}} (\sphinxstyleliteralemphasis{\sphinxupquote{int}}) \textendash{} Number of periods defining the existence of the energy system.

\item {} 
\sphinxAtStartPar
\sphinxstyleliteralstrong{\sphinxupquote{lifetime}} (\sphinxstyleliteralemphasis{\sphinxupquote{int}}\sphinxstyleliteralemphasis{\sphinxupquote{, }}\sphinxstyleliteralemphasis{\sphinxupquote{optional}}) \textendash{} Number of periods defining the existence of the component.
If specified, overwrite the “Lifetime” property.

\item {} 
\sphinxAtStartPar
\sphinxstyleliteralstrong{\sphinxupquote{discount\_rate}} (\sphinxstyleliteralemphasis{\sphinxupquote{float}}) \textendash{} In percent (\%). Describes the importance of the economic amortization process, per period.
If specified, overwrite the “Discount rate (\%)” property.

\end{itemize}

\sphinxlineitem{Returns}
\sphinxAtStartPar
\begin{itemize}
\item {} 
\sphinxAtStartPar
\sphinxstyleemphasis{A 3\sphinxhyphen{}tuple (total\_cost, CAPEX\_share, OPEX\_share) where}

\item {} 
\sphinxAtStartPar
* CAPEX\_share is the share of total cost related to \sphinxtitleref{CAPEX}

\item {} 
\sphinxAtStartPar
* OPEX\_share is the share of total cost related to \sphinxtitleref{OPEX}

\item {} 
\sphinxAtStartPar
\sphinxstyleemphasis{* total\_cost = CAPEX\_share + OPEX\_share}

\end{itemize}


\end{description}\end{quote}
\subsubsection*{Notes}

\sphinxAtStartPar
Takes into account residual value of component in the case \sphinxtitleref{system\_lifetime} is not a multiple of \sphinxtitleref{lifetime}.
In this case, the last replacement occuring at period replacement\_period is paid in proportion of ‘CAPEX’
depending linearly on the number of periods left:
CAPEX * (system\_lifetime \sphinxhyphen{} replacement\_period) / lifetime

\end{fulllineitems}

\index{eco\_count (tamos.production.CompHP property)@\spxentry{eco\_count}\spxextra{tamos.production.CompHP property}}

\begin{fulllineitems}
\phantomsection\label{\detokenize{generated/tamos.production.CompHP:tamos.production.CompHP.eco_count}}
\pysigstartsignatures
\pysigline{\sphinxbfcode{\sphinxupquote{property\DUrole{w}{  }}}\sphinxbfcode{\sphinxupquote{eco\_count}}}
\pysigstopsignatures
\sphinxAtStartPar
Whether this instance contributes to the system “Eco” KPI.
bool

\end{fulllineitems}

\index{efficiency (tamos.production.CompHP property)@\spxentry{efficiency}\spxextra{tamos.production.CompHP property}}

\begin{fulllineitems}
\phantomsection\label{\detokenize{generated/tamos.production.CompHP:tamos.production.CompHP.efficiency}}
\pysigstartsignatures
\pysigline{\sphinxbfcode{\sphinxupquote{property\DUrole{w}{  }}}\sphinxbfcode{\sphinxupquote{efficiency}}}
\pysigstopsignatures
\sphinxAtStartPar
Defines explicitly the efficiency of the heat pump, i.e. the coefficient of performance (COP).

\sphinxAtStartPar
The COP is a heating COP, i.e. it is used according to the two following constraints:
1. energy\_from\_source(t) + energy\_from\_drive(t) = energy\_to\_sink(t)
2. energy\_from\_sink(t) = energy\_from\_drive(t) * COP(t)

\sphinxAtStartPar
If called, replaces the definition of the efficiency using \sphinxtitleref{set\_efficiency\_model} (default).
int, float or numpy.ndarray

\end{fulllineitems}

\index{energy\_drive (tamos.production.CompHP property)@\spxentry{energy\_drive}\spxextra{tamos.production.CompHP property}}

\begin{fulllineitems}
\phantomsection\label{\detokenize{generated/tamos.production.CompHP:tamos.production.CompHP.energy_drive}}
\pysigstartsignatures
\pysigline{\sphinxbfcode{\sphinxupquote{property\DUrole{w}{  }}}\sphinxbfcode{\sphinxupquote{energy\_drive}}}
\pysigstopsignatures
\sphinxAtStartPar
Element enabling the pressurization of the heat pump refrigerant.

\end{fulllineitems}

\index{energy\_sink (tamos.production.CompHP property)@\spxentry{energy\_sink}\spxextra{tamos.production.CompHP property}}

\begin{fulllineitems}
\phantomsection\label{\detokenize{generated/tamos.production.CompHP:tamos.production.CompHP.energy_sink}}
\pysigstartsignatures
\pysigline{\sphinxbfcode{\sphinxupquote{property\DUrole{w}{  }}}\sphinxbfcode{\sphinxupquote{energy\_sink}}}
\pysigstopsignatures
\sphinxAtStartPar
Element that receives thermal energy.
Must be warmed up if ThermalVectorPair.                ——\sphinxhyphen{}

\end{fulllineitems}

\index{energy\_source (tamos.production.CompHP property)@\spxentry{energy\_source}\spxextra{tamos.production.CompHP property}}

\begin{fulllineitems}
\phantomsection\label{\detokenize{generated/tamos.production.CompHP:tamos.production.CompHP.energy_source}}
\pysigstartsignatures
\pysigline{\sphinxbfcode{\sphinxupquote{property\DUrole{w}{  }}}\sphinxbfcode{\sphinxupquote{energy\_source}}}
\pysigstopsignatures
\sphinxAtStartPar
Element that gives thermal energy.
Must be cooled down if ThermalVectorPair.

\end{fulllineitems}

\index{given\_sizing (tamos.production.CompHP property)@\spxentry{given\_sizing}\spxextra{tamos.production.CompHP property}}

\begin{fulllineitems}
\phantomsection\label{\detokenize{generated/tamos.production.CompHP:tamos.production.CompHP.given_sizing}}
\pysigstartsignatures
\pysigline{\sphinxbfcode{\sphinxupquote{property\DUrole{w}{  }}}\sphinxbfcode{\sphinxupquote{given\_sizing}}}
\pysigstopsignatures
\sphinxAtStartPar
The maximum capacity of the component, in kW.
Relates to decision variable ‘SP\_P’.
int or float

\end{fulllineitems}

\index{name (tamos.production.CompHP property)@\spxentry{name}\spxextra{tamos.production.CompHP property}}

\begin{fulllineitems}
\phantomsection\label{\detokenize{generated/tamos.production.CompHP:tamos.production.CompHP.name}}
\pysigstartsignatures
\pysigline{\sphinxbfcode{\sphinxupquote{property\DUrole{w}{  }}}\sphinxbfcode{\sphinxupquote{name}}}
\pysigstopsignatures
\sphinxAtStartPar
str.
This name is used in MILP model description.
names must not exceed 255 characters,
all of which must be alphanumeric (a\sphinxhyphen{}z, A\sphinxhyphen{}Z, 0\sphinxhyphen{}9) or one of these symbols:
! ” \# \$ \% \& , . ; ? @ \_ ‘ ’ \{ \} \textasciitilde{}.
\begin{quote}\begin{description}
\sphinxlineitem{Type}
\sphinxAtStartPar
Name of the instance

\end{description}\end{quote}

\end{fulllineitems}

\index{ref\_production (tamos.production.CompHP property)@\spxentry{ref\_production}\spxextra{tamos.production.CompHP property}}

\begin{fulllineitems}
\phantomsection\label{\detokenize{generated/tamos.production.CompHP:tamos.production.CompHP.ref_production}}
\pysigstartsignatures
\pysigline{\sphinxbfcode{\sphinxupquote{property\DUrole{w}{  }}}\sphinxbfcode{\sphinxupquote{ref\_production}}}
\pysigstopsignatures
\sphinxAtStartPar
Defines whether \sphinxtitleref{energy\_sink} or \sphinxtitleref{energy\_source} is the reference production.
Either \sphinxtitleref{energy\_sink} or \sphinxtitleref{energy\_source}.

\sphinxAtStartPar
The reference production defines decision variable ‘Q\_P’ and thus ‘SP\_P’,
which defines the cost of the component.
If costs properties are typical values for a heat pump use, set \sphinxtitleref{HP.ref\_production=HP.energy\_sink}.
If the costs properties are given for a chiller use, set \sphinxtitleref{HP.ref\_production=HP.energy\_source}.

\end{fulllineitems}

\index{set\_efficiency\_model() (tamos.production.CompHP method)@\spxentry{set\_efficiency\_model()}\spxextra{tamos.production.CompHP method}}

\begin{fulllineitems}
\phantomsection\label{\detokenize{generated/tamos.production.CompHP:tamos.production.CompHP.set_efficiency_model}}
\pysigstartsignatures
\pysiglinewithargsret{\sphinxbfcode{\sphinxupquote{set\_efficiency\_model}}}{\emph{\DUrole{n}{source\_pinch}}, \emph{\DUrole{n}{sink\_pinch}}, \emph{\DUrole{n}{as\_HEx}\DUrole{o}{=}\DUrole{default_value}{True}}, \emph{\DUrole{o}{**}\DUrole{n}{model\_kwargs}}}{}
\pysigstopsignatures
\sphinxAtStartPar
Defines the efficiency of the heat pump, i.e. the coefficient of performance (COP).

\sphinxAtStartPar
The COP is a heating COP, i.e. it is used according to the two following constraints:
1. energy\_from\_source(t) + energy\_from\_drive(t) = energy\_to\_sink(t)
2. energy\_from\_sink(t) = energy\_from\_drive(t) * COP(t)
\begin{quote}\begin{description}
\sphinxlineitem{Parameters}\begin{itemize}
\item {} 
\sphinxAtStartPar
\sphinxstyleliteralstrong{\sphinxupquote{source\_pinch}} (\sphinxstyleliteralemphasis{\sphinxupquote{int}}\sphinxstyleliteralemphasis{\sphinxupquote{, }}\sphinxstyleliteralemphasis{\sphinxupquote{float}}\sphinxstyleliteralemphasis{\sphinxupquote{ or }}\sphinxstyleliteralemphasis{\sphinxupquote{numpy.ndarray}}) \textendash{} Temperature difference between \sphinxtitleref{energy\_source} and the refrigerant fluid of the heat pump (evaporator side).
If \sphinxtitleref{energy\_source} is a ThermalVectorPair, the relevant temperature is the one of the cold vector (outcoming).

\item {} 
\sphinxAtStartPar
\sphinxstyleliteralstrong{\sphinxupquote{sink\_pinch}} (\sphinxstyleliteralemphasis{\sphinxupquote{int}}\sphinxstyleliteralemphasis{\sphinxupquote{, }}\sphinxstyleliteralemphasis{\sphinxupquote{float}}\sphinxstyleliteralemphasis{\sphinxupquote{ or }}\sphinxstyleliteralemphasis{\sphinxupquote{numpy.ndarray}}) \textendash{} Temperature difference between \sphinxtitleref{energy\_sink} and the refrigerant fluid of the heat pump (condenser side).
If \sphinxtitleref{energy\_sink} is a ThermalVectorPair, the relevant temperature is the one of the warm vector (outcoming).

\item {} 
\sphinxAtStartPar
\sphinxstyleliteralstrong{\sphinxupquote{as\_HEx}} (\sphinxstyleliteralemphasis{\sphinxupquote{bool}}\sphinxstyleliteralemphasis{\sphinxupquote{, }}\sphinxstyleliteralemphasis{\sphinxupquote{optional}}\sphinxstyleliteralemphasis{\sphinxupquote{, }}\sphinxstyleliteralemphasis{\sphinxupquote{default True}}) \textendash{} 
\sphinxAtStartPar
Describes the behavior of the component when the temperature of the heat source exceeds the one of the heat sink.
\begin{itemize}
\item {} 
\sphinxAtStartPar
If True, the heat pump behaves similarly as a heat exchanger since its COP is very high.
This high COP is calculated by defining equal temperatures for energy source and sink:
new\_source\_temperature(t) = new\_sink\_temperature(t) = (source\_temperature(t)+sink\_temperature(t))/2

\item {} 
\sphinxAtStartPar
If False, COP model is applied as of, which could lead to non physical results.

\end{itemize}


\item {} 
\sphinxAtStartPar
\sphinxstyleliteralstrong{\sphinxupquote{model\_kwargs}} (\sphinxstyleliteralemphasis{\sphinxupquote{keyword arguments passed to the COP calculation function.}}) \textendash{} \begin{itemize}
\item {} 
\sphinxAtStartPar
For CompHP components, these can be:
\begin{itemize}
\item {} 
\sphinxAtStartPar
’fluid’: \{‘ammonia’, ‘isobutane’\}, default ‘ammonia’
Refrigerant fluid of the heat pump.

\item {} 
\sphinxAtStartPar
’eta\_is’, float, default 0.75
Isentropic efficiency of the compression.

\item {} 
\sphinxAtStartPar
’f\_Q’: float, default 0.2
Compressor heat loss ratio.

\end{itemize}

\item {} 
\sphinxAtStartPar
For AbsHP components, these can be:
\begin{itemize}
\item {} 
\sphinxAtStartPar
’couple’: \{‘NH3/H2O’, ‘H2O/LiBr’\}, default ‘H2O/LiBr’
Refrigerant and absorbent fluids.

\end{itemize}

\end{itemize}


\end{itemize}

\end{description}\end{quote}
\subsubsection*{Notes}

\sphinxAtStartPar
See class \sphinxtitleref{COPModels} from tamos.production.

\end{fulllineitems}

\index{units\_number\_lb (tamos.production.CompHP property)@\spxentry{units\_number\_lb}\spxextra{tamos.production.CompHP property}}

\begin{fulllineitems}
\phantomsection\label{\detokenize{generated/tamos.production.CompHP:tamos.production.CompHP.units_number_lb}}
\pysigstartsignatures
\pysigline{\sphinxbfcode{\sphinxupquote{property\DUrole{w}{  }}}\sphinxbfcode{\sphinxupquote{units\_number\_lb}}}
\pysigstopsignatures
\sphinxAtStartPar
The lower bound of the number of real components that this instance aims to stand for.
Setting \sphinxtitleref{units\_number\_lb} has a meaning if “LB max output power (kW)” property is different from 0.
int

\end{fulllineitems}

\index{units\_number\_ub (tamos.production.CompHP property)@\spxentry{units\_number\_ub}\spxextra{tamos.production.CompHP property}}

\begin{fulllineitems}
\phantomsection\label{\detokenize{generated/tamos.production.CompHP:tamos.production.CompHP.units_number_ub}}
\pysigstartsignatures
\pysigline{\sphinxbfcode{\sphinxupquote{property\DUrole{w}{  }}}\sphinxbfcode{\sphinxupquote{units\_number\_ub}}}
\pysigstopsignatures
\sphinxAtStartPar
The upper bound of the number of real components that this instance aims to stand for.
Setting \sphinxtitleref{units\_number\_ub} has a meaning if “LB max output power (kW)” property is different from 0.
int

\end{fulllineitems}

\index{used\_elements (tamos.production.CompHP property)@\spxentry{used\_elements}\spxextra{tamos.production.CompHP property}}

\begin{fulllineitems}
\phantomsection\label{\detokenize{generated/tamos.production.CompHP:tamos.production.CompHP.used_elements}}
\pysigstartsignatures
\pysigline{\sphinxbfcode{\sphinxupquote{property\DUrole{w}{  }}}\sphinxbfcode{\sphinxupquote{used\_elements}}}
\pysigstopsignatures
\sphinxAtStartPar
Elements used by the component.

\end{fulllineitems}


\end{fulllineitems}


\sphinxstepscope


\section{tamos.production.DryCooler}
\label{\detokenize{generated/tamos.production.DryCooler:tamos-production-drycooler}}\label{\detokenize{generated/tamos.production.DryCooler::doc}}\index{DryCooler (class in tamos.production)@\spxentry{DryCooler}\spxextra{class in tamos.production}}

\begin{fulllineitems}
\phantomsection\label{\detokenize{generated/tamos.production.DryCooler:tamos.production.DryCooler}}
\pysigstartsignatures
\pysiglinewithargsret{\sphinxbfcode{\sphinxupquote{class\DUrole{w}{  }}}\sphinxcode{\sphinxupquote{tamos.production.}}\sphinxbfcode{\sphinxupquote{DryCooler}}}{\emph{\DUrole{n}{energy\_drive}}, \emph{\DUrole{n}{energy\_source}}, \emph{\DUrole{n}{energy\_sink}}, \emph{\DUrole{n}{properties}}, \emph{\DUrole{n}{efficiency}\DUrole{o}{=}\DUrole{default_value}{36.36363636363637}}, \emph{\DUrole{n}{given\_sizing}\DUrole{o}{=}\DUrole{default_value}{None}}, \emph{\DUrole{n}{name}\DUrole{o}{=}\DUrole{default_value}{None}}, \emph{\DUrole{n}{units\_number\_ub}\DUrole{o}{=}\DUrole{default_value}{1}}, \emph{\DUrole{n}{units\_number\_lb}\DUrole{o}{=}\DUrole{default_value}{1}}, \emph{\DUrole{n}{eco\_count}\DUrole{o}{=}\DUrole{default_value}{True}}}{}
\pysigstopsignatures\index{\_\_init\_\_() (tamos.production.DryCooler method)@\spxentry{\_\_init\_\_()}\spxextra{tamos.production.DryCooler method}}

\begin{fulllineitems}
\phantomsection\label{\detokenize{generated/tamos.production.DryCooler:tamos.production.DryCooler.__init__}}
\pysigstartsignatures
\pysiglinewithargsret{\sphinxbfcode{\sphinxupquote{\_\_init\_\_}}}{\emph{\DUrole{n}{energy\_drive}}, \emph{\DUrole{n}{energy\_source}}, \emph{\DUrole{n}{energy\_sink}}, \emph{\DUrole{n}{properties}}, \emph{\DUrole{n}{efficiency}\DUrole{o}{=}\DUrole{default_value}{36.36363636363637}}, \emph{\DUrole{n}{given\_sizing}\DUrole{o}{=}\DUrole{default_value}{None}}, \emph{\DUrole{n}{name}\DUrole{o}{=}\DUrole{default_value}{None}}, \emph{\DUrole{n}{units\_number\_ub}\DUrole{o}{=}\DUrole{default_value}{1}}, \emph{\DUrole{n}{units\_number\_lb}\DUrole{o}{=}\DUrole{default_value}{1}}, \emph{\DUrole{n}{eco\_count}\DUrole{o}{=}\DUrole{default_value}{True}}}{}
\pysigstopsignatures
\sphinxAtStartPar
DryCooler components dissipate thermal energy using fans having an electrical consumption.

\sphinxAtStartPar
This component declares the following exported decision variables:
\begin{itemize}
\item {} 
\sphinxAtStartPar
X\_P, binary.
Whether the component is used by the hub.

\item {} 
\sphinxAtStartPar
SP\_P, continuous, in kW.
The maximum capacity of the component. Defines the investment costs.

\item {} 
\sphinxAtStartPar
For all t, for all element e, F\_P(e, t), continuous, in kW.
The power related to element e entering the component (i.e. leaving the hub interface).

\item {} 
\sphinxAtStartPar
For all t, Q\_P(t), continuous, in kW.
The reference power related to the component. Defines the variable cost.
This power is a lower bound of SP\_P.
There exists one element e such that Q\_P(t) = F\_P(e, t) or Q\_P(t) = \sphinxhyphen{} F\_P(e, t).
For this component, e is \sphinxtitleref{energy\_source}.

\end{itemize}

\sphinxAtStartPar
This component declares the following KPIs:
\begin{itemize}
\item {} 
\sphinxAtStartPar
\sphinxtitleref{COST\_production}
In euros.
Contributes to the “Eco” objective function.

\end{itemize}
\begin{quote}\begin{description}
\sphinxlineitem{Parameters}\begin{itemize}
\item {} 
\sphinxAtStartPar
\sphinxstyleliteralstrong{\sphinxupquote{energy\_drive}} ({\hyperref[\detokenize{generated/tamos.element.ElectricityVector:tamos.element.ElectricityVector}]{\sphinxcrossref{\sphinxstyleliteralemphasis{\sphinxupquote{ElectricityVector}}}}}) \textendash{} Electricty used to facilitate energy transfer (e.g. using fans).

\item {} 
\sphinxAtStartPar
\sphinxstyleliteralstrong{\sphinxupquote{energy\_source}} (\sphinxstyleliteralemphasis{\sphinxupquote{ThermalVectorPair}}\sphinxstyleliteralemphasis{\sphinxupquote{, }}{\hyperref[\detokenize{generated/tamos.element.ThermalVector:tamos.element.ThermalVector}]{\sphinxcrossref{\sphinxstyleliteralemphasis{\sphinxupquote{ThermalVector}}}}}) \textendash{} Element which must be dissipated.
Must be cooled down if ThermalVectorPair.

\item {} 
\sphinxAtStartPar
\sphinxstyleliteralstrong{\sphinxupquote{energy\_sink}} (\sphinxstyleliteralemphasis{\sphinxupquote{ThermalVectorPair}}\sphinxstyleliteralemphasis{\sphinxupquote{, }}{\hyperref[\detokenize{generated/tamos.element.ThermalVector:tamos.element.ThermalVector}]{\sphinxcrossref{\sphinxstyleliteralemphasis{\sphinxupquote{ThermalVector}}}}}) \textendash{} Element receiving thermal energy of the dissipation of \sphinxtitleref{energy\_source}.
Must be warmed up if ThermalVectorPair.

\item {} 
\sphinxAtStartPar
\sphinxstyleliteralstrong{\sphinxupquote{properties}} (\sphinxstyleliteralemphasis{\sphinxupquote{dict \{str: int}}\sphinxstyleliteralemphasis{\sphinxupquote{ | }}\sphinxstyleliteralemphasis{\sphinxupquote{float\}}}) \textendash{} 
\sphinxAtStartPar
Techno\sphinxhyphen{}economic properties of the component.
The \sphinxtitleref{properties} attribute must include the following keys:
\begin{itemize}
\item {} 
\sphinxAtStartPar
”LB max output power (kW)”

\item {} 
\sphinxAtStartPar
”UB max output power (kW)”

\item {} 
\sphinxAtStartPar
”CAPEX (EUR/kW)”

\item {} 
\sphinxAtStartPar
”OPEX (\%CAPEX)”

\item {} 
\sphinxAtStartPar
”Variable OPEX (EUR/MWh)”

\end{itemize}


\item {} 
\sphinxAtStartPar
\sphinxstyleliteralstrong{\sphinxupquote{efficiency}} (\sphinxstyleliteralemphasis{\sphinxupquote{int}}\sphinxstyleliteralemphasis{\sphinxupquote{, }}\sphinxstyleliteralemphasis{\sphinxupquote{float}}\sphinxstyleliteralemphasis{\sphinxupquote{ or }}\sphinxstyleliteralemphasis{\sphinxupquote{numpy.ndarray}}\sphinxstyleliteralemphasis{\sphinxupquote{, }}\sphinxstyleliteralemphasis{\sphinxupquote{optional}}\sphinxstyleliteralemphasis{\sphinxupquote{, }}\sphinxstyleliteralemphasis{\sphinxupquote{default 1/0.0275}}) \textendash{} Number of units of dissipated heat per unit of electricity.

\item {} 
\sphinxAtStartPar
\sphinxstyleliteralstrong{\sphinxupquote{given\_sizing}} (\sphinxstyleliteralemphasis{\sphinxupquote{int}}\sphinxstyleliteralemphasis{\sphinxupquote{ or }}\sphinxstyleliteralemphasis{\sphinxupquote{float}}\sphinxstyleliteralemphasis{\sphinxupquote{, }}\sphinxstyleliteralemphasis{\sphinxupquote{optional}}) \textendash{} The maximum capacity of the component, in kW.
Relates to decision variable ‘SP\_P’.
If specified, only the operation of this component is performed by the MILP solver.
If let unknown, both sizing and operation are performed.

\item {} 
\sphinxAtStartPar
\sphinxstyleliteralstrong{\sphinxupquote{name}} (\sphinxstyleliteralemphasis{\sphinxupquote{str}}\sphinxstyleliteralemphasis{\sphinxupquote{, }}\sphinxstyleliteralemphasis{\sphinxupquote{optional}}) \textendash{} 

\item {} 
\sphinxAtStartPar
\sphinxstyleliteralstrong{\sphinxupquote{units\_number\_lb}} (\sphinxstyleliteralemphasis{\sphinxupquote{int}}\sphinxstyleliteralemphasis{\sphinxupquote{, }}\sphinxstyleliteralemphasis{\sphinxupquote{optional}}\sphinxstyleliteralemphasis{\sphinxupquote{, }}\sphinxstyleliteralemphasis{\sphinxupquote{default 1}}) \textendash{} The lower bound (upper bound) of the number of real components that this instance aims to stand for.
Setting \sphinxtitleref{units\_number\_lb} (\sphinxtitleref{units\_number\_ub}) has a meaning if “LB max output power (kW)” property is
different from 0.

\item {} 
\sphinxAtStartPar
\sphinxstyleliteralstrong{\sphinxupquote{units\_number\_ub}} (\sphinxstyleliteralemphasis{\sphinxupquote{int}}\sphinxstyleliteralemphasis{\sphinxupquote{, }}\sphinxstyleliteralemphasis{\sphinxupquote{optional}}\sphinxstyleliteralemphasis{\sphinxupquote{, }}\sphinxstyleliteralemphasis{\sphinxupquote{default 1}}) \textendash{} The lower bound (upper bound) of the number of real components that this instance aims to stand for.
Setting \sphinxtitleref{units\_number\_lb} (\sphinxtitleref{units\_number\_ub}) has a meaning if “LB max output power (kW)” property is
different from 0.

\item {} 
\sphinxAtStartPar
\sphinxstyleliteralstrong{\sphinxupquote{eco\_count}} (\sphinxstyleliteralemphasis{\sphinxupquote{bool}}\sphinxstyleliteralemphasis{\sphinxupquote{, }}\sphinxstyleliteralemphasis{\sphinxupquote{optional}}\sphinxstyleliteralemphasis{\sphinxupquote{, }}\sphinxstyleliteralemphasis{\sphinxupquote{default True}}) \textendash{} Whether this instance contributes to the system “Eco” KPI.

\end{itemize}

\end{description}\end{quote}
\subsubsection*{Notes}

\sphinxAtStartPar
Flows of \sphinxtitleref{energy\_source} and \sphinxtitleref{energy\_sink} are equal.
\subsubsection*{Examples}

\begin{sphinxVerbatim}[commandchars=\\\{\}]
\PYG{g+gp}{\PYGZgt{}\PYGZgt{}\PYGZgt{} }\PYG{n}{air} \PYG{o}{=} \PYG{n}{ThermalVector}\PYG{p}{(}\PYG{n}{temperature} \PYG{o}{=} \PYG{l+m+mi}{273} \PYG{o}{+} \PYG{l+m+mi}{20}\PYG{p}{,} \PYG{n}{name}\PYG{o}{=}\PYG{l+s+s2}{\PYGZdq{}}\PYG{l+s+s2}{Ambiant air}\PYG{l+s+s2}{\PYGZdq{}}\PYG{p}{)}
\PYG{g+gp}{\PYGZgt{}\PYGZgt{}\PYGZgt{} }\PYG{n}{excess\PYGZus{}heat} \PYG{o}{=} \PYG{n}{ThermalVectorPair}\PYG{p}{(}\PYG{n}{in\PYGZus{}TV}\PYG{o}{=}\PYG{n}{in\PYGZus{}TV}\PYG{p}{,} \PYG{n}{out\PYGZus{}TV}\PYG{o}{=}\PYG{n}{out\PYGZus{}TV}\PYG{p}{)}
\PYG{g+gp}{\PYGZgt{}\PYGZgt{}\PYGZgt{} }\PYG{n}{excess\PYGZus{}heat}\PYG{o}{.}\PYG{n}{is\PYGZus{}cooled}
\PYG{g+go}{    True}
\PYG{g+gp}{\PYGZgt{}\PYGZgt{}\PYGZgt{} }\PYG{n}{DryCooler}\PYG{p}{(}\PYG{n}{electricity}\PYG{p}{,} \PYG{n}{excess\PYGZus{}heat}\PYG{p}{,} \PYG{n}{air}\PYG{p}{,} \PYG{n}{properties}\PYG{p}{,} \PYG{n}{efficiency} \PYG{o}{=} \PYG{l+m+mi}{20}\PYG{p}{)}
\end{sphinxVerbatim}

\sphinxAtStartPar
Heat from \sphinxtitleref{excess\_heat} is converted (i.e. dissipated) to \sphinxtitleref{air} with an electricity consumption being 20 times
smaller than the converted thermal power.

\end{fulllineitems}

\subsubsection*{Methods}


\begin{savenotes}\sphinxattablestart
\centering
\begin{tabulary}{\linewidth}[t]{\X{1}{2}\X{1}{2}}
\hline

\sphinxAtStartPar
{\hyperref[\detokenize{generated/tamos.production.DryCooler:tamos.production.DryCooler.__init__}]{\sphinxcrossref{\sphinxcode{\sphinxupquote{\_\_init\_\_}}}}}(energy\_drive, energy\_source, ...{[}, ...{]})
&
\sphinxAtStartPar
DryCooler components dissipate thermal energy using fans having an electrical consumption.
\\
\hline
\sphinxAtStartPar
{\hyperref[\detokenize{generated/tamos.production.DryCooler:tamos.production.DryCooler.compute_actualized_cost}]{\sphinxcrossref{\sphinxcode{\sphinxupquote{compute\_actualized\_cost}}}}}(CAPEX, OPEX, ...{[}, ...{]})
&
\sphinxAtStartPar
Computes the cost of a component using its \textquotesingle{}Lifetime\textquotesingle{} and \textquotesingle{}Discount rate (\%)\textquotesingle{} properties.
\\
\hline
\end{tabulary}
\par
\sphinxattableend\end{savenotes}
\subsubsection*{Attributes}


\begin{savenotes}\sphinxattablestart
\centering
\begin{tabulary}{\linewidth}[t]{\X{1}{2}\X{1}{2}}
\hline

\sphinxAtStartPar
{\hyperref[\detokenize{generated/tamos.production.DryCooler:tamos.production.DryCooler.eco_count}]{\sphinxcrossref{\sphinxcode{\sphinxupquote{eco\_count}}}}}
&
\sphinxAtStartPar
Whether this instance contributes to the system "Eco" KPI.
\\
\hline
\sphinxAtStartPar
{\hyperref[\detokenize{generated/tamos.production.DryCooler:tamos.production.DryCooler.efficiency}]{\sphinxcrossref{\sphinxcode{\sphinxupquote{efficiency}}}}}
&
\sphinxAtStartPar
Number of units of dissipated heat per unit of electricity.
\\
\hline
\sphinxAtStartPar
{\hyperref[\detokenize{generated/tamos.production.DryCooler:tamos.production.DryCooler.energy_drive}]{\sphinxcrossref{\sphinxcode{\sphinxupquote{energy\_drive}}}}}
&
\sphinxAtStartPar
Electricty used to facilitate energy transfer (e.g.
\\
\hline
\sphinxAtStartPar
{\hyperref[\detokenize{generated/tamos.production.DryCooler:tamos.production.DryCooler.energy_sink}]{\sphinxcrossref{\sphinxcode{\sphinxupquote{energy\_sink}}}}}
&
\sphinxAtStartPar
Element receiving thermal energy of the dissipation of \sphinxtitleref{energy\_source}.
\\
\hline
\sphinxAtStartPar
{\hyperref[\detokenize{generated/tamos.production.DryCooler:tamos.production.DryCooler.energy_source}]{\sphinxcrossref{\sphinxcode{\sphinxupquote{energy\_source}}}}}
&
\sphinxAtStartPar
Element which must be dissipated.
\\
\hline
\sphinxAtStartPar
{\hyperref[\detokenize{generated/tamos.production.DryCooler:tamos.production.DryCooler.given_sizing}]{\sphinxcrossref{\sphinxcode{\sphinxupquote{given\_sizing}}}}}
&
\sphinxAtStartPar
The maximum capacity of the component, in kW.
\\
\hline
\sphinxAtStartPar
{\hyperref[\detokenize{generated/tamos.production.DryCooler:tamos.production.DryCooler.name}]{\sphinxcrossref{\sphinxcode{\sphinxupquote{name}}}}}
&
\sphinxAtStartPar
str.
\\
\hline
\sphinxAtStartPar
{\hyperref[\detokenize{generated/tamos.production.DryCooler:tamos.production.DryCooler.units_number_lb}]{\sphinxcrossref{\sphinxcode{\sphinxupquote{units\_number\_lb}}}}}
&
\sphinxAtStartPar
The lower bound of the number of real components that this instance aims to stand for.
\\
\hline
\sphinxAtStartPar
{\hyperref[\detokenize{generated/tamos.production.DryCooler:tamos.production.DryCooler.units_number_ub}]{\sphinxcrossref{\sphinxcode{\sphinxupquote{units\_number\_ub}}}}}
&
\sphinxAtStartPar
The upper bound of the number of real components that this instance aims to stand for.
\\
\hline
\sphinxAtStartPar
{\hyperref[\detokenize{generated/tamos.production.DryCooler:tamos.production.DryCooler.used_elements}]{\sphinxcrossref{\sphinxcode{\sphinxupquote{used\_elements}}}}}
&
\sphinxAtStartPar
Elements used by the component.
\\
\hline
\end{tabulary}
\par
\sphinxattableend\end{savenotes}
\index{compute\_actualized\_cost() (tamos.production.DryCooler method)@\spxentry{compute\_actualized\_cost()}\spxextra{tamos.production.DryCooler method}}

\begin{fulllineitems}
\phantomsection\label{\detokenize{generated/tamos.production.DryCooler:tamos.production.DryCooler.compute_actualized_cost}}
\pysigstartsignatures
\pysiglinewithargsret{\sphinxbfcode{\sphinxupquote{compute\_actualized\_cost}}}{\emph{\DUrole{n}{CAPEX}}, \emph{\DUrole{n}{OPEX}}, \emph{\DUrole{n}{system\_lifetime}}, \emph{\DUrole{n}{lifetime}\DUrole{o}{=}\DUrole{default_value}{None}}, \emph{\DUrole{n}{discount\_rate}\DUrole{o}{=}\DUrole{default_value}{None}}}{}
\pysigstopsignatures
\sphinxAtStartPar
Computes the cost of a component using its ‘Lifetime’ and ‘Discount rate (\%)’ properties.
\begin{quote}\begin{description}
\sphinxlineitem{Parameters}\begin{itemize}
\item {} 
\sphinxAtStartPar
\sphinxstyleliteralstrong{\sphinxupquote{CAPEX}} (\sphinxstyleliteralemphasis{\sphinxupquote{float}}) \textendash{} Capital Expenditure. Cost in euros paid every \sphinxtitleref{technical\_lifetime} periods.

\item {} 
\sphinxAtStartPar
\sphinxstyleliteralstrong{\sphinxupquote{OPEX}} (\sphinxstyleliteralemphasis{\sphinxupquote{float}}) \textendash{} Operational Expenditure. Cost in euros paid each period.

\item {} 
\sphinxAtStartPar
\sphinxstyleliteralstrong{\sphinxupquote{system\_lifetime}} (\sphinxstyleliteralemphasis{\sphinxupquote{int}}) \textendash{} Number of periods defining the existence of the energy system.

\item {} 
\sphinxAtStartPar
\sphinxstyleliteralstrong{\sphinxupquote{lifetime}} (\sphinxstyleliteralemphasis{\sphinxupquote{int}}\sphinxstyleliteralemphasis{\sphinxupquote{, }}\sphinxstyleliteralemphasis{\sphinxupquote{optional}}) \textendash{} Number of periods defining the existence of the component.
If specified, overwrite the “Lifetime” property.

\item {} 
\sphinxAtStartPar
\sphinxstyleliteralstrong{\sphinxupquote{discount\_rate}} (\sphinxstyleliteralemphasis{\sphinxupquote{float}}) \textendash{} In percent (\%). Describes the importance of the economic amortization process, per period.
If specified, overwrite the “Discount rate (\%)” property.

\end{itemize}

\sphinxlineitem{Returns}
\sphinxAtStartPar
\begin{itemize}
\item {} 
\sphinxAtStartPar
\sphinxstyleemphasis{A 3\sphinxhyphen{}tuple (total\_cost, CAPEX\_share, OPEX\_share) where}

\item {} 
\sphinxAtStartPar
* CAPEX\_share is the share of total cost related to \sphinxtitleref{CAPEX}

\item {} 
\sphinxAtStartPar
* OPEX\_share is the share of total cost related to \sphinxtitleref{OPEX}

\item {} 
\sphinxAtStartPar
\sphinxstyleemphasis{* total\_cost = CAPEX\_share + OPEX\_share}

\end{itemize}


\end{description}\end{quote}
\subsubsection*{Notes}

\sphinxAtStartPar
Takes into account residual value of component in the case \sphinxtitleref{system\_lifetime} is not a multiple of \sphinxtitleref{lifetime}.
In this case, the last replacement occuring at period replacement\_period is paid in proportion of ‘CAPEX’
depending linearly on the number of periods left:
CAPEX * (system\_lifetime \sphinxhyphen{} replacement\_period) / lifetime

\end{fulllineitems}

\index{eco\_count (tamos.production.DryCooler property)@\spxentry{eco\_count}\spxextra{tamos.production.DryCooler property}}

\begin{fulllineitems}
\phantomsection\label{\detokenize{generated/tamos.production.DryCooler:tamos.production.DryCooler.eco_count}}
\pysigstartsignatures
\pysigline{\sphinxbfcode{\sphinxupquote{property\DUrole{w}{  }}}\sphinxbfcode{\sphinxupquote{eco\_count}}}
\pysigstopsignatures
\sphinxAtStartPar
Whether this instance contributes to the system “Eco” KPI.
bool

\end{fulllineitems}

\index{efficiency (tamos.production.DryCooler property)@\spxentry{efficiency}\spxextra{tamos.production.DryCooler property}}

\begin{fulllineitems}
\phantomsection\label{\detokenize{generated/tamos.production.DryCooler:tamos.production.DryCooler.efficiency}}
\pysigstartsignatures
\pysigline{\sphinxbfcode{\sphinxupquote{property\DUrole{w}{  }}}\sphinxbfcode{\sphinxupquote{efficiency}}}
\pysigstopsignatures
\sphinxAtStartPar
Number of units of dissipated heat per unit of electricity.
int, float or numpy.ndarray

\end{fulllineitems}

\index{energy\_drive (tamos.production.DryCooler property)@\spxentry{energy\_drive}\spxextra{tamos.production.DryCooler property}}

\begin{fulllineitems}
\phantomsection\label{\detokenize{generated/tamos.production.DryCooler:tamos.production.DryCooler.energy_drive}}
\pysigstartsignatures
\pysigline{\sphinxbfcode{\sphinxupquote{property\DUrole{w}{  }}}\sphinxbfcode{\sphinxupquote{energy\_drive}}}
\pysigstopsignatures
\sphinxAtStartPar
Electricty used to facilitate energy transfer (e.g. using fans).

\end{fulllineitems}

\index{energy\_sink (tamos.production.DryCooler property)@\spxentry{energy\_sink}\spxextra{tamos.production.DryCooler property}}

\begin{fulllineitems}
\phantomsection\label{\detokenize{generated/tamos.production.DryCooler:tamos.production.DryCooler.energy_sink}}
\pysigstartsignatures
\pysigline{\sphinxbfcode{\sphinxupquote{property\DUrole{w}{  }}}\sphinxbfcode{\sphinxupquote{energy\_sink}}}
\pysigstopsignatures
\sphinxAtStartPar
Element receiving thermal energy of the dissipation of \sphinxtitleref{energy\_source}.
Must be warmed up if ThermalVectorPair.
ThermalVectorPair, ThermalVector

\end{fulllineitems}

\index{energy\_source (tamos.production.DryCooler property)@\spxentry{energy\_source}\spxextra{tamos.production.DryCooler property}}

\begin{fulllineitems}
\phantomsection\label{\detokenize{generated/tamos.production.DryCooler:tamos.production.DryCooler.energy_source}}
\pysigstartsignatures
\pysigline{\sphinxbfcode{\sphinxupquote{property\DUrole{w}{  }}}\sphinxbfcode{\sphinxupquote{energy\_source}}}
\pysigstopsignatures
\sphinxAtStartPar
Element which must be dissipated.
Must be cooled down if ThermalVectorPair.
ThermalVectorPair, ThermalVector

\end{fulllineitems}

\index{given\_sizing (tamos.production.DryCooler property)@\spxentry{given\_sizing}\spxextra{tamos.production.DryCooler property}}

\begin{fulllineitems}
\phantomsection\label{\detokenize{generated/tamos.production.DryCooler:tamos.production.DryCooler.given_sizing}}
\pysigstartsignatures
\pysigline{\sphinxbfcode{\sphinxupquote{property\DUrole{w}{  }}}\sphinxbfcode{\sphinxupquote{given\_sizing}}}
\pysigstopsignatures
\sphinxAtStartPar
The maximum capacity of the component, in kW.
Relates to decision variable ‘SP\_P’.
int or float

\end{fulllineitems}

\index{name (tamos.production.DryCooler property)@\spxentry{name}\spxextra{tamos.production.DryCooler property}}

\begin{fulllineitems}
\phantomsection\label{\detokenize{generated/tamos.production.DryCooler:tamos.production.DryCooler.name}}
\pysigstartsignatures
\pysigline{\sphinxbfcode{\sphinxupquote{property\DUrole{w}{  }}}\sphinxbfcode{\sphinxupquote{name}}}
\pysigstopsignatures
\sphinxAtStartPar
str.
This name is used in MILP model description.
names must not exceed 255 characters,
all of which must be alphanumeric (a\sphinxhyphen{}z, A\sphinxhyphen{}Z, 0\sphinxhyphen{}9) or one of these symbols:
! ” \# \$ \% \& , . ; ? @ \_ ‘ ’ \{ \} \textasciitilde{}.
\begin{quote}\begin{description}
\sphinxlineitem{Type}
\sphinxAtStartPar
Name of the instance

\end{description}\end{quote}

\end{fulllineitems}

\index{units\_number\_lb (tamos.production.DryCooler property)@\spxentry{units\_number\_lb}\spxextra{tamos.production.DryCooler property}}

\begin{fulllineitems}
\phantomsection\label{\detokenize{generated/tamos.production.DryCooler:tamos.production.DryCooler.units_number_lb}}
\pysigstartsignatures
\pysigline{\sphinxbfcode{\sphinxupquote{property\DUrole{w}{  }}}\sphinxbfcode{\sphinxupquote{units\_number\_lb}}}
\pysigstopsignatures
\sphinxAtStartPar
The lower bound of the number of real components that this instance aims to stand for.
Setting \sphinxtitleref{units\_number\_lb} has a meaning if “LB max output power (kW)” property is different from 0.
int

\end{fulllineitems}

\index{units\_number\_ub (tamos.production.DryCooler property)@\spxentry{units\_number\_ub}\spxextra{tamos.production.DryCooler property}}

\begin{fulllineitems}
\phantomsection\label{\detokenize{generated/tamos.production.DryCooler:tamos.production.DryCooler.units_number_ub}}
\pysigstartsignatures
\pysigline{\sphinxbfcode{\sphinxupquote{property\DUrole{w}{  }}}\sphinxbfcode{\sphinxupquote{units\_number\_ub}}}
\pysigstopsignatures
\sphinxAtStartPar
The upper bound of the number of real components that this instance aims to stand for.
Setting \sphinxtitleref{units\_number\_ub} has a meaning if “LB max output power (kW)” property is different from 0.
int

\end{fulllineitems}

\index{used\_elements (tamos.production.DryCooler property)@\spxentry{used\_elements}\spxextra{tamos.production.DryCooler property}}

\begin{fulllineitems}
\phantomsection\label{\detokenize{generated/tamos.production.DryCooler:tamos.production.DryCooler.used_elements}}
\pysigstartsignatures
\pysigline{\sphinxbfcode{\sphinxupquote{property\DUrole{w}{  }}}\sphinxbfcode{\sphinxupquote{used\_elements}}}
\pysigstopsignatures
\sphinxAtStartPar
Elements used by the component.

\end{fulllineitems}


\end{fulllineitems}


\sphinxstepscope


\section{tamos.production.ElectricHeater}
\label{\detokenize{generated/tamos.production.ElectricHeater:tamos-production-electricheater}}\label{\detokenize{generated/tamos.production.ElectricHeater::doc}}\index{ElectricHeater (class in tamos.production)@\spxentry{ElectricHeater}\spxextra{class in tamos.production}}

\begin{fulllineitems}
\phantomsection\label{\detokenize{generated/tamos.production.ElectricHeater:tamos.production.ElectricHeater}}
\pysigstartsignatures
\pysiglinewithargsret{\sphinxbfcode{\sphinxupquote{class\DUrole{w}{  }}}\sphinxcode{\sphinxupquote{tamos.production.}}\sphinxbfcode{\sphinxupquote{ElectricHeater}}}{\emph{\DUrole{n}{energy\_source}}, \emph{\DUrole{n}{energy\_sink}}, \emph{\DUrole{n}{properties}}, \emph{\DUrole{n}{given\_sizing}\DUrole{o}{=}\DUrole{default_value}{None}}, \emph{\DUrole{n}{name}\DUrole{o}{=}\DUrole{default_value}{None}}, \emph{\DUrole{n}{units\_number\_ub}\DUrole{o}{=}\DUrole{default_value}{1}}, \emph{\DUrole{n}{units\_number\_lb}\DUrole{o}{=}\DUrole{default_value}{1}}, \emph{\DUrole{n}{eco\_count}\DUrole{o}{=}\DUrole{default_value}{True}}}{}
\pysigstopsignatures\index{\_\_init\_\_() (tamos.production.ElectricHeater method)@\spxentry{\_\_init\_\_()}\spxextra{tamos.production.ElectricHeater method}}

\begin{fulllineitems}
\phantomsection\label{\detokenize{generated/tamos.production.ElectricHeater:tamos.production.ElectricHeater.__init__}}
\pysigstartsignatures
\pysiglinewithargsret{\sphinxbfcode{\sphinxupquote{\_\_init\_\_}}}{\emph{\DUrole{n}{energy\_source}}, \emph{\DUrole{n}{energy\_sink}}, \emph{\DUrole{n}{properties}}, \emph{\DUrole{n}{given\_sizing}\DUrole{o}{=}\DUrole{default_value}{None}}, \emph{\DUrole{n}{name}\DUrole{o}{=}\DUrole{default_value}{None}}, \emph{\DUrole{n}{units\_number\_ub}\DUrole{o}{=}\DUrole{default_value}{1}}, \emph{\DUrole{n}{units\_number\_lb}\DUrole{o}{=}\DUrole{default_value}{1}}, \emph{\DUrole{n}{eco\_count}\DUrole{o}{=}\DUrole{default_value}{True}}}{}
\pysigstopsignatures
\sphinxAtStartPar
ElectricHeater components produce heat following a unit efficiency.

\sphinxAtStartPar
This component declares the following exported decision variables:
\begin{itemize}
\item {} 
\sphinxAtStartPar
X\_P, binary.
Whether the component is used by the hub.

\item {} 
\sphinxAtStartPar
SP\_P, continuous, in kW.
The maximum capacity of the component. Defines the investment costs.

\item {} 
\sphinxAtStartPar
For all t, for all element e, F\_P(e, t), continuous, in kW.
The power related to element e entering the component (i.e. leaving the hub interface).

\item {} 
\sphinxAtStartPar
For all t, Q\_P(t), continuous, in kW.
The reference power related to the component. Defines the variable cost.
This power is a lower bound of SP\_P.
There exists one element e such that Q\_P(t) = F\_P(e, t) or Q\_P(t) = \sphinxhyphen{} F\_P(e, t).
For this component, e is \sphinxtitleref{energy\_sink}.

\end{itemize}

\sphinxAtStartPar
This component declares the following KPIs:
\begin{itemize}
\item {} 
\sphinxAtStartPar
\sphinxtitleref{COST\_production}
In euros.
Contributes to the “Eco” objective function.

\end{itemize}
\begin{quote}\begin{description}
\sphinxlineitem{Parameters}\begin{itemize}
\item {} 
\sphinxAtStartPar
\sphinxstyleliteralstrong{\sphinxupquote{energy\_source}} ({\hyperref[\detokenize{generated/tamos.element.ElectricityVector:tamos.element.ElectricityVector}]{\sphinxcrossref{\sphinxstyleliteralemphasis{\sphinxupquote{ElectricityVector}}}}}) \textendash{} 

\item {} 
\sphinxAtStartPar
\sphinxstyleliteralstrong{\sphinxupquote{energy\_sink}} (\sphinxstyleliteralemphasis{\sphinxupquote{ThermalVectorPair}}) \textendash{} Thermal flow that is warmed up by the boiler.

\item {} 
\sphinxAtStartPar
\sphinxstyleliteralstrong{\sphinxupquote{properties}} (\sphinxstyleliteralemphasis{\sphinxupquote{dict \{str: int}}\sphinxstyleliteralemphasis{\sphinxupquote{ | }}\sphinxstyleliteralemphasis{\sphinxupquote{float\}}}) \textendash{} 
\sphinxAtStartPar
Techno\sphinxhyphen{}economic properties of the component.
The \sphinxtitleref{properties} attribute must include the following keys:
\begin{itemize}
\item {} 
\sphinxAtStartPar
”LB max output power (kW)”

\item {} 
\sphinxAtStartPar
”UB max output power (kW)”

\item {} 
\sphinxAtStartPar
”CAPEX (EUR/kW)”

\item {} 
\sphinxAtStartPar
”OPEX (\%CAPEX)”

\item {} 
\sphinxAtStartPar
”Variable OPEX (EUR/MWh)”

\end{itemize}


\item {} 
\sphinxAtStartPar
\sphinxstyleliteralstrong{\sphinxupquote{given\_sizing}} (\sphinxstyleliteralemphasis{\sphinxupquote{int}}\sphinxstyleliteralemphasis{\sphinxupquote{ or }}\sphinxstyleliteralemphasis{\sphinxupquote{float}}\sphinxstyleliteralemphasis{\sphinxupquote{, }}\sphinxstyleliteralemphasis{\sphinxupquote{optional}}) \textendash{} The maximum capacity of the component, in kW.
Relates to decision variable ‘SP\_P’.
If specified, only the operation of this component is performed by the MILP solver.
If let unknown, both sizing and operation are performed.

\item {} 
\sphinxAtStartPar
\sphinxstyleliteralstrong{\sphinxupquote{name}} (\sphinxstyleliteralemphasis{\sphinxupquote{str}}\sphinxstyleliteralemphasis{\sphinxupquote{, }}\sphinxstyleliteralemphasis{\sphinxupquote{optional}}) \textendash{} 

\item {} 
\sphinxAtStartPar
\sphinxstyleliteralstrong{\sphinxupquote{units\_number\_lb}} (\sphinxstyleliteralemphasis{\sphinxupquote{int}}\sphinxstyleliteralemphasis{\sphinxupquote{, }}\sphinxstyleliteralemphasis{\sphinxupquote{optional}}\sphinxstyleliteralemphasis{\sphinxupquote{, }}\sphinxstyleliteralemphasis{\sphinxupquote{default 1}}) \textendash{} The lower bound (upper bound) of the number of real components that this instance aims to stand for.
Setting \sphinxtitleref{units\_number\_lb} (\sphinxtitleref{units\_number\_ub}) has a meaning if “LB max output power (kW)” property is
different from 0.

\item {} 
\sphinxAtStartPar
\sphinxstyleliteralstrong{\sphinxupquote{units\_number\_ub}} (\sphinxstyleliteralemphasis{\sphinxupquote{int}}\sphinxstyleliteralemphasis{\sphinxupquote{, }}\sphinxstyleliteralemphasis{\sphinxupquote{optional}}\sphinxstyleliteralemphasis{\sphinxupquote{, }}\sphinxstyleliteralemphasis{\sphinxupquote{default 1}}) \textendash{} The lower bound (upper bound) of the number of real components that this instance aims to stand for.
Setting \sphinxtitleref{units\_number\_lb} (\sphinxtitleref{units\_number\_ub}) has a meaning if “LB max output power (kW)” property is
different from 0.

\item {} 
\sphinxAtStartPar
\sphinxstyleliteralstrong{\sphinxupquote{eco\_count}} (\sphinxstyleliteralemphasis{\sphinxupquote{bool}}\sphinxstyleliteralemphasis{\sphinxupquote{, }}\sphinxstyleliteralemphasis{\sphinxupquote{optional}}\sphinxstyleliteralemphasis{\sphinxupquote{, }}\sphinxstyleliteralemphasis{\sphinxupquote{default True}}) \textendash{} Whether this instance contributes to the system “Eco” KPI.

\end{itemize}

\end{description}\end{quote}

\end{fulllineitems}

\subsubsection*{Methods}


\begin{savenotes}\sphinxattablestart
\centering
\begin{tabulary}{\linewidth}[t]{\X{1}{2}\X{1}{2}}
\hline

\sphinxAtStartPar
{\hyperref[\detokenize{generated/tamos.production.ElectricHeater:tamos.production.ElectricHeater.__init__}]{\sphinxcrossref{\sphinxcode{\sphinxupquote{\_\_init\_\_}}}}}(energy\_source, energy\_sink, properties)
&
\sphinxAtStartPar
ElectricHeater components produce heat following a unit efficiency.
\\
\hline
\sphinxAtStartPar
{\hyperref[\detokenize{generated/tamos.production.ElectricHeater:tamos.production.ElectricHeater.compute_actualized_cost}]{\sphinxcrossref{\sphinxcode{\sphinxupquote{compute\_actualized\_cost}}}}}(CAPEX, OPEX, ...{[}, ...{]})
&
\sphinxAtStartPar
Computes the cost of a component using its \textquotesingle{}Lifetime\textquotesingle{} and \textquotesingle{}Discount rate (\%)\textquotesingle{} properties.
\\
\hline
\sphinxAtStartPar
{\hyperref[\detokenize{generated/tamos.production.ElectricHeater:tamos.production.ElectricHeater.set_efficiency_model}]{\sphinxcrossref{\sphinxcode{\sphinxupquote{set\_efficiency\_model}}}}}(efficiency\_function, pinch)
&
\sphinxAtStartPar
Defines the efficiency of the conversion of \sphinxtitleref{energy\_source} to \sphinxtitleref{energy\_sink} using a function of the cold temperature of \sphinxtitleref{energy\_sink}.
\\
\hline
\end{tabulary}
\par
\sphinxattableend\end{savenotes}
\subsubsection*{Attributes}


\begin{savenotes}\sphinxattablestart
\centering
\begin{tabulary}{\linewidth}[t]{\X{1}{2}\X{1}{2}}
\hline

\sphinxAtStartPar
{\hyperref[\detokenize{generated/tamos.production.ElectricHeater:tamos.production.ElectricHeater.eco_count}]{\sphinxcrossref{\sphinxcode{\sphinxupquote{eco\_count}}}}}
&
\sphinxAtStartPar
Whether this instance contributes to the system "Eco" KPI.
\\
\hline
\sphinxAtStartPar
{\hyperref[\detokenize{generated/tamos.production.ElectricHeater:tamos.production.ElectricHeater.efficiency}]{\sphinxcrossref{\sphinxcode{\sphinxupquote{efficiency}}}}}
&
\sphinxAtStartPar
Defines explicitly the efficiency of the conversion of \sphinxtitleref{energy\_source} to \sphinxtitleref{energy\_sink}.
\\
\hline
\sphinxAtStartPar
{\hyperref[\detokenize{generated/tamos.production.ElectricHeater:tamos.production.ElectricHeater.energy_sink}]{\sphinxcrossref{\sphinxcode{\sphinxupquote{energy\_sink}}}}}
&
\sphinxAtStartPar
Thermal flow that is warmed up by the boiler.
\\
\hline
\sphinxAtStartPar
\sphinxcode{\sphinxupquote{energy\_source}}
&
\sphinxAtStartPar

\\
\hline
\sphinxAtStartPar
{\hyperref[\detokenize{generated/tamos.production.ElectricHeater:tamos.production.ElectricHeater.given_sizing}]{\sphinxcrossref{\sphinxcode{\sphinxupquote{given\_sizing}}}}}
&
\sphinxAtStartPar
The maximum capacity of the component, in kW.
\\
\hline
\sphinxAtStartPar
{\hyperref[\detokenize{generated/tamos.production.ElectricHeater:tamos.production.ElectricHeater.name}]{\sphinxcrossref{\sphinxcode{\sphinxupquote{name}}}}}
&
\sphinxAtStartPar
str.
\\
\hline
\sphinxAtStartPar
{\hyperref[\detokenize{generated/tamos.production.ElectricHeater:tamos.production.ElectricHeater.units_number_lb}]{\sphinxcrossref{\sphinxcode{\sphinxupquote{units\_number\_lb}}}}}
&
\sphinxAtStartPar
The lower bound of the number of real components that this instance aims to stand for.
\\
\hline
\sphinxAtStartPar
{\hyperref[\detokenize{generated/tamos.production.ElectricHeater:tamos.production.ElectricHeater.units_number_ub}]{\sphinxcrossref{\sphinxcode{\sphinxupquote{units\_number\_ub}}}}}
&
\sphinxAtStartPar
The upper bound of the number of real components that this instance aims to stand for.
\\
\hline
\sphinxAtStartPar
{\hyperref[\detokenize{generated/tamos.production.ElectricHeater:tamos.production.ElectricHeater.used_elements}]{\sphinxcrossref{\sphinxcode{\sphinxupquote{used\_elements}}}}}
&
\sphinxAtStartPar
Elements used by the component.
\\
\hline
\end{tabulary}
\par
\sphinxattableend\end{savenotes}
\index{compute\_actualized\_cost() (tamos.production.ElectricHeater method)@\spxentry{compute\_actualized\_cost()}\spxextra{tamos.production.ElectricHeater method}}

\begin{fulllineitems}
\phantomsection\label{\detokenize{generated/tamos.production.ElectricHeater:tamos.production.ElectricHeater.compute_actualized_cost}}
\pysigstartsignatures
\pysiglinewithargsret{\sphinxbfcode{\sphinxupquote{compute\_actualized\_cost}}}{\emph{\DUrole{n}{CAPEX}}, \emph{\DUrole{n}{OPEX}}, \emph{\DUrole{n}{system\_lifetime}}, \emph{\DUrole{n}{lifetime}\DUrole{o}{=}\DUrole{default_value}{None}}, \emph{\DUrole{n}{discount\_rate}\DUrole{o}{=}\DUrole{default_value}{None}}}{}
\pysigstopsignatures
\sphinxAtStartPar
Computes the cost of a component using its ‘Lifetime’ and ‘Discount rate (\%)’ properties.
\begin{quote}\begin{description}
\sphinxlineitem{Parameters}\begin{itemize}
\item {} 
\sphinxAtStartPar
\sphinxstyleliteralstrong{\sphinxupquote{CAPEX}} (\sphinxstyleliteralemphasis{\sphinxupquote{float}}) \textendash{} Capital Expenditure. Cost in euros paid every \sphinxtitleref{technical\_lifetime} periods.

\item {} 
\sphinxAtStartPar
\sphinxstyleliteralstrong{\sphinxupquote{OPEX}} (\sphinxstyleliteralemphasis{\sphinxupquote{float}}) \textendash{} Operational Expenditure. Cost in euros paid each period.

\item {} 
\sphinxAtStartPar
\sphinxstyleliteralstrong{\sphinxupquote{system\_lifetime}} (\sphinxstyleliteralemphasis{\sphinxupquote{int}}) \textendash{} Number of periods defining the existence of the energy system.

\item {} 
\sphinxAtStartPar
\sphinxstyleliteralstrong{\sphinxupquote{lifetime}} (\sphinxstyleliteralemphasis{\sphinxupquote{int}}\sphinxstyleliteralemphasis{\sphinxupquote{, }}\sphinxstyleliteralemphasis{\sphinxupquote{optional}}) \textendash{} Number of periods defining the existence of the component.
If specified, overwrite the “Lifetime” property.

\item {} 
\sphinxAtStartPar
\sphinxstyleliteralstrong{\sphinxupquote{discount\_rate}} (\sphinxstyleliteralemphasis{\sphinxupquote{float}}) \textendash{} In percent (\%). Describes the importance of the economic amortization process, per period.
If specified, overwrite the “Discount rate (\%)” property.

\end{itemize}

\sphinxlineitem{Returns}
\sphinxAtStartPar
\begin{itemize}
\item {} 
\sphinxAtStartPar
\sphinxstyleemphasis{A 3\sphinxhyphen{}tuple (total\_cost, CAPEX\_share, OPEX\_share) where}

\item {} 
\sphinxAtStartPar
* CAPEX\_share is the share of total cost related to \sphinxtitleref{CAPEX}

\item {} 
\sphinxAtStartPar
* OPEX\_share is the share of total cost related to \sphinxtitleref{OPEX}

\item {} 
\sphinxAtStartPar
\sphinxstyleemphasis{* total\_cost = CAPEX\_share + OPEX\_share}

\end{itemize}


\end{description}\end{quote}
\subsubsection*{Notes}

\sphinxAtStartPar
Takes into account residual value of component in the case \sphinxtitleref{system\_lifetime} is not a multiple of \sphinxtitleref{lifetime}.
In this case, the last replacement occuring at period replacement\_period is paid in proportion of ‘CAPEX’
depending linearly on the number of periods left:
CAPEX * (system\_lifetime \sphinxhyphen{} replacement\_period) / lifetime

\end{fulllineitems}

\index{eco\_count (tamos.production.ElectricHeater property)@\spxentry{eco\_count}\spxextra{tamos.production.ElectricHeater property}}

\begin{fulllineitems}
\phantomsection\label{\detokenize{generated/tamos.production.ElectricHeater:tamos.production.ElectricHeater.eco_count}}
\pysigstartsignatures
\pysigline{\sphinxbfcode{\sphinxupquote{property\DUrole{w}{  }}}\sphinxbfcode{\sphinxupquote{eco\_count}}}
\pysigstopsignatures
\sphinxAtStartPar
Whether this instance contributes to the system “Eco” KPI.
bool

\end{fulllineitems}

\index{efficiency (tamos.production.ElectricHeater property)@\spxentry{efficiency}\spxextra{tamos.production.ElectricHeater property}}

\begin{fulllineitems}
\phantomsection\label{\detokenize{generated/tamos.production.ElectricHeater:tamos.production.ElectricHeater.efficiency}}
\pysigstartsignatures
\pysigline{\sphinxbfcode{\sphinxupquote{property\DUrole{w}{  }}}\sphinxbfcode{\sphinxupquote{efficiency}}}
\pysigstopsignatures
\sphinxAtStartPar
Defines explicitly the efficiency of the conversion of \sphinxtitleref{energy\_source} to \sphinxtitleref{energy\_sink}.
If called, replaces the definition of the efficiency using \sphinxtitleref{set\_efficiency\_model} (default).
int, float or numpy.ndarray

\end{fulllineitems}

\index{energy\_sink (tamos.production.ElectricHeater property)@\spxentry{energy\_sink}\spxextra{tamos.production.ElectricHeater property}}

\begin{fulllineitems}
\phantomsection\label{\detokenize{generated/tamos.production.ElectricHeater:tamos.production.ElectricHeater.energy_sink}}
\pysigstartsignatures
\pysigline{\sphinxbfcode{\sphinxupquote{property\DUrole{w}{  }}}\sphinxbfcode{\sphinxupquote{energy\_sink}}}
\pysigstopsignatures
\sphinxAtStartPar
Thermal flow that is warmed up by the boiler.

\end{fulllineitems}

\index{given\_sizing (tamos.production.ElectricHeater property)@\spxentry{given\_sizing}\spxextra{tamos.production.ElectricHeater property}}

\begin{fulllineitems}
\phantomsection\label{\detokenize{generated/tamos.production.ElectricHeater:tamos.production.ElectricHeater.given_sizing}}
\pysigstartsignatures
\pysigline{\sphinxbfcode{\sphinxupquote{property\DUrole{w}{  }}}\sphinxbfcode{\sphinxupquote{given\_sizing}}}
\pysigstopsignatures
\sphinxAtStartPar
The maximum capacity of the component, in kW.
Relates to decision variable ‘SP\_P’.
int or float

\end{fulllineitems}

\index{name (tamos.production.ElectricHeater property)@\spxentry{name}\spxextra{tamos.production.ElectricHeater property}}

\begin{fulllineitems}
\phantomsection\label{\detokenize{generated/tamos.production.ElectricHeater:tamos.production.ElectricHeater.name}}
\pysigstartsignatures
\pysigline{\sphinxbfcode{\sphinxupquote{property\DUrole{w}{  }}}\sphinxbfcode{\sphinxupquote{name}}}
\pysigstopsignatures
\sphinxAtStartPar
str.
This name is used in MILP model description.
names must not exceed 255 characters,
all of which must be alphanumeric (a\sphinxhyphen{}z, A\sphinxhyphen{}Z, 0\sphinxhyphen{}9) or one of these symbols:
! ” \# \$ \% \& , . ; ? @ \_ ‘ ’ \{ \} \textasciitilde{}.
\begin{quote}\begin{description}
\sphinxlineitem{Type}
\sphinxAtStartPar
Name of the instance

\end{description}\end{quote}

\end{fulllineitems}

\index{set\_efficiency\_model() (tamos.production.ElectricHeater method)@\spxentry{set\_efficiency\_model()}\spxextra{tamos.production.ElectricHeater method}}

\begin{fulllineitems}
\phantomsection\label{\detokenize{generated/tamos.production.ElectricHeater:tamos.production.ElectricHeater.set_efficiency_model}}
\pysigstartsignatures
\pysiglinewithargsret{\sphinxbfcode{\sphinxupquote{set\_efficiency\_model}}}{\emph{\DUrole{n}{efficiency\_function}}, \emph{\DUrole{n}{pinch}}}{}
\pysigstopsignatures
\sphinxAtStartPar
Defines the efficiency of the conversion of \sphinxtitleref{energy\_source} to \sphinxtitleref{energy\_sink} using
a function of the cold temperature of \sphinxtitleref{energy\_sink}.
\begin{quote}\begin{description}
\sphinxlineitem{Parameters}\begin{itemize}
\item {} 
\sphinxAtStartPar
\sphinxstyleliteralstrong{\sphinxupquote{efficiency\_function}} (\sphinxstyleliteralemphasis{\sphinxupquote{callable f}}\sphinxstyleliteralemphasis{\sphinxupquote{(}}\sphinxstyleliteralemphasis{\sphinxupquote{T}}\sphinxstyleliteralemphasis{\sphinxupquote{)}}) \textendash{} T is the temperature of the cold vector of \sphinxtitleref{energy\_sink}, in Kelvins (K).

\item {} 
\sphinxAtStartPar
\sphinxstyleliteralstrong{\sphinxupquote{pinch}} (\sphinxstyleliteralemphasis{\sphinxupquote{int}}\sphinxstyleliteralemphasis{\sphinxupquote{, }}\sphinxstyleliteralemphasis{\sphinxupquote{float}}\sphinxstyleliteralemphasis{\sphinxupquote{ or }}\sphinxstyleliteralemphasis{\sphinxupquote{numpy.ndarray}}) \textendash{} Temperature difference between the flue gases of the boiler and the cold vector of \sphinxtitleref{energy\_sink}, in Kelvins (K).

\end{itemize}

\end{description}\end{quote}
\subsubsection*{Notes}

\sphinxAtStartPar
By default, set\_efficiency\_model is called with the \sphinxtitleref{default\_efficiency} attribute of this instance and pinch = 2.

\end{fulllineitems}

\index{units\_number\_lb (tamos.production.ElectricHeater property)@\spxentry{units\_number\_lb}\spxextra{tamos.production.ElectricHeater property}}

\begin{fulllineitems}
\phantomsection\label{\detokenize{generated/tamos.production.ElectricHeater:tamos.production.ElectricHeater.units_number_lb}}
\pysigstartsignatures
\pysigline{\sphinxbfcode{\sphinxupquote{property\DUrole{w}{  }}}\sphinxbfcode{\sphinxupquote{units\_number\_lb}}}
\pysigstopsignatures
\sphinxAtStartPar
The lower bound of the number of real components that this instance aims to stand for.
Setting \sphinxtitleref{units\_number\_lb} has a meaning if “LB max output power (kW)” property is different from 0.
int

\end{fulllineitems}

\index{units\_number\_ub (tamos.production.ElectricHeater property)@\spxentry{units\_number\_ub}\spxextra{tamos.production.ElectricHeater property}}

\begin{fulllineitems}
\phantomsection\label{\detokenize{generated/tamos.production.ElectricHeater:tamos.production.ElectricHeater.units_number_ub}}
\pysigstartsignatures
\pysigline{\sphinxbfcode{\sphinxupquote{property\DUrole{w}{  }}}\sphinxbfcode{\sphinxupquote{units\_number\_ub}}}
\pysigstopsignatures
\sphinxAtStartPar
The upper bound of the number of real components that this instance aims to stand for.
Setting \sphinxtitleref{units\_number\_ub} has a meaning if “LB max output power (kW)” property is different from 0.
int

\end{fulllineitems}

\index{used\_elements (tamos.production.ElectricHeater property)@\spxentry{used\_elements}\spxextra{tamos.production.ElectricHeater property}}

\begin{fulllineitems}
\phantomsection\label{\detokenize{generated/tamos.production.ElectricHeater:tamos.production.ElectricHeater.used_elements}}
\pysigstartsignatures
\pysigline{\sphinxbfcode{\sphinxupquote{property\DUrole{w}{  }}}\sphinxbfcode{\sphinxupquote{used\_elements}}}
\pysigstopsignatures
\sphinxAtStartPar
Elements used by the component.

\end{fulllineitems}


\end{fulllineitems}


\sphinxstepscope


\section{tamos.production.ElementConverter}
\label{\detokenize{generated/tamos.production.ElementConverter:tamos-production-elementconverter}}\label{\detokenize{generated/tamos.production.ElementConverter::doc}}\index{ElementConverter (class in tamos.production)@\spxentry{ElementConverter}\spxextra{class in tamos.production}}

\begin{fulllineitems}
\phantomsection\label{\detokenize{generated/tamos.production.ElementConverter:tamos.production.ElementConverter}}
\pysigstartsignatures
\pysiglinewithargsret{\sphinxbfcode{\sphinxupquote{class\DUrole{w}{  }}}\sphinxcode{\sphinxupquote{tamos.production.}}\sphinxbfcode{\sphinxupquote{ElementConverter}}}{\emph{\DUrole{n}{element\_1}}, \emph{\DUrole{n}{element\_2}}, \emph{\DUrole{n}{direction}}, \emph{\DUrole{n}{name}\DUrole{o}{=}\DUrole{default_value}{None}}}{}
\pysigstopsignatures\index{\_\_init\_\_() (tamos.production.ElementConverter method)@\spxentry{\_\_init\_\_()}\spxextra{tamos.production.ElementConverter method}}

\begin{fulllineitems}
\phantomsection\label{\detokenize{generated/tamos.production.ElementConverter:tamos.production.ElementConverter.__init__}}
\pysigstartsignatures
\pysiglinewithargsret{\sphinxbfcode{\sphinxupquote{\_\_init\_\_}}}{\emph{\DUrole{n}{element\_1}}, \emph{\DUrole{n}{element\_2}}, \emph{\DUrole{n}{direction}}, \emph{\DUrole{n}{name}\DUrole{o}{=}\DUrole{default_value}{None}}}{}
\pysigstopsignatures
\sphinxAtStartPar
ElementConverter components convert an element into another.

\sphinxAtStartPar
This component declares the following exported decision variables:
\begin{itemize}
\item {} 
\sphinxAtStartPar
X\_P, binary.
Whether the component is used by the hub.

\item {} 
\sphinxAtStartPar
SP\_P, continuous, in kW.
The maximum capacity of the component.

\item {} 
\sphinxAtStartPar
For all t, for all element e, F\_P(e, t), continuous, in kW.
The power related to element e entering the component (i.e. leaving the hub interface).

\item {} 
\sphinxAtStartPar
For all t, Q\_P(t), continuous, in kW.
The reference power related to the component.
This power is a lower bound of SP\_P.
There exists one element e such that Q\_P(t) = F\_P(e, t) or Q\_P(t) = \sphinxhyphen{} F\_P(e, t).
For this component, e is either \sphinxtitleref{element\_1} or \sphinxtitleref{element\_2} depending on the \sphinxtitleref{direction} attribute.

\end{itemize}

\sphinxAtStartPar
This component does not declare any KPI.
\begin{quote}\begin{description}
\sphinxlineitem{Parameters}\begin{itemize}
\item {} 
\sphinxAtStartPar
\sphinxstyleliteralstrong{\sphinxupquote{element\_1}} ({\hyperref[\detokenize{generated/tamos.element.ElectricityVector:tamos.element.ElectricityVector}]{\sphinxcrossref{\sphinxstyleliteralemphasis{\sphinxupquote{ElectricityVector}}}}}\sphinxstyleliteralemphasis{\sphinxupquote{, }}{\hyperref[\detokenize{generated/tamos.element.FuelVector:tamos.element.FuelVector}]{\sphinxcrossref{\sphinxstyleliteralemphasis{\sphinxupquote{FuelVector}}}}}\sphinxstyleliteralemphasis{\sphinxupquote{, }}\sphinxstyleliteralemphasis{\sphinxupquote{ThermalVectorPair}}\sphinxstyleliteralemphasis{\sphinxupquote{, }}{\hyperref[\detokenize{generated/tamos.element.ThermalVector:tamos.element.ThermalVector}]{\sphinxcrossref{\sphinxstyleliteralemphasis{\sphinxupquote{ThermalVector}}}}}) \textendash{} 

\item {} 
\sphinxAtStartPar
\sphinxstyleliteralstrong{\sphinxupquote{element\_2}} ({\hyperref[\detokenize{generated/tamos.element.ElectricityVector:tamos.element.ElectricityVector}]{\sphinxcrossref{\sphinxstyleliteralemphasis{\sphinxupquote{ElectricityVector}}}}}\sphinxstyleliteralemphasis{\sphinxupquote{, }}{\hyperref[\detokenize{generated/tamos.element.FuelVector:tamos.element.FuelVector}]{\sphinxcrossref{\sphinxstyleliteralemphasis{\sphinxupquote{FuelVector}}}}}\sphinxstyleliteralemphasis{\sphinxupquote{, }}\sphinxstyleliteralemphasis{\sphinxupquote{ThermalVectorPair}}\sphinxstyleliteralemphasis{\sphinxupquote{, }}{\hyperref[\detokenize{generated/tamos.element.ThermalVector:tamos.element.ThermalVector}]{\sphinxcrossref{\sphinxstyleliteralemphasis{\sphinxupquote{ThermalVector}}}}}) \textendash{} 

\item {} 
\sphinxAtStartPar
\sphinxstyleliteralstrong{\sphinxupquote{direction}} (\sphinxstyleliteralemphasis{\sphinxupquote{\{\textquotesingle{}produced\textquotesingle{}}}\sphinxstyleliteralemphasis{\sphinxupquote{, }}\sphinxstyleliteralemphasis{\sphinxupquote{\textquotesingle{}consumed\textquotesingle{}}}\sphinxstyleliteralemphasis{\sphinxupquote{, }}\sphinxstyleliteralemphasis{\sphinxupquote{\textquotesingle{}both\textquotesingle{}\}}}) \textendash{} 
\sphinxAtStartPar
Related to \sphinxtitleref{element\_1}.
\begin{itemize}
\item {} 
\sphinxAtStartPar
’produced’: a flow of \sphinxtitleref{element\_1} is produced, a flow of \sphinxtitleref{element\_2} is consumed

\item {} 
\sphinxAtStartPar
’consumed’: a flow of \sphinxtitleref{element\_2} is produced, a flow of \sphinxtitleref{element\_1} is consumed

\item {} \begin{description}
\sphinxlineitem{’both’: depending on the time step t, a flow of \sphinxtitleref{element\_1} is produced or consumed}
\sphinxAtStartPar
In this mode, decision variable ‘Q\_P’ has no upper bound.

\end{description}

\end{itemize}


\item {} 
\sphinxAtStartPar
\sphinxstyleliteralstrong{\sphinxupquote{name}} (\sphinxstyleliteralemphasis{\sphinxupquote{str}}\sphinxstyleliteralemphasis{\sphinxupquote{, }}\sphinxstyleliteralemphasis{\sphinxupquote{optional}}) \textendash{} 

\end{itemize}

\end{description}\end{quote}
\subsubsection*{Notes}

\sphinxAtStartPar
No \sphinxtitleref{properties} argument must be specified for this component:
\begin{itemize}
\item {} 
\sphinxAtStartPar
“LB max output power (kW)” is set to 0 kW

\item {} 
\sphinxAtStartPar
“UB max output power (kW)” is set to 1 GW

\item {} 
\sphinxAtStartPar
no costs are associated with the component

\end{itemize}

\end{fulllineitems}

\subsubsection*{Methods}


\begin{savenotes}\sphinxattablestart
\centering
\begin{tabulary}{\linewidth}[t]{\X{1}{2}\X{1}{2}}
\hline

\sphinxAtStartPar
{\hyperref[\detokenize{generated/tamos.production.ElementConverter:tamos.production.ElementConverter.__init__}]{\sphinxcrossref{\sphinxcode{\sphinxupquote{\_\_init\_\_}}}}}(element\_1, element\_2, direction{[}, name{]})
&
\sphinxAtStartPar
ElementConverter components convert an element into another.
\\
\hline
\sphinxAtStartPar
{\hyperref[\detokenize{generated/tamos.production.ElementConverter:tamos.production.ElementConverter.compute_actualized_cost}]{\sphinxcrossref{\sphinxcode{\sphinxupquote{compute\_actualized\_cost}}}}}(CAPEX, OPEX, ...{[}, ...{]})
&
\sphinxAtStartPar
Computes the cost of a component using its \textquotesingle{}Lifetime\textquotesingle{} and \textquotesingle{}Discount rate (\%)\textquotesingle{} properties.
\\
\hline
\end{tabulary}
\par
\sphinxattableend\end{savenotes}
\subsubsection*{Attributes}


\begin{savenotes}\sphinxattablestart
\centering
\begin{tabulary}{\linewidth}[t]{\X{1}{2}\X{1}{2}}
\hline

\sphinxAtStartPar
{\hyperref[\detokenize{generated/tamos.production.ElementConverter:tamos.production.ElementConverter.direction}]{\sphinxcrossref{\sphinxcode{\sphinxupquote{direction}}}}}
&
\sphinxAtStartPar
Related to \sphinxtitleref{element\_1}.
\\
\hline
\sphinxAtStartPar
{\hyperref[\detokenize{generated/tamos.production.ElementConverter:tamos.production.ElementConverter.eco_count}]{\sphinxcrossref{\sphinxcode{\sphinxupquote{eco\_count}}}}}
&
\sphinxAtStartPar
Whether this instance contributes to the system "Eco" KPI.
\\
\hline
\sphinxAtStartPar
{\hyperref[\detokenize{generated/tamos.production.ElementConverter:tamos.production.ElementConverter.element_1}]{\sphinxcrossref{\sphinxcode{\sphinxupquote{element\_1}}}}}
&
\sphinxAtStartPar
ElectricityVector, FuelVector, ThermalVectorPair, ThermalVector
\\
\hline
\sphinxAtStartPar
{\hyperref[\detokenize{generated/tamos.production.ElementConverter:tamos.production.ElementConverter.element_2}]{\sphinxcrossref{\sphinxcode{\sphinxupquote{element\_2}}}}}
&
\sphinxAtStartPar
ElectricityVector, FuelVector, ThermalVectorPair, ThermalVector
\\
\hline
\sphinxAtStartPar
{\hyperref[\detokenize{generated/tamos.production.ElementConverter:tamos.production.ElementConverter.given_sizing}]{\sphinxcrossref{\sphinxcode{\sphinxupquote{given\_sizing}}}}}
&
\sphinxAtStartPar
The maximum capacity of the component, in kW.
\\
\hline
\sphinxAtStartPar
{\hyperref[\detokenize{generated/tamos.production.ElementConverter:tamos.production.ElementConverter.name}]{\sphinxcrossref{\sphinxcode{\sphinxupquote{name}}}}}
&
\sphinxAtStartPar
str.
\\
\hline
\sphinxAtStartPar
{\hyperref[\detokenize{generated/tamos.production.ElementConverter:tamos.production.ElementConverter.units_number_lb}]{\sphinxcrossref{\sphinxcode{\sphinxupquote{units\_number\_lb}}}}}
&
\sphinxAtStartPar
The lower bound of the number of real components that this instance aims to stand for.
\\
\hline
\sphinxAtStartPar
{\hyperref[\detokenize{generated/tamos.production.ElementConverter:tamos.production.ElementConverter.units_number_ub}]{\sphinxcrossref{\sphinxcode{\sphinxupquote{units\_number\_ub}}}}}
&
\sphinxAtStartPar
The upper bound of the number of real components that this instance aims to stand for.
\\
\hline
\sphinxAtStartPar
{\hyperref[\detokenize{generated/tamos.production.ElementConverter:tamos.production.ElementConverter.used_elements}]{\sphinxcrossref{\sphinxcode{\sphinxupquote{used\_elements}}}}}
&
\sphinxAtStartPar
Elements used by the component.
\\
\hline
\end{tabulary}
\par
\sphinxattableend\end{savenotes}
\index{compute\_actualized\_cost() (tamos.production.ElementConverter method)@\spxentry{compute\_actualized\_cost()}\spxextra{tamos.production.ElementConverter method}}

\begin{fulllineitems}
\phantomsection\label{\detokenize{generated/tamos.production.ElementConverter:tamos.production.ElementConverter.compute_actualized_cost}}
\pysigstartsignatures
\pysiglinewithargsret{\sphinxbfcode{\sphinxupquote{compute\_actualized\_cost}}}{\emph{\DUrole{n}{CAPEX}}, \emph{\DUrole{n}{OPEX}}, \emph{\DUrole{n}{system\_lifetime}}, \emph{\DUrole{n}{lifetime}\DUrole{o}{=}\DUrole{default_value}{None}}, \emph{\DUrole{n}{discount\_rate}\DUrole{o}{=}\DUrole{default_value}{None}}}{}
\pysigstopsignatures
\sphinxAtStartPar
Computes the cost of a component using its ‘Lifetime’ and ‘Discount rate (\%)’ properties.
\begin{quote}\begin{description}
\sphinxlineitem{Parameters}\begin{itemize}
\item {} 
\sphinxAtStartPar
\sphinxstyleliteralstrong{\sphinxupquote{CAPEX}} (\sphinxstyleliteralemphasis{\sphinxupquote{float}}) \textendash{} Capital Expenditure. Cost in euros paid every \sphinxtitleref{technical\_lifetime} periods.

\item {} 
\sphinxAtStartPar
\sphinxstyleliteralstrong{\sphinxupquote{OPEX}} (\sphinxstyleliteralemphasis{\sphinxupquote{float}}) \textendash{} Operational Expenditure. Cost in euros paid each period.

\item {} 
\sphinxAtStartPar
\sphinxstyleliteralstrong{\sphinxupquote{system\_lifetime}} (\sphinxstyleliteralemphasis{\sphinxupquote{int}}) \textendash{} Number of periods defining the existence of the energy system.

\item {} 
\sphinxAtStartPar
\sphinxstyleliteralstrong{\sphinxupquote{lifetime}} (\sphinxstyleliteralemphasis{\sphinxupquote{int}}\sphinxstyleliteralemphasis{\sphinxupquote{, }}\sphinxstyleliteralemphasis{\sphinxupquote{optional}}) \textendash{} Number of periods defining the existence of the component.
If specified, overwrite the “Lifetime” property.

\item {} 
\sphinxAtStartPar
\sphinxstyleliteralstrong{\sphinxupquote{discount\_rate}} (\sphinxstyleliteralemphasis{\sphinxupquote{float}}) \textendash{} In percent (\%). Describes the importance of the economic amortization process, per period.
If specified, overwrite the “Discount rate (\%)” property.

\end{itemize}

\sphinxlineitem{Returns}
\sphinxAtStartPar
\begin{itemize}
\item {} 
\sphinxAtStartPar
\sphinxstyleemphasis{A 3\sphinxhyphen{}tuple (total\_cost, CAPEX\_share, OPEX\_share) where}

\item {} 
\sphinxAtStartPar
* CAPEX\_share is the share of total cost related to \sphinxtitleref{CAPEX}

\item {} 
\sphinxAtStartPar
* OPEX\_share is the share of total cost related to \sphinxtitleref{OPEX}

\item {} 
\sphinxAtStartPar
\sphinxstyleemphasis{* total\_cost = CAPEX\_share + OPEX\_share}

\end{itemize}


\end{description}\end{quote}
\subsubsection*{Notes}

\sphinxAtStartPar
Takes into account residual value of component in the case \sphinxtitleref{system\_lifetime} is not a multiple of \sphinxtitleref{lifetime}.
In this case, the last replacement occuring at period replacement\_period is paid in proportion of ‘CAPEX’
depending linearly on the number of periods left:
CAPEX * (system\_lifetime \sphinxhyphen{} replacement\_period) / lifetime

\end{fulllineitems}

\index{direction (tamos.production.ElementConverter property)@\spxentry{direction}\spxextra{tamos.production.ElementConverter property}}

\begin{fulllineitems}
\phantomsection\label{\detokenize{generated/tamos.production.ElementConverter:tamos.production.ElementConverter.direction}}
\pysigstartsignatures
\pysigline{\sphinxbfcode{\sphinxupquote{property\DUrole{w}{  }}}\sphinxbfcode{\sphinxupquote{direction}}}
\pysigstopsignatures
\sphinxAtStartPar
Related to \sphinxtitleref{element\_1}.
\begin{itemize}
\item {} 
\sphinxAtStartPar
‘produced’: a flow of \sphinxtitleref{element\_1} is produced, a flow of \sphinxtitleref{element\_2} is consumed

\item {} 
\sphinxAtStartPar
‘consumed’: a flow of \sphinxtitleref{element\_2} is produced, a flow of \sphinxtitleref{element\_1} is consumed

\item {} \begin{description}
\sphinxlineitem{‘both’: depending on the time step t, a flow of \sphinxtitleref{element\_1} is produced or consumed}
\sphinxAtStartPar
In this mode, decision variable ‘Q\_P’ has no upper bound.

\end{description}

\end{itemize}

\sphinxAtStartPar
\{‘produced’, ‘consumed’, ‘both’\}

\end{fulllineitems}

\index{eco\_count (tamos.production.ElementConverter property)@\spxentry{eco\_count}\spxextra{tamos.production.ElementConverter property}}

\begin{fulllineitems}
\phantomsection\label{\detokenize{generated/tamos.production.ElementConverter:tamos.production.ElementConverter.eco_count}}
\pysigstartsignatures
\pysigline{\sphinxbfcode{\sphinxupquote{property\DUrole{w}{  }}}\sphinxbfcode{\sphinxupquote{eco\_count}}}
\pysigstopsignatures
\sphinxAtStartPar
Whether this instance contributes to the system “Eco” KPI.
bool

\end{fulllineitems}

\index{element\_1 (tamos.production.ElementConverter property)@\spxentry{element\_1}\spxextra{tamos.production.ElementConverter property}}

\begin{fulllineitems}
\phantomsection\label{\detokenize{generated/tamos.production.ElementConverter:tamos.production.ElementConverter.element_1}}
\pysigstartsignatures
\pysigline{\sphinxbfcode{\sphinxupquote{property\DUrole{w}{  }}}\sphinxbfcode{\sphinxupquote{element\_1}}}
\pysigstopsignatures
\sphinxAtStartPar
ElectricityVector, FuelVector, ThermalVectorPair, ThermalVector

\end{fulllineitems}

\index{element\_2 (tamos.production.ElementConverter property)@\spxentry{element\_2}\spxextra{tamos.production.ElementConverter property}}

\begin{fulllineitems}
\phantomsection\label{\detokenize{generated/tamos.production.ElementConverter:tamos.production.ElementConverter.element_2}}
\pysigstartsignatures
\pysigline{\sphinxbfcode{\sphinxupquote{property\DUrole{w}{  }}}\sphinxbfcode{\sphinxupquote{element\_2}}}
\pysigstopsignatures
\sphinxAtStartPar
ElectricityVector, FuelVector, ThermalVectorPair, ThermalVector

\end{fulllineitems}

\index{given\_sizing (tamos.production.ElementConverter property)@\spxentry{given\_sizing}\spxextra{tamos.production.ElementConverter property}}

\begin{fulllineitems}
\phantomsection\label{\detokenize{generated/tamos.production.ElementConverter:tamos.production.ElementConverter.given_sizing}}
\pysigstartsignatures
\pysigline{\sphinxbfcode{\sphinxupquote{property\DUrole{w}{  }}}\sphinxbfcode{\sphinxupquote{given\_sizing}}}
\pysigstopsignatures
\sphinxAtStartPar
The maximum capacity of the component, in kW.
Relates to decision variable ‘SP\_P’.
int or float

\end{fulllineitems}

\index{name (tamos.production.ElementConverter property)@\spxentry{name}\spxextra{tamos.production.ElementConverter property}}

\begin{fulllineitems}
\phantomsection\label{\detokenize{generated/tamos.production.ElementConverter:tamos.production.ElementConverter.name}}
\pysigstartsignatures
\pysigline{\sphinxbfcode{\sphinxupquote{property\DUrole{w}{  }}}\sphinxbfcode{\sphinxupquote{name}}}
\pysigstopsignatures
\sphinxAtStartPar
str.
This name is used in MILP model description.
names must not exceed 255 characters,
all of which must be alphanumeric (a\sphinxhyphen{}z, A\sphinxhyphen{}Z, 0\sphinxhyphen{}9) or one of these symbols:
! ” \# \$ \% \& , . ; ? @ \_ ‘ ’ \{ \} \textasciitilde{}.
\begin{quote}\begin{description}
\sphinxlineitem{Type}
\sphinxAtStartPar
Name of the instance

\end{description}\end{quote}

\end{fulllineitems}

\index{units\_number\_lb (tamos.production.ElementConverter property)@\spxentry{units\_number\_lb}\spxextra{tamos.production.ElementConverter property}}

\begin{fulllineitems}
\phantomsection\label{\detokenize{generated/tamos.production.ElementConverter:tamos.production.ElementConverter.units_number_lb}}
\pysigstartsignatures
\pysigline{\sphinxbfcode{\sphinxupquote{property\DUrole{w}{  }}}\sphinxbfcode{\sphinxupquote{units\_number\_lb}}}
\pysigstopsignatures
\sphinxAtStartPar
The lower bound of the number of real components that this instance aims to stand for.
Setting \sphinxtitleref{units\_number\_lb} has a meaning if “LB max output power (kW)” property is different from 0.
int

\end{fulllineitems}

\index{units\_number\_ub (tamos.production.ElementConverter property)@\spxentry{units\_number\_ub}\spxextra{tamos.production.ElementConverter property}}

\begin{fulllineitems}
\phantomsection\label{\detokenize{generated/tamos.production.ElementConverter:tamos.production.ElementConverter.units_number_ub}}
\pysigstartsignatures
\pysigline{\sphinxbfcode{\sphinxupquote{property\DUrole{w}{  }}}\sphinxbfcode{\sphinxupquote{units\_number\_ub}}}
\pysigstopsignatures
\sphinxAtStartPar
The upper bound of the number of real components that this instance aims to stand for.
Setting \sphinxtitleref{units\_number\_ub} has a meaning if “LB max output power (kW)” property is different from 0.
int

\end{fulllineitems}

\index{used\_elements (tamos.production.ElementConverter property)@\spxentry{used\_elements}\spxextra{tamos.production.ElementConverter property}}

\begin{fulllineitems}
\phantomsection\label{\detokenize{generated/tamos.production.ElementConverter:tamos.production.ElementConverter.used_elements}}
\pysigstartsignatures
\pysigline{\sphinxbfcode{\sphinxupquote{property\DUrole{w}{  }}}\sphinxbfcode{\sphinxupquote{used\_elements}}}
\pysigstopsignatures
\sphinxAtStartPar
Elements used by the component.

\end{fulllineitems}


\end{fulllineitems}


\sphinxstepscope


\section{tamos.production.FPSolar}
\label{\detokenize{generated/tamos.production.FPSolar:tamos-production-fpsolar}}\label{\detokenize{generated/tamos.production.FPSolar::doc}}\index{FPSolar (class in tamos.production)@\spxentry{FPSolar}\spxextra{class in tamos.production}}

\begin{fulllineitems}
\phantomsection\label{\detokenize{generated/tamos.production.FPSolar:tamos.production.FPSolar}}
\pysigstartsignatures
\pysiglinewithargsret{\sphinxbfcode{\sphinxupquote{class\DUrole{w}{  }}}\sphinxcode{\sphinxupquote{tamos.production.}}\sphinxbfcode{\sphinxupquote{FPSolar}}}{\emph{\DUrole{n}{energy\_sink}}, \emph{\DUrole{n}{air\_temperature}}, \emph{\DUrole{n}{total\_irradiance}}, \emph{\DUrole{n}{properties}}, \emph{\DUrole{n}{pinch}\DUrole{o}{=}\DUrole{default_value}{3}}, \emph{\DUrole{n}{given\_sizing}\DUrole{o}{=}\DUrole{default_value}{None}}, \emph{\DUrole{n}{name}\DUrole{o}{=}\DUrole{default_value}{None}}, \emph{\DUrole{n}{units\_number\_ub}\DUrole{o}{=}\DUrole{default_value}{1}}, \emph{\DUrole{n}{units\_number\_lb}\DUrole{o}{=}\DUrole{default_value}{1}}, \emph{\DUrole{n}{eco\_count}\DUrole{o}{=}\DUrole{default_value}{True}}}{}
\pysigstopsignatures\index{\_\_init\_\_() (tamos.production.FPSolar method)@\spxentry{\_\_init\_\_()}\spxextra{tamos.production.FPSolar method}}

\begin{fulllineitems}
\phantomsection\label{\detokenize{generated/tamos.production.FPSolar:tamos.production.FPSolar.__init__}}
\pysigstartsignatures
\pysiglinewithargsret{\sphinxbfcode{\sphinxupquote{\_\_init\_\_}}}{\emph{\DUrole{n}{energy\_sink}}, \emph{\DUrole{n}{air\_temperature}}, \emph{\DUrole{n}{total\_irradiance}}, \emph{\DUrole{n}{properties}}, \emph{\DUrole{n}{pinch}\DUrole{o}{=}\DUrole{default_value}{3}}, \emph{\DUrole{n}{given\_sizing}\DUrole{o}{=}\DUrole{default_value}{None}}, \emph{\DUrole{n}{name}\DUrole{o}{=}\DUrole{default_value}{None}}, \emph{\DUrole{n}{units\_number\_ub}\DUrole{o}{=}\DUrole{default_value}{1}}, \emph{\DUrole{n}{units\_number\_lb}\DUrole{o}{=}\DUrole{default_value}{1}}, \emph{\DUrole{n}{eco\_count}\DUrole{o}{=}\DUrole{default_value}{True}}}{}
\pysigstopsignatures
\sphinxAtStartPar
FPSolar components convert solar irradiance to heat.

\sphinxAtStartPar
This component declares the following exported decision variables:
\begin{itemize}
\item {} 
\sphinxAtStartPar
X\_P, binary.
Whether the component is used by the hub.

\item {} 
\sphinxAtStartPar
SP\_P, continuous, in kW.
The maximum capacity of the component. Defines the investment costs.

\item {} 
\sphinxAtStartPar
For all t, for all element e, F\_P(e, t), continuous, in kW.
The power related to element e entering the component (i.e. leaving the hub interface).

\item {} 
\sphinxAtStartPar
For all t, Q\_P(t), continuous, in kW.
The reference power related to the component. Defines the variable cost.
This power is a lower bound of SP\_P.
There exists one element e such that Q\_P(t) = F\_P(e, t) or Q\_P(t) = \sphinxhyphen{} F\_P(e, t).
For this component, e is \sphinxtitleref{energy\_sink}.

\end{itemize}

\sphinxAtStartPar
This component declares the following KPIs:
\begin{itemize}
\item {} 
\sphinxAtStartPar
\sphinxtitleref{COST\_production}
In euros.
Contributes to the “Eco” objective function.

\end{itemize}
\begin{quote}\begin{description}
\sphinxlineitem{Parameters}\begin{itemize}
\item {} 
\sphinxAtStartPar
\sphinxstyleliteralstrong{\sphinxupquote{energy\_sink}} (\sphinxstyleliteralemphasis{\sphinxupquote{ThermalVectorPair}}) \textendash{} Thermal flow that is warmed up.

\item {} 
\sphinxAtStartPar
\sphinxstyleliteralstrong{\sphinxupquote{air\_temperature}} (\sphinxstyleliteralemphasis{\sphinxupquote{int}}\sphinxstyleliteralemphasis{\sphinxupquote{, }}\sphinxstyleliteralemphasis{\sphinxupquote{float}}\sphinxstyleliteralemphasis{\sphinxupquote{ or }}\sphinxstyleliteralemphasis{\sphinxupquote{numpy.ndarray}}) \textendash{} In Kelvins (K).
Temperature of the air surrounding the solar panels.

\item {} 
\sphinxAtStartPar
\sphinxstyleliteralstrong{\sphinxupquote{total\_irradiance}} (\sphinxstyleliteralemphasis{\sphinxupquote{int}}\sphinxstyleliteralemphasis{\sphinxupquote{, }}\sphinxstyleliteralemphasis{\sphinxupquote{float}}\sphinxstyleliteralemphasis{\sphinxupquote{ or }}\sphinxstyleliteralemphasis{\sphinxupquote{numpy.ndarray}}) \textendash{} In kW/m\(\sp{\text{2}}\).
Solar irradiance received on the normal of the solar panels.

\item {} 
\sphinxAtStartPar
\sphinxstyleliteralstrong{\sphinxupquote{properties}} (\sphinxstyleliteralemphasis{\sphinxupquote{dict \{str: int}}\sphinxstyleliteralemphasis{\sphinxupquote{ | }}\sphinxstyleliteralemphasis{\sphinxupquote{float\}}}) \textendash{} 
\sphinxAtStartPar
Techno\sphinxhyphen{}economic properties of the component.
The \sphinxtitleref{properties} attribute must include the following keys:
\begin{itemize}
\item {} 
\sphinxAtStartPar
”LB max output power (kW)”

\item {} 
\sphinxAtStartPar
”UB max output power (kW)”

\item {} 
\sphinxAtStartPar
”CAPEX (EUR/m2)”

\item {} 
\sphinxAtStartPar
”OPEX (\%CAPEX)”

\item {} 
\sphinxAtStartPar
”Variable OPEX (EUR/MWh)”

\item {} 
\sphinxAtStartPar
”eta0”

\item {} 
\sphinxAtStartPar
”a1 (W/(m2.K))”

\item {} 
\sphinxAtStartPar
”a2 (W/(m2.K2))”

\item {} 
\sphinxAtStartPar
”a5 (J/(m2.K))”

\item {} 
\sphinxAtStartPar
”UB area (m2)”

\end{itemize}


\item {} 
\sphinxAtStartPar
\sphinxstyleliteralstrong{\sphinxupquote{pinch}} (\sphinxstyleliteralemphasis{\sphinxupquote{int}}\sphinxstyleliteralemphasis{\sphinxupquote{, }}\sphinxstyleliteralemphasis{\sphinxupquote{float}}\sphinxstyleliteralemphasis{\sphinxupquote{ or }}\sphinxstyleliteralemphasis{\sphinxupquote{numpy.ndarray}}\sphinxstyleliteralemphasis{\sphinxupquote{, }}\sphinxstyleliteralemphasis{\sphinxupquote{optional}}\sphinxstyleliteralemphasis{\sphinxupquote{, }}\sphinxstyleliteralemphasis{\sphinxupquote{default 3}}) \textendash{} Difference between the temperature of the fluid circulating in the solar panels and the one of \sphinxtitleref{energy\_sink}.

\item {} 
\sphinxAtStartPar
\sphinxstyleliteralstrong{\sphinxupquote{given\_sizing}} (\sphinxstyleliteralemphasis{\sphinxupquote{int}}\sphinxstyleliteralemphasis{\sphinxupquote{ or }}\sphinxstyleliteralemphasis{\sphinxupquote{float}}\sphinxstyleliteralemphasis{\sphinxupquote{, }}\sphinxstyleliteralemphasis{\sphinxupquote{optional}}) \textendash{} The maximum capacity of the component, in kW.
Relates to decision variable ‘SP\_P’.
If specified, only the operation of this component is performed by the MILP solver.
If let unknown, both sizing and operation are performed.

\item {} 
\sphinxAtStartPar
\sphinxstyleliteralstrong{\sphinxupquote{name}} (\sphinxstyleliteralemphasis{\sphinxupquote{str}}\sphinxstyleliteralemphasis{\sphinxupquote{, }}\sphinxstyleliteralemphasis{\sphinxupquote{optional}}) \textendash{} 

\item {} 
\sphinxAtStartPar
\sphinxstyleliteralstrong{\sphinxupquote{units\_number\_lb}} (\sphinxstyleliteralemphasis{\sphinxupquote{int}}\sphinxstyleliteralemphasis{\sphinxupquote{, }}\sphinxstyleliteralemphasis{\sphinxupquote{optional}}\sphinxstyleliteralemphasis{\sphinxupquote{, }}\sphinxstyleliteralemphasis{\sphinxupquote{default 1}}) \textendash{} The lower bound (upper bound) of the number of real components that this instance aims to stand for.
Setting \sphinxtitleref{units\_number\_lb} (\sphinxtitleref{units\_number\_ub}) has a meaning if “LB max output power (kW)” property is
different from 0.

\item {} 
\sphinxAtStartPar
\sphinxstyleliteralstrong{\sphinxupquote{units\_number\_ub}} (\sphinxstyleliteralemphasis{\sphinxupquote{int}}\sphinxstyleliteralemphasis{\sphinxupquote{, }}\sphinxstyleliteralemphasis{\sphinxupquote{optional}}\sphinxstyleliteralemphasis{\sphinxupquote{, }}\sphinxstyleliteralemphasis{\sphinxupquote{default 1}}) \textendash{} The lower bound (upper bound) of the number of real components that this instance aims to stand for.
Setting \sphinxtitleref{units\_number\_lb} (\sphinxtitleref{units\_number\_ub}) has a meaning if “LB max output power (kW)” property is
different from 0.

\item {} 
\sphinxAtStartPar
\sphinxstyleliteralstrong{\sphinxupquote{eco\_count}} (\sphinxstyleliteralemphasis{\sphinxupquote{bool}}\sphinxstyleliteralemphasis{\sphinxupquote{, }}\sphinxstyleliteralemphasis{\sphinxupquote{optional}}\sphinxstyleliteralemphasis{\sphinxupquote{, }}\sphinxstyleliteralemphasis{\sphinxupquote{default True}}) \textendash{} Whether this instance contributes to the system “Eco” KPI.

\end{itemize}

\end{description}\end{quote}
\subsubsection*{Notes}
\begin{enumerate}
\sphinxsetlistlabels{\arabic}{enumi}{enumii}{}{.}%
\item {} 
\sphinxAtStartPar
Parameters “eta0”, “a1 (W/(m2.K))”, “a2 (W/(m2.K2))” and “a5 (J/(m2.K))” can be found in databases like the
Solar Keymark database.

\item {} 
\sphinxAtStartPar
‘FPSolar’ stands for ‘Flat plate solar thermal’.

\end{enumerate}

\end{fulllineitems}

\subsubsection*{Methods}


\begin{savenotes}\sphinxattablestart
\centering
\begin{tabulary}{\linewidth}[t]{\X{1}{2}\X{1}{2}}
\hline

\sphinxAtStartPar
{\hyperref[\detokenize{generated/tamos.production.FPSolar:tamos.production.FPSolar.__init__}]{\sphinxcrossref{\sphinxcode{\sphinxupquote{\_\_init\_\_}}}}}(energy\_sink, air\_temperature, ...)
&
\sphinxAtStartPar
FPSolar components convert solar irradiance to heat.
\\
\hline
\sphinxAtStartPar
{\hyperref[\detokenize{generated/tamos.production.FPSolar:tamos.production.FPSolar.compute_actualized_cost}]{\sphinxcrossref{\sphinxcode{\sphinxupquote{compute\_actualized\_cost}}}}}(CAPEX, OPEX, ...{[}, ...{]})
&
\sphinxAtStartPar
Computes the cost of a component using its \textquotesingle{}Lifetime\textquotesingle{} and \textquotesingle{}Discount rate (\%)\textquotesingle{} properties.
\\
\hline
\end{tabulary}
\par
\sphinxattableend\end{savenotes}
\subsubsection*{Attributes}


\begin{savenotes}\sphinxattablestart
\centering
\begin{tabulary}{\linewidth}[t]{\X{1}{2}\X{1}{2}}
\hline

\sphinxAtStartPar
{\hyperref[\detokenize{generated/tamos.production.FPSolar:tamos.production.FPSolar.air_temperature}]{\sphinxcrossref{\sphinxcode{\sphinxupquote{air\_temperature}}}}}
&
\sphinxAtStartPar
Temperature of the air surrounding the solar panels.
\\
\hline
\sphinxAtStartPar
{\hyperref[\detokenize{generated/tamos.production.FPSolar:tamos.production.FPSolar.eco_count}]{\sphinxcrossref{\sphinxcode{\sphinxupquote{eco\_count}}}}}
&
\sphinxAtStartPar
Whether this instance contributes to the system "Eco" KPI.
\\
\hline
\sphinxAtStartPar
{\hyperref[\detokenize{generated/tamos.production.FPSolar:tamos.production.FPSolar.energy_sink}]{\sphinxcrossref{\sphinxcode{\sphinxupquote{energy\_sink}}}}}
&
\sphinxAtStartPar
Thermal flow that is warmed up.
\\
\hline
\sphinxAtStartPar
{\hyperref[\detokenize{generated/tamos.production.FPSolar:tamos.production.FPSolar.given_sizing}]{\sphinxcrossref{\sphinxcode{\sphinxupquote{given\_sizing}}}}}
&
\sphinxAtStartPar
The maximum capacity of the component, in kW.
\\
\hline
\sphinxAtStartPar
{\hyperref[\detokenize{generated/tamos.production.FPSolar:tamos.production.FPSolar.name}]{\sphinxcrossref{\sphinxcode{\sphinxupquote{name}}}}}
&
\sphinxAtStartPar
str.
\\
\hline
\sphinxAtStartPar
{\hyperref[\detokenize{generated/tamos.production.FPSolar:tamos.production.FPSolar.pinch}]{\sphinxcrossref{\sphinxcode{\sphinxupquote{pinch}}}}}
&
\sphinxAtStartPar
Difference between the temperature of the fluid circulating in the solar panels and the one of \sphinxtitleref{energy\_sink}.
\\
\hline
\sphinxAtStartPar
{\hyperref[\detokenize{generated/tamos.production.FPSolar:tamos.production.FPSolar.total_irradiance}]{\sphinxcrossref{\sphinxcode{\sphinxupquote{total\_irradiance}}}}}
&
\sphinxAtStartPar
Solar irradiance received on the normal of the solar panels.
\\
\hline
\sphinxAtStartPar
{\hyperref[\detokenize{generated/tamos.production.FPSolar:tamos.production.FPSolar.units_number_lb}]{\sphinxcrossref{\sphinxcode{\sphinxupquote{units\_number\_lb}}}}}
&
\sphinxAtStartPar
The lower bound of the number of real components that this instance aims to stand for.
\\
\hline
\sphinxAtStartPar
{\hyperref[\detokenize{generated/tamos.production.FPSolar:tamos.production.FPSolar.units_number_ub}]{\sphinxcrossref{\sphinxcode{\sphinxupquote{units\_number\_ub}}}}}
&
\sphinxAtStartPar
The upper bound of the number of real components that this instance aims to stand for.
\\
\hline
\sphinxAtStartPar
{\hyperref[\detokenize{generated/tamos.production.FPSolar:tamos.production.FPSolar.used_elements}]{\sphinxcrossref{\sphinxcode{\sphinxupquote{used\_elements}}}}}
&
\sphinxAtStartPar
Elements used by the component.
\\
\hline
\end{tabulary}
\par
\sphinxattableend\end{savenotes}
\index{air\_temperature (tamos.production.FPSolar property)@\spxentry{air\_temperature}\spxextra{tamos.production.FPSolar property}}

\begin{fulllineitems}
\phantomsection\label{\detokenize{generated/tamos.production.FPSolar:tamos.production.FPSolar.air_temperature}}
\pysigstartsignatures
\pysigline{\sphinxbfcode{\sphinxupquote{property\DUrole{w}{  }}}\sphinxbfcode{\sphinxupquote{air\_temperature}}}
\pysigstopsignatures
\sphinxAtStartPar
Temperature of the air surrounding the solar panels.
In Kelvins (K).
int, float or numpy.ndarray

\end{fulllineitems}

\index{compute\_actualized\_cost() (tamos.production.FPSolar method)@\spxentry{compute\_actualized\_cost()}\spxextra{tamos.production.FPSolar method}}

\begin{fulllineitems}
\phantomsection\label{\detokenize{generated/tamos.production.FPSolar:tamos.production.FPSolar.compute_actualized_cost}}
\pysigstartsignatures
\pysiglinewithargsret{\sphinxbfcode{\sphinxupquote{compute\_actualized\_cost}}}{\emph{\DUrole{n}{CAPEX}}, \emph{\DUrole{n}{OPEX}}, \emph{\DUrole{n}{system\_lifetime}}, \emph{\DUrole{n}{lifetime}\DUrole{o}{=}\DUrole{default_value}{None}}, \emph{\DUrole{n}{discount\_rate}\DUrole{o}{=}\DUrole{default_value}{None}}}{}
\pysigstopsignatures
\sphinxAtStartPar
Computes the cost of a component using its ‘Lifetime’ and ‘Discount rate (\%)’ properties.
\begin{quote}\begin{description}
\sphinxlineitem{Parameters}\begin{itemize}
\item {} 
\sphinxAtStartPar
\sphinxstyleliteralstrong{\sphinxupquote{CAPEX}} (\sphinxstyleliteralemphasis{\sphinxupquote{float}}) \textendash{} Capital Expenditure. Cost in euros paid every \sphinxtitleref{technical\_lifetime} periods.

\item {} 
\sphinxAtStartPar
\sphinxstyleliteralstrong{\sphinxupquote{OPEX}} (\sphinxstyleliteralemphasis{\sphinxupquote{float}}) \textendash{} Operational Expenditure. Cost in euros paid each period.

\item {} 
\sphinxAtStartPar
\sphinxstyleliteralstrong{\sphinxupquote{system\_lifetime}} (\sphinxstyleliteralemphasis{\sphinxupquote{int}}) \textendash{} Number of periods defining the existence of the energy system.

\item {} 
\sphinxAtStartPar
\sphinxstyleliteralstrong{\sphinxupquote{lifetime}} (\sphinxstyleliteralemphasis{\sphinxupquote{int}}\sphinxstyleliteralemphasis{\sphinxupquote{, }}\sphinxstyleliteralemphasis{\sphinxupquote{optional}}) \textendash{} Number of periods defining the existence of the component.
If specified, overwrite the “Lifetime” property.

\item {} 
\sphinxAtStartPar
\sphinxstyleliteralstrong{\sphinxupquote{discount\_rate}} (\sphinxstyleliteralemphasis{\sphinxupquote{float}}) \textendash{} In percent (\%). Describes the importance of the economic amortization process, per period.
If specified, overwrite the “Discount rate (\%)” property.

\end{itemize}

\sphinxlineitem{Returns}
\sphinxAtStartPar
\begin{itemize}
\item {} 
\sphinxAtStartPar
\sphinxstyleemphasis{A 3\sphinxhyphen{}tuple (total\_cost, CAPEX\_share, OPEX\_share) where}

\item {} 
\sphinxAtStartPar
* CAPEX\_share is the share of total cost related to \sphinxtitleref{CAPEX}

\item {} 
\sphinxAtStartPar
* OPEX\_share is the share of total cost related to \sphinxtitleref{OPEX}

\item {} 
\sphinxAtStartPar
\sphinxstyleemphasis{* total\_cost = CAPEX\_share + OPEX\_share}

\end{itemize}


\end{description}\end{quote}
\subsubsection*{Notes}

\sphinxAtStartPar
Takes into account residual value of component in the case \sphinxtitleref{system\_lifetime} is not a multiple of \sphinxtitleref{lifetime}.
In this case, the last replacement occuring at period replacement\_period is paid in proportion of ‘CAPEX’
depending linearly on the number of periods left:
CAPEX * (system\_lifetime \sphinxhyphen{} replacement\_period) / lifetime

\end{fulllineitems}

\index{eco\_count (tamos.production.FPSolar property)@\spxentry{eco\_count}\spxextra{tamos.production.FPSolar property}}

\begin{fulllineitems}
\phantomsection\label{\detokenize{generated/tamos.production.FPSolar:tamos.production.FPSolar.eco_count}}
\pysigstartsignatures
\pysigline{\sphinxbfcode{\sphinxupquote{property\DUrole{w}{  }}}\sphinxbfcode{\sphinxupquote{eco\_count}}}
\pysigstopsignatures
\sphinxAtStartPar
Whether this instance contributes to the system “Eco” KPI.
bool

\end{fulllineitems}

\index{energy\_sink (tamos.production.FPSolar property)@\spxentry{energy\_sink}\spxextra{tamos.production.FPSolar property}}

\begin{fulllineitems}
\phantomsection\label{\detokenize{generated/tamos.production.FPSolar:tamos.production.FPSolar.energy_sink}}
\pysigstartsignatures
\pysigline{\sphinxbfcode{\sphinxupquote{property\DUrole{w}{  }}}\sphinxbfcode{\sphinxupquote{energy\_sink}}}
\pysigstopsignatures
\sphinxAtStartPar
Thermal flow that is warmed up.
ThermalVectorPair

\end{fulllineitems}

\index{given\_sizing (tamos.production.FPSolar property)@\spxentry{given\_sizing}\spxextra{tamos.production.FPSolar property}}

\begin{fulllineitems}
\phantomsection\label{\detokenize{generated/tamos.production.FPSolar:tamos.production.FPSolar.given_sizing}}
\pysigstartsignatures
\pysigline{\sphinxbfcode{\sphinxupquote{property\DUrole{w}{  }}}\sphinxbfcode{\sphinxupquote{given\_sizing}}}
\pysigstopsignatures
\sphinxAtStartPar
The maximum capacity of the component, in kW.
Relates to decision variable ‘SP\_P’.
int or float

\end{fulllineitems}

\index{name (tamos.production.FPSolar property)@\spxentry{name}\spxextra{tamos.production.FPSolar property}}

\begin{fulllineitems}
\phantomsection\label{\detokenize{generated/tamos.production.FPSolar:tamos.production.FPSolar.name}}
\pysigstartsignatures
\pysigline{\sphinxbfcode{\sphinxupquote{property\DUrole{w}{  }}}\sphinxbfcode{\sphinxupquote{name}}}
\pysigstopsignatures
\sphinxAtStartPar
str.
This name is used in MILP model description.
names must not exceed 255 characters,
all of which must be alphanumeric (a\sphinxhyphen{}z, A\sphinxhyphen{}Z, 0\sphinxhyphen{}9) or one of these symbols:
! ” \# \$ \% \& , . ; ? @ \_ ‘ ’ \{ \} \textasciitilde{}.
\begin{quote}\begin{description}
\sphinxlineitem{Type}
\sphinxAtStartPar
Name of the instance

\end{description}\end{quote}

\end{fulllineitems}

\index{pinch (tamos.production.FPSolar property)@\spxentry{pinch}\spxextra{tamos.production.FPSolar property}}

\begin{fulllineitems}
\phantomsection\label{\detokenize{generated/tamos.production.FPSolar:tamos.production.FPSolar.pinch}}
\pysigstartsignatures
\pysigline{\sphinxbfcode{\sphinxupquote{property\DUrole{w}{  }}}\sphinxbfcode{\sphinxupquote{pinch}}}
\pysigstopsignatures
\sphinxAtStartPar
Difference between the temperature of the fluid circulating in the solar panels and the one of \sphinxtitleref{energy\_sink}.
int, float or numpy.ndarray

\end{fulllineitems}

\index{total\_irradiance (tamos.production.FPSolar property)@\spxentry{total\_irradiance}\spxextra{tamos.production.FPSolar property}}

\begin{fulllineitems}
\phantomsection\label{\detokenize{generated/tamos.production.FPSolar:tamos.production.FPSolar.total_irradiance}}
\pysigstartsignatures
\pysigline{\sphinxbfcode{\sphinxupquote{property\DUrole{w}{  }}}\sphinxbfcode{\sphinxupquote{total\_irradiance}}}
\pysigstopsignatures
\sphinxAtStartPar
Solar irradiance received on the normal of the solar panels.
In kW/m\(\sp{\text{2}}\).
int, float or numpy.ndarray

\end{fulllineitems}

\index{units\_number\_lb (tamos.production.FPSolar property)@\spxentry{units\_number\_lb}\spxextra{tamos.production.FPSolar property}}

\begin{fulllineitems}
\phantomsection\label{\detokenize{generated/tamos.production.FPSolar:tamos.production.FPSolar.units_number_lb}}
\pysigstartsignatures
\pysigline{\sphinxbfcode{\sphinxupquote{property\DUrole{w}{  }}}\sphinxbfcode{\sphinxupquote{units\_number\_lb}}}
\pysigstopsignatures
\sphinxAtStartPar
The lower bound of the number of real components that this instance aims to stand for.
Setting \sphinxtitleref{units\_number\_lb} has a meaning if “LB max output power (kW)” property is different from 0.
int

\end{fulllineitems}

\index{units\_number\_ub (tamos.production.FPSolar property)@\spxentry{units\_number\_ub}\spxextra{tamos.production.FPSolar property}}

\begin{fulllineitems}
\phantomsection\label{\detokenize{generated/tamos.production.FPSolar:tamos.production.FPSolar.units_number_ub}}
\pysigstartsignatures
\pysigline{\sphinxbfcode{\sphinxupquote{property\DUrole{w}{  }}}\sphinxbfcode{\sphinxupquote{units\_number\_ub}}}
\pysigstopsignatures
\sphinxAtStartPar
The upper bound of the number of real components that this instance aims to stand for.
Setting \sphinxtitleref{units\_number\_ub} has a meaning if “LB max output power (kW)” property is different from 0.
int

\end{fulllineitems}

\index{used\_elements (tamos.production.FPSolar property)@\spxentry{used\_elements}\spxextra{tamos.production.FPSolar property}}

\begin{fulllineitems}
\phantomsection\label{\detokenize{generated/tamos.production.FPSolar:tamos.production.FPSolar.used_elements}}
\pysigstartsignatures
\pysigline{\sphinxbfcode{\sphinxupquote{property\DUrole{w}{  }}}\sphinxbfcode{\sphinxupquote{used\_elements}}}
\pysigstopsignatures
\sphinxAtStartPar
Elements used by the component.

\end{fulllineitems}


\end{fulllineitems}


\sphinxstepscope


\section{tamos.production.GasBoiler}
\label{\detokenize{generated/tamos.production.GasBoiler:tamos-production-gasboiler}}\label{\detokenize{generated/tamos.production.GasBoiler::doc}}\index{GasBoiler (class in tamos.production)@\spxentry{GasBoiler}\spxextra{class in tamos.production}}

\begin{fulllineitems}
\phantomsection\label{\detokenize{generated/tamos.production.GasBoiler:tamos.production.GasBoiler}}
\pysigstartsignatures
\pysiglinewithargsret{\sphinxbfcode{\sphinxupquote{class\DUrole{w}{  }}}\sphinxcode{\sphinxupquote{tamos.production.}}\sphinxbfcode{\sphinxupquote{GasBoiler}}}{\emph{\DUrole{n}{energy\_source}}, \emph{\DUrole{n}{energy\_sink}}, \emph{\DUrole{n}{properties}}, \emph{\DUrole{n}{given\_sizing}\DUrole{o}{=}\DUrole{default_value}{None}}, \emph{\DUrole{n}{name}\DUrole{o}{=}\DUrole{default_value}{None}}, \emph{\DUrole{n}{units\_number\_ub}\DUrole{o}{=}\DUrole{default_value}{1}}, \emph{\DUrole{n}{units\_number\_lb}\DUrole{o}{=}\DUrole{default_value}{1}}, \emph{\DUrole{n}{eco\_count}\DUrole{o}{=}\DUrole{default_value}{True}}}{}
\pysigstopsignatures\index{\_\_init\_\_() (tamos.production.GasBoiler method)@\spxentry{\_\_init\_\_()}\spxextra{tamos.production.GasBoiler method}}

\begin{fulllineitems}
\phantomsection\label{\detokenize{generated/tamos.production.GasBoiler:tamos.production.GasBoiler.__init__}}
\pysigstartsignatures
\pysiglinewithargsret{\sphinxbfcode{\sphinxupquote{\_\_init\_\_}}}{\emph{\DUrole{n}{energy\_source}}, \emph{\DUrole{n}{energy\_sink}}, \emph{\DUrole{n}{properties}}, \emph{\DUrole{n}{given\_sizing}\DUrole{o}{=}\DUrole{default_value}{None}}, \emph{\DUrole{n}{name}\DUrole{o}{=}\DUrole{default_value}{None}}, \emph{\DUrole{n}{units\_number\_ub}\DUrole{o}{=}\DUrole{default_value}{1}}, \emph{\DUrole{n}{units\_number\_lb}\DUrole{o}{=}\DUrole{default_value}{1}}, \emph{\DUrole{n}{eco\_count}\DUrole{o}{=}\DUrole{default_value}{True}}}{}
\pysigstopsignatures
\sphinxAtStartPar
GasBoiler components produce heat given an energy efficiency of typical gas boilers.
This efficiency takes into account the condensing property of flue gases.

\sphinxAtStartPar
This component declares the following exported decision variables:
\begin{itemize}
\item {} 
\sphinxAtStartPar
X\_P, binary.
Whether the component is used by the hub.

\item {} 
\sphinxAtStartPar
SP\_P, continuous, in kW.
The maximum capacity of the component. Defines the investment costs.

\item {} 
\sphinxAtStartPar
For all t, for all element e, F\_P(e, t), continuous, in kW.
The power related to element e entering the component (i.e. leaving the hub interface).

\item {} 
\sphinxAtStartPar
For all t, Q\_P(t), continuous, in kW.
The reference power related to the component. Defines the variable cost.
This power is a lower bound of SP\_P.
There exists one element e such that Q\_P(t) = F\_P(e, t) or Q\_P(t) = \sphinxhyphen{} F\_P(e, t).
For this component, e is \sphinxtitleref{energy\_sink}.

\end{itemize}

\sphinxAtStartPar
This component declares the following KPIs:
\begin{itemize}
\item {} 
\sphinxAtStartPar
\sphinxtitleref{COST\_production}
In euros.
Contributes to the “Eco” objective function.

\end{itemize}
\begin{quote}\begin{description}
\sphinxlineitem{Parameters}\begin{itemize}
\item {} 
\sphinxAtStartPar
\sphinxstyleliteralstrong{\sphinxupquote{energy\_source}} ({\hyperref[\detokenize{generated/tamos.element.FuelVector:tamos.element.FuelVector}]{\sphinxcrossref{\sphinxstyleliteralemphasis{\sphinxupquote{FuelVector}}}}}) \textendash{} Natural gas that is consumed.

\item {} 
\sphinxAtStartPar
\sphinxstyleliteralstrong{\sphinxupquote{energy\_sink}} (\sphinxstyleliteralemphasis{\sphinxupquote{ThermalVectorPair}}) \textendash{} Thermal flow that is warmed up by the boiler.

\item {} 
\sphinxAtStartPar
\sphinxstyleliteralstrong{\sphinxupquote{properties}} (\sphinxstyleliteralemphasis{\sphinxupquote{dict \{str: int}}\sphinxstyleliteralemphasis{\sphinxupquote{ | }}\sphinxstyleliteralemphasis{\sphinxupquote{float\}}}) \textendash{} 
\sphinxAtStartPar
Techno\sphinxhyphen{}economic properties of the component.
The \sphinxtitleref{properties} attribute must include the following keys:
\begin{itemize}
\item {} 
\sphinxAtStartPar
”LB max output power (kW)”

\item {} 
\sphinxAtStartPar
”UB max output power (kW)”

\item {} 
\sphinxAtStartPar
”CAPEX (EUR/kW)”

\item {} 
\sphinxAtStartPar
”OPEX (\%CAPEX)”

\item {} 
\sphinxAtStartPar
”Variable OPEX (EUR/MWh)”

\end{itemize}


\item {} 
\sphinxAtStartPar
\sphinxstyleliteralstrong{\sphinxupquote{given\_sizing}} (\sphinxstyleliteralemphasis{\sphinxupquote{int}}\sphinxstyleliteralemphasis{\sphinxupquote{ or }}\sphinxstyleliteralemphasis{\sphinxupquote{float}}\sphinxstyleliteralemphasis{\sphinxupquote{, }}\sphinxstyleliteralemphasis{\sphinxupquote{optional}}) \textendash{} The maximum capacity of the component, in kW.
Relates to decision variable ‘SP\_P’.
If specified, only the operation of this component is performed by the MILP solver.
If let unknown, both sizing and operation are performed.

\item {} 
\sphinxAtStartPar
\sphinxstyleliteralstrong{\sphinxupquote{name}} (\sphinxstyleliteralemphasis{\sphinxupquote{str}}\sphinxstyleliteralemphasis{\sphinxupquote{, }}\sphinxstyleliteralemphasis{\sphinxupquote{optional}}) \textendash{} 

\item {} 
\sphinxAtStartPar
\sphinxstyleliteralstrong{\sphinxupquote{units\_number\_lb}} (\sphinxstyleliteralemphasis{\sphinxupquote{int}}\sphinxstyleliteralemphasis{\sphinxupquote{, }}\sphinxstyleliteralemphasis{\sphinxupquote{optional}}\sphinxstyleliteralemphasis{\sphinxupquote{, }}\sphinxstyleliteralemphasis{\sphinxupquote{default 1}}) \textendash{} The lower bound (upper bound) of the number of real components that this instance aims to stand for.
Setting \sphinxtitleref{units\_number\_lb} (\sphinxtitleref{units\_number\_ub}) has a meaning if “LB max output power (kW)” property is
different from 0.

\item {} 
\sphinxAtStartPar
\sphinxstyleliteralstrong{\sphinxupquote{units\_number\_ub}} (\sphinxstyleliteralemphasis{\sphinxupquote{int}}\sphinxstyleliteralemphasis{\sphinxupquote{, }}\sphinxstyleliteralemphasis{\sphinxupquote{optional}}\sphinxstyleliteralemphasis{\sphinxupquote{, }}\sphinxstyleliteralemphasis{\sphinxupquote{default 1}}) \textendash{} The lower bound (upper bound) of the number of real components that this instance aims to stand for.
Setting \sphinxtitleref{units\_number\_lb} (\sphinxtitleref{units\_number\_ub}) has a meaning if “LB max output power (kW)” property is
different from 0.

\item {} 
\sphinxAtStartPar
\sphinxstyleliteralstrong{\sphinxupquote{eco\_count}} (\sphinxstyleliteralemphasis{\sphinxupquote{bool}}\sphinxstyleliteralemphasis{\sphinxupquote{, }}\sphinxstyleliteralemphasis{\sphinxupquote{optional}}\sphinxstyleliteralemphasis{\sphinxupquote{, }}\sphinxstyleliteralemphasis{\sphinxupquote{default True}}) \textendash{} Whether this instance contributes to the system “Eco” KPI.

\end{itemize}

\end{description}\end{quote}

\end{fulllineitems}

\subsubsection*{Methods}


\begin{savenotes}\sphinxattablestart
\centering
\begin{tabulary}{\linewidth}[t]{\X{1}{2}\X{1}{2}}
\hline

\sphinxAtStartPar
{\hyperref[\detokenize{generated/tamos.production.GasBoiler:tamos.production.GasBoiler.__init__}]{\sphinxcrossref{\sphinxcode{\sphinxupquote{\_\_init\_\_}}}}}(energy\_source, energy\_sink, properties)
&
\sphinxAtStartPar
GasBoiler components produce heat given an energy efficiency of typical gas boilers.
\\
\hline
\sphinxAtStartPar
{\hyperref[\detokenize{generated/tamos.production.GasBoiler:tamos.production.GasBoiler.compute_actualized_cost}]{\sphinxcrossref{\sphinxcode{\sphinxupquote{compute\_actualized\_cost}}}}}(CAPEX, OPEX, ...{[}, ...{]})
&
\sphinxAtStartPar
Computes the cost of a component using its \textquotesingle{}Lifetime\textquotesingle{} and \textquotesingle{}Discount rate (\%)\textquotesingle{} properties.
\\
\hline
\sphinxAtStartPar
\sphinxcode{\sphinxupquote{default\_efficiency}}(T)
&
\sphinxAtStartPar

\\
\hline
\sphinxAtStartPar
{\hyperref[\detokenize{generated/tamos.production.GasBoiler:tamos.production.GasBoiler.set_efficiency_model}]{\sphinxcrossref{\sphinxcode{\sphinxupquote{set\_efficiency\_model}}}}}(efficiency\_function, pinch)
&
\sphinxAtStartPar
Defines the efficiency of the conversion of \sphinxtitleref{energy\_source} to \sphinxtitleref{energy\_sink} using a function of the cold temperature of \sphinxtitleref{energy\_sink}.
\\
\hline
\end{tabulary}
\par
\sphinxattableend\end{savenotes}
\subsubsection*{Attributes}


\begin{savenotes}\sphinxattablestart
\centering
\begin{tabulary}{\linewidth}[t]{\X{1}{2}\X{1}{2}}
\hline

\sphinxAtStartPar
{\hyperref[\detokenize{generated/tamos.production.GasBoiler:tamos.production.GasBoiler.eco_count}]{\sphinxcrossref{\sphinxcode{\sphinxupquote{eco\_count}}}}}
&
\sphinxAtStartPar
Whether this instance contributes to the system "Eco" KPI.
\\
\hline
\sphinxAtStartPar
{\hyperref[\detokenize{generated/tamos.production.GasBoiler:tamos.production.GasBoiler.efficiency}]{\sphinxcrossref{\sphinxcode{\sphinxupquote{efficiency}}}}}
&
\sphinxAtStartPar
Defines explicitly the efficiency of the conversion of \sphinxtitleref{energy\_source} to \sphinxtitleref{energy\_sink}.
\\
\hline
\sphinxAtStartPar
{\hyperref[\detokenize{generated/tamos.production.GasBoiler:tamos.production.GasBoiler.energy_sink}]{\sphinxcrossref{\sphinxcode{\sphinxupquote{energy\_sink}}}}}
&
\sphinxAtStartPar
Thermal flow that is warmed up by the boiler.
\\
\hline
\sphinxAtStartPar
\sphinxcode{\sphinxupquote{energy\_source}}
&
\sphinxAtStartPar

\\
\hline
\sphinxAtStartPar
{\hyperref[\detokenize{generated/tamos.production.GasBoiler:tamos.production.GasBoiler.given_sizing}]{\sphinxcrossref{\sphinxcode{\sphinxupquote{given\_sizing}}}}}
&
\sphinxAtStartPar
The maximum capacity of the component, in kW.
\\
\hline
\sphinxAtStartPar
{\hyperref[\detokenize{generated/tamos.production.GasBoiler:tamos.production.GasBoiler.name}]{\sphinxcrossref{\sphinxcode{\sphinxupquote{name}}}}}
&
\sphinxAtStartPar
str.
\\
\hline
\sphinxAtStartPar
{\hyperref[\detokenize{generated/tamos.production.GasBoiler:tamos.production.GasBoiler.units_number_lb}]{\sphinxcrossref{\sphinxcode{\sphinxupquote{units\_number\_lb}}}}}
&
\sphinxAtStartPar
The lower bound of the number of real components that this instance aims to stand for.
\\
\hline
\sphinxAtStartPar
{\hyperref[\detokenize{generated/tamos.production.GasBoiler:tamos.production.GasBoiler.units_number_ub}]{\sphinxcrossref{\sphinxcode{\sphinxupquote{units\_number\_ub}}}}}
&
\sphinxAtStartPar
The upper bound of the number of real components that this instance aims to stand for.
\\
\hline
\sphinxAtStartPar
{\hyperref[\detokenize{generated/tamos.production.GasBoiler:tamos.production.GasBoiler.used_elements}]{\sphinxcrossref{\sphinxcode{\sphinxupquote{used\_elements}}}}}
&
\sphinxAtStartPar
Elements used by the component.
\\
\hline
\end{tabulary}
\par
\sphinxattableend\end{savenotes}
\index{compute\_actualized\_cost() (tamos.production.GasBoiler method)@\spxentry{compute\_actualized\_cost()}\spxextra{tamos.production.GasBoiler method}}

\begin{fulllineitems}
\phantomsection\label{\detokenize{generated/tamos.production.GasBoiler:tamos.production.GasBoiler.compute_actualized_cost}}
\pysigstartsignatures
\pysiglinewithargsret{\sphinxbfcode{\sphinxupquote{compute\_actualized\_cost}}}{\emph{\DUrole{n}{CAPEX}}, \emph{\DUrole{n}{OPEX}}, \emph{\DUrole{n}{system\_lifetime}}, \emph{\DUrole{n}{lifetime}\DUrole{o}{=}\DUrole{default_value}{None}}, \emph{\DUrole{n}{discount\_rate}\DUrole{o}{=}\DUrole{default_value}{None}}}{}
\pysigstopsignatures
\sphinxAtStartPar
Computes the cost of a component using its ‘Lifetime’ and ‘Discount rate (\%)’ properties.
\begin{quote}\begin{description}
\sphinxlineitem{Parameters}\begin{itemize}
\item {} 
\sphinxAtStartPar
\sphinxstyleliteralstrong{\sphinxupquote{CAPEX}} (\sphinxstyleliteralemphasis{\sphinxupquote{float}}) \textendash{} Capital Expenditure. Cost in euros paid every \sphinxtitleref{technical\_lifetime} periods.

\item {} 
\sphinxAtStartPar
\sphinxstyleliteralstrong{\sphinxupquote{OPEX}} (\sphinxstyleliteralemphasis{\sphinxupquote{float}}) \textendash{} Operational Expenditure. Cost in euros paid each period.

\item {} 
\sphinxAtStartPar
\sphinxstyleliteralstrong{\sphinxupquote{system\_lifetime}} (\sphinxstyleliteralemphasis{\sphinxupquote{int}}) \textendash{} Number of periods defining the existence of the energy system.

\item {} 
\sphinxAtStartPar
\sphinxstyleliteralstrong{\sphinxupquote{lifetime}} (\sphinxstyleliteralemphasis{\sphinxupquote{int}}\sphinxstyleliteralemphasis{\sphinxupquote{, }}\sphinxstyleliteralemphasis{\sphinxupquote{optional}}) \textendash{} Number of periods defining the existence of the component.
If specified, overwrite the “Lifetime” property.

\item {} 
\sphinxAtStartPar
\sphinxstyleliteralstrong{\sphinxupquote{discount\_rate}} (\sphinxstyleliteralemphasis{\sphinxupquote{float}}) \textendash{} In percent (\%). Describes the importance of the economic amortization process, per period.
If specified, overwrite the “Discount rate (\%)” property.

\end{itemize}

\sphinxlineitem{Returns}
\sphinxAtStartPar
\begin{itemize}
\item {} 
\sphinxAtStartPar
\sphinxstyleemphasis{A 3\sphinxhyphen{}tuple (total\_cost, CAPEX\_share, OPEX\_share) where}

\item {} 
\sphinxAtStartPar
* CAPEX\_share is the share of total cost related to \sphinxtitleref{CAPEX}

\item {} 
\sphinxAtStartPar
* OPEX\_share is the share of total cost related to \sphinxtitleref{OPEX}

\item {} 
\sphinxAtStartPar
\sphinxstyleemphasis{* total\_cost = CAPEX\_share + OPEX\_share}

\end{itemize}


\end{description}\end{quote}
\subsubsection*{Notes}

\sphinxAtStartPar
Takes into account residual value of component in the case \sphinxtitleref{system\_lifetime} is not a multiple of \sphinxtitleref{lifetime}.
In this case, the last replacement occuring at period replacement\_period is paid in proportion of ‘CAPEX’
depending linearly on the number of periods left:
CAPEX * (system\_lifetime \sphinxhyphen{} replacement\_period) / lifetime

\end{fulllineitems}

\index{eco\_count (tamos.production.GasBoiler property)@\spxentry{eco\_count}\spxextra{tamos.production.GasBoiler property}}

\begin{fulllineitems}
\phantomsection\label{\detokenize{generated/tamos.production.GasBoiler:tamos.production.GasBoiler.eco_count}}
\pysigstartsignatures
\pysigline{\sphinxbfcode{\sphinxupquote{property\DUrole{w}{  }}}\sphinxbfcode{\sphinxupquote{eco\_count}}}
\pysigstopsignatures
\sphinxAtStartPar
Whether this instance contributes to the system “Eco” KPI.
bool

\end{fulllineitems}

\index{efficiency (tamos.production.GasBoiler property)@\spxentry{efficiency}\spxextra{tamos.production.GasBoiler property}}

\begin{fulllineitems}
\phantomsection\label{\detokenize{generated/tamos.production.GasBoiler:tamos.production.GasBoiler.efficiency}}
\pysigstartsignatures
\pysigline{\sphinxbfcode{\sphinxupquote{property\DUrole{w}{  }}}\sphinxbfcode{\sphinxupquote{efficiency}}}
\pysigstopsignatures
\sphinxAtStartPar
Defines explicitly the efficiency of the conversion of \sphinxtitleref{energy\_source} to \sphinxtitleref{energy\_sink}.
If called, replaces the definition of the efficiency using \sphinxtitleref{set\_efficiency\_model} (default).
int, float or numpy.ndarray

\end{fulllineitems}

\index{energy\_sink (tamos.production.GasBoiler property)@\spxentry{energy\_sink}\spxextra{tamos.production.GasBoiler property}}

\begin{fulllineitems}
\phantomsection\label{\detokenize{generated/tamos.production.GasBoiler:tamos.production.GasBoiler.energy_sink}}
\pysigstartsignatures
\pysigline{\sphinxbfcode{\sphinxupquote{property\DUrole{w}{  }}}\sphinxbfcode{\sphinxupquote{energy\_sink}}}
\pysigstopsignatures
\sphinxAtStartPar
Thermal flow that is warmed up by the boiler.

\end{fulllineitems}

\index{given\_sizing (tamos.production.GasBoiler property)@\spxentry{given\_sizing}\spxextra{tamos.production.GasBoiler property}}

\begin{fulllineitems}
\phantomsection\label{\detokenize{generated/tamos.production.GasBoiler:tamos.production.GasBoiler.given_sizing}}
\pysigstartsignatures
\pysigline{\sphinxbfcode{\sphinxupquote{property\DUrole{w}{  }}}\sphinxbfcode{\sphinxupquote{given\_sizing}}}
\pysigstopsignatures
\sphinxAtStartPar
The maximum capacity of the component, in kW.
Relates to decision variable ‘SP\_P’.
int or float

\end{fulllineitems}

\index{name (tamos.production.GasBoiler property)@\spxentry{name}\spxextra{tamos.production.GasBoiler property}}

\begin{fulllineitems}
\phantomsection\label{\detokenize{generated/tamos.production.GasBoiler:tamos.production.GasBoiler.name}}
\pysigstartsignatures
\pysigline{\sphinxbfcode{\sphinxupquote{property\DUrole{w}{  }}}\sphinxbfcode{\sphinxupquote{name}}}
\pysigstopsignatures
\sphinxAtStartPar
str.
This name is used in MILP model description.
names must not exceed 255 characters,
all of which must be alphanumeric (a\sphinxhyphen{}z, A\sphinxhyphen{}Z, 0\sphinxhyphen{}9) or one of these symbols:
! ” \# \$ \% \& , . ; ? @ \_ ‘ ’ \{ \} \textasciitilde{}.
\begin{quote}\begin{description}
\sphinxlineitem{Type}
\sphinxAtStartPar
Name of the instance

\end{description}\end{quote}

\end{fulllineitems}

\index{set\_efficiency\_model() (tamos.production.GasBoiler method)@\spxentry{set\_efficiency\_model()}\spxextra{tamos.production.GasBoiler method}}

\begin{fulllineitems}
\phantomsection\label{\detokenize{generated/tamos.production.GasBoiler:tamos.production.GasBoiler.set_efficiency_model}}
\pysigstartsignatures
\pysiglinewithargsret{\sphinxbfcode{\sphinxupquote{set\_efficiency\_model}}}{\emph{\DUrole{n}{efficiency\_function}}, \emph{\DUrole{n}{pinch}}}{}
\pysigstopsignatures
\sphinxAtStartPar
Defines the efficiency of the conversion of \sphinxtitleref{energy\_source} to \sphinxtitleref{energy\_sink} using
a function of the cold temperature of \sphinxtitleref{energy\_sink}.
\begin{quote}\begin{description}
\sphinxlineitem{Parameters}\begin{itemize}
\item {} 
\sphinxAtStartPar
\sphinxstyleliteralstrong{\sphinxupquote{efficiency\_function}} (\sphinxstyleliteralemphasis{\sphinxupquote{callable f}}\sphinxstyleliteralemphasis{\sphinxupquote{(}}\sphinxstyleliteralemphasis{\sphinxupquote{T}}\sphinxstyleliteralemphasis{\sphinxupquote{)}}) \textendash{} T is the temperature of the cold vector of \sphinxtitleref{energy\_sink}, in Kelvins (K).

\item {} 
\sphinxAtStartPar
\sphinxstyleliteralstrong{\sphinxupquote{pinch}} (\sphinxstyleliteralemphasis{\sphinxupquote{int}}\sphinxstyleliteralemphasis{\sphinxupquote{, }}\sphinxstyleliteralemphasis{\sphinxupquote{float}}\sphinxstyleliteralemphasis{\sphinxupquote{ or }}\sphinxstyleliteralemphasis{\sphinxupquote{numpy.ndarray}}) \textendash{} Temperature difference between the flue gases of the boiler and the cold vector of \sphinxtitleref{energy\_sink}, in Kelvins (K).

\end{itemize}

\end{description}\end{quote}
\subsubsection*{Notes}

\sphinxAtStartPar
By default, set\_efficiency\_model is called with the \sphinxtitleref{default\_efficiency} attribute of this instance and pinch = 2.

\end{fulllineitems}

\index{units\_number\_lb (tamos.production.GasBoiler property)@\spxentry{units\_number\_lb}\spxextra{tamos.production.GasBoiler property}}

\begin{fulllineitems}
\phantomsection\label{\detokenize{generated/tamos.production.GasBoiler:tamos.production.GasBoiler.units_number_lb}}
\pysigstartsignatures
\pysigline{\sphinxbfcode{\sphinxupquote{property\DUrole{w}{  }}}\sphinxbfcode{\sphinxupquote{units\_number\_lb}}}
\pysigstopsignatures
\sphinxAtStartPar
The lower bound of the number of real components that this instance aims to stand for.
Setting \sphinxtitleref{units\_number\_lb} has a meaning if “LB max output power (kW)” property is different from 0.
int

\end{fulllineitems}

\index{units\_number\_ub (tamos.production.GasBoiler property)@\spxentry{units\_number\_ub}\spxextra{tamos.production.GasBoiler property}}

\begin{fulllineitems}
\phantomsection\label{\detokenize{generated/tamos.production.GasBoiler:tamos.production.GasBoiler.units_number_ub}}
\pysigstartsignatures
\pysigline{\sphinxbfcode{\sphinxupquote{property\DUrole{w}{  }}}\sphinxbfcode{\sphinxupquote{units\_number\_ub}}}
\pysigstopsignatures
\sphinxAtStartPar
The upper bound of the number of real components that this instance aims to stand for.
Setting \sphinxtitleref{units\_number\_ub} has a meaning if “LB max output power (kW)” property is different from 0.
int

\end{fulllineitems}

\index{used\_elements (tamos.production.GasBoiler property)@\spxentry{used\_elements}\spxextra{tamos.production.GasBoiler property}}

\begin{fulllineitems}
\phantomsection\label{\detokenize{generated/tamos.production.GasBoiler:tamos.production.GasBoiler.used_elements}}
\pysigstartsignatures
\pysigline{\sphinxbfcode{\sphinxupquote{property\DUrole{w}{  }}}\sphinxbfcode{\sphinxupquote{used\_elements}}}
\pysigstopsignatures
\sphinxAtStartPar
Elements used by the component.

\end{fulllineitems}


\end{fulllineitems}


\sphinxstepscope


\section{tamos.production.HeatExchanger}
\label{\detokenize{generated/tamos.production.HeatExchanger:tamos-production-heatexchanger}}\label{\detokenize{generated/tamos.production.HeatExchanger::doc}}\index{HeatExchanger (class in tamos.production)@\spxentry{HeatExchanger}\spxextra{class in tamos.production}}

\begin{fulllineitems}
\phantomsection\label{\detokenize{generated/tamos.production.HeatExchanger:tamos.production.HeatExchanger}}
\pysigstartsignatures
\pysiglinewithargsret{\sphinxbfcode{\sphinxupquote{class\DUrole{w}{  }}}\sphinxcode{\sphinxupquote{tamos.production.}}\sphinxbfcode{\sphinxupquote{HeatExchanger}}}{\emph{\DUrole{n}{energy\_source}}, \emph{\DUrole{n}{energy\_sink}}, \emph{\DUrole{n}{properties}}, \emph{\DUrole{n}{given\_sizing}\DUrole{o}{=}\DUrole{default_value}{None}}, \emph{\DUrole{n}{efficiency}\DUrole{o}{=}\DUrole{default_value}{0.95}}, \emph{\DUrole{n}{name}\DUrole{o}{=}\DUrole{default_value}{None}}, \emph{\DUrole{n}{units\_number\_ub}\DUrole{o}{=}\DUrole{default_value}{1}}, \emph{\DUrole{n}{units\_number\_lb}\DUrole{o}{=}\DUrole{default_value}{1}}, \emph{\DUrole{n}{eco\_count}\DUrole{o}{=}\DUrole{default_value}{True}}}{}
\pysigstopsignatures\index{\_\_init\_\_() (tamos.production.HeatExchanger method)@\spxentry{\_\_init\_\_()}\spxextra{tamos.production.HeatExchanger method}}

\begin{fulllineitems}
\phantomsection\label{\detokenize{generated/tamos.production.HeatExchanger:tamos.production.HeatExchanger.__init__}}
\pysigstartsignatures
\pysiglinewithargsret{\sphinxbfcode{\sphinxupquote{\_\_init\_\_}}}{\emph{\DUrole{n}{energy\_source}}, \emph{\DUrole{n}{energy\_sink}}, \emph{\DUrole{n}{properties}}, \emph{\DUrole{n}{given\_sizing}\DUrole{o}{=}\DUrole{default_value}{None}}, \emph{\DUrole{n}{efficiency}\DUrole{o}{=}\DUrole{default_value}{0.95}}, \emph{\DUrole{n}{name}\DUrole{o}{=}\DUrole{default_value}{None}}, \emph{\DUrole{n}{units\_number\_ub}\DUrole{o}{=}\DUrole{default_value}{1}}, \emph{\DUrole{n}{units\_number\_lb}\DUrole{o}{=}\DUrole{default_value}{1}}, \emph{\DUrole{n}{eco\_count}\DUrole{o}{=}\DUrole{default_value}{True}}}{}
\pysigstopsignatures
\sphinxAtStartPar
HeatExchanger components convert heat in a passive way.

\sphinxAtStartPar
This component declares the following exported decision variables:
\begin{itemize}
\item {} 
\sphinxAtStartPar
X\_P, binary.
Whether the component is used by the hub.

\item {} 
\sphinxAtStartPar
SP\_P, continuous, in kW.
The maximum capacity of the component. Defines the investment costs.

\item {} 
\sphinxAtStartPar
For all t, for all element e, F\_P(e, t), continuous, in kW.
The power related to element e entering the component (i.e. leaving the hub interface).

\item {} 
\sphinxAtStartPar
For all t, Q\_P(t), continuous, in kW.
The reference power related to the component. Defines the variable cost.
This power is a lower bound of SP\_P.
There exists one element e such that Q\_P(t) = F\_P(e, t) or Q\_P(t) = \sphinxhyphen{} F\_P(e, t).
For this component, e is \sphinxtitleref{energy\_sink}.

\end{itemize}

\sphinxAtStartPar
This component declares the following KPIs:
\begin{itemize}
\item {} 
\sphinxAtStartPar
\sphinxtitleref{COST\_production}
In euros.
Contributes to the “Eco” objective function.

\end{itemize}
\begin{quote}\begin{description}
\sphinxlineitem{Parameters}\begin{itemize}
\item {} 
\sphinxAtStartPar
\sphinxstyleliteralstrong{\sphinxupquote{energy\_source}} (\sphinxstyleliteralemphasis{\sphinxupquote{ThermalVectorPair}}\sphinxstyleliteralemphasis{\sphinxupquote{, }}{\hyperref[\detokenize{generated/tamos.element.ThermalVector:tamos.element.ThermalVector}]{\sphinxcrossref{\sphinxstyleliteralemphasis{\sphinxupquote{ThermalVector}}}}}) \textendash{} Element that gives thermal energy.
Must be cooled down if ThermalVectorPair.

\item {} 
\sphinxAtStartPar
\sphinxstyleliteralstrong{\sphinxupquote{energy\_sink}} (\sphinxstyleliteralemphasis{\sphinxupquote{ThermalVectorPair}}\sphinxstyleliteralemphasis{\sphinxupquote{, }}{\hyperref[\detokenize{generated/tamos.element.ThermalVector:tamos.element.ThermalVector}]{\sphinxcrossref{\sphinxstyleliteralemphasis{\sphinxupquote{ThermalVector}}}}}) \textendash{} Element that receives thermal energy.
Must be warmed up if ThermalVectorPair.

\item {} 
\sphinxAtStartPar
\sphinxstyleliteralstrong{\sphinxupquote{properties}} (\sphinxstyleliteralemphasis{\sphinxupquote{dict \{str: int}}\sphinxstyleliteralemphasis{\sphinxupquote{ | }}\sphinxstyleliteralemphasis{\sphinxupquote{float\}}}) \textendash{} 
\sphinxAtStartPar
Techno\sphinxhyphen{}economic properties of the component.
The \sphinxtitleref{properties} attribute must include the following keys:
\begin{itemize}
\item {} 
\sphinxAtStartPar
”LB max output power (kW)”

\item {} 
\sphinxAtStartPar
”UB max output power (kW)”

\item {} 
\sphinxAtStartPar
”CAPEX (EUR/kW)”

\item {} 
\sphinxAtStartPar
”OPEX (\%CAPEX)”

\item {} 
\sphinxAtStartPar
”Variable OPEX (EUR/MWh)”

\end{itemize}


\item {} 
\sphinxAtStartPar
\sphinxstyleliteralstrong{\sphinxupquote{given\_sizing}} (\sphinxstyleliteralemphasis{\sphinxupquote{int}}\sphinxstyleliteralemphasis{\sphinxupquote{ or }}\sphinxstyleliteralemphasis{\sphinxupquote{float}}\sphinxstyleliteralemphasis{\sphinxupquote{, }}\sphinxstyleliteralemphasis{\sphinxupquote{optional}}) \textendash{} The maximum capacity of the component, in kW.
Relates to decision variable ‘SP\_P’.
If specified, only the operation of this component is performed by the MILP solver.
If let unknown, both sizing and operation are performed.

\item {} 
\sphinxAtStartPar
\sphinxstyleliteralstrong{\sphinxupquote{efficiency}} (\sphinxstyleliteralemphasis{\sphinxupquote{int}}\sphinxstyleliteralemphasis{\sphinxupquote{, }}\sphinxstyleliteralemphasis{\sphinxupquote{float}}\sphinxstyleliteralemphasis{\sphinxupquote{ or }}\sphinxstyleliteralemphasis{\sphinxupquote{numpy.ndarray}}\sphinxstyleliteralemphasis{\sphinxupquote{, }}\sphinxstyleliteralemphasis{\sphinxupquote{optional}}\sphinxstyleliteralemphasis{\sphinxupquote{, }}\sphinxstyleliteralemphasis{\sphinxupquote{default 0.95}}) \textendash{} The amount of energy received by \sphinxtitleref{energy\_sink} for a unit of energy from \sphinxtitleref{energy\_source}.

\item {} 
\sphinxAtStartPar
\sphinxstyleliteralstrong{\sphinxupquote{name}} (\sphinxstyleliteralemphasis{\sphinxupquote{str}}\sphinxstyleliteralemphasis{\sphinxupquote{, }}\sphinxstyleliteralemphasis{\sphinxupquote{optional}}) \textendash{} 

\item {} 
\sphinxAtStartPar
\sphinxstyleliteralstrong{\sphinxupquote{units\_number\_lb}} (\sphinxstyleliteralemphasis{\sphinxupquote{int}}\sphinxstyleliteralemphasis{\sphinxupquote{, }}\sphinxstyleliteralemphasis{\sphinxupquote{optional}}\sphinxstyleliteralemphasis{\sphinxupquote{, }}\sphinxstyleliteralemphasis{\sphinxupquote{default 1}}) \textendash{} The lower bound (upper bound) of the number of real components that this instance aims to stand for.
Setting \sphinxtitleref{units\_number\_lb} (\sphinxtitleref{units\_number\_ub}) has a meaning if “LB max output power (kW)” property is
different from 0.

\item {} 
\sphinxAtStartPar
\sphinxstyleliteralstrong{\sphinxupquote{units\_number\_ub}} (\sphinxstyleliteralemphasis{\sphinxupquote{int}}\sphinxstyleliteralemphasis{\sphinxupquote{, }}\sphinxstyleliteralemphasis{\sphinxupquote{optional}}\sphinxstyleliteralemphasis{\sphinxupquote{, }}\sphinxstyleliteralemphasis{\sphinxupquote{default 1}}) \textendash{} The lower bound (upper bound) of the number of real components that this instance aims to stand for.
Setting \sphinxtitleref{units\_number\_lb} (\sphinxtitleref{units\_number\_ub}) has a meaning if “LB max output power (kW)” property is
different from 0.

\item {} 
\sphinxAtStartPar
\sphinxstyleliteralstrong{\sphinxupquote{eco\_count}} (\sphinxstyleliteralemphasis{\sphinxupquote{bool}}\sphinxstyleliteralemphasis{\sphinxupquote{, }}\sphinxstyleliteralemphasis{\sphinxupquote{optional}}\sphinxstyleliteralemphasis{\sphinxupquote{, }}\sphinxstyleliteralemphasis{\sphinxupquote{default True}}) \textendash{} Whether this instance contributes to the system “Eco” KPI.

\end{itemize}

\end{description}\end{quote}

\end{fulllineitems}

\subsubsection*{Methods}


\begin{savenotes}\sphinxattablestart
\centering
\begin{tabulary}{\linewidth}[t]{\X{1}{2}\X{1}{2}}
\hline

\sphinxAtStartPar
{\hyperref[\detokenize{generated/tamos.production.HeatExchanger:tamos.production.HeatExchanger.__init__}]{\sphinxcrossref{\sphinxcode{\sphinxupquote{\_\_init\_\_}}}}}(energy\_source, energy\_sink, properties)
&
\sphinxAtStartPar
HeatExchanger components convert heat in a passive way.
\\
\hline
\sphinxAtStartPar
{\hyperref[\detokenize{generated/tamos.production.HeatExchanger:tamos.production.HeatExchanger.compute_actualized_cost}]{\sphinxcrossref{\sphinxcode{\sphinxupquote{compute\_actualized\_cost}}}}}(CAPEX, OPEX, ...{[}, ...{]})
&
\sphinxAtStartPar
Computes the cost of a component using its \textquotesingle{}Lifetime\textquotesingle{} and \textquotesingle{}Discount rate (\%)\textquotesingle{} properties.
\\
\hline
\end{tabulary}
\par
\sphinxattableend\end{savenotes}
\subsubsection*{Attributes}


\begin{savenotes}\sphinxattablestart
\centering
\begin{tabulary}{\linewidth}[t]{\X{1}{2}\X{1}{2}}
\hline

\sphinxAtStartPar
{\hyperref[\detokenize{generated/tamos.production.HeatExchanger:tamos.production.HeatExchanger.eco_count}]{\sphinxcrossref{\sphinxcode{\sphinxupquote{eco\_count}}}}}
&
\sphinxAtStartPar
Whether this instance contributes to the system "Eco" KPI.
\\
\hline
\sphinxAtStartPar
{\hyperref[\detokenize{generated/tamos.production.HeatExchanger:tamos.production.HeatExchanger.efficiency}]{\sphinxcrossref{\sphinxcode{\sphinxupquote{efficiency}}}}}
&
\sphinxAtStartPar
The amount of energy received by \sphinxtitleref{energy\_sink} for a unit of energy from \sphinxtitleref{energy\_source}.
\\
\hline
\sphinxAtStartPar
{\hyperref[\detokenize{generated/tamos.production.HeatExchanger:tamos.production.HeatExchanger.energy_sink}]{\sphinxcrossref{\sphinxcode{\sphinxupquote{energy\_sink}}}}}
&
\sphinxAtStartPar
Element that receives thermal energy.
\\
\hline
\sphinxAtStartPar
{\hyperref[\detokenize{generated/tamos.production.HeatExchanger:tamos.production.HeatExchanger.energy_source}]{\sphinxcrossref{\sphinxcode{\sphinxupquote{energy\_source}}}}}
&
\sphinxAtStartPar
Element that gives thermal energy.
\\
\hline
\sphinxAtStartPar
{\hyperref[\detokenize{generated/tamos.production.HeatExchanger:tamos.production.HeatExchanger.given_sizing}]{\sphinxcrossref{\sphinxcode{\sphinxupquote{given\_sizing}}}}}
&
\sphinxAtStartPar
The maximum capacity of the component, in kW.
\\
\hline
\sphinxAtStartPar
{\hyperref[\detokenize{generated/tamos.production.HeatExchanger:tamos.production.HeatExchanger.name}]{\sphinxcrossref{\sphinxcode{\sphinxupquote{name}}}}}
&
\sphinxAtStartPar
str.
\\
\hline
\sphinxAtStartPar
{\hyperref[\detokenize{generated/tamos.production.HeatExchanger:tamos.production.HeatExchanger.units_number_lb}]{\sphinxcrossref{\sphinxcode{\sphinxupquote{units\_number\_lb}}}}}
&
\sphinxAtStartPar
The lower bound of the number of real components that this instance aims to stand for.
\\
\hline
\sphinxAtStartPar
{\hyperref[\detokenize{generated/tamos.production.HeatExchanger:tamos.production.HeatExchanger.units_number_ub}]{\sphinxcrossref{\sphinxcode{\sphinxupquote{units\_number\_ub}}}}}
&
\sphinxAtStartPar
The upper bound of the number of real components that this instance aims to stand for.
\\
\hline
\sphinxAtStartPar
{\hyperref[\detokenize{generated/tamos.production.HeatExchanger:tamos.production.HeatExchanger.used_elements}]{\sphinxcrossref{\sphinxcode{\sphinxupquote{used\_elements}}}}}
&
\sphinxAtStartPar
Elements used by the component.
\\
\hline
\end{tabulary}
\par
\sphinxattableend\end{savenotes}
\index{compute\_actualized\_cost() (tamos.production.HeatExchanger method)@\spxentry{compute\_actualized\_cost()}\spxextra{tamos.production.HeatExchanger method}}

\begin{fulllineitems}
\phantomsection\label{\detokenize{generated/tamos.production.HeatExchanger:tamos.production.HeatExchanger.compute_actualized_cost}}
\pysigstartsignatures
\pysiglinewithargsret{\sphinxbfcode{\sphinxupquote{compute\_actualized\_cost}}}{\emph{\DUrole{n}{CAPEX}}, \emph{\DUrole{n}{OPEX}}, \emph{\DUrole{n}{system\_lifetime}}, \emph{\DUrole{n}{lifetime}\DUrole{o}{=}\DUrole{default_value}{None}}, \emph{\DUrole{n}{discount\_rate}\DUrole{o}{=}\DUrole{default_value}{None}}}{}
\pysigstopsignatures
\sphinxAtStartPar
Computes the cost of a component using its ‘Lifetime’ and ‘Discount rate (\%)’ properties.
\begin{quote}\begin{description}
\sphinxlineitem{Parameters}\begin{itemize}
\item {} 
\sphinxAtStartPar
\sphinxstyleliteralstrong{\sphinxupquote{CAPEX}} (\sphinxstyleliteralemphasis{\sphinxupquote{float}}) \textendash{} Capital Expenditure. Cost in euros paid every \sphinxtitleref{technical\_lifetime} periods.

\item {} 
\sphinxAtStartPar
\sphinxstyleliteralstrong{\sphinxupquote{OPEX}} (\sphinxstyleliteralemphasis{\sphinxupquote{float}}) \textendash{} Operational Expenditure. Cost in euros paid each period.

\item {} 
\sphinxAtStartPar
\sphinxstyleliteralstrong{\sphinxupquote{system\_lifetime}} (\sphinxstyleliteralemphasis{\sphinxupquote{int}}) \textendash{} Number of periods defining the existence of the energy system.

\item {} 
\sphinxAtStartPar
\sphinxstyleliteralstrong{\sphinxupquote{lifetime}} (\sphinxstyleliteralemphasis{\sphinxupquote{int}}\sphinxstyleliteralemphasis{\sphinxupquote{, }}\sphinxstyleliteralemphasis{\sphinxupquote{optional}}) \textendash{} Number of periods defining the existence of the component.
If specified, overwrite the “Lifetime” property.

\item {} 
\sphinxAtStartPar
\sphinxstyleliteralstrong{\sphinxupquote{discount\_rate}} (\sphinxstyleliteralemphasis{\sphinxupquote{float}}) \textendash{} In percent (\%). Describes the importance of the economic amortization process, per period.
If specified, overwrite the “Discount rate (\%)” property.

\end{itemize}

\sphinxlineitem{Returns}
\sphinxAtStartPar
\begin{itemize}
\item {} 
\sphinxAtStartPar
\sphinxstyleemphasis{A 3\sphinxhyphen{}tuple (total\_cost, CAPEX\_share, OPEX\_share) where}

\item {} 
\sphinxAtStartPar
* CAPEX\_share is the share of total cost related to \sphinxtitleref{CAPEX}

\item {} 
\sphinxAtStartPar
* OPEX\_share is the share of total cost related to \sphinxtitleref{OPEX}

\item {} 
\sphinxAtStartPar
\sphinxstyleemphasis{* total\_cost = CAPEX\_share + OPEX\_share}

\end{itemize}


\end{description}\end{quote}
\subsubsection*{Notes}

\sphinxAtStartPar
Takes into account residual value of component in the case \sphinxtitleref{system\_lifetime} is not a multiple of \sphinxtitleref{lifetime}.
In this case, the last replacement occuring at period replacement\_period is paid in proportion of ‘CAPEX’
depending linearly on the number of periods left:
CAPEX * (system\_lifetime \sphinxhyphen{} replacement\_period) / lifetime

\end{fulllineitems}

\index{eco\_count (tamos.production.HeatExchanger property)@\spxentry{eco\_count}\spxextra{tamos.production.HeatExchanger property}}

\begin{fulllineitems}
\phantomsection\label{\detokenize{generated/tamos.production.HeatExchanger:tamos.production.HeatExchanger.eco_count}}
\pysigstartsignatures
\pysigline{\sphinxbfcode{\sphinxupquote{property\DUrole{w}{  }}}\sphinxbfcode{\sphinxupquote{eco\_count}}}
\pysigstopsignatures
\sphinxAtStartPar
Whether this instance contributes to the system “Eco” KPI.
bool

\end{fulllineitems}

\index{efficiency (tamos.production.HeatExchanger property)@\spxentry{efficiency}\spxextra{tamos.production.HeatExchanger property}}

\begin{fulllineitems}
\phantomsection\label{\detokenize{generated/tamos.production.HeatExchanger:tamos.production.HeatExchanger.efficiency}}
\pysigstartsignatures
\pysigline{\sphinxbfcode{\sphinxupquote{property\DUrole{w}{  }}}\sphinxbfcode{\sphinxupquote{efficiency}}}
\pysigstopsignatures
\sphinxAtStartPar
The amount of energy received by \sphinxtitleref{energy\_sink} for a unit of energy from \sphinxtitleref{energy\_source}.
int, float or numpy.ndarray

\end{fulllineitems}

\index{energy\_sink (tamos.production.HeatExchanger property)@\spxentry{energy\_sink}\spxextra{tamos.production.HeatExchanger property}}

\begin{fulllineitems}
\phantomsection\label{\detokenize{generated/tamos.production.HeatExchanger:tamos.production.HeatExchanger.energy_sink}}
\pysigstartsignatures
\pysigline{\sphinxbfcode{\sphinxupquote{property\DUrole{w}{  }}}\sphinxbfcode{\sphinxupquote{energy\_sink}}}
\pysigstopsignatures
\sphinxAtStartPar
Element that receives thermal energy.
Must be warmed up if ThermalVectorPair.
ThermalVectorPair, ThermalVector

\end{fulllineitems}

\index{energy\_source (tamos.production.HeatExchanger property)@\spxentry{energy\_source}\spxextra{tamos.production.HeatExchanger property}}

\begin{fulllineitems}
\phantomsection\label{\detokenize{generated/tamos.production.HeatExchanger:tamos.production.HeatExchanger.energy_source}}
\pysigstartsignatures
\pysigline{\sphinxbfcode{\sphinxupquote{property\DUrole{w}{  }}}\sphinxbfcode{\sphinxupquote{energy\_source}}}
\pysigstopsignatures
\sphinxAtStartPar
Element that gives thermal energy.
Must be cooled down if ThermalVectorPair.
ThermalVectorPair, ThermalVector

\end{fulllineitems}

\index{given\_sizing (tamos.production.HeatExchanger property)@\spxentry{given\_sizing}\spxextra{tamos.production.HeatExchanger property}}

\begin{fulllineitems}
\phantomsection\label{\detokenize{generated/tamos.production.HeatExchanger:tamos.production.HeatExchanger.given_sizing}}
\pysigstartsignatures
\pysigline{\sphinxbfcode{\sphinxupquote{property\DUrole{w}{  }}}\sphinxbfcode{\sphinxupquote{given\_sizing}}}
\pysigstopsignatures
\sphinxAtStartPar
The maximum capacity of the component, in kW.
Relates to decision variable ‘SP\_P’.
int or float

\end{fulllineitems}

\index{name (tamos.production.HeatExchanger property)@\spxentry{name}\spxextra{tamos.production.HeatExchanger property}}

\begin{fulllineitems}
\phantomsection\label{\detokenize{generated/tamos.production.HeatExchanger:tamos.production.HeatExchanger.name}}
\pysigstartsignatures
\pysigline{\sphinxbfcode{\sphinxupquote{property\DUrole{w}{  }}}\sphinxbfcode{\sphinxupquote{name}}}
\pysigstopsignatures
\sphinxAtStartPar
str.
This name is used in MILP model description.
names must not exceed 255 characters,
all of which must be alphanumeric (a\sphinxhyphen{}z, A\sphinxhyphen{}Z, 0\sphinxhyphen{}9) or one of these symbols:
! ” \# \$ \% \& , . ; ? @ \_ ‘ ’ \{ \} \textasciitilde{}.
\begin{quote}\begin{description}
\sphinxlineitem{Type}
\sphinxAtStartPar
Name of the instance

\end{description}\end{quote}

\end{fulllineitems}

\index{units\_number\_lb (tamos.production.HeatExchanger property)@\spxentry{units\_number\_lb}\spxextra{tamos.production.HeatExchanger property}}

\begin{fulllineitems}
\phantomsection\label{\detokenize{generated/tamos.production.HeatExchanger:tamos.production.HeatExchanger.units_number_lb}}
\pysigstartsignatures
\pysigline{\sphinxbfcode{\sphinxupquote{property\DUrole{w}{  }}}\sphinxbfcode{\sphinxupquote{units\_number\_lb}}}
\pysigstopsignatures
\sphinxAtStartPar
The lower bound of the number of real components that this instance aims to stand for.
Setting \sphinxtitleref{units\_number\_lb} has a meaning if “LB max output power (kW)” property is different from 0.
int

\end{fulllineitems}

\index{units\_number\_ub (tamos.production.HeatExchanger property)@\spxentry{units\_number\_ub}\spxextra{tamos.production.HeatExchanger property}}

\begin{fulllineitems}
\phantomsection\label{\detokenize{generated/tamos.production.HeatExchanger:tamos.production.HeatExchanger.units_number_ub}}
\pysigstartsignatures
\pysigline{\sphinxbfcode{\sphinxupquote{property\DUrole{w}{  }}}\sphinxbfcode{\sphinxupquote{units\_number\_ub}}}
\pysigstopsignatures
\sphinxAtStartPar
The upper bound of the number of real components that this instance aims to stand for.
Setting \sphinxtitleref{units\_number\_ub} has a meaning if “LB max output power (kW)” property is different from 0.
int

\end{fulllineitems}

\index{used\_elements (tamos.production.HeatExchanger property)@\spxentry{used\_elements}\spxextra{tamos.production.HeatExchanger property}}

\begin{fulllineitems}
\phantomsection\label{\detokenize{generated/tamos.production.HeatExchanger:tamos.production.HeatExchanger.used_elements}}
\pysigstartsignatures
\pysigline{\sphinxbfcode{\sphinxupquote{property\DUrole{w}{  }}}\sphinxbfcode{\sphinxupquote{used\_elements}}}
\pysigstopsignatures
\sphinxAtStartPar
Elements used by the component.

\end{fulllineitems}


\end{fulllineitems}


\sphinxstepscope


\section{tamos.production.Pump}
\label{\detokenize{generated/tamos.production.Pump:tamos-production-pump}}\label{\detokenize{generated/tamos.production.Pump::doc}}\index{Pump (class in tamos.production)@\spxentry{Pump}\spxextra{class in tamos.production}}

\begin{fulllineitems}
\phantomsection\label{\detokenize{generated/tamos.production.Pump:tamos.production.Pump}}
\pysigstartsignatures
\pysiglinewithargsret{\sphinxbfcode{\sphinxupquote{class\DUrole{w}{  }}}\sphinxcode{\sphinxupquote{tamos.production.}}\sphinxbfcode{\sphinxupquote{Pump}}}{\emph{\DUrole{n}{element\_1}}, \emph{\DUrole{n}{element\_2}}, \emph{\DUrole{n}{energy\_drive}}, \emph{\DUrole{n}{properties}}, \emph{\DUrole{n}{direction}\DUrole{o}{=}\DUrole{default_value}{\textquotesingle{}both\textquotesingle{}}}, \emph{\DUrole{n}{pump\_consumption}\DUrole{o}{=}\DUrole{default_value}{0.005}}, \emph{\DUrole{n}{given\_sizing}\DUrole{o}{=}\DUrole{default_value}{None}}, \emph{\DUrole{n}{name}\DUrole{o}{=}\DUrole{default_value}{None}}, \emph{\DUrole{n}{units\_number\_ub}\DUrole{o}{=}\DUrole{default_value}{1}}, \emph{\DUrole{n}{units\_number\_lb}\DUrole{o}{=}\DUrole{default_value}{1}}, \emph{\DUrole{n}{eco\_count}\DUrole{o}{=}\DUrole{default_value}{True}}}{}
\pysigstopsignatures\index{\_\_init\_\_() (tamos.production.Pump method)@\spxentry{\_\_init\_\_()}\spxextra{tamos.production.Pump method}}

\begin{fulllineitems}
\phantomsection\label{\detokenize{generated/tamos.production.Pump:tamos.production.Pump.__init__}}
\pysigstartsignatures
\pysiglinewithargsret{\sphinxbfcode{\sphinxupquote{\_\_init\_\_}}}{\emph{\DUrole{n}{element\_1}}, \emph{\DUrole{n}{element\_2}}, \emph{\DUrole{n}{energy\_drive}}, \emph{\DUrole{n}{properties}}, \emph{\DUrole{n}{direction}\DUrole{o}{=}\DUrole{default_value}{\textquotesingle{}both\textquotesingle{}}}, \emph{\DUrole{n}{pump\_consumption}\DUrole{o}{=}\DUrole{default_value}{0.005}}, \emph{\DUrole{n}{given\_sizing}\DUrole{o}{=}\DUrole{default_value}{None}}, \emph{\DUrole{n}{name}\DUrole{o}{=}\DUrole{default_value}{None}}, \emph{\DUrole{n}{units\_number\_ub}\DUrole{o}{=}\DUrole{default_value}{1}}, \emph{\DUrole{n}{units\_number\_lb}\DUrole{o}{=}\DUrole{default_value}{1}}, \emph{\DUrole{n}{eco\_count}\DUrole{o}{=}\DUrole{default_value}{True}}}{}
\pysigstopsignatures
\sphinxAtStartPar
Pump components convert a ThermalVectorPair into another ThermalVectorPair.
Electricity is consumed in proportion of the pumped power.

\sphinxAtStartPar
This component declares the following exported decision variables:
\begin{itemize}
\item {} 
\sphinxAtStartPar
X\_P, binary.
Whether the component is used by the hub.

\item {} 
\sphinxAtStartPar
SP\_P, continuous, in kW.
The maximum capacity of the component. Defines the investment costs.

\item {} 
\sphinxAtStartPar
For all t, for all element e, F\_P(e, t), continuous, in kW.
The power related to element e entering the component (i.e. leaving the hub interface).

\item {} 
\sphinxAtStartPar
For all t, Q\_P(t), continuous, in kW.
The reference power related to the component. Defines the variable cost.
This power is a lower bound of SP\_P.
There exists one element e such that Q\_P(t) = F\_P(e, t) or Q\_P(t) = \sphinxhyphen{} F\_P(e, t).
For this component, e is \sphinxtitleref{energy\_drive}.

\end{itemize}

\sphinxAtStartPar
This component declares the following KPIs:
\begin{itemize}
\item {} 
\sphinxAtStartPar
\sphinxtitleref{COST\_production}
In euros.
Contributes to the “Eco” objective function.

\end{itemize}
\begin{quote}\begin{description}
\sphinxlineitem{Parameters}\begin{itemize}
\item {} 
\sphinxAtStartPar
\sphinxstyleliteralstrong{\sphinxupquote{element\_1}} (\sphinxstyleliteralemphasis{\sphinxupquote{ThermalVectorPair}}) \textendash{} 

\item {} 
\sphinxAtStartPar
\sphinxstyleliteralstrong{\sphinxupquote{element\_2}} (\sphinxstyleliteralemphasis{\sphinxupquote{ThermalVectorPair}}) \textendash{} 

\item {} 
\sphinxAtStartPar
\sphinxstyleliteralstrong{\sphinxupquote{energy\_drive}} ({\hyperref[\detokenize{generated/tamos.element.ElectricityVector:tamos.element.ElectricityVector}]{\sphinxcrossref{\sphinxstyleliteralemphasis{\sphinxupquote{ElectricityVector}}}}}) \textendash{} 

\item {} 
\sphinxAtStartPar
\sphinxstyleliteralstrong{\sphinxupquote{properties}} (\sphinxstyleliteralemphasis{\sphinxupquote{dict \{str: int}}\sphinxstyleliteralemphasis{\sphinxupquote{ | }}\sphinxstyleliteralemphasis{\sphinxupquote{float\}}}) \textendash{} 
\sphinxAtStartPar
Techno\sphinxhyphen{}economic properties of the component.
The \sphinxtitleref{properties} attribute must include the following keys:
\begin{itemize}
\item {} 
\sphinxAtStartPar
”LB max output power (kW)”

\item {} 
\sphinxAtStartPar
”UB max output power (kW)”

\item {} 
\sphinxAtStartPar
”CAPEX (EUR/kW)”

\item {} 
\sphinxAtStartPar
”OPEX (\%CAPEX)”

\item {} 
\sphinxAtStartPar
”Variable OPEX (EUR/MWh)”

\end{itemize}


\item {} 
\sphinxAtStartPar
\sphinxstyleliteralstrong{\sphinxupquote{direction}} (\sphinxstyleliteralemphasis{\sphinxupquote{\{\textquotesingle{}produced\textquotesingle{}}}\sphinxstyleliteralemphasis{\sphinxupquote{, }}\sphinxstyleliteralemphasis{\sphinxupquote{\textquotesingle{}consumed\textquotesingle{}}}\sphinxstyleliteralemphasis{\sphinxupquote{, }}\sphinxstyleliteralemphasis{\sphinxupquote{\textquotesingle{}both\textquotesingle{}\}}}\sphinxstyleliteralemphasis{\sphinxupquote{, }}\sphinxstyleliteralemphasis{\sphinxupquote{optional}}\sphinxstyleliteralemphasis{\sphinxupquote{, }}\sphinxstyleliteralemphasis{\sphinxupquote{default \textquotesingle{}both\textquotesingle{}}}) \textendash{} 
\sphinxAtStartPar
Related to \sphinxtitleref{element\_1}.
\begin{itemize}
\item {} 
\sphinxAtStartPar
’produced’: a flow of \sphinxtitleref{element\_1} is produced, a flow of \sphinxtitleref{element\_2} is consumed

\item {} 
\sphinxAtStartPar
’consumed’: a flow of \sphinxtitleref{element\_2} is produced, a flow of \sphinxtitleref{element\_1} is consumed

\item {} \begin{description}
\sphinxlineitem{’both’: depending on the time step t, a flow of \sphinxtitleref{element\_1} is produced or consumed}
\sphinxAtStartPar
In this mode, decision variable ‘Q\_P’ has no upper bound.
In ‘Eco’ optimization, the component cost constrains \sphinxtitleref{Q\_P}.

\end{description}

\end{itemize}


\item {} 
\sphinxAtStartPar
\sphinxstyleliteralstrong{\sphinxupquote{pump\_consumption}} (\sphinxstyleliteralemphasis{\sphinxupquote{int}}\sphinxstyleliteralemphasis{\sphinxupquote{, }}\sphinxstyleliteralemphasis{\sphinxupquote{float}}\sphinxstyleliteralemphasis{\sphinxupquote{ or }}\sphinxstyleliteralemphasis{\sphinxupquote{numpy.ndarray}}\sphinxstyleliteralemphasis{\sphinxupquote{, }}\sphinxstyleliteralemphasis{\sphinxupquote{optional}}\sphinxstyleliteralemphasis{\sphinxupquote{, }}\sphinxstyleliteralemphasis{\sphinxupquote{default 0.005}}) \textendash{} Instantaneous electrical consumption of the component in proportion of the pumped power.

\item {} 
\sphinxAtStartPar
\sphinxstyleliteralstrong{\sphinxupquote{name}} (\sphinxstyleliteralemphasis{\sphinxupquote{str}}\sphinxstyleliteralemphasis{\sphinxupquote{, }}\sphinxstyleliteralemphasis{\sphinxupquote{optional}}) \textendash{} 

\end{itemize}

\end{description}\end{quote}

\end{fulllineitems}

\subsubsection*{Methods}


\begin{savenotes}\sphinxattablestart
\centering
\begin{tabulary}{\linewidth}[t]{\X{1}{2}\X{1}{2}}
\hline

\sphinxAtStartPar
{\hyperref[\detokenize{generated/tamos.production.Pump:tamos.production.Pump.__init__}]{\sphinxcrossref{\sphinxcode{\sphinxupquote{\_\_init\_\_}}}}}(element\_1, element\_2, energy\_drive, ...)
&
\sphinxAtStartPar
Pump components convert a ThermalVectorPair into another ThermalVectorPair.
\\
\hline
\sphinxAtStartPar
{\hyperref[\detokenize{generated/tamos.production.Pump:tamos.production.Pump.compute_actualized_cost}]{\sphinxcrossref{\sphinxcode{\sphinxupquote{compute\_actualized\_cost}}}}}(CAPEX, OPEX, ...{[}, ...{]})
&
\sphinxAtStartPar
Computes the cost of a component using its \textquotesingle{}Lifetime\textquotesingle{} and \textquotesingle{}Discount rate (\%)\textquotesingle{} properties.
\\
\hline
\end{tabulary}
\par
\sphinxattableend\end{savenotes}
\subsubsection*{Attributes}


\begin{savenotes}\sphinxattablestart
\centering
\begin{tabulary}{\linewidth}[t]{\X{1}{2}\X{1}{2}}
\hline

\sphinxAtStartPar
{\hyperref[\detokenize{generated/tamos.production.Pump:tamos.production.Pump.direction}]{\sphinxcrossref{\sphinxcode{\sphinxupquote{direction}}}}}
&
\sphinxAtStartPar
Related to \sphinxtitleref{element\_1}.
\\
\hline
\sphinxAtStartPar
{\hyperref[\detokenize{generated/tamos.production.Pump:tamos.production.Pump.eco_count}]{\sphinxcrossref{\sphinxcode{\sphinxupquote{eco\_count}}}}}
&
\sphinxAtStartPar
Whether this instance contributes to the system "Eco" KPI.
\\
\hline
\sphinxAtStartPar
{\hyperref[\detokenize{generated/tamos.production.Pump:tamos.production.Pump.element_1}]{\sphinxcrossref{\sphinxcode{\sphinxupquote{element\_1}}}}}
&
\sphinxAtStartPar
ThermalVectorPair
\\
\hline
\sphinxAtStartPar
{\hyperref[\detokenize{generated/tamos.production.Pump:tamos.production.Pump.element_2}]{\sphinxcrossref{\sphinxcode{\sphinxupquote{element\_2}}}}}
&
\sphinxAtStartPar
ThermalVectorPair
\\
\hline
\sphinxAtStartPar
{\hyperref[\detokenize{generated/tamos.production.Pump:tamos.production.Pump.energy_drive}]{\sphinxcrossref{\sphinxcode{\sphinxupquote{energy\_drive}}}}}
&
\sphinxAtStartPar
ElectricityVector
\\
\hline
\sphinxAtStartPar
{\hyperref[\detokenize{generated/tamos.production.Pump:tamos.production.Pump.given_sizing}]{\sphinxcrossref{\sphinxcode{\sphinxupquote{given\_sizing}}}}}
&
\sphinxAtStartPar
The maximum capacity of the component, in kW.
\\
\hline
\sphinxAtStartPar
{\hyperref[\detokenize{generated/tamos.production.Pump:tamos.production.Pump.name}]{\sphinxcrossref{\sphinxcode{\sphinxupquote{name}}}}}
&
\sphinxAtStartPar
str.
\\
\hline
\sphinxAtStartPar
\sphinxcode{\sphinxupquote{pump\_consumption}}
&
\sphinxAtStartPar

\\
\hline
\sphinxAtStartPar
{\hyperref[\detokenize{generated/tamos.production.Pump:tamos.production.Pump.units_number_lb}]{\sphinxcrossref{\sphinxcode{\sphinxupquote{units\_number\_lb}}}}}
&
\sphinxAtStartPar
The lower bound of the number of real components that this instance aims to stand for.
\\
\hline
\sphinxAtStartPar
{\hyperref[\detokenize{generated/tamos.production.Pump:tamos.production.Pump.units_number_ub}]{\sphinxcrossref{\sphinxcode{\sphinxupquote{units\_number\_ub}}}}}
&
\sphinxAtStartPar
The upper bound of the number of real components that this instance aims to stand for.
\\
\hline
\sphinxAtStartPar
{\hyperref[\detokenize{generated/tamos.production.Pump:tamos.production.Pump.used_elements}]{\sphinxcrossref{\sphinxcode{\sphinxupquote{used\_elements}}}}}
&
\sphinxAtStartPar
Elements used by the component.
\\
\hline
\end{tabulary}
\par
\sphinxattableend\end{savenotes}
\index{compute\_actualized\_cost() (tamos.production.Pump method)@\spxentry{compute\_actualized\_cost()}\spxextra{tamos.production.Pump method}}

\begin{fulllineitems}
\phantomsection\label{\detokenize{generated/tamos.production.Pump:tamos.production.Pump.compute_actualized_cost}}
\pysigstartsignatures
\pysiglinewithargsret{\sphinxbfcode{\sphinxupquote{compute\_actualized\_cost}}}{\emph{\DUrole{n}{CAPEX}}, \emph{\DUrole{n}{OPEX}}, \emph{\DUrole{n}{system\_lifetime}}, \emph{\DUrole{n}{lifetime}\DUrole{o}{=}\DUrole{default_value}{None}}, \emph{\DUrole{n}{discount\_rate}\DUrole{o}{=}\DUrole{default_value}{None}}}{}
\pysigstopsignatures
\sphinxAtStartPar
Computes the cost of a component using its ‘Lifetime’ and ‘Discount rate (\%)’ properties.
\begin{quote}\begin{description}
\sphinxlineitem{Parameters}\begin{itemize}
\item {} 
\sphinxAtStartPar
\sphinxstyleliteralstrong{\sphinxupquote{CAPEX}} (\sphinxstyleliteralemphasis{\sphinxupquote{float}}) \textendash{} Capital Expenditure. Cost in euros paid every \sphinxtitleref{technical\_lifetime} periods.

\item {} 
\sphinxAtStartPar
\sphinxstyleliteralstrong{\sphinxupquote{OPEX}} (\sphinxstyleliteralemphasis{\sphinxupquote{float}}) \textendash{} Operational Expenditure. Cost in euros paid each period.

\item {} 
\sphinxAtStartPar
\sphinxstyleliteralstrong{\sphinxupquote{system\_lifetime}} (\sphinxstyleliteralemphasis{\sphinxupquote{int}}) \textendash{} Number of periods defining the existence of the energy system.

\item {} 
\sphinxAtStartPar
\sphinxstyleliteralstrong{\sphinxupquote{lifetime}} (\sphinxstyleliteralemphasis{\sphinxupquote{int}}\sphinxstyleliteralemphasis{\sphinxupquote{, }}\sphinxstyleliteralemphasis{\sphinxupquote{optional}}) \textendash{} Number of periods defining the existence of the component.
If specified, overwrite the “Lifetime” property.

\item {} 
\sphinxAtStartPar
\sphinxstyleliteralstrong{\sphinxupquote{discount\_rate}} (\sphinxstyleliteralemphasis{\sphinxupquote{float}}) \textendash{} In percent (\%). Describes the importance of the economic amortization process, per period.
If specified, overwrite the “Discount rate (\%)” property.

\end{itemize}

\sphinxlineitem{Returns}
\sphinxAtStartPar
\begin{itemize}
\item {} 
\sphinxAtStartPar
\sphinxstyleemphasis{A 3\sphinxhyphen{}tuple (total\_cost, CAPEX\_share, OPEX\_share) where}

\item {} 
\sphinxAtStartPar
* CAPEX\_share is the share of total cost related to \sphinxtitleref{CAPEX}

\item {} 
\sphinxAtStartPar
* OPEX\_share is the share of total cost related to \sphinxtitleref{OPEX}

\item {} 
\sphinxAtStartPar
\sphinxstyleemphasis{* total\_cost = CAPEX\_share + OPEX\_share}

\end{itemize}


\end{description}\end{quote}
\subsubsection*{Notes}

\sphinxAtStartPar
Takes into account residual value of component in the case \sphinxtitleref{system\_lifetime} is not a multiple of \sphinxtitleref{lifetime}.
In this case, the last replacement occuring at period replacement\_period is paid in proportion of ‘CAPEX’
depending linearly on the number of periods left:
CAPEX * (system\_lifetime \sphinxhyphen{} replacement\_period) / lifetime

\end{fulllineitems}

\index{direction (tamos.production.Pump property)@\spxentry{direction}\spxextra{tamos.production.Pump property}}

\begin{fulllineitems}
\phantomsection\label{\detokenize{generated/tamos.production.Pump:tamos.production.Pump.direction}}
\pysigstartsignatures
\pysigline{\sphinxbfcode{\sphinxupquote{property\DUrole{w}{  }}}\sphinxbfcode{\sphinxupquote{direction}}}
\pysigstopsignatures
\sphinxAtStartPar
Related to \sphinxtitleref{element\_1}.
\begin{itemize}
\item {} 
\sphinxAtStartPar
‘produced’: a flow of \sphinxtitleref{element\_1} is produced, a flow of \sphinxtitleref{element\_2} is consumed

\item {} 
\sphinxAtStartPar
‘consumed’: a flow of \sphinxtitleref{element\_2} is produced, a flow of \sphinxtitleref{element\_1} is consumed

\item {} \begin{description}
\sphinxlineitem{‘both’: depending on the time step t, a flow of \sphinxtitleref{element\_1} is produced or consumed}
\sphinxAtStartPar
In this mode, decision variable ‘Q\_P’ has no upper bound.

\end{description}

\end{itemize}

\sphinxAtStartPar
\{‘produced’, ‘consumed’, ‘both’\}

\end{fulllineitems}

\index{eco\_count (tamos.production.Pump property)@\spxentry{eco\_count}\spxextra{tamos.production.Pump property}}

\begin{fulllineitems}
\phantomsection\label{\detokenize{generated/tamos.production.Pump:tamos.production.Pump.eco_count}}
\pysigstartsignatures
\pysigline{\sphinxbfcode{\sphinxupquote{property\DUrole{w}{  }}}\sphinxbfcode{\sphinxupquote{eco\_count}}}
\pysigstopsignatures
\sphinxAtStartPar
Whether this instance contributes to the system “Eco” KPI.
bool

\end{fulllineitems}

\index{element\_1 (tamos.production.Pump property)@\spxentry{element\_1}\spxextra{tamos.production.Pump property}}

\begin{fulllineitems}
\phantomsection\label{\detokenize{generated/tamos.production.Pump:tamos.production.Pump.element_1}}
\pysigstartsignatures
\pysigline{\sphinxbfcode{\sphinxupquote{property\DUrole{w}{  }}}\sphinxbfcode{\sphinxupquote{element\_1}}}
\pysigstopsignatures
\sphinxAtStartPar
ThermalVectorPair

\end{fulllineitems}

\index{element\_2 (tamos.production.Pump property)@\spxentry{element\_2}\spxextra{tamos.production.Pump property}}

\begin{fulllineitems}
\phantomsection\label{\detokenize{generated/tamos.production.Pump:tamos.production.Pump.element_2}}
\pysigstartsignatures
\pysigline{\sphinxbfcode{\sphinxupquote{property\DUrole{w}{  }}}\sphinxbfcode{\sphinxupquote{element\_2}}}
\pysigstopsignatures
\sphinxAtStartPar
ThermalVectorPair

\end{fulllineitems}

\index{energy\_drive (tamos.production.Pump property)@\spxentry{energy\_drive}\spxextra{tamos.production.Pump property}}

\begin{fulllineitems}
\phantomsection\label{\detokenize{generated/tamos.production.Pump:tamos.production.Pump.energy_drive}}
\pysigstartsignatures
\pysigline{\sphinxbfcode{\sphinxupquote{property\DUrole{w}{  }}}\sphinxbfcode{\sphinxupquote{energy\_drive}}}
\pysigstopsignatures
\sphinxAtStartPar
ElectricityVector

\end{fulllineitems}

\index{given\_sizing (tamos.production.Pump property)@\spxentry{given\_sizing}\spxextra{tamos.production.Pump property}}

\begin{fulllineitems}
\phantomsection\label{\detokenize{generated/tamos.production.Pump:tamos.production.Pump.given_sizing}}
\pysigstartsignatures
\pysigline{\sphinxbfcode{\sphinxupquote{property\DUrole{w}{  }}}\sphinxbfcode{\sphinxupquote{given\_sizing}}}
\pysigstopsignatures
\sphinxAtStartPar
The maximum capacity of the component, in kW.
Relates to decision variable ‘SP\_P’.
int or float

\end{fulllineitems}

\index{name (tamos.production.Pump property)@\spxentry{name}\spxextra{tamos.production.Pump property}}

\begin{fulllineitems}
\phantomsection\label{\detokenize{generated/tamos.production.Pump:tamos.production.Pump.name}}
\pysigstartsignatures
\pysigline{\sphinxbfcode{\sphinxupquote{property\DUrole{w}{  }}}\sphinxbfcode{\sphinxupquote{name}}}
\pysigstopsignatures
\sphinxAtStartPar
str.
This name is used in MILP model description.
names must not exceed 255 characters,
all of which must be alphanumeric (a\sphinxhyphen{}z, A\sphinxhyphen{}Z, 0\sphinxhyphen{}9) or one of these symbols:
! ” \# \$ \% \& , . ; ? @ \_ ‘ ’ \{ \} \textasciitilde{}.
\begin{quote}\begin{description}
\sphinxlineitem{Type}
\sphinxAtStartPar
Name of the instance

\end{description}\end{quote}

\end{fulllineitems}

\index{units\_number\_lb (tamos.production.Pump property)@\spxentry{units\_number\_lb}\spxextra{tamos.production.Pump property}}

\begin{fulllineitems}
\phantomsection\label{\detokenize{generated/tamos.production.Pump:tamos.production.Pump.units_number_lb}}
\pysigstartsignatures
\pysigline{\sphinxbfcode{\sphinxupquote{property\DUrole{w}{  }}}\sphinxbfcode{\sphinxupquote{units\_number\_lb}}}
\pysigstopsignatures
\sphinxAtStartPar
The lower bound of the number of real components that this instance aims to stand for.
Setting \sphinxtitleref{units\_number\_lb} has a meaning if “LB max output power (kW)” property is different from 0.
int

\end{fulllineitems}

\index{units\_number\_ub (tamos.production.Pump property)@\spxentry{units\_number\_ub}\spxextra{tamos.production.Pump property}}

\begin{fulllineitems}
\phantomsection\label{\detokenize{generated/tamos.production.Pump:tamos.production.Pump.units_number_ub}}
\pysigstartsignatures
\pysigline{\sphinxbfcode{\sphinxupquote{property\DUrole{w}{  }}}\sphinxbfcode{\sphinxupquote{units\_number\_ub}}}
\pysigstopsignatures
\sphinxAtStartPar
The upper bound of the number of real components that this instance aims to stand for.
Setting \sphinxtitleref{units\_number\_ub} has a meaning if “LB max output power (kW)” property is different from 0.
int

\end{fulllineitems}

\index{used\_elements (tamos.production.Pump property)@\spxentry{used\_elements}\spxextra{tamos.production.Pump property}}

\begin{fulllineitems}
\phantomsection\label{\detokenize{generated/tamos.production.Pump:tamos.production.Pump.used_elements}}
\pysigstartsignatures
\pysigline{\sphinxbfcode{\sphinxupquote{property\DUrole{w}{  }}}\sphinxbfcode{\sphinxupquote{used\_elements}}}
\pysigstopsignatures
\sphinxAtStartPar
Elements used by the component.

\end{fulllineitems}


\end{fulllineitems}


\sphinxstepscope


\chapter{Storage components}
\label{\detokenize{storage_components:storage-components}}\label{\detokenize{storage_components::doc}}

\begin{savenotes}\sphinxattablestart
\centering
\begin{tabulary}{\linewidth}[t]{\X{1}{2}\X{1}{2}}
\hline

\sphinxAtStartPar
{\hyperref[\detokenize{generated/tamos.storage.OneVector:tamos.storage.OneVector}]{\sphinxcrossref{\sphinxcode{\sphinxupquote{tamos.storage.OneVector}}}}}(vector, properties)
&
\sphinxAtStartPar

\\
\hline
\sphinxAtStartPar
{\hyperref[\detokenize{generated/tamos.storage.Thermocline:tamos.storage.Thermocline}]{\sphinxcrossref{\sphinxcode{\sphinxupquote{tamos.storage.Thermocline}}}}}(stored\_TVP{[}, ...{]})
&
\sphinxAtStartPar

\\
\hline
\end{tabulary}
\par
\sphinxattableend\end{savenotes}

\sphinxstepscope


\section{tamos.storage.OneVector}
\label{\detokenize{generated/tamos.storage.OneVector:tamos-storage-onevector}}\label{\detokenize{generated/tamos.storage.OneVector::doc}}\index{OneVector (class in tamos.storage)@\spxentry{OneVector}\spxextra{class in tamos.storage}}

\begin{fulllineitems}
\phantomsection\label{\detokenize{generated/tamos.storage.OneVector:tamos.storage.OneVector}}
\pysigstartsignatures
\pysiglinewithargsret{\sphinxbfcode{\sphinxupquote{class\DUrole{w}{  }}}\sphinxcode{\sphinxupquote{tamos.storage.}}\sphinxbfcode{\sphinxupquote{OneVector}}}{\emph{\DUrole{n}{vector}}, \emph{\DUrole{n}{properties}}, \emph{\DUrole{n}{given\_sizing}\DUrole{o}{=}\DUrole{default_value}{None}}, \emph{\DUrole{n}{name}\DUrole{o}{=}\DUrole{default_value}{None}}, \emph{\DUrole{n}{units\_number\_ub}\DUrole{o}{=}\DUrole{default_value}{1}}, \emph{\DUrole{n}{units\_number\_lb}\DUrole{o}{=}\DUrole{default_value}{1}}, \emph{\DUrole{n}{eco\_count}\DUrole{o}{=}\DUrole{default_value}{True}}}{}
\pysigstopsignatures\index{\_\_init\_\_() (tamos.storage.OneVector method)@\spxentry{\_\_init\_\_()}\spxextra{tamos.storage.OneVector method}}

\begin{fulllineitems}
\phantomsection\label{\detokenize{generated/tamos.storage.OneVector:tamos.storage.OneVector.__init__}}
\pysigstartsignatures
\pysiglinewithargsret{\sphinxbfcode{\sphinxupquote{\_\_init\_\_}}}{\emph{\DUrole{n}{vector}}, \emph{\DUrole{n}{properties}}, \emph{\DUrole{n}{given\_sizing}\DUrole{o}{=}\DUrole{default_value}{None}}, \emph{\DUrole{n}{name}\DUrole{o}{=}\DUrole{default_value}{None}}, \emph{\DUrole{n}{units\_number\_ub}\DUrole{o}{=}\DUrole{default_value}{1}}, \emph{\DUrole{n}{units\_number\_lb}\DUrole{o}{=}\DUrole{default_value}{1}}, \emph{\DUrole{n}{eco\_count}\DUrole{o}{=}\DUrole{default_value}{True}}}{}
\pysigstopsignatures
\sphinxAtStartPar
OneVector components store Electricity or FuelVector elements.

\sphinxAtStartPar
This component declares the following exported decision variables:
\begin{itemize}
\item {} 
\sphinxAtStartPar
X\_S, binary.
Whether the component is used by the hub.

\item {} 
\sphinxAtStartPar
SE\_S, continuous, in kWh.
The maximum energetical capacity of the component.

\item {} 
\sphinxAtStartPar
For all t, E\_S(t), continuous, in kWh.
State of charge of the storage.

\item {} 
\sphinxAtStartPar
For all t, for all element e, F\_S(e, t), continuous, in kW.
The power related to element e entering the component (i.e. leaving the hub interface).

\end{itemize}

\sphinxAtStartPar
This component declares the following KPIs:
\begin{itemize}
\item {} 
\sphinxAtStartPar
\sphinxtitleref{COST\_storage}
In euros.
Contributes to the “Eco” objective function.

\end{itemize}
\begin{quote}\begin{description}
\sphinxlineitem{Parameters}\begin{itemize}
\item {} 
\sphinxAtStartPar
\sphinxstyleliteralstrong{\sphinxupquote{vector}} ({\hyperref[\detokenize{generated/tamos.element.ElectricityVector:tamos.element.ElectricityVector}]{\sphinxcrossref{\sphinxstyleliteralemphasis{\sphinxupquote{ElectricityVector}}}}}\sphinxstyleliteralemphasis{\sphinxupquote{ or }}{\hyperref[\detokenize{generated/tamos.element.FuelVector:tamos.element.FuelVector}]{\sphinxcrossref{\sphinxstyleliteralemphasis{\sphinxupquote{FuelVector}}}}}) \textendash{} Stored element.

\item {} 
\sphinxAtStartPar
\sphinxstyleliteralstrong{\sphinxupquote{properties}} (\sphinxstyleliteralemphasis{\sphinxupquote{dict \{str: int}}\sphinxstyleliteralemphasis{\sphinxupquote{ | }}\sphinxstyleliteralemphasis{\sphinxupquote{float\}}}) \textendash{} 
\sphinxAtStartPar
Techno\sphinxhyphen{}economic properties of the component.
The \sphinxtitleref{properties} attribute must include the following keys:
\begin{itemize}
\item {} 
\sphinxAtStartPar
”LB max energy (kWh)”

\item {} 
\sphinxAtStartPar
”UB max energy (kWh)”

\item {} 
\sphinxAtStartPar
”Charge/discharge delay (h)”

\item {} 
\sphinxAtStartPar
”Energy conservation (/h)”

\item {} 
\sphinxAtStartPar
”CAPEX energy (EUR/kWh)”

\item {} 
\sphinxAtStartPar
”OPEX energy (\%CAPEX)”

\end{itemize}


\item {} 
\sphinxAtStartPar
\sphinxstyleliteralstrong{\sphinxupquote{given\_sizing}} (\sphinxstyleliteralemphasis{\sphinxupquote{int}}\sphinxstyleliteralemphasis{\sphinxupquote{ or }}\sphinxstyleliteralemphasis{\sphinxupquote{float}}\sphinxstyleliteralemphasis{\sphinxupquote{, }}\sphinxstyleliteralemphasis{\sphinxupquote{optional}}) \textendash{} The maximum capacity of the component, in kWh.
Relates to decision variable ‘SE\_S’.
If specified, only the operation of this component is performed by the MILP solver.
If let unknown, both sizing and operation are performed.

\item {} 
\sphinxAtStartPar
\sphinxstyleliteralstrong{\sphinxupquote{name}} (\sphinxstyleliteralemphasis{\sphinxupquote{str}}\sphinxstyleliteralemphasis{\sphinxupquote{, }}\sphinxstyleliteralemphasis{\sphinxupquote{optional}}) \textendash{} 

\item {} 
\sphinxAtStartPar
\sphinxstyleliteralstrong{\sphinxupquote{units\_number\_lb}} (\sphinxstyleliteralemphasis{\sphinxupquote{int}}\sphinxstyleliteralemphasis{\sphinxupquote{, }}\sphinxstyleliteralemphasis{\sphinxupquote{optional}}\sphinxstyleliteralemphasis{\sphinxupquote{, }}\sphinxstyleliteralemphasis{\sphinxupquote{default 1}}) \textendash{} The lower bound (upper bound) of the number of real components that this instance aims to stand for.
Setting \sphinxtitleref{units\_number\_lb} (\sphinxtitleref{units\_number\_ub}) has a meaning if “LB max energy (kWh)” property is
different from 0.

\item {} 
\sphinxAtStartPar
\sphinxstyleliteralstrong{\sphinxupquote{units\_number\_ub}} (\sphinxstyleliteralemphasis{\sphinxupquote{int}}\sphinxstyleliteralemphasis{\sphinxupquote{, }}\sphinxstyleliteralemphasis{\sphinxupquote{optional}}\sphinxstyleliteralemphasis{\sphinxupquote{, }}\sphinxstyleliteralemphasis{\sphinxupquote{default 1}}) \textendash{} The lower bound (upper bound) of the number of real components that this instance aims to stand for.
Setting \sphinxtitleref{units\_number\_lb} (\sphinxtitleref{units\_number\_ub}) has a meaning if “LB max energy (kWh)” property is
different from 0.

\item {} 
\sphinxAtStartPar
\sphinxstyleliteralstrong{\sphinxupquote{eco\_count}} (\sphinxstyleliteralemphasis{\sphinxupquote{bool}}\sphinxstyleliteralemphasis{\sphinxupquote{, }}\sphinxstyleliteralemphasis{\sphinxupquote{optional}}\sphinxstyleliteralemphasis{\sphinxupquote{, }}\sphinxstyleliteralemphasis{\sphinxupquote{default True}}) \textendash{} Whether this instance contributes to the system “Eco” KPI.

\end{itemize}

\end{description}\end{quote}

\end{fulllineitems}

\subsubsection*{Methods}


\begin{savenotes}\sphinxattablestart
\centering
\begin{tabulary}{\linewidth}[t]{\X{1}{2}\X{1}{2}}
\hline

\sphinxAtStartPar
{\hyperref[\detokenize{generated/tamos.storage.OneVector:tamos.storage.OneVector.__init__}]{\sphinxcrossref{\sphinxcode{\sphinxupquote{\_\_init\_\_}}}}}(vector, properties{[}, given\_sizing, ...{]})
&
\sphinxAtStartPar
OneVector components store Electricity or FuelVector elements.
\\
\hline
\sphinxAtStartPar
{\hyperref[\detokenize{generated/tamos.storage.OneVector:tamos.storage.OneVector.compute_actualized_cost}]{\sphinxcrossref{\sphinxcode{\sphinxupquote{compute\_actualized\_cost}}}}}(CAPEX, OPEX, ...{[}, ...{]})
&
\sphinxAtStartPar
Computes the cost of a component using its \textquotesingle{}Lifetime\textquotesingle{} and \textquotesingle{}Discount rate (\%)\textquotesingle{} properties.
\\
\hline
\end{tabulary}
\par
\sphinxattableend\end{savenotes}
\subsubsection*{Attributes}


\begin{savenotes}\sphinxattablestart
\centering
\begin{tabulary}{\linewidth}[t]{\X{1}{2}\X{1}{2}}
\hline

\sphinxAtStartPar
{\hyperref[\detokenize{generated/tamos.storage.OneVector:tamos.storage.OneVector.eco_count}]{\sphinxcrossref{\sphinxcode{\sphinxupquote{eco\_count}}}}}
&
\sphinxAtStartPar
Whether this instance contributes to the system "Eco" KPI.
\\
\hline
\sphinxAtStartPar
{\hyperref[\detokenize{generated/tamos.storage.OneVector:tamos.storage.OneVector.given_sizing}]{\sphinxcrossref{\sphinxcode{\sphinxupquote{given\_sizing}}}}}
&
\sphinxAtStartPar
The maximum capacity of the component, in kg (Thermocline) or kWh (OneVector).
\\
\hline
\sphinxAtStartPar
{\hyperref[\detokenize{generated/tamos.storage.OneVector:tamos.storage.OneVector.name}]{\sphinxcrossref{\sphinxcode{\sphinxupquote{name}}}}}
&
\sphinxAtStartPar
str.
\\
\hline
\sphinxAtStartPar
{\hyperref[\detokenize{generated/tamos.storage.OneVector:tamos.storage.OneVector.units_number_lb}]{\sphinxcrossref{\sphinxcode{\sphinxupquote{units\_number\_lb}}}}}
&
\sphinxAtStartPar
The lower bound of the number of real components that this instance aims to stand for.
\\
\hline
\sphinxAtStartPar
{\hyperref[\detokenize{generated/tamos.storage.OneVector:tamos.storage.OneVector.units_number_ub}]{\sphinxcrossref{\sphinxcode{\sphinxupquote{units\_number\_ub}}}}}
&
\sphinxAtStartPar
The upper bound of the number of real components that this instance aims to stand for.
\\
\hline
\sphinxAtStartPar
{\hyperref[\detokenize{generated/tamos.storage.OneVector:tamos.storage.OneVector.used_elements}]{\sphinxcrossref{\sphinxcode{\sphinxupquote{used\_elements}}}}}
&
\sphinxAtStartPar
Elements used by the component.
\\
\hline
\sphinxAtStartPar
{\hyperref[\detokenize{generated/tamos.storage.OneVector:tamos.storage.OneVector.vector}]{\sphinxcrossref{\sphinxcode{\sphinxupquote{vector}}}}}
&
\sphinxAtStartPar
Stored element
\\
\hline
\end{tabulary}
\par
\sphinxattableend\end{savenotes}
\index{compute\_actualized\_cost() (tamos.storage.OneVector method)@\spxentry{compute\_actualized\_cost()}\spxextra{tamos.storage.OneVector method}}

\begin{fulllineitems}
\phantomsection\label{\detokenize{generated/tamos.storage.OneVector:tamos.storage.OneVector.compute_actualized_cost}}
\pysigstartsignatures
\pysiglinewithargsret{\sphinxbfcode{\sphinxupquote{compute\_actualized\_cost}}}{\emph{\DUrole{n}{CAPEX}}, \emph{\DUrole{n}{OPEX}}, \emph{\DUrole{n}{system\_lifetime}}, \emph{\DUrole{n}{lifetime}\DUrole{o}{=}\DUrole{default_value}{None}}, \emph{\DUrole{n}{discount\_rate}\DUrole{o}{=}\DUrole{default_value}{None}}}{}
\pysigstopsignatures
\sphinxAtStartPar
Computes the cost of a component using its ‘Lifetime’ and ‘Discount rate (\%)’ properties.
\begin{quote}\begin{description}
\sphinxlineitem{Parameters}\begin{itemize}
\item {} 
\sphinxAtStartPar
\sphinxstyleliteralstrong{\sphinxupquote{CAPEX}} (\sphinxstyleliteralemphasis{\sphinxupquote{float}}) \textendash{} Capital Expenditure. Cost in euros paid every \sphinxtitleref{technical\_lifetime} periods.

\item {} 
\sphinxAtStartPar
\sphinxstyleliteralstrong{\sphinxupquote{OPEX}} (\sphinxstyleliteralemphasis{\sphinxupquote{float}}) \textendash{} Operational Expenditure. Cost in euros paid each period.

\item {} 
\sphinxAtStartPar
\sphinxstyleliteralstrong{\sphinxupquote{system\_lifetime}} (\sphinxstyleliteralemphasis{\sphinxupquote{int}}) \textendash{} Number of periods defining the existence of the energy system.

\item {} 
\sphinxAtStartPar
\sphinxstyleliteralstrong{\sphinxupquote{lifetime}} (\sphinxstyleliteralemphasis{\sphinxupquote{int}}\sphinxstyleliteralemphasis{\sphinxupquote{, }}\sphinxstyleliteralemphasis{\sphinxupquote{optional}}) \textendash{} Number of periods defining the existence of the component.
If specified, overwrite the “Lifetime” property.

\item {} 
\sphinxAtStartPar
\sphinxstyleliteralstrong{\sphinxupquote{discount\_rate}} (\sphinxstyleliteralemphasis{\sphinxupquote{float}}) \textendash{} In percent (\%). Describes the importance of the economic amortization process, per period.
If specified, overwrite the “Discount rate (\%)” property.

\end{itemize}

\sphinxlineitem{Returns}
\sphinxAtStartPar
\begin{itemize}
\item {} 
\sphinxAtStartPar
\sphinxstyleemphasis{A 3\sphinxhyphen{}tuple (total\_cost, CAPEX\_share, OPEX\_share) where}

\item {} 
\sphinxAtStartPar
* CAPEX\_share is the share of total cost related to \sphinxtitleref{CAPEX}

\item {} 
\sphinxAtStartPar
* OPEX\_share is the share of total cost related to \sphinxtitleref{OPEX}

\item {} 
\sphinxAtStartPar
\sphinxstyleemphasis{* total\_cost = CAPEX\_share + OPEX\_share}

\end{itemize}


\end{description}\end{quote}
\subsubsection*{Notes}

\sphinxAtStartPar
Takes into account residual value of component in the case \sphinxtitleref{system\_lifetime} is not a multiple of \sphinxtitleref{lifetime}.
In this case, the last replacement occuring at period replacement\_period is paid in proportion of ‘CAPEX’
depending linearly on the number of periods left:
CAPEX * (system\_lifetime \sphinxhyphen{} replacement\_period) / lifetime

\end{fulllineitems}

\index{eco\_count (tamos.storage.OneVector property)@\spxentry{eco\_count}\spxextra{tamos.storage.OneVector property}}

\begin{fulllineitems}
\phantomsection\label{\detokenize{generated/tamos.storage.OneVector:tamos.storage.OneVector.eco_count}}
\pysigstartsignatures
\pysigline{\sphinxbfcode{\sphinxupquote{property\DUrole{w}{  }}}\sphinxbfcode{\sphinxupquote{eco\_count}}}
\pysigstopsignatures
\sphinxAtStartPar
Whether this instance contributes to the system “Eco” KPI.
bool

\end{fulllineitems}

\index{given\_sizing (tamos.storage.OneVector property)@\spxentry{given\_sizing}\spxextra{tamos.storage.OneVector property}}

\begin{fulllineitems}
\phantomsection\label{\detokenize{generated/tamos.storage.OneVector:tamos.storage.OneVector.given_sizing}}
\pysigstartsignatures
\pysigline{\sphinxbfcode{\sphinxupquote{property\DUrole{w}{  }}}\sphinxbfcode{\sphinxupquote{given\_sizing}}}
\pysigstopsignatures
\sphinxAtStartPar
The maximum capacity of the component, in kg (Thermocline) or kWh (OneVector).
Relates to decision variable ‘SE\_S’.
If specified, only the operation of this component is performed by the MILP solver.
If let unknown, both sizing and operation are performed.
int or float

\end{fulllineitems}

\index{name (tamos.storage.OneVector property)@\spxentry{name}\spxextra{tamos.storage.OneVector property}}

\begin{fulllineitems}
\phantomsection\label{\detokenize{generated/tamos.storage.OneVector:tamos.storage.OneVector.name}}
\pysigstartsignatures
\pysigline{\sphinxbfcode{\sphinxupquote{property\DUrole{w}{  }}}\sphinxbfcode{\sphinxupquote{name}}}
\pysigstopsignatures
\sphinxAtStartPar
str.
This name is used in MILP model description.
names must not exceed 255 characters,
all of which must be alphanumeric (a\sphinxhyphen{}z, A\sphinxhyphen{}Z, 0\sphinxhyphen{}9) or one of these symbols:
! ” \# \$ \% \& , . ; ? @ \_ ‘ ’ \{ \} \textasciitilde{}.
\begin{quote}\begin{description}
\sphinxlineitem{Type}
\sphinxAtStartPar
Name of the instance

\end{description}\end{quote}

\end{fulllineitems}

\index{units\_number\_lb (tamos.storage.OneVector property)@\spxentry{units\_number\_lb}\spxextra{tamos.storage.OneVector property}}

\begin{fulllineitems}
\phantomsection\label{\detokenize{generated/tamos.storage.OneVector:tamos.storage.OneVector.units_number_lb}}
\pysigstartsignatures
\pysigline{\sphinxbfcode{\sphinxupquote{property\DUrole{w}{  }}}\sphinxbfcode{\sphinxupquote{units\_number\_lb}}}
\pysigstopsignatures
\sphinxAtStartPar
The lower bound of the number of real components that this instance aims to stand for.
Setting \sphinxtitleref{units\_number\_lb} has a meaning if “LB max output power (kW)” property is different from 0.
int

\end{fulllineitems}

\index{units\_number\_ub (tamos.storage.OneVector property)@\spxentry{units\_number\_ub}\spxextra{tamos.storage.OneVector property}}

\begin{fulllineitems}
\phantomsection\label{\detokenize{generated/tamos.storage.OneVector:tamos.storage.OneVector.units_number_ub}}
\pysigstartsignatures
\pysigline{\sphinxbfcode{\sphinxupquote{property\DUrole{w}{  }}}\sphinxbfcode{\sphinxupquote{units\_number\_ub}}}
\pysigstopsignatures
\sphinxAtStartPar
The upper bound of the number of real components that this instance aims to stand for.
Setting \sphinxtitleref{units\_number\_ub} has a meaning if “LB max output power (kW)” property is different from 0.
int

\end{fulllineitems}

\index{used\_elements (tamos.storage.OneVector property)@\spxentry{used\_elements}\spxextra{tamos.storage.OneVector property}}

\begin{fulllineitems}
\phantomsection\label{\detokenize{generated/tamos.storage.OneVector:tamos.storage.OneVector.used_elements}}
\pysigstartsignatures
\pysigline{\sphinxbfcode{\sphinxupquote{property\DUrole{w}{  }}}\sphinxbfcode{\sphinxupquote{used\_elements}}}
\pysigstopsignatures
\sphinxAtStartPar
Elements used by the component.

\end{fulllineitems}

\index{vector (tamos.storage.OneVector property)@\spxentry{vector}\spxextra{tamos.storage.OneVector property}}

\begin{fulllineitems}
\phantomsection\label{\detokenize{generated/tamos.storage.OneVector:tamos.storage.OneVector.vector}}
\pysigstartsignatures
\pysigline{\sphinxbfcode{\sphinxupquote{property\DUrole{w}{  }}}\sphinxbfcode{\sphinxupquote{vector}}}
\pysigstopsignatures
\sphinxAtStartPar
Stored element

\end{fulllineitems}


\end{fulllineitems}


\sphinxstepscope


\section{tamos.storage.Thermocline}
\label{\detokenize{generated/tamos.storage.Thermocline:tamos-storage-thermocline}}\label{\detokenize{generated/tamos.storage.Thermocline::doc}}\index{Thermocline (class in tamos.storage)@\spxentry{Thermocline}\spxextra{class in tamos.storage}}

\begin{fulllineitems}
\phantomsection\label{\detokenize{generated/tamos.storage.Thermocline:tamos.storage.Thermocline}}
\pysigstartsignatures
\pysiglinewithargsret{\sphinxbfcode{\sphinxupquote{class\DUrole{w}{  }}}\sphinxcode{\sphinxupquote{tamos.storage.}}\sphinxbfcode{\sphinxupquote{Thermocline}}}{\emph{\DUrole{n}{stored\_TVP}}, \emph{\DUrole{n}{properties}\DUrole{o}{=}\DUrole{default_value}{None}}, \emph{\DUrole{n}{given\_sizing}\DUrole{o}{=}\DUrole{default_value}{None}}, \emph{\DUrole{n}{units\_number\_ub}\DUrole{o}{=}\DUrole{default_value}{1}}, \emph{\DUrole{n}{units\_number\_lb}\DUrole{o}{=}\DUrole{default_value}{1}}, \emph{\DUrole{n}{name}\DUrole{o}{=}\DUrole{default_value}{None}}, \emph{\DUrole{n}{eco\_count}\DUrole{o}{=}\DUrole{default_value}{True}}}{}
\pysigstopsignatures\index{\_\_init\_\_() (tamos.storage.Thermocline method)@\spxentry{\_\_init\_\_()}\spxextra{tamos.storage.Thermocline method}}

\begin{fulllineitems}
\phantomsection\label{\detokenize{generated/tamos.storage.Thermocline:tamos.storage.Thermocline.__init__}}
\pysigstartsignatures
\pysiglinewithargsret{\sphinxbfcode{\sphinxupquote{\_\_init\_\_}}}{\emph{\DUrole{n}{stored\_TVP}}, \emph{\DUrole{n}{properties}\DUrole{o}{=}\DUrole{default_value}{None}}, \emph{\DUrole{n}{given\_sizing}\DUrole{o}{=}\DUrole{default_value}{None}}, \emph{\DUrole{n}{units\_number\_ub}\DUrole{o}{=}\DUrole{default_value}{1}}, \emph{\DUrole{n}{units\_number\_lb}\DUrole{o}{=}\DUrole{default_value}{1}}, \emph{\DUrole{n}{name}\DUrole{o}{=}\DUrole{default_value}{None}}, \emph{\DUrole{n}{eco\_count}\DUrole{o}{=}\DUrole{default_value}{True}}}{}
\pysigstopsignatures
\sphinxAtStartPar
Thermocline components model a perfectly stratified thermal energy storage.

\sphinxAtStartPar
This component declares the following exported decision variables:
\begin{itemize}
\item {} 
\sphinxAtStartPar
X\_S, binary.
Whether the component is used by the hub.

\item {} 
\sphinxAtStartPar
SE\_S, continuous, in kg.
The maximum energetical capacity of the component.

\item {} 
\sphinxAtStartPar
For all t, E\_S(t), continuous, in kg.
State of charge of the storage.

\item {} 
\sphinxAtStartPar
For all t, for all element e, F\_S(e, t), continuous, in kW.
The power related to element e entering the component (i.e. leaving the hub interface).

\end{itemize}

\sphinxAtStartPar
This component declares the following KPIs:
\begin{itemize}
\item {} 
\sphinxAtStartPar
\sphinxtitleref{COST\_storage}
In euros.
Contributes to the “Eco” objective function.

\end{itemize}
\begin{quote}\begin{description}
\sphinxlineitem{Parameters}\begin{itemize}
\item {} 
\sphinxAtStartPar
\sphinxstyleliteralstrong{\sphinxupquote{stored\_TVP}} (\sphinxstyleliteralemphasis{\sphinxupquote{ThermalVectorPair}}) \textendash{} Stored element.
Storage is full when its entire content is the incoming vector of \sphinxtitleref{stored\_TVP}.
Please see note 2) for an explanation of the restriction regarding \sphinxtitleref{stored\_TVP}.

\item {} 
\sphinxAtStartPar
\sphinxstyleliteralstrong{\sphinxupquote{properties}} (\sphinxstyleliteralemphasis{\sphinxupquote{dict \{str: int}}\sphinxstyleliteralemphasis{\sphinxupquote{ | }}\sphinxstyleliteralemphasis{\sphinxupquote{float\}}}) \textendash{} 
\sphinxAtStartPar
Techno\sphinxhyphen{}economic properties of the component.

\sphinxAtStartPar
The \sphinxtitleref{properties} attribute must include the following keys:
\begin{itemize}
\item {} 
\sphinxAtStartPar
”LB max energy (kg)”

\item {} 
\sphinxAtStartPar
”UB max energy (kg)”

\item {} 
\sphinxAtStartPar
”Charge/discharge delay (h)”

\item {} 
\sphinxAtStartPar
”Energy conservation (/h)”

\item {} 
\sphinxAtStartPar
”CAPEX energy (EUR/kg)”

\item {} 
\sphinxAtStartPar
”OPEX energy (\%CAPEX)”

\end{itemize}


\item {} 
\sphinxAtStartPar
\sphinxstyleliteralstrong{\sphinxupquote{given\_sizing}} (\sphinxstyleliteralemphasis{\sphinxupquote{int}}\sphinxstyleliteralemphasis{\sphinxupquote{ or }}\sphinxstyleliteralemphasis{\sphinxupquote{float}}\sphinxstyleliteralemphasis{\sphinxupquote{, }}\sphinxstyleliteralemphasis{\sphinxupquote{optional}}) \textendash{} The maximum capacity of the component, in kg.
Relates to decision variable ‘SE\_S’.
If specified, only the operation of this component is performed by the MILP solver.
If let unknown, both sizing and operation are performed.

\item {} 
\sphinxAtStartPar
\sphinxstyleliteralstrong{\sphinxupquote{name}} (\sphinxstyleliteralemphasis{\sphinxupquote{str}}\sphinxstyleliteralemphasis{\sphinxupquote{, }}\sphinxstyleliteralemphasis{\sphinxupquote{optional}}) \textendash{} 

\item {} 
\sphinxAtStartPar
\sphinxstyleliteralstrong{\sphinxupquote{units\_number\_lb}} (\sphinxstyleliteralemphasis{\sphinxupquote{int}}\sphinxstyleliteralemphasis{\sphinxupquote{, }}\sphinxstyleliteralemphasis{\sphinxupquote{optional}}\sphinxstyleliteralemphasis{\sphinxupquote{, }}\sphinxstyleliteralemphasis{\sphinxupquote{default 1}}) \textendash{} The lower bound (upper bound) of the number of real components that this instance aims to stand for.
Setting \sphinxtitleref{units\_number\_lb} (\sphinxtitleref{units\_number\_ub}) has a meaning if “LB max energy (kg)” property is
different from 0.

\item {} 
\sphinxAtStartPar
\sphinxstyleliteralstrong{\sphinxupquote{units\_number\_ub}} (\sphinxstyleliteralemphasis{\sphinxupquote{int}}\sphinxstyleliteralemphasis{\sphinxupquote{, }}\sphinxstyleliteralemphasis{\sphinxupquote{optional}}\sphinxstyleliteralemphasis{\sphinxupquote{, }}\sphinxstyleliteralemphasis{\sphinxupquote{default 1}}) \textendash{} The lower bound (upper bound) of the number of real components that this instance aims to stand for.
Setting \sphinxtitleref{units\_number\_lb} (\sphinxtitleref{units\_number\_ub}) has a meaning if “LB max energy (kg)” property is
different from 0.

\item {} 
\sphinxAtStartPar
\sphinxstyleliteralstrong{\sphinxupquote{eco\_count}} (\sphinxstyleliteralemphasis{\sphinxupquote{bool}}\sphinxstyleliteralemphasis{\sphinxupquote{, }}\sphinxstyleliteralemphasis{\sphinxupquote{optional}}\sphinxstyleliteralemphasis{\sphinxupquote{, }}\sphinxstyleliteralemphasis{\sphinxupquote{default True}}) \textendash{} Whether this instance contributes to the system “Eco” KPI.

\end{itemize}

\end{description}\end{quote}
\subsubsection*{Notes}
\begin{enumerate}
\sphinxsetlistlabels{\arabic}{enumi}{enumii}{}{.}%
\item {} 
\sphinxAtStartPar
This modelisation is equivalent to 2 OneVector storages with the constraint
state\_of\_charge\_1 + state\_of\_charge\_2 = state\_of\_charge
(i.e.: the discharge of one side leads to the charge of the other side).

\item {} 
\sphinxAtStartPar
The instantaneous energetical state of charge of the storage depends on the product m(t) x Cp(t) x DT(t) where:
\begin{itemize}
\item {} 
\sphinxAtStartPar
m(t) is the instantaneous mass state of charge of the storage (decision variable: E\_S)

\item {} 
\sphinxAtStartPar
Cp(t) is the specific heat capacity of \sphinxtitleref{stored\_TVP}

\item {} 
\sphinxAtStartPar
DT(t) is the temperature difference between incoming and outcoming flows of \sphinxtitleref{stored\_TVP}

\end{itemize}

\sphinxAtStartPar
Thus, in the general case, the storage can be charged when Cp(t) x DT(t) is low (requiring low charging power, decision variable: F\_S)
and discharged when Cp(t) x DT(t) is high; which has no physical meaning.
For this reason, the Thermocline component must not be used with a ThermalVectorPair \sphinxtitleref{stored\_TVP} having a variable Cp(t) x DT(t) product.

\end{enumerate}

\end{fulllineitems}

\subsubsection*{Methods}


\begin{savenotes}\sphinxattablestart
\centering
\begin{tabulary}{\linewidth}[t]{\X{1}{2}\X{1}{2}}
\hline

\sphinxAtStartPar
{\hyperref[\detokenize{generated/tamos.storage.Thermocline:tamos.storage.Thermocline.__init__}]{\sphinxcrossref{\sphinxcode{\sphinxupquote{\_\_init\_\_}}}}}(stored\_TVP{[}, properties, ...{]})
&
\sphinxAtStartPar
Thermocline components model a perfectly stratified thermal energy storage.
\\
\hline
\sphinxAtStartPar
{\hyperref[\detokenize{generated/tamos.storage.Thermocline:tamos.storage.Thermocline.compute_actualized_cost}]{\sphinxcrossref{\sphinxcode{\sphinxupquote{compute\_actualized\_cost}}}}}(CAPEX, OPEX, ...{[}, ...{]})
&
\sphinxAtStartPar
Computes the cost of a component using its \textquotesingle{}Lifetime\textquotesingle{} and \textquotesingle{}Discount rate (\%)\textquotesingle{} properties.
\\
\hline
\end{tabulary}
\par
\sphinxattableend\end{savenotes}
\subsubsection*{Attributes}


\begin{savenotes}\sphinxattablestart
\centering
\begin{tabulary}{\linewidth}[t]{\X{1}{2}\X{1}{2}}
\hline

\sphinxAtStartPar
{\hyperref[\detokenize{generated/tamos.storage.Thermocline:tamos.storage.Thermocline.eco_count}]{\sphinxcrossref{\sphinxcode{\sphinxupquote{eco\_count}}}}}
&
\sphinxAtStartPar
Whether this instance contributes to the system "Eco" KPI.
\\
\hline
\sphinxAtStartPar
{\hyperref[\detokenize{generated/tamos.storage.Thermocline:tamos.storage.Thermocline.given_sizing}]{\sphinxcrossref{\sphinxcode{\sphinxupquote{given\_sizing}}}}}
&
\sphinxAtStartPar
The maximum capacity of the component, in kg (Thermocline) or kWh (OneVector).
\\
\hline
\sphinxAtStartPar
{\hyperref[\detokenize{generated/tamos.storage.Thermocline:tamos.storage.Thermocline.name}]{\sphinxcrossref{\sphinxcode{\sphinxupquote{name}}}}}
&
\sphinxAtStartPar
str.
\\
\hline
\sphinxAtStartPar
{\hyperref[\detokenize{generated/tamos.storage.Thermocline:tamos.storage.Thermocline.stored_TVP}]{\sphinxcrossref{\sphinxcode{\sphinxupquote{stored\_TVP}}}}}
&
\sphinxAtStartPar
Stored element.
\\
\hline
\sphinxAtStartPar
{\hyperref[\detokenize{generated/tamos.storage.Thermocline:tamos.storage.Thermocline.units_number_lb}]{\sphinxcrossref{\sphinxcode{\sphinxupquote{units\_number\_lb}}}}}
&
\sphinxAtStartPar
The lower bound of the number of real components that this instance aims to stand for.
\\
\hline
\sphinxAtStartPar
{\hyperref[\detokenize{generated/tamos.storage.Thermocline:tamos.storage.Thermocline.units_number_ub}]{\sphinxcrossref{\sphinxcode{\sphinxupquote{units\_number\_ub}}}}}
&
\sphinxAtStartPar
The upper bound of the number of real components that this instance aims to stand for.
\\
\hline
\sphinxAtStartPar
{\hyperref[\detokenize{generated/tamos.storage.Thermocline:tamos.storage.Thermocline.used_elements}]{\sphinxcrossref{\sphinxcode{\sphinxupquote{used\_elements}}}}}
&
\sphinxAtStartPar
Elements used by the component.
\\
\hline
\end{tabulary}
\par
\sphinxattableend\end{savenotes}
\index{compute\_actualized\_cost() (tamos.storage.Thermocline method)@\spxentry{compute\_actualized\_cost()}\spxextra{tamos.storage.Thermocline method}}

\begin{fulllineitems}
\phantomsection\label{\detokenize{generated/tamos.storage.Thermocline:tamos.storage.Thermocline.compute_actualized_cost}}
\pysigstartsignatures
\pysiglinewithargsret{\sphinxbfcode{\sphinxupquote{compute\_actualized\_cost}}}{\emph{\DUrole{n}{CAPEX}}, \emph{\DUrole{n}{OPEX}}, \emph{\DUrole{n}{system\_lifetime}}, \emph{\DUrole{n}{lifetime}\DUrole{o}{=}\DUrole{default_value}{None}}, \emph{\DUrole{n}{discount\_rate}\DUrole{o}{=}\DUrole{default_value}{None}}}{}
\pysigstopsignatures
\sphinxAtStartPar
Computes the cost of a component using its ‘Lifetime’ and ‘Discount rate (\%)’ properties.
\begin{quote}\begin{description}
\sphinxlineitem{Parameters}\begin{itemize}
\item {} 
\sphinxAtStartPar
\sphinxstyleliteralstrong{\sphinxupquote{CAPEX}} (\sphinxstyleliteralemphasis{\sphinxupquote{float}}) \textendash{} Capital Expenditure. Cost in euros paid every \sphinxtitleref{technical\_lifetime} periods.

\item {} 
\sphinxAtStartPar
\sphinxstyleliteralstrong{\sphinxupquote{OPEX}} (\sphinxstyleliteralemphasis{\sphinxupquote{float}}) \textendash{} Operational Expenditure. Cost in euros paid each period.

\item {} 
\sphinxAtStartPar
\sphinxstyleliteralstrong{\sphinxupquote{system\_lifetime}} (\sphinxstyleliteralemphasis{\sphinxupquote{int}}) \textendash{} Number of periods defining the existence of the energy system.

\item {} 
\sphinxAtStartPar
\sphinxstyleliteralstrong{\sphinxupquote{lifetime}} (\sphinxstyleliteralemphasis{\sphinxupquote{int}}\sphinxstyleliteralemphasis{\sphinxupquote{, }}\sphinxstyleliteralemphasis{\sphinxupquote{optional}}) \textendash{} Number of periods defining the existence of the component.
If specified, overwrite the “Lifetime” property.

\item {} 
\sphinxAtStartPar
\sphinxstyleliteralstrong{\sphinxupquote{discount\_rate}} (\sphinxstyleliteralemphasis{\sphinxupquote{float}}) \textendash{} In percent (\%). Describes the importance of the economic amortization process, per period.
If specified, overwrite the “Discount rate (\%)” property.

\end{itemize}

\sphinxlineitem{Returns}
\sphinxAtStartPar
\begin{itemize}
\item {} 
\sphinxAtStartPar
\sphinxstyleemphasis{A 3\sphinxhyphen{}tuple (total\_cost, CAPEX\_share, OPEX\_share) where}

\item {} 
\sphinxAtStartPar
* CAPEX\_share is the share of total cost related to \sphinxtitleref{CAPEX}

\item {} 
\sphinxAtStartPar
* OPEX\_share is the share of total cost related to \sphinxtitleref{OPEX}

\item {} 
\sphinxAtStartPar
\sphinxstyleemphasis{* total\_cost = CAPEX\_share + OPEX\_share}

\end{itemize}


\end{description}\end{quote}
\subsubsection*{Notes}

\sphinxAtStartPar
Takes into account residual value of component in the case \sphinxtitleref{system\_lifetime} is not a multiple of \sphinxtitleref{lifetime}.
In this case, the last replacement occuring at period replacement\_period is paid in proportion of ‘CAPEX’
depending linearly on the number of periods left:
CAPEX * (system\_lifetime \sphinxhyphen{} replacement\_period) / lifetime

\end{fulllineitems}

\index{eco\_count (tamos.storage.Thermocline property)@\spxentry{eco\_count}\spxextra{tamos.storage.Thermocline property}}

\begin{fulllineitems}
\phantomsection\label{\detokenize{generated/tamos.storage.Thermocline:tamos.storage.Thermocline.eco_count}}
\pysigstartsignatures
\pysigline{\sphinxbfcode{\sphinxupquote{property\DUrole{w}{  }}}\sphinxbfcode{\sphinxupquote{eco\_count}}}
\pysigstopsignatures
\sphinxAtStartPar
Whether this instance contributes to the system “Eco” KPI.
bool

\end{fulllineitems}

\index{given\_sizing (tamos.storage.Thermocline property)@\spxentry{given\_sizing}\spxextra{tamos.storage.Thermocline property}}

\begin{fulllineitems}
\phantomsection\label{\detokenize{generated/tamos.storage.Thermocline:tamos.storage.Thermocline.given_sizing}}
\pysigstartsignatures
\pysigline{\sphinxbfcode{\sphinxupquote{property\DUrole{w}{  }}}\sphinxbfcode{\sphinxupquote{given\_sizing}}}
\pysigstopsignatures
\sphinxAtStartPar
The maximum capacity of the component, in kg (Thermocline) or kWh (OneVector).
Relates to decision variable ‘SE\_S’.
If specified, only the operation of this component is performed by the MILP solver.
If let unknown, both sizing and operation are performed.
int or float

\end{fulllineitems}

\index{name (tamos.storage.Thermocline property)@\spxentry{name}\spxextra{tamos.storage.Thermocline property}}

\begin{fulllineitems}
\phantomsection\label{\detokenize{generated/tamos.storage.Thermocline:tamos.storage.Thermocline.name}}
\pysigstartsignatures
\pysigline{\sphinxbfcode{\sphinxupquote{property\DUrole{w}{  }}}\sphinxbfcode{\sphinxupquote{name}}}
\pysigstopsignatures
\sphinxAtStartPar
str.
This name is used in MILP model description.
names must not exceed 255 characters,
all of which must be alphanumeric (a\sphinxhyphen{}z, A\sphinxhyphen{}Z, 0\sphinxhyphen{}9) or one of these symbols:
! ” \# \$ \% \& , . ; ? @ \_ ‘ ’ \{ \} \textasciitilde{}.
\begin{quote}\begin{description}
\sphinxlineitem{Type}
\sphinxAtStartPar
Name of the instance

\end{description}\end{quote}

\end{fulllineitems}

\index{stored\_TVP (tamos.storage.Thermocline property)@\spxentry{stored\_TVP}\spxextra{tamos.storage.Thermocline property}}

\begin{fulllineitems}
\phantomsection\label{\detokenize{generated/tamos.storage.Thermocline:tamos.storage.Thermocline.stored_TVP}}
\pysigstartsignatures
\pysigline{\sphinxbfcode{\sphinxupquote{property\DUrole{w}{  }}}\sphinxbfcode{\sphinxupquote{stored\_TVP}}}
\pysigstopsignatures
\sphinxAtStartPar
Stored element.
Storage is full when its entire content is the incoming vector of \sphinxtitleref{stored\_TVP}.

\end{fulllineitems}

\index{units\_number\_lb (tamos.storage.Thermocline property)@\spxentry{units\_number\_lb}\spxextra{tamos.storage.Thermocline property}}

\begin{fulllineitems}
\phantomsection\label{\detokenize{generated/tamos.storage.Thermocline:tamos.storage.Thermocline.units_number_lb}}
\pysigstartsignatures
\pysigline{\sphinxbfcode{\sphinxupquote{property\DUrole{w}{  }}}\sphinxbfcode{\sphinxupquote{units\_number\_lb}}}
\pysigstopsignatures
\sphinxAtStartPar
The lower bound of the number of real components that this instance aims to stand for.
Setting \sphinxtitleref{units\_number\_lb} has a meaning if “LB max output power (kW)” property is different from 0.
int

\end{fulllineitems}

\index{units\_number\_ub (tamos.storage.Thermocline property)@\spxentry{units\_number\_ub}\spxextra{tamos.storage.Thermocline property}}

\begin{fulllineitems}
\phantomsection\label{\detokenize{generated/tamos.storage.Thermocline:tamos.storage.Thermocline.units_number_ub}}
\pysigstartsignatures
\pysigline{\sphinxbfcode{\sphinxupquote{property\DUrole{w}{  }}}\sphinxbfcode{\sphinxupquote{units\_number\_ub}}}
\pysigstopsignatures
\sphinxAtStartPar
The upper bound of the number of real components that this instance aims to stand for.
Setting \sphinxtitleref{units\_number\_ub} has a meaning if “LB max output power (kW)” property is different from 0.
int

\end{fulllineitems}

\index{used\_elements (tamos.storage.Thermocline property)@\spxentry{used\_elements}\spxextra{tamos.storage.Thermocline property}}

\begin{fulllineitems}
\phantomsection\label{\detokenize{generated/tamos.storage.Thermocline:tamos.storage.Thermocline.used_elements}}
\pysigstartsignatures
\pysigline{\sphinxbfcode{\sphinxupquote{property\DUrole{w}{  }}}\sphinxbfcode{\sphinxupquote{used\_elements}}}
\pysigstopsignatures
\sphinxAtStartPar
Elements used by the component.

\end{fulllineitems}


\end{fulllineitems}


\sphinxstepscope


\chapter{ElementIO components}
\label{\detokenize{element_IO_components:elementio-components}}\label{\detokenize{element_IO_components::doc}}

\begin{savenotes}\sphinxattablestart
\centering
\begin{tabulary}{\linewidth}[t]{\X{1}{2}\X{1}{2}}
\hline

\sphinxAtStartPar
{\hyperref[\detokenize{generated/tamos.elementIO.Cost:tamos.elementIO.Cost}]{\sphinxcrossref{\sphinxcode{\sphinxupquote{tamos.elementIO.Cost}}}}}({[}cost, carbon\_cost, name{]})
&
\sphinxAtStartPar

\\
\hline
\sphinxAtStartPar
{\hyperref[\detokenize{generated/tamos.elementIO.Grid:tamos.elementIO.Grid}]{\sphinxcrossref{\sphinxcode{\sphinxupquote{tamos.elementIO.Grid}}}}}(element{[}, emissions, ...{]})
&
\sphinxAtStartPar

\\
\hline
\sphinxAtStartPar
{\hyperref[\detokenize{generated/tamos.elementIO.Load:tamos.elementIO.Load}]{\sphinxcrossref{\sphinxcode{\sphinxupquote{tamos.elementIO.Load}}}}}(load, element{[}, ...{]})
&
\sphinxAtStartPar

\\
\hline
\end{tabulary}
\par
\sphinxattableend\end{savenotes}

\sphinxstepscope


\section{tamos.elementIO.Cost}
\label{\detokenize{generated/tamos.elementIO.Cost:tamos-elementio-cost}}\label{\detokenize{generated/tamos.elementIO.Cost::doc}}\index{Cost (class in tamos.elementIO)@\spxentry{Cost}\spxextra{class in tamos.elementIO}}

\begin{fulllineitems}
\phantomsection\label{\detokenize{generated/tamos.elementIO.Cost:tamos.elementIO.Cost}}
\pysigstartsignatures
\pysiglinewithargsret{\sphinxbfcode{\sphinxupquote{class\DUrole{w}{  }}}\sphinxcode{\sphinxupquote{tamos.elementIO.}}\sphinxbfcode{\sphinxupquote{Cost}}}{\emph{\DUrole{n}{cost}\DUrole{o}{=}\DUrole{default_value}{None}}, \emph{\DUrole{n}{carbon\_cost}\DUrole{o}{=}\DUrole{default_value}{None}}, \emph{\DUrole{n}{name}\DUrole{o}{=}\DUrole{default_value}{None}}}{}
\pysigstopsignatures\index{\_\_init\_\_() (tamos.elementIO.Cost method)@\spxentry{\_\_init\_\_()}\spxextra{tamos.elementIO.Cost method}}

\begin{fulllineitems}
\phantomsection\label{\detokenize{generated/tamos.elementIO.Cost:tamos.elementIO.Cost.__init__}}
\pysigstartsignatures
\pysiglinewithargsret{\sphinxbfcode{\sphinxupquote{\_\_init\_\_}}}{\emph{\DUrole{n}{cost}\DUrole{o}{=}\DUrole{default_value}{None}}, \emph{\DUrole{n}{carbon\_cost}\DUrole{o}{=}\DUrole{default_value}{None}}, \emph{\DUrole{n}{name}\DUrole{o}{=}\DUrole{default_value}{None}}}{}
\pysigstopsignatures
\sphinxAtStartPar
Defines the cost of buying energy from or selling energy to the outside of the energy system.
A Cost instance must be passed to a Grid or Load instance to be effective.
All costs are defined relatively to the flow of the \sphinxtitleref{element} attribute of the Grid or Load instance.

\sphinxAtStartPar
Cost instances declare the following KPIs:
\begin{itemize}
\item {} 
\sphinxAtStartPar
\sphinxtitleref{COST\_element}
In euros.
Contributes to the “Eco” objective function. Related to the \sphinxtitleref{cost} attribute.

\item {} 
\sphinxAtStartPar
\sphinxtitleref{Carbon tax}
In euros.
Contributes to the “Eco” objective function. Related to the \sphinxtitleref{carbon\_cost} attribute.

\end{itemize}
\begin{quote}\begin{description}
\sphinxlineitem{Parameters}\begin{itemize}
\item {} 
\sphinxAtStartPar
\sphinxstyleliteralstrong{\sphinxupquote{cost}} (\sphinxstyleliteralemphasis{\sphinxupquote{int}}\sphinxstyleliteralemphasis{\sphinxupquote{, }}\sphinxstyleliteralemphasis{\sphinxupquote{float}}\sphinxstyleliteralemphasis{\sphinxupquote{ or }}\sphinxstyleliteralemphasis{\sphinxupquote{numpy.ndarray}}\sphinxstyleliteralemphasis{\sphinxupquote{, }}\sphinxstyleliteralemphasis{\sphinxupquote{optional}}) \textendash{} Cost associated with a positive flow of \sphinxtitleref{element}.
In euros/kWh.
Usually negative.

\item {} 
\sphinxAtStartPar
\sphinxstyleliteralstrong{\sphinxupquote{carbon\_cost}} (\sphinxstyleliteralemphasis{\sphinxupquote{int}}\sphinxstyleliteralemphasis{\sphinxupquote{, }}\sphinxstyleliteralemphasis{\sphinxupquote{float}}\sphinxstyleliteralemphasis{\sphinxupquote{ or }}\sphinxstyleliteralemphasis{\sphinxupquote{numpy.ndarray}}\sphinxstyleliteralemphasis{\sphinxupquote{, }}\sphinxstyleliteralemphasis{\sphinxupquote{optional}}) \textendash{} Cost associated with positive CO2 emissions regarding the flow of ‘element’.
In euros/kgEqCO2.
Usually positive.

\item {} 
\sphinxAtStartPar
\sphinxstyleliteralstrong{\sphinxupquote{name}} (\sphinxstyleliteralemphasis{\sphinxupquote{str}}\sphinxstyleliteralemphasis{\sphinxupquote{, }}\sphinxstyleliteralemphasis{\sphinxupquote{optional}}) \textendash{} 

\end{itemize}

\end{description}\end{quote}
\subsubsection*{Examples}

\begin{sphinxVerbatim}[commandchars=\\\{\}]
\PYG{g+gp}{\PYGZgt{}\PYGZgt{}\PYGZgt{} }\PYG{n}{grid}\PYG{o}{=}  \PYG{n}{Grid}\PYG{p}{(}\PYG{n}{element}\PYG{o}{=}\PYG{n}{electricity}\PYG{p}{,} \PYG{n}{emissions}\PYG{o}{=}\PYG{l+m+mf}{0.3}\PYG{p}{)}
\PYG{g+gp}{\PYGZgt{}\PYGZgt{}\PYGZgt{} }\PYG{n}{cost} \PYG{o}{=} \PYG{n}{Cost}\PYG{p}{(}\PYG{n}{cost}\PYG{o}{=}\PYG{o}{\PYGZhy{}}\PYG{l+m+mf}{0.2}\PYG{p}{,} \PYG{n}{carbon\PYGZus{}cost}\PYG{o}{=}\PYG{l+m+mf}{0.1}\PYG{p}{)}
\PYG{g+gp}{\PYGZgt{}\PYGZgt{}\PYGZgt{} }\PYG{n}{grid}\PYG{o}{.}\PYG{n}{element\PYGZus{}cost} \PYG{o}{=} \PYG{n}{cost}
\end{sphinxVerbatim}

\sphinxAtStartPar
The energy system paies 0.2\texteuro{} to buy 1 kWh of \sphinxtitleref{electricity} from \sphinxtitleref{grid},
added to 0.3 * 0.1 = 0.03\texteuro{} paid as a carbon tax.

\end{fulllineitems}

\subsubsection*{Methods}


\begin{savenotes}\sphinxattablestart
\centering
\begin{tabulary}{\linewidth}[t]{\X{1}{2}\X{1}{2}}
\hline

\sphinxAtStartPar
{\hyperref[\detokenize{generated/tamos.elementIO.Cost:tamos.elementIO.Cost.__init__}]{\sphinxcrossref{\sphinxcode{\sphinxupquote{\_\_init\_\_}}}}}({[}cost, carbon\_cost, name{]})
&
\sphinxAtStartPar
Defines the cost of buying energy from or selling energy to the outside of the energy system.
\\
\hline
\end{tabulary}
\par
\sphinxattableend\end{savenotes}
\subsubsection*{Attributes}


\begin{savenotes}\sphinxattablestart
\centering
\begin{tabulary}{\linewidth}[t]{\X{1}{2}\X{1}{2}}
\hline

\sphinxAtStartPar
{\hyperref[\detokenize{generated/tamos.elementIO.Cost:tamos.elementIO.Cost.carbon_cost}]{\sphinxcrossref{\sphinxcode{\sphinxupquote{carbon\_cost}}}}}
&
\sphinxAtStartPar
Cost associated with positive CO2 emissions regarding the flow of \textquotesingle{}element\textquotesingle{}.
\\
\hline
\sphinxAtStartPar
{\hyperref[\detokenize{generated/tamos.elementIO.Cost:tamos.elementIO.Cost.cost}]{\sphinxcrossref{\sphinxcode{\sphinxupquote{cost}}}}}
&
\sphinxAtStartPar
Cost associated with a positive flow of \sphinxtitleref{element}.
\\
\hline
\sphinxAtStartPar
{\hyperref[\detokenize{generated/tamos.elementIO.Cost:tamos.elementIO.Cost.name}]{\sphinxcrossref{\sphinxcode{\sphinxupquote{name}}}}}
&
\sphinxAtStartPar
str.
\\
\hline
\end{tabulary}
\par
\sphinxattableend\end{savenotes}
\index{carbon\_cost (tamos.elementIO.Cost property)@\spxentry{carbon\_cost}\spxextra{tamos.elementIO.Cost property}}

\begin{fulllineitems}
\phantomsection\label{\detokenize{generated/tamos.elementIO.Cost:tamos.elementIO.Cost.carbon_cost}}
\pysigstartsignatures
\pysigline{\sphinxbfcode{\sphinxupquote{property\DUrole{w}{  }}}\sphinxbfcode{\sphinxupquote{carbon\_cost}}}
\pysigstopsignatures
\sphinxAtStartPar
Cost associated with positive CO2 emissions regarding the flow of ‘element’.
In euros/kgEqCO2.
Usually positive.
int, float or numpy.ndarray

\end{fulllineitems}

\index{cost (tamos.elementIO.Cost property)@\spxentry{cost}\spxextra{tamos.elementIO.Cost property}}

\begin{fulllineitems}
\phantomsection\label{\detokenize{generated/tamos.elementIO.Cost:tamos.elementIO.Cost.cost}}
\pysigstartsignatures
\pysigline{\sphinxbfcode{\sphinxupquote{property\DUrole{w}{  }}}\sphinxbfcode{\sphinxupquote{cost}}}
\pysigstopsignatures
\sphinxAtStartPar
Cost associated with a positive flow of \sphinxtitleref{element}.
In euros/kWh.
Usually negative.
int, float or numpy.ndarray

\end{fulllineitems}

\index{name (tamos.elementIO.Cost property)@\spxentry{name}\spxextra{tamos.elementIO.Cost property}}

\begin{fulllineitems}
\phantomsection\label{\detokenize{generated/tamos.elementIO.Cost:tamos.elementIO.Cost.name}}
\pysigstartsignatures
\pysigline{\sphinxbfcode{\sphinxupquote{property\DUrole{w}{  }}}\sphinxbfcode{\sphinxupquote{name}}}
\pysigstopsignatures
\sphinxAtStartPar
str.
This name is used in MILP model description.
names must not exceed 255 characters,
all of which must be alphanumeric (a\sphinxhyphen{}z, A\sphinxhyphen{}Z, 0\sphinxhyphen{}9) or one of these symbols:
! ” \# \$ \% \& , . ; ? @ \_ ‘ ’ \{ \} \textasciitilde{}.
\begin{quote}\begin{description}
\sphinxlineitem{Type}
\sphinxAtStartPar
Name of the instance

\end{description}\end{quote}

\end{fulllineitems}


\end{fulllineitems}


\sphinxstepscope


\section{tamos.elementIO.Grid}
\label{\detokenize{generated/tamos.elementIO.Grid:tamos-elementio-grid}}\label{\detokenize{generated/tamos.elementIO.Grid::doc}}\index{Grid (class in tamos.elementIO)@\spxentry{Grid}\spxextra{class in tamos.elementIO}}

\begin{fulllineitems}
\phantomsection\label{\detokenize{generated/tamos.elementIO.Grid:tamos.elementIO.Grid}}
\pysigstartsignatures
\pysiglinewithargsret{\sphinxbfcode{\sphinxupquote{class\DUrole{w}{  }}}\sphinxcode{\sphinxupquote{tamos.elementIO.}}\sphinxbfcode{\sphinxupquote{Grid}}}{\emph{\DUrole{n}{element}}, \emph{\DUrole{n}{emissions}\DUrole{o}{=}\DUrole{default_value}{None}}, \emph{\DUrole{n}{element\_cost}\DUrole{o}{=}\DUrole{default_value}{None}}, \emph{\DUrole{n}{exergy\_count}\DUrole{o}{=}\DUrole{default_value}{True}}, \emph{\DUrole{n}{name}\DUrole{o}{=}\DUrole{default_value}{None}}}{}
\pysigstopsignatures\index{\_\_init\_\_() (tamos.elementIO.Grid method)@\spxentry{\_\_init\_\_()}\spxextra{tamos.elementIO.Grid method}}

\begin{fulllineitems}
\phantomsection\label{\detokenize{generated/tamos.elementIO.Grid:tamos.elementIO.Grid.__init__}}
\pysigstartsignatures
\pysiglinewithargsret{\sphinxbfcode{\sphinxupquote{\_\_init\_\_}}}{\emph{\DUrole{n}{element}}, \emph{\DUrole{n}{emissions}\DUrole{o}{=}\DUrole{default_value}{None}}, \emph{\DUrole{n}{element\_cost}\DUrole{o}{=}\DUrole{default_value}{None}}, \emph{\DUrole{n}{exergy\_count}\DUrole{o}{=}\DUrole{default_value}{True}}, \emph{\DUrole{n}{name}\DUrole{o}{=}\DUrole{default_value}{None}}}{}
\pysigstopsignatures
\sphinxAtStartPar
Allows unconstrained element exchanges between the energy system and its environment.

\sphinxAtStartPar
Grid components are associated with the following exported decision variables:
\begin{itemize}
\item {} 
\sphinxAtStartPar
X\_EXT, binary.
Whether the Grid instance is used by the hub.

\item {} 
\sphinxAtStartPar
For all t, F\_EXT(t), continuous, in kW.
The power related to \sphinxtitleref{element} entering the grid (i.e. leaving the hub interface).

\end{itemize}

\sphinxAtStartPar
Grid components declare the following KPIs:
\begin{itemize}
\item {} 
\sphinxAtStartPar
\sphinxtitleref{ElementIO CO2}
In kgEqCO2.
Defines the “CO2” objective function. Related to the \sphinxtitleref{emissions} attribute.

\item {} 
\sphinxAtStartPar
\sphinxtitleref{ElementIO Exergy}
In kWh.
Defines the “Exergy” objective function. Related to the \sphinxtitleref{exergy\_factor} attribute of \sphinxtitleref{element}.

\end{itemize}
\begin{quote}\begin{description}
\sphinxlineitem{Parameters}\begin{itemize}
\item {} 
\sphinxAtStartPar
\sphinxstyleliteralstrong{\sphinxupquote{element}} ({\hyperref[\detokenize{generated/tamos.element.ElectricityVector:tamos.element.ElectricityVector}]{\sphinxcrossref{\sphinxstyleliteralemphasis{\sphinxupquote{ElectricityVector}}}}}\sphinxstyleliteralemphasis{\sphinxupquote{, }}{\hyperref[\detokenize{generated/tamos.element.FuelVector:tamos.element.FuelVector}]{\sphinxcrossref{\sphinxstyleliteralemphasis{\sphinxupquote{FuelVector}}}}}\sphinxstyleliteralemphasis{\sphinxupquote{, }}{\hyperref[\detokenize{generated/tamos.element.ThermalVector:tamos.element.ThermalVector}]{\sphinxcrossref{\sphinxstyleliteralemphasis{\sphinxupquote{ThermalVector}}}}}\sphinxstyleliteralemphasis{\sphinxupquote{ or }}\sphinxstyleliteralemphasis{\sphinxupquote{ThermalVectorPair}}) \textendash{} 

\item {} 
\sphinxAtStartPar
\sphinxstyleliteralstrong{\sphinxupquote{emissions}} (\sphinxstyleliteralemphasis{\sphinxupquote{int}}\sphinxstyleliteralemphasis{\sphinxupquote{, }}\sphinxstyleliteralemphasis{\sphinxupquote{float}}\sphinxstyleliteralemphasis{\sphinxupquote{ or }}\sphinxstyleliteralemphasis{\sphinxupquote{numpy.ndarray}}\sphinxstyleliteralemphasis{\sphinxupquote{, }}\sphinxstyleliteralemphasis{\sphinxupquote{optional}}) \textendash{} Quantity of CO2 associated with a positive flow of \sphinxtitleref{element}.
In kgEqCO2/kWh.
Usually negative.
If None, emissions are not accounted for.

\item {} 
\sphinxAtStartPar
\sphinxstyleliteralstrong{\sphinxupquote{element\_cost}} ({\hyperref[\detokenize{generated/tamos.elementIO.Cost:tamos.elementIO.Cost}]{\sphinxcrossref{\sphinxstyleliteralemphasis{\sphinxupquote{Cost}}}}}\sphinxstyleliteralemphasis{\sphinxupquote{, }}\sphinxstyleliteralemphasis{\sphinxupquote{optional}}) \textendash{} Cost associated with a positive flow of ‘element’.
If None, no costs are taken into account.

\item {} 
\sphinxAtStartPar
\sphinxstyleliteralstrong{\sphinxupquote{exergy\_count}} (\sphinxstyleliteralemphasis{\sphinxupquote{bool}}\sphinxstyleliteralemphasis{\sphinxupquote{, }}\sphinxstyleliteralemphasis{\sphinxupquote{optional}}\sphinxstyleliteralemphasis{\sphinxupquote{, }}\sphinxstyleliteralemphasis{\sphinxupquote{default True}}) \textendash{} Whether this instance contributes to the system “Exergy” KPI.

\item {} 
\sphinxAtStartPar
\sphinxstyleliteralstrong{\sphinxupquote{name}} (\sphinxstyleliteralemphasis{\sphinxupquote{str}}\sphinxstyleliteralemphasis{\sphinxupquote{, }}\sphinxstyleliteralemphasis{\sphinxupquote{optional}}) \textendash{} 

\end{itemize}

\end{description}\end{quote}

\end{fulllineitems}

\subsubsection*{Methods}


\begin{savenotes}\sphinxattablestart
\centering
\begin{tabulary}{\linewidth}[t]{\X{1}{2}\X{1}{2}}
\hline

\sphinxAtStartPar
{\hyperref[\detokenize{generated/tamos.elementIO.Grid:tamos.elementIO.Grid.__init__}]{\sphinxcrossref{\sphinxcode{\sphinxupquote{\_\_init\_\_}}}}}(element{[}, emissions, element\_cost, ...{]})
&
\sphinxAtStartPar
Allows unconstrained element exchanges between the energy system and its environment.
\\
\hline
\sphinxAtStartPar
{\hyperref[\detokenize{generated/tamos.elementIO.Grid:tamos.elementIO.Grid.compute_actualized_cost}]{\sphinxcrossref{\sphinxcode{\sphinxupquote{compute\_actualized\_cost}}}}}(CAPEX, OPEX, ...{[}, ...{]})
&
\sphinxAtStartPar
Computes the cost of a component using its \textquotesingle{}Lifetime\textquotesingle{} and \textquotesingle{}Discount rate (\%)\textquotesingle{} properties.
\\
\hline
\end{tabulary}
\par
\sphinxattableend\end{savenotes}
\subsubsection*{Attributes}


\begin{savenotes}\sphinxattablestart
\centering
\begin{tabulary}{\linewidth}[t]{\X{1}{2}\X{1}{2}}
\hline

\sphinxAtStartPar
{\hyperref[\detokenize{generated/tamos.elementIO.Grid:tamos.elementIO.Grid.element}]{\sphinxcrossref{\sphinxcode{\sphinxupquote{element}}}}}
&
\sphinxAtStartPar
Element exchanged between the hub interface and this instance.
\\
\hline
\sphinxAtStartPar
{\hyperref[\detokenize{generated/tamos.elementIO.Grid:tamos.elementIO.Grid.element_cost}]{\sphinxcrossref{\sphinxcode{\sphinxupquote{element\_cost}}}}}
&
\sphinxAtStartPar
Cost instance associated with a positive flow of \textquotesingle{}element\textquotesingle{}.
\\
\hline
\sphinxAtStartPar
{\hyperref[\detokenize{generated/tamos.elementIO.Grid:tamos.elementIO.Grid.emissions}]{\sphinxcrossref{\sphinxcode{\sphinxupquote{emissions}}}}}
&
\sphinxAtStartPar
Quantity of CO2 associated with a positive power flow of \sphinxtitleref{element}.
\\
\hline
\sphinxAtStartPar
{\hyperref[\detokenize{generated/tamos.elementIO.Grid:tamos.elementIO.Grid.exergy_count}]{\sphinxcrossref{\sphinxcode{\sphinxupquote{exergy\_count}}}}}
&
\sphinxAtStartPar
Whether this instance contributes to the system "Exergy" KPI.
\\
\hline
\sphinxAtStartPar
{\hyperref[\detokenize{generated/tamos.elementIO.Grid:tamos.elementIO.Grid.name}]{\sphinxcrossref{\sphinxcode{\sphinxupquote{name}}}}}
&
\sphinxAtStartPar
str.
\\
\hline
\sphinxAtStartPar
{\hyperref[\detokenize{generated/tamos.elementIO.Grid:tamos.elementIO.Grid.used_elements}]{\sphinxcrossref{\sphinxcode{\sphinxupquote{used\_elements}}}}}
&
\sphinxAtStartPar
Elements used by the component.
\\
\hline
\end{tabulary}
\par
\sphinxattableend\end{savenotes}
\index{compute\_actualized\_cost() (tamos.elementIO.Grid method)@\spxentry{compute\_actualized\_cost()}\spxextra{tamos.elementIO.Grid method}}

\begin{fulllineitems}
\phantomsection\label{\detokenize{generated/tamos.elementIO.Grid:tamos.elementIO.Grid.compute_actualized_cost}}
\pysigstartsignatures
\pysiglinewithargsret{\sphinxbfcode{\sphinxupquote{compute\_actualized\_cost}}}{\emph{\DUrole{n}{CAPEX}}, \emph{\DUrole{n}{OPEX}}, \emph{\DUrole{n}{system\_lifetime}}, \emph{\DUrole{n}{lifetime}\DUrole{o}{=}\DUrole{default_value}{None}}, \emph{\DUrole{n}{discount\_rate}\DUrole{o}{=}\DUrole{default_value}{None}}}{}
\pysigstopsignatures
\sphinxAtStartPar
Computes the cost of a component using its ‘Lifetime’ and ‘Discount rate (\%)’ properties.
\begin{quote}\begin{description}
\sphinxlineitem{Parameters}\begin{itemize}
\item {} 
\sphinxAtStartPar
\sphinxstyleliteralstrong{\sphinxupquote{CAPEX}} (\sphinxstyleliteralemphasis{\sphinxupquote{float}}) \textendash{} Capital Expenditure. Cost in euros paid every \sphinxtitleref{technical\_lifetime} periods.

\item {} 
\sphinxAtStartPar
\sphinxstyleliteralstrong{\sphinxupquote{OPEX}} (\sphinxstyleliteralemphasis{\sphinxupquote{float}}) \textendash{} Operational Expenditure. Cost in euros paid each period.

\item {} 
\sphinxAtStartPar
\sphinxstyleliteralstrong{\sphinxupquote{system\_lifetime}} (\sphinxstyleliteralemphasis{\sphinxupquote{int}}) \textendash{} Number of periods defining the existence of the energy system.

\item {} 
\sphinxAtStartPar
\sphinxstyleliteralstrong{\sphinxupquote{lifetime}} (\sphinxstyleliteralemphasis{\sphinxupquote{int}}\sphinxstyleliteralemphasis{\sphinxupquote{, }}\sphinxstyleliteralemphasis{\sphinxupquote{optional}}) \textendash{} Number of periods defining the existence of the component.
If specified, overwrite the “Lifetime” property.

\item {} 
\sphinxAtStartPar
\sphinxstyleliteralstrong{\sphinxupquote{discount\_rate}} (\sphinxstyleliteralemphasis{\sphinxupquote{float}}) \textendash{} In percent (\%). Describes the importance of the economic amortization process, per period.
If specified, overwrite the “Discount rate (\%)” property.

\end{itemize}

\sphinxlineitem{Returns}
\sphinxAtStartPar
\begin{itemize}
\item {} 
\sphinxAtStartPar
\sphinxstyleemphasis{A 3\sphinxhyphen{}tuple (total\_cost, CAPEX\_share, OPEX\_share) where}

\item {} 
\sphinxAtStartPar
* CAPEX\_share is the share of total cost related to \sphinxtitleref{CAPEX}

\item {} 
\sphinxAtStartPar
* OPEX\_share is the share of total cost related to \sphinxtitleref{OPEX}

\item {} 
\sphinxAtStartPar
\sphinxstyleemphasis{* total\_cost = CAPEX\_share + OPEX\_share}

\end{itemize}


\end{description}\end{quote}
\subsubsection*{Notes}

\sphinxAtStartPar
Takes into account residual value of component in the case \sphinxtitleref{system\_lifetime} is not a multiple of \sphinxtitleref{lifetime}.
In this case, the last replacement occuring at period replacement\_period is paid in proportion of ‘CAPEX’
depending linearly on the number of periods left:
CAPEX * (system\_lifetime \sphinxhyphen{} replacement\_period) / lifetime

\end{fulllineitems}

\index{element (tamos.elementIO.Grid property)@\spxentry{element}\spxextra{tamos.elementIO.Grid property}}

\begin{fulllineitems}
\phantomsection\label{\detokenize{generated/tamos.elementIO.Grid:tamos.elementIO.Grid.element}}
\pysigstartsignatures
\pysigline{\sphinxbfcode{\sphinxupquote{property\DUrole{w}{  }}}\sphinxbfcode{\sphinxupquote{element}}}
\pysigstopsignatures
\sphinxAtStartPar
Element exchanged between the hub interface and this instance.

\end{fulllineitems}

\index{element\_cost (tamos.elementIO.Grid property)@\spxentry{element\_cost}\spxextra{tamos.elementIO.Grid property}}

\begin{fulllineitems}
\phantomsection\label{\detokenize{generated/tamos.elementIO.Grid:tamos.elementIO.Grid.element_cost}}
\pysigstartsignatures
\pysigline{\sphinxbfcode{\sphinxupquote{property\DUrole{w}{  }}}\sphinxbfcode{\sphinxupquote{element\_cost}}}
\pysigstopsignatures
\sphinxAtStartPar
Cost instance associated with a positive flow of ‘element’.
Cost instance

\end{fulllineitems}

\index{emissions (tamos.elementIO.Grid property)@\spxentry{emissions}\spxextra{tamos.elementIO.Grid property}}

\begin{fulllineitems}
\phantomsection\label{\detokenize{generated/tamos.elementIO.Grid:tamos.elementIO.Grid.emissions}}
\pysigstartsignatures
\pysigline{\sphinxbfcode{\sphinxupquote{property\DUrole{w}{  }}}\sphinxbfcode{\sphinxupquote{emissions}}}
\pysigstopsignatures
\sphinxAtStartPar
Quantity of CO2 associated with a positive power flow of \sphinxtitleref{element}.
In kgEqCO2/kWh.
Usually negative.
int, float or numpy.ndarray
\subsubsection*{Examples}

\sphinxAtStartPar
Examples below are for a Grid component, but Load components behaves similarly.

\begin{sphinxVerbatim}[commandchars=\\\{\}]
\PYG{g+gp}{\PYGZgt{}\PYGZgt{}\PYGZgt{} }\PYG{n}{natural\PYGZus{}gas\PYGZus{}grid} \PYG{o}{=} \PYG{n}{Grid}\PYG{p}{(}\PYG{n}{element}\PYG{o}{=}\PYG{n}{natural\PYGZus{}gas}\PYG{p}{)}
\PYG{g+gp}{\PYGZgt{}\PYGZgt{}\PYGZgt{} }\PYG{n}{natural\PYGZus{}gas\PYGZus{}grid}\PYG{o}{.}\PYG{n}{emissions} \PYG{o}{=} \PYG{o}{\PYGZhy{}} \PYG{l+m+mf}{0.25}
\end{sphinxVerbatim}

\sphinxAtStartPar
When the Grid component \sphinxtitleref{natural\_gas\_grid} receives 1 kWh of natural gas from the hub interface
the net CO2 emissions of the energy system decrease of 0.250 kgEqCO2.
Conversely, if the energy\_system receives 1 kWh of natural gas from \sphinxtitleref{natural\_gas\_grid},
its CO2 emissions increase of 0.250 kgEqCO2.

\begin{sphinxVerbatim}[commandchars=\\\{\}]
\PYG{g+gp}{\PYGZgt{}\PYGZgt{}\PYGZgt{} }\PYG{n}{thermal\PYGZus{}grid} \PYG{o}{=} \PYG{n}{Grid}\PYG{p}{(}\PYG{n}{element}\PYG{o}{=}\PYG{n}{thermal\PYGZus{}source}\PYG{p}{)}
\PYG{g+gp}{\PYGZgt{}\PYGZgt{}\PYGZgt{} }\PYG{n}{thermal\PYGZus{}grid}\PYG{o}{.}\PYG{n}{emissions} \PYG{o}{=} \PYG{o}{\PYGZhy{}} \PYG{l+m+mf}{0.05}
\end{sphinxVerbatim}

\sphinxAtStartPar
\sphinxtitleref{thermal\_grid} is such that when the power flow of \sphinxtitleref{thermal\_source} is 1 kWh (positive):
\begin{itemize}
\item {} 
\sphinxAtStartPar
thermal\_source.in\_TV enters the grid (i.e leaves the energy system)
and thermal\_source.out\_TV leaves the grid (i.e enters the energy\_system)

\item {} 
\sphinxAtStartPar
the net CO2 emissions of the energy system decrease of 50 gEqCO2.

\end{itemize}

\end{fulllineitems}

\index{exergy\_count (tamos.elementIO.Grid property)@\spxentry{exergy\_count}\spxextra{tamos.elementIO.Grid property}}

\begin{fulllineitems}
\phantomsection\label{\detokenize{generated/tamos.elementIO.Grid:tamos.elementIO.Grid.exergy_count}}
\pysigstartsignatures
\pysigline{\sphinxbfcode{\sphinxupquote{property\DUrole{w}{  }}}\sphinxbfcode{\sphinxupquote{exergy\_count}}}
\pysigstopsignatures
\sphinxAtStartPar
Whether this instance contributes to the system “Exergy” KPI.
bool

\end{fulllineitems}

\index{name (tamos.elementIO.Grid property)@\spxentry{name}\spxextra{tamos.elementIO.Grid property}}

\begin{fulllineitems}
\phantomsection\label{\detokenize{generated/tamos.elementIO.Grid:tamos.elementIO.Grid.name}}
\pysigstartsignatures
\pysigline{\sphinxbfcode{\sphinxupquote{property\DUrole{w}{  }}}\sphinxbfcode{\sphinxupquote{name}}}
\pysigstopsignatures
\sphinxAtStartPar
str.
This name is used in MILP model description.
names must not exceed 255 characters,
all of which must be alphanumeric (a\sphinxhyphen{}z, A\sphinxhyphen{}Z, 0\sphinxhyphen{}9) or one of these symbols:
! ” \# \$ \% \& , . ; ? @ \_ ‘ ’ \{ \} \textasciitilde{}.
\begin{quote}\begin{description}
\sphinxlineitem{Type}
\sphinxAtStartPar
Name of the instance

\end{description}\end{quote}

\end{fulllineitems}

\index{used\_elements (tamos.elementIO.Grid property)@\spxentry{used\_elements}\spxextra{tamos.elementIO.Grid property}}

\begin{fulllineitems}
\phantomsection\label{\detokenize{generated/tamos.elementIO.Grid:tamos.elementIO.Grid.used_elements}}
\pysigstartsignatures
\pysigline{\sphinxbfcode{\sphinxupquote{property\DUrole{w}{  }}}\sphinxbfcode{\sphinxupquote{used\_elements}}}
\pysigstopsignatures
\sphinxAtStartPar
Elements used by the component.

\end{fulllineitems}


\end{fulllineitems}


\sphinxstepscope


\section{tamos.elementIO.Load}
\label{\detokenize{generated/tamos.elementIO.Load:tamos-elementio-load}}\label{\detokenize{generated/tamos.elementIO.Load::doc}}\index{Load (class in tamos.elementIO)@\spxentry{Load}\spxextra{class in tamos.elementIO}}

\begin{fulllineitems}
\phantomsection\label{\detokenize{generated/tamos.elementIO.Load:tamos.elementIO.Load}}
\pysigstartsignatures
\pysiglinewithargsret{\sphinxbfcode{\sphinxupquote{class\DUrole{w}{  }}}\sphinxcode{\sphinxupquote{tamos.elementIO.}}\sphinxbfcode{\sphinxupquote{Load}}}{\emph{\DUrole{n}{load}}, \emph{\DUrole{n}{element}}, \emph{\DUrole{n}{emissions}\DUrole{o}{=}\DUrole{default_value}{None}}, \emph{\DUrole{n}{element\_cost}\DUrole{o}{=}\DUrole{default_value}{None}}, \emph{\DUrole{n}{exergy\_count}\DUrole{o}{=}\DUrole{default_value}{True}}, \emph{\DUrole{n}{name}\DUrole{o}{=}\DUrole{default_value}{None}}}{}
\pysigstopsignatures\index{\_\_init\_\_() (tamos.elementIO.Load method)@\spxentry{\_\_init\_\_()}\spxextra{tamos.elementIO.Load method}}

\begin{fulllineitems}
\phantomsection\label{\detokenize{generated/tamos.elementIO.Load:tamos.elementIO.Load.__init__}}
\pysigstartsignatures
\pysiglinewithargsret{\sphinxbfcode{\sphinxupquote{\_\_init\_\_}}}{\emph{\DUrole{n}{load}}, \emph{\DUrole{n}{element}}, \emph{\DUrole{n}{emissions}\DUrole{o}{=}\DUrole{default_value}{None}}, \emph{\DUrole{n}{element\_cost}\DUrole{o}{=}\DUrole{default_value}{None}}, \emph{\DUrole{n}{exergy\_count}\DUrole{o}{=}\DUrole{default_value}{True}}, \emph{\DUrole{n}{name}\DUrole{o}{=}\DUrole{default_value}{None}}}{}
\pysigstopsignatures
\sphinxAtStartPar
Allows constrained element exchanges between the energy system and its environment.

\sphinxAtStartPar
Load components are associated with the following exported decision variables:
\begin{itemize}
\item {} 
\sphinxAtStartPar
X\_EXT, binary.
Whether the Load instance is used by the hub.
To force the use of this instance by a hub, set components\_assemblies at Hub or MILPModel level.
For instance:

\begin{sphinxVerbatim}[commandchars=\\\{\}]
\PYG{g+gp}{\PYGZgt{}\PYGZgt{}\PYGZgt{} }\PYG{n}{hub}\PYG{o}{.}\PYG{n}{components\PYGZus{}assemblies} \PYG{o}{=} \PYG{p}{[}\PYG{p}{(}\PYG{l+m+mi}{1}\PYG{p}{,} \PYG{l+m+mi}{1}\PYG{p}{,} \PYG{n}{heating\PYGZus{}load}\PYG{p}{)}\PYG{p}{]}
\end{sphinxVerbatim}

\item {} 
\sphinxAtStartPar
For all t, F\_EXT(t), continuous, in kW.
The power related to \sphinxtitleref{element} entering the load (i.e. leaving the hub).

\end{itemize}

\sphinxAtStartPar
Load components declare the following KPIs:
\begin{itemize}
\item {} \begin{description}
\sphinxlineitem{\sphinxtitleref{ElementIO CO2}}
\sphinxAtStartPar
In kgEqCO2.
Defines the “CO2” objective function. Related to the \sphinxtitleref{emissions} attribute.

\end{description}

\item {} \begin{description}
\sphinxlineitem{\sphinxtitleref{ElementIO Exergy}}
\sphinxAtStartPar
In kWh.
Defines the “Exergy” objective function. Related to the \sphinxtitleref{exergy\_factor} attribute of \sphinxtitleref{element}.

\end{description}

\end{itemize}
\begin{quote}\begin{description}
\sphinxlineitem{Parameters}\begin{itemize}
\item {} 
\sphinxAtStartPar
\sphinxstyleliteralstrong{\sphinxupquote{load}} (\sphinxstyleliteralemphasis{\sphinxupquote{int}}\sphinxstyleliteralemphasis{\sphinxupquote{, }}\sphinxstyleliteralemphasis{\sphinxupquote{float}}\sphinxstyleliteralemphasis{\sphinxupquote{ or }}\sphinxstyleliteralemphasis{\sphinxupquote{numpy.ndarray}}) \textendash{} The power pattern that constrains element exchanges.
In kW.
Same sign than the flow of ‘element’.

\item {} 
\sphinxAtStartPar
\sphinxstyleliteralstrong{\sphinxupquote{element}} ({\hyperref[\detokenize{generated/tamos.element.ElectricityVector:tamos.element.ElectricityVector}]{\sphinxcrossref{\sphinxstyleliteralemphasis{\sphinxupquote{ElectricityVector}}}}}\sphinxstyleliteralemphasis{\sphinxupquote{, }}{\hyperref[\detokenize{generated/tamos.element.FuelVector:tamos.element.FuelVector}]{\sphinxcrossref{\sphinxstyleliteralemphasis{\sphinxupquote{FuelVector}}}}}\sphinxstyleliteralemphasis{\sphinxupquote{, }}{\hyperref[\detokenize{generated/tamos.element.ThermalVector:tamos.element.ThermalVector}]{\sphinxcrossref{\sphinxstyleliteralemphasis{\sphinxupquote{ThermalVector}}}}}\sphinxstyleliteralemphasis{\sphinxupquote{ or }}\sphinxstyleliteralemphasis{\sphinxupquote{ThermalVectorPair}}) \textendash{} 

\item {} 
\sphinxAtStartPar
\sphinxstyleliteralstrong{\sphinxupquote{emissions}} (\sphinxstyleliteralemphasis{\sphinxupquote{int}}\sphinxstyleliteralemphasis{\sphinxupquote{, }}\sphinxstyleliteralemphasis{\sphinxupquote{float}}\sphinxstyleliteralemphasis{\sphinxupquote{ or }}\sphinxstyleliteralemphasis{\sphinxupquote{numpy.ndarray}}\sphinxstyleliteralemphasis{\sphinxupquote{, }}\sphinxstyleliteralemphasis{\sphinxupquote{optional}}) \textendash{} Quantity of CO2 associated with a positive flow of \sphinxtitleref{element}.
In kgEqCO2/kWh.
Usually negative.
If None, emissions are not accounted for.

\item {} 
\sphinxAtStartPar
\sphinxstyleliteralstrong{\sphinxupquote{element\_cost}} ({\hyperref[\detokenize{generated/tamos.elementIO.Cost:tamos.elementIO.Cost}]{\sphinxcrossref{\sphinxstyleliteralemphasis{\sphinxupquote{Cost}}}}}\sphinxstyleliteralemphasis{\sphinxupquote{, }}\sphinxstyleliteralemphasis{\sphinxupquote{optional}}) \textendash{} Cost associated with a positive flow of ‘element’.
If None, no costs are taken into account.

\item {} 
\sphinxAtStartPar
\sphinxstyleliteralstrong{\sphinxupquote{exergy\_count}} (\sphinxstyleliteralemphasis{\sphinxupquote{bool}}\sphinxstyleliteralemphasis{\sphinxupquote{, }}\sphinxstyleliteralemphasis{\sphinxupquote{optional}}\sphinxstyleliteralemphasis{\sphinxupquote{, }}\sphinxstyleliteralemphasis{\sphinxupquote{default True}}) \textendash{} Whether this instance contributes to the system “Exergy” KPI.

\item {} 
\sphinxAtStartPar
\sphinxstyleliteralstrong{\sphinxupquote{name}} (\sphinxstyleliteralemphasis{\sphinxupquote{str}}\sphinxstyleliteralemphasis{\sphinxupquote{, }}\sphinxstyleliteralemphasis{\sphinxupquote{optional}}) \textendash{} 

\end{itemize}

\end{description}\end{quote}
\subsubsection*{Notes}

\sphinxAtStartPar
The Load class defines a multiplication operation to speed up the definition of several Load instances.
Multiplication makes a copy of \sphinxtitleref{load} attribute, so that mutable numpy.ndarray are made independent.
For instance:

\begin{sphinxVerbatim}[commandchars=\\\{\}]
\PYG{g+gp}{\PYGZgt{}\PYGZgt{}\PYGZgt{} }\PYG{n}{load\PYGZus{}1} \PYG{o}{=} \PYG{n}{Load}\PYG{p}{(}\PYG{n}{load}\PYG{o}{=}\PYG{o}{\PYGZhy{}}\PYG{l+m+mi}{4}\PYG{p}{,} \PYG{n}{element}\PYG{o}{=}\PYG{n}{element1}\PYG{p}{,} \PYG{n}{emissions}\PYG{o}{=}\PYG{o}{\PYGZhy{}}\PYG{l+m+mf}{0.2}\PYG{p}{,} \PYG{n}{exergy\PYGZus{}count}\PYG{o}{=}\PYG{k+kc}{False}\PYG{p}{)}
\PYG{g+gp}{\PYGZgt{}\PYGZgt{}\PYGZgt{} }\PYG{n}{load\PYGZus{}2} \PYG{o}{=} \PYG{l+m+mi}{3} \PYG{o}{*} \PYG{n}{load\PYGZus{}1}
\PYG{g+gp}{\PYGZgt{}\PYGZgt{}\PYGZgt{} }\PYG{n}{load\PYGZus{}2}\PYG{o}{.}\PYG{n}{load}
\PYG{g+go}{ \PYGZhy{}12}
\PYG{g+gp}{\PYGZgt{}\PYGZgt{}\PYGZgt{} }\PYG{n}{load\PYGZus{}2}\PYG{o}{.}\PYG{n}{element} \PYG{o+ow}{is} \PYG{n}{element1}
\PYG{g+go}{ True}
\end{sphinxVerbatim}
\subsubsection*{Examples}

\begin{sphinxVerbatim}[commandchars=\\\{\}]
\PYG{g+gp}{\PYGZgt{}\PYGZgt{}\PYGZgt{} }\PYG{n}{Load}\PYG{p}{(}\PYG{n}{load}\PYG{o}{=}\PYG{l+m+mi}{10}\PYG{p}{,} \PYG{n}{element}\PYG{o}{=}\PYG{n}{electricity}\PYG{p}{)}
\end{sphinxVerbatim}

\sphinxAtStartPar
A constant electrical power of 10 kW will leave the hub interface.

\begin{sphinxVerbatim}[commandchars=\\\{\}]
\PYG{g+gp}{\PYGZgt{}\PYGZgt{}\PYGZgt{} }\PYG{n}{thermal\PYGZus{}source} \PYG{o}{=} \PYG{n}{fetch\PYGZus{}TVP}\PYG{p}{(}\PYG{n}{in\PYGZus{}TV}\PYG{o}{=}\PYG{n}{in\PYGZus{}TV}\PYG{p}{,} \PYG{n}{out\PYGZus{}TV}\PYG{o}{=}\PYG{n}{out\PYGZus{}TV}\PYG{p}{)}
\PYG{g+gp}{\PYGZgt{}\PYGZgt{}\PYGZgt{} }\PYG{n}{load\PYGZus{}1} \PYG{o}{=} \PYG{n}{Load}\PYG{p}{(}\PYG{n}{load}\PYG{o}{=}\PYG{o}{\PYGZhy{}}\PYG{l+m+mi}{5}\PYG{p}{,} \PYG{n}{element}\PYG{o}{=}\PYG{n}{thermal\PYGZus{}source}\PYG{p}{)}
\PYG{g+gp}{\PYGZgt{}\PYGZgt{}\PYGZgt{} }\PYG{n}{load\PYGZus{}2} \PYG{o}{=} \PYG{n}{Load}\PYG{p}{(}\PYG{n}{load}\PYG{o}{=}\PYG{l+m+mi}{5}\PYG{p}{,} \PYG{n}{element}\PYG{o}{=}\PYG{o}{\PYGZti{}}\PYG{n}{thermal\PYGZus{}source}\PYG{p}{)}
\end{sphinxVerbatim}

\sphinxAtStartPar
In both cases of load\_1 and load\_2, a constant power of 5 kW is exchanged between the hub interface and the Load component,
with \sphinxtitleref{out\_TV} entering the hub interface (i.e. leaving the Load component).

\end{fulllineitems}

\subsubsection*{Methods}


\begin{savenotes}\sphinxattablestart
\centering
\begin{tabulary}{\linewidth}[t]{\X{1}{2}\X{1}{2}}
\hline

\sphinxAtStartPar
{\hyperref[\detokenize{generated/tamos.elementIO.Load:tamos.elementIO.Load.__init__}]{\sphinxcrossref{\sphinxcode{\sphinxupquote{\_\_init\_\_}}}}}(load, element{[}, emissions, ...{]})
&
\sphinxAtStartPar
Allows constrained element exchanges between the energy system and its environment.
\\
\hline
\sphinxAtStartPar
{\hyperref[\detokenize{generated/tamos.elementIO.Load:tamos.elementIO.Load.compute_actualized_cost}]{\sphinxcrossref{\sphinxcode{\sphinxupquote{compute\_actualized\_cost}}}}}(CAPEX, OPEX, ...{[}, ...{]})
&
\sphinxAtStartPar
Computes the cost of a component using its \textquotesingle{}Lifetime\textquotesingle{} and \textquotesingle{}Discount rate (\%)\textquotesingle{} properties.
\\
\hline
\end{tabulary}
\par
\sphinxattableend\end{savenotes}
\subsubsection*{Attributes}


\begin{savenotes}\sphinxattablestart
\centering
\begin{tabulary}{\linewidth}[t]{\X{1}{2}\X{1}{2}}
\hline

\sphinxAtStartPar
{\hyperref[\detokenize{generated/tamos.elementIO.Load:tamos.elementIO.Load.element}]{\sphinxcrossref{\sphinxcode{\sphinxupquote{element}}}}}
&
\sphinxAtStartPar
Element exchanged between the hub interface and this instance.
\\
\hline
\sphinxAtStartPar
{\hyperref[\detokenize{generated/tamos.elementIO.Load:tamos.elementIO.Load.element_cost}]{\sphinxcrossref{\sphinxcode{\sphinxupquote{element\_cost}}}}}
&
\sphinxAtStartPar
Cost instance associated with a positive flow of \textquotesingle{}element\textquotesingle{}.
\\
\hline
\sphinxAtStartPar
{\hyperref[\detokenize{generated/tamos.elementIO.Load:tamos.elementIO.Load.emissions}]{\sphinxcrossref{\sphinxcode{\sphinxupquote{emissions}}}}}
&
\sphinxAtStartPar
Quantity of CO2 associated with a positive power flow of \sphinxtitleref{element}.
\\
\hline
\sphinxAtStartPar
{\hyperref[\detokenize{generated/tamos.elementIO.Load:tamos.elementIO.Load.exergy_count}]{\sphinxcrossref{\sphinxcode{\sphinxupquote{exergy\_count}}}}}
&
\sphinxAtStartPar
Whether this instance contributes to the system "Exergy" KPI.
\\
\hline
\sphinxAtStartPar
{\hyperref[\detokenize{generated/tamos.elementIO.Load:tamos.elementIO.Load.load}]{\sphinxcrossref{\sphinxcode{\sphinxupquote{load}}}}}
&
\sphinxAtStartPar
The power pattern that constrains element exchanges.
\\
\hline
\sphinxAtStartPar
{\hyperref[\detokenize{generated/tamos.elementIO.Load:tamos.elementIO.Load.name}]{\sphinxcrossref{\sphinxcode{\sphinxupquote{name}}}}}
&
\sphinxAtStartPar
str.
\\
\hline
\sphinxAtStartPar
{\hyperref[\detokenize{generated/tamos.elementIO.Load:tamos.elementIO.Load.used_elements}]{\sphinxcrossref{\sphinxcode{\sphinxupquote{used\_elements}}}}}
&
\sphinxAtStartPar
Elements used by the component.
\\
\hline
\end{tabulary}
\par
\sphinxattableend\end{savenotes}
\index{compute\_actualized\_cost() (tamos.elementIO.Load method)@\spxentry{compute\_actualized\_cost()}\spxextra{tamos.elementIO.Load method}}

\begin{fulllineitems}
\phantomsection\label{\detokenize{generated/tamos.elementIO.Load:tamos.elementIO.Load.compute_actualized_cost}}
\pysigstartsignatures
\pysiglinewithargsret{\sphinxbfcode{\sphinxupquote{compute\_actualized\_cost}}}{\emph{\DUrole{n}{CAPEX}}, \emph{\DUrole{n}{OPEX}}, \emph{\DUrole{n}{system\_lifetime}}, \emph{\DUrole{n}{lifetime}\DUrole{o}{=}\DUrole{default_value}{None}}, \emph{\DUrole{n}{discount\_rate}\DUrole{o}{=}\DUrole{default_value}{None}}}{}
\pysigstopsignatures
\sphinxAtStartPar
Computes the cost of a component using its ‘Lifetime’ and ‘Discount rate (\%)’ properties.
\begin{quote}\begin{description}
\sphinxlineitem{Parameters}\begin{itemize}
\item {} 
\sphinxAtStartPar
\sphinxstyleliteralstrong{\sphinxupquote{CAPEX}} (\sphinxstyleliteralemphasis{\sphinxupquote{float}}) \textendash{} Capital Expenditure. Cost in euros paid every \sphinxtitleref{technical\_lifetime} periods.

\item {} 
\sphinxAtStartPar
\sphinxstyleliteralstrong{\sphinxupquote{OPEX}} (\sphinxstyleliteralemphasis{\sphinxupquote{float}}) \textendash{} Operational Expenditure. Cost in euros paid each period.

\item {} 
\sphinxAtStartPar
\sphinxstyleliteralstrong{\sphinxupquote{system\_lifetime}} (\sphinxstyleliteralemphasis{\sphinxupquote{int}}) \textendash{} Number of periods defining the existence of the energy system.

\item {} 
\sphinxAtStartPar
\sphinxstyleliteralstrong{\sphinxupquote{lifetime}} (\sphinxstyleliteralemphasis{\sphinxupquote{int}}\sphinxstyleliteralemphasis{\sphinxupquote{, }}\sphinxstyleliteralemphasis{\sphinxupquote{optional}}) \textendash{} Number of periods defining the existence of the component.
If specified, overwrite the “Lifetime” property.

\item {} 
\sphinxAtStartPar
\sphinxstyleliteralstrong{\sphinxupquote{discount\_rate}} (\sphinxstyleliteralemphasis{\sphinxupquote{float}}) \textendash{} In percent (\%). Describes the importance of the economic amortization process, per period.
If specified, overwrite the “Discount rate (\%)” property.

\end{itemize}

\sphinxlineitem{Returns}
\sphinxAtStartPar
\begin{itemize}
\item {} 
\sphinxAtStartPar
\sphinxstyleemphasis{A 3\sphinxhyphen{}tuple (total\_cost, CAPEX\_share, OPEX\_share) where}

\item {} 
\sphinxAtStartPar
* CAPEX\_share is the share of total cost related to \sphinxtitleref{CAPEX}

\item {} 
\sphinxAtStartPar
* OPEX\_share is the share of total cost related to \sphinxtitleref{OPEX}

\item {} 
\sphinxAtStartPar
\sphinxstyleemphasis{* total\_cost = CAPEX\_share + OPEX\_share}

\end{itemize}


\end{description}\end{quote}
\subsubsection*{Notes}

\sphinxAtStartPar
Takes into account residual value of component in the case \sphinxtitleref{system\_lifetime} is not a multiple of \sphinxtitleref{lifetime}.
In this case, the last replacement occuring at period replacement\_period is paid in proportion of ‘CAPEX’
depending linearly on the number of periods left:
CAPEX * (system\_lifetime \sphinxhyphen{} replacement\_period) / lifetime

\end{fulllineitems}

\index{element (tamos.elementIO.Load property)@\spxentry{element}\spxextra{tamos.elementIO.Load property}}

\begin{fulllineitems}
\phantomsection\label{\detokenize{generated/tamos.elementIO.Load:tamos.elementIO.Load.element}}
\pysigstartsignatures
\pysigline{\sphinxbfcode{\sphinxupquote{property\DUrole{w}{  }}}\sphinxbfcode{\sphinxupquote{element}}}
\pysigstopsignatures
\sphinxAtStartPar
Element exchanged between the hub interface and this instance.

\end{fulllineitems}

\index{element\_cost (tamos.elementIO.Load property)@\spxentry{element\_cost}\spxextra{tamos.elementIO.Load property}}

\begin{fulllineitems}
\phantomsection\label{\detokenize{generated/tamos.elementIO.Load:tamos.elementIO.Load.element_cost}}
\pysigstartsignatures
\pysigline{\sphinxbfcode{\sphinxupquote{property\DUrole{w}{  }}}\sphinxbfcode{\sphinxupquote{element\_cost}}}
\pysigstopsignatures
\sphinxAtStartPar
Cost instance associated with a positive flow of ‘element’.
Cost instance

\end{fulllineitems}

\index{emissions (tamos.elementIO.Load property)@\spxentry{emissions}\spxextra{tamos.elementIO.Load property}}

\begin{fulllineitems}
\phantomsection\label{\detokenize{generated/tamos.elementIO.Load:tamos.elementIO.Load.emissions}}
\pysigstartsignatures
\pysigline{\sphinxbfcode{\sphinxupquote{property\DUrole{w}{  }}}\sphinxbfcode{\sphinxupquote{emissions}}}
\pysigstopsignatures
\sphinxAtStartPar
Quantity of CO2 associated with a positive power flow of \sphinxtitleref{element}.
In kgEqCO2/kWh.
Usually negative.
int, float or numpy.ndarray
\subsubsection*{Examples}

\sphinxAtStartPar
Examples below are for a Grid component, but Load components behaves similarly.

\begin{sphinxVerbatim}[commandchars=\\\{\}]
\PYG{g+gp}{\PYGZgt{}\PYGZgt{}\PYGZgt{} }\PYG{n}{natural\PYGZus{}gas\PYGZus{}grid} \PYG{o}{=} \PYG{n}{Grid}\PYG{p}{(}\PYG{n}{element}\PYG{o}{=}\PYG{n}{natural\PYGZus{}gas}\PYG{p}{)}
\PYG{g+gp}{\PYGZgt{}\PYGZgt{}\PYGZgt{} }\PYG{n}{natural\PYGZus{}gas\PYGZus{}grid}\PYG{o}{.}\PYG{n}{emissions} \PYG{o}{=} \PYG{o}{\PYGZhy{}} \PYG{l+m+mf}{0.25}
\end{sphinxVerbatim}

\sphinxAtStartPar
When the Grid component \sphinxtitleref{natural\_gas\_grid} receives 1 kWh of natural gas from the hub interface
the net CO2 emissions of the energy system decrease of 0.250 kgEqCO2.
Conversely, if the energy\_system receives 1 kWh of natural gas from \sphinxtitleref{natural\_gas\_grid},
its CO2 emissions increase of 0.250 kgEqCO2.

\begin{sphinxVerbatim}[commandchars=\\\{\}]
\PYG{g+gp}{\PYGZgt{}\PYGZgt{}\PYGZgt{} }\PYG{n}{thermal\PYGZus{}grid} \PYG{o}{=} \PYG{n}{Grid}\PYG{p}{(}\PYG{n}{element}\PYG{o}{=}\PYG{n}{thermal\PYGZus{}source}\PYG{p}{)}
\PYG{g+gp}{\PYGZgt{}\PYGZgt{}\PYGZgt{} }\PYG{n}{thermal\PYGZus{}grid}\PYG{o}{.}\PYG{n}{emissions} \PYG{o}{=} \PYG{o}{\PYGZhy{}} \PYG{l+m+mf}{0.05}
\end{sphinxVerbatim}

\sphinxAtStartPar
\sphinxtitleref{thermal\_grid} is such that when the power flow of \sphinxtitleref{thermal\_source} is 1 kWh (positive):
\begin{itemize}
\item {} 
\sphinxAtStartPar
thermal\_source.in\_TV enters the grid (i.e leaves the energy system)
and thermal\_source.out\_TV leaves the grid (i.e enters the energy\_system)

\item {} 
\sphinxAtStartPar
the net CO2 emissions of the energy system decrease of 50 gEqCO2.

\end{itemize}

\end{fulllineitems}

\index{exergy\_count (tamos.elementIO.Load property)@\spxentry{exergy\_count}\spxextra{tamos.elementIO.Load property}}

\begin{fulllineitems}
\phantomsection\label{\detokenize{generated/tamos.elementIO.Load:tamos.elementIO.Load.exergy_count}}
\pysigstartsignatures
\pysigline{\sphinxbfcode{\sphinxupquote{property\DUrole{w}{  }}}\sphinxbfcode{\sphinxupquote{exergy\_count}}}
\pysigstopsignatures
\sphinxAtStartPar
Whether this instance contributes to the system “Exergy” KPI.
bool

\end{fulllineitems}

\index{load (tamos.elementIO.Load property)@\spxentry{load}\spxextra{tamos.elementIO.Load property}}

\begin{fulllineitems}
\phantomsection\label{\detokenize{generated/tamos.elementIO.Load:tamos.elementIO.Load.load}}
\pysigstartsignatures
\pysigline{\sphinxbfcode{\sphinxupquote{property\DUrole{w}{  }}}\sphinxbfcode{\sphinxupquote{load}}}
\pysigstopsignatures
\sphinxAtStartPar
The power pattern that constrains element exchanges.
In kW.
Same sign than the flow of ‘element’.
int, float or numpy.ndarray

\end{fulllineitems}

\index{name (tamos.elementIO.Load property)@\spxentry{name}\spxextra{tamos.elementIO.Load property}}

\begin{fulllineitems}
\phantomsection\label{\detokenize{generated/tamos.elementIO.Load:tamos.elementIO.Load.name}}
\pysigstartsignatures
\pysigline{\sphinxbfcode{\sphinxupquote{property\DUrole{w}{  }}}\sphinxbfcode{\sphinxupquote{name}}}
\pysigstopsignatures
\sphinxAtStartPar
str.
This name is used in MILP model description.
names must not exceed 255 characters,
all of which must be alphanumeric (a\sphinxhyphen{}z, A\sphinxhyphen{}Z, 0\sphinxhyphen{}9) or one of these symbols:
! ” \# \$ \% \& , . ; ? @ \_ ‘ ’ \{ \} \textasciitilde{}.
\begin{quote}\begin{description}
\sphinxlineitem{Type}
\sphinxAtStartPar
Name of the instance

\end{description}\end{quote}

\end{fulllineitems}

\index{used\_elements (tamos.elementIO.Load property)@\spxentry{used\_elements}\spxextra{tamos.elementIO.Load property}}

\begin{fulllineitems}
\phantomsection\label{\detokenize{generated/tamos.elementIO.Load:tamos.elementIO.Load.used_elements}}
\pysigstartsignatures
\pysigline{\sphinxbfcode{\sphinxupquote{property\DUrole{w}{  }}}\sphinxbfcode{\sphinxupquote{used\_elements}}}
\pysigstopsignatures
\sphinxAtStartPar
Elements used by the component.

\end{fulllineitems}


\end{fulllineitems}


\sphinxstepscope


\chapter{Network components}
\label{\detokenize{network_components:network-components}}\label{\detokenize{network_components::doc}}

\section{Components}
\label{\detokenize{network_components:components}}

\begin{savenotes}\sphinxattablestart
\centering
\begin{tabulary}{\linewidth}[t]{\X{1}{2}\X{1}{2}}
\hline

\sphinxAtStartPar
{\hyperref[\detokenize{generated/tamos.network.NonThermalNetwork:tamos.network.NonThermalNetwork}]{\sphinxcrossref{\sphinxcode{\sphinxupquote{tamos.network.NonThermalNetwork}}}}}(...{[}, ...{]})
&
\sphinxAtStartPar

\\
\hline
\sphinxAtStartPar
{\hyperref[\detokenize{generated/tamos.network.ThermalNetwork:tamos.network.ThermalNetwork}]{\sphinxcrossref{\sphinxcode{\sphinxupquote{tamos.network.ThermalNetwork}}}}}(hubs\_locations, ...)
&
\sphinxAtStartPar

\\
\hline
\sphinxAtStartPar
{\hyperref[\detokenize{generated/tamos.network.HREThermalNetwork:tamos.network.HREThermalNetwork}]{\sphinxcrossref{\sphinxcode{\sphinxupquote{tamos.network.HREThermalNetwork}}}}}(...{[}, ...{]})
&
\sphinxAtStartPar

\\
\hline
\end{tabulary}
\par
\sphinxattableend\end{savenotes}

\sphinxstepscope


\subsection{tamos.network.NonThermalNetwork}
\label{\detokenize{generated/tamos.network.NonThermalNetwork:tamos-network-nonthermalnetwork}}\label{\detokenize{generated/tamos.network.NonThermalNetwork::doc}}\index{NonThermalNetwork (class in tamos.network)@\spxentry{NonThermalNetwork}\spxextra{class in tamos.network}}

\begin{fulllineitems}
\phantomsection\label{\detokenize{generated/tamos.network.NonThermalNetwork:tamos.network.NonThermalNetwork}}
\pysigstartsignatures
\pysiglinewithargsret{\sphinxbfcode{\sphinxupquote{class\DUrole{w}{  }}}\sphinxcode{\sphinxupquote{tamos.network.}}\sphinxbfcode{\sphinxupquote{NonThermalNetwork}}}{\emph{\DUrole{n}{hubs\_locations}}, \emph{\DUrole{n}{element}}, \emph{\DUrole{n}{properties}}, \emph{\DUrole{n}{production\_hub}}, \emph{\DUrole{n}{scale\_factor}\DUrole{o}{=}\DUrole{default_value}{1}}, \emph{\DUrole{n}{eco\_count}\DUrole{o}{=}\DUrole{default_value}{True}}, \emph{\DUrole{n}{name}\DUrole{o}{=}\DUrole{default_value}{None}}}{}
\pysigstopsignatures\index{\_\_init\_\_() (tamos.network.NonThermalNetwork method)@\spxentry{\_\_init\_\_()}\spxextra{tamos.network.NonThermalNetwork method}}

\begin{fulllineitems}
\phantomsection\label{\detokenize{generated/tamos.network.NonThermalNetwork:tamos.network.NonThermalNetwork.__init__}}
\pysigstartsignatures
\pysiglinewithargsret{\sphinxbfcode{\sphinxupquote{\_\_init\_\_}}}{\emph{\DUrole{n}{hubs\_locations}}, \emph{\DUrole{n}{element}}, \emph{\DUrole{n}{properties}}, \emph{\DUrole{n}{production\_hub}}, \emph{\DUrole{n}{scale\_factor}\DUrole{o}{=}\DUrole{default_value}{1}}, \emph{\DUrole{n}{eco\_count}\DUrole{o}{=}\DUrole{default_value}{True}}, \emph{\DUrole{n}{name}\DUrole{o}{=}\DUrole{default_value}{None}}}{}
\pysigstopsignatures
\sphinxAtStartPar
NonThermalNetwork instances makes possible to share ElectricityVector or FuelVector elements between hubs.

\sphinxAtStartPar
Power is exchanged between two hubs, given a distribution loss related to the “Losses (\%/km)” property.
All distribution losses must be compensated for by an additional power in \sphinxtitleref{production\_hub}.

\sphinxAtStartPar
NonThermalNetwork components are associated with the following exported decision variables:
\begin{itemize}
\item {} 
\sphinxAtStartPar
X\_N(hub\_1, hub\_2), binary.
Whether a connection from hub \sphinxtitleref{hub\_1} to hub \sphinxtitleref{hub\_2} exists and allows a flow of \sphinxtitleref{element}.
Note that X\_N(hub\_1, hub\_2) is different from X\_N(hub\_2, hub\_1).

\item {} 
\sphinxAtStartPar
Y\_N(hub\_1, hub\_2), binary.
Whether a connection between hubs \sphinxtitleref{hub\_1} and \sphinxtitleref{hub\_2} exists and allows a flow of \sphinxtitleref{element}.

\item {} 
\sphinxAtStartPar
X\_SYS(hub), binary.
Whether the hub \sphinxtitleref{hub} is connected to at least one other hub, no matter the direction of the connection.

\item {} 
\sphinxAtStartPar
For all t, F\_SYS(hub, t), continuous, in kW.
The power related to \sphinxtitleref{element} going from hub \sphinxtitleref{hub} to the network.

\item {} 
\sphinxAtStartPar
For all t, F\_N(hub\_1, hub\_2, t), continuous, in kW.
The power related to \sphinxtitleref{element} going from hub \sphinxtitleref{hub\_1} to hub \sphinxtitleref{hub\_2} through the network.
Note that F\_N(hub\_1, hub\_2, t) is the opposite of F\_N(hub\_2, hub\_1, t).

\end{itemize}

\sphinxAtStartPar
NonThermalNetwork components declare the following KPIs:
\begin{itemize}
\item {} 
\sphinxAtStartPar
\sphinxtitleref{COST\_network}
In euros.
Contributes to the “Eco” objective function.

\end{itemize}
\begin{quote}\begin{description}
\sphinxlineitem{Parameters}\begin{itemize}
\item {} 
\sphinxAtStartPar
\sphinxstyleliteralstrong{\sphinxupquote{hubs\_locations}} (\sphinxstyleliteralemphasis{\sphinxupquote{dict \{Hub:}}\sphinxstyleliteralemphasis{\sphinxupquote{ (}}\sphinxstyleliteralemphasis{\sphinxupquote{float}}\sphinxstyleliteralemphasis{\sphinxupquote{, }}\sphinxstyleliteralemphasis{\sphinxupquote{float}}\sphinxstyleliteralemphasis{\sphinxupquote{)}}\sphinxstyleliteralemphasis{\sphinxupquote{\}}}) \textendash{} \begin{itemize}
\item {} 
\sphinxAtStartPar
Keys of \sphinxtitleref{hubs\_locations} are the hubs possibly connected by the network.
They define the \sphinxtitleref{hubs} attribute.

\item {} 
\sphinxAtStartPar
Values of \sphinxtitleref{hubs\_locations} define x and y coordinates in space.
In km.
They describe the position of the hub given the absolute reference (0, 0).
Used to calculate distance between two hubs. The used distance function can be accessed by the
\sphinxtitleref{get\_distance\_function} function and set using \sphinxtitleref{set\_distance\_function} from tamos.network.

\end{itemize}


\item {} 
\sphinxAtStartPar
\sphinxstyleliteralstrong{\sphinxupquote{element}} ({\hyperref[\detokenize{generated/tamos.element.ElectricityVector:tamos.element.ElectricityVector}]{\sphinxcrossref{\sphinxstyleliteralemphasis{\sphinxupquote{ElectricityVector}}}}}\sphinxstyleliteralemphasis{\sphinxupquote{, }}{\hyperref[\detokenize{generated/tamos.element.FuelVector:tamos.element.FuelVector}]{\sphinxcrossref{\sphinxstyleliteralemphasis{\sphinxupquote{FuelVector}}}}}) \textendash{} Element exchanged between \sphinxtitleref{hubs}.

\item {} 
\sphinxAtStartPar
\sphinxstyleliteralstrong{\sphinxupquote{properties}} (\sphinxstyleliteralemphasis{\sphinxupquote{dict \{str: int}}\sphinxstyleliteralemphasis{\sphinxupquote{ | }}\sphinxstyleliteralemphasis{\sphinxupquote{float\}}}) \textendash{} 
\sphinxAtStartPar
Techno\sphinxhyphen{}economic properties of the network.
The \sphinxtitleref{properties} attribute must include the following keys:
\begin{itemize}
\item {} 
\sphinxAtStartPar
”Losses (\%/km)”

\item {} 
\sphinxAtStartPar
”CAPEX (EUR/km)”

\item {} 
\sphinxAtStartPar
”OPEX (\%CAPEX)”

\end{itemize}


\item {} 
\sphinxAtStartPar
\sphinxstyleliteralstrong{\sphinxupquote{production\_hub}} ({\hyperref[\detokenize{generated/tamos.Hub:tamos.Hub}]{\sphinxcrossref{\sphinxstyleliteralemphasis{\sphinxupquote{Hub}}}}}) \textendash{} The hub that bears all the distribution losses of the network.
Must be one of \sphinxtitleref{hubs} and must be able to send \sphinxtitleref{element} to the network.

\item {} 
\sphinxAtStartPar
\sphinxstyleliteralstrong{\sphinxupquote{scale\_factor}} (\sphinxstyleliteralemphasis{\sphinxupquote{float}}\sphinxstyleliteralemphasis{\sphinxupquote{, }}\sphinxstyleliteralemphasis{\sphinxupquote{optional}}\sphinxstyleliteralemphasis{\sphinxupquote{, }}\sphinxstyleliteralemphasis{\sphinxupquote{default 1}}) \textendash{} Multiplies the two coordinates of each hub in \sphinxtitleref{hubs\_locations}.
A scale factor \textgreater{}1 tends to increase the distance between two hubs.

\item {} 
\sphinxAtStartPar
\sphinxstyleliteralstrong{\sphinxupquote{eco\_count}} (\sphinxstyleliteralemphasis{\sphinxupquote{bool}}\sphinxstyleliteralemphasis{\sphinxupquote{, }}\sphinxstyleliteralemphasis{\sphinxupquote{optional}}\sphinxstyleliteralemphasis{\sphinxupquote{, }}\sphinxstyleliteralemphasis{\sphinxupquote{default True}}) \textendash{} Whether this instance contributes to the system “Eco” KPI.

\item {} 
\sphinxAtStartPar
\sphinxstyleliteralstrong{\sphinxupquote{name}} (\sphinxstyleliteralemphasis{\sphinxupquote{str}}\sphinxstyleliteralemphasis{\sphinxupquote{, }}\sphinxstyleliteralemphasis{\sphinxupquote{optional}}) \textendash{} 

\end{itemize}

\end{description}\end{quote}
\subsubsection*{Notes}
\begin{enumerate}
\sphinxsetlistlabels{\arabic}{enumi}{enumii}{}{.}%
\item {} 
\sphinxAtStartPar
A network component describe an oriented graph where each node is a hub and each edge is a connection between two hubs.

\item {} 
\sphinxAtStartPar
Connection status between two hubs can be defined using the \sphinxtitleref{set\_connection\_status} and \sphinxtitleref{set\_node\_status} methods.
Per default, all pairs of hubs are connected according to the status \sphinxtitleref{no\_connection}.

\item {} 
\sphinxAtStartPar
The connection of production hub to every other hub using the network is not mandatory,
i.e. nothing constrains the network graph to be connected.

\item {} 
\sphinxAtStartPar
The way distribution losses are declared and applied to \sphinxtitleref{production\_hub} in the MILP formalism gives no upper bound for
these losses. This implementation is chosen to prevent the use of the absolute value of a continuous variable,
which comes with a binary variable.
The consequence of this implementation is that problem where large losses benefit to the minimization of
the objective function (unlikely to happen) might get a solution with no physical meaning.

\end{enumerate}

\end{fulllineitems}

\subsubsection*{Methods}


\begin{savenotes}\sphinxattablestart
\centering
\begin{tabulary}{\linewidth}[t]{\X{1}{2}\X{1}{2}}
\hline

\sphinxAtStartPar
{\hyperref[\detokenize{generated/tamos.network.NonThermalNetwork:tamos.network.NonThermalNetwork.__init__}]{\sphinxcrossref{\sphinxcode{\sphinxupquote{\_\_init\_\_}}}}}(hubs\_locations, element, ...{[}, ...{]})
&
\sphinxAtStartPar
NonThermalNetwork instances makes possible to share ElectricityVector or FuelVector elements between hubs.
\\
\hline
\sphinxAtStartPar
{\hyperref[\detokenize{generated/tamos.network.NonThermalNetwork:tamos.network.NonThermalNetwork.compute_actualized_cost}]{\sphinxcrossref{\sphinxcode{\sphinxupquote{compute\_actualized\_cost}}}}}(CAPEX, OPEX, ...{[}, ...{]})
&
\sphinxAtStartPar
Computes the cost of a component using its \textquotesingle{}Lifetime\textquotesingle{} and \textquotesingle{}Discount rate (\%)\textquotesingle{} properties.
\\
\hline
\sphinxAtStartPar
{\hyperref[\detokenize{generated/tamos.network.NonThermalNetwork:tamos.network.NonThermalNetwork.generate_MSP}]{\sphinxcrossref{\sphinxcode{\sphinxupquote{generate\_MSP}}}}}({[}excluded\_hubs{]})
&
\sphinxAtStartPar
Defines a minimum spanning tree (MSP) linking all hubs of \sphinxtitleref{hubs} that are not in \sphinxtitleref{excluded\_hubs}.
\\
\hline
\sphinxAtStartPar
{\hyperref[\detokenize{generated/tamos.network.NonThermalNetwork:tamos.network.NonThermalNetwork.get_connection_power_bounds}]{\sphinxcrossref{\sphinxcode{\sphinxupquote{get\_connection\_power\_bounds}}}}}(hub\_1, hub\_2)
&
\sphinxAtStartPar
Returns the power limits of the flow of element from hub \sphinxtitleref{hub\_1} to hub \sphinxtitleref{hub\_2}.
\\
\hline
\sphinxAtStartPar
{\hyperref[\detokenize{generated/tamos.network.NonThermalNetwork:tamos.network.NonThermalNetwork.get_connection_status}]{\sphinxcrossref{\sphinxcode{\sphinxupquote{get\_connection\_status}}}}}(hub\_1, hub\_2)
&
\sphinxAtStartPar
Returns the status of the directional connection between two hubs.
\\
\hline
\sphinxAtStartPar
{\hyperref[\detokenize{generated/tamos.network.NonThermalNetwork:tamos.network.NonThermalNetwork.get_distance}]{\sphinxcrossref{\sphinxcode{\sphinxupquote{get\_distance}}}}}(hub\_1, hub\_2)
&
\sphinxAtStartPar
Returns the distance between two hubs.
\\
\hline
\sphinxAtStartPar
{\hyperref[\detokenize{generated/tamos.network.NonThermalNetwork:tamos.network.NonThermalNetwork.might_connect}]{\sphinxcrossref{\sphinxcode{\sphinxupquote{might\_connect}}}}}(hub)
&
\sphinxAtStartPar
Checks whether a hub is able to connect to the network.
\\
\hline
\sphinxAtStartPar
{\hyperref[\detokenize{generated/tamos.network.NonThermalNetwork:tamos.network.NonThermalNetwork.plot}]{\sphinxcrossref{\sphinxcode{\sphinxupquote{plot}}}}}()
&
\sphinxAtStartPar
Plots a representation of the network hubs and connections in the (x, y) space.
\\
\hline
\sphinxAtStartPar
{\hyperref[\detokenize{generated/tamos.network.NonThermalNetwork:tamos.network.NonThermalNetwork.set_connection_power_bounds}]{\sphinxcrossref{\sphinxcode{\sphinxupquote{set\_connection\_power\_bounds}}}}}(hub\_1, hub\_2{[}, ...{]})
&
\sphinxAtStartPar
Sets power limits on the flow of element from hub \sphinxtitleref{hub\_1} to hub \sphinxtitleref{hub\_2}.
\\
\hline
\sphinxAtStartPar
{\hyperref[\detokenize{generated/tamos.network.NonThermalNetwork:tamos.network.NonThermalNetwork.set_connection_status}]{\sphinxcrossref{\sphinxcode{\sphinxupquote{set\_connection\_status}}}}}(hub\_1, hub\_2, status)
&
\sphinxAtStartPar
Defines the connection status of the edge going from \sphinxtitleref{hub\_1} to \sphinxtitleref{hub\_2}.
\\
\hline
\sphinxAtStartPar
{\hyperref[\detokenize{generated/tamos.network.NonThermalNetwork:tamos.network.NonThermalNetwork.set_node_status}]{\sphinxcrossref{\sphinxcode{\sphinxupquote{set\_node\_status}}}}}(hub, status)
&
\sphinxAtStartPar
Defines the connection status of every incoming and outcoming edge of a hub.
\\
\hline
\sphinxAtStartPar
{\hyperref[\detokenize{generated/tamos.network.NonThermalNetwork:tamos.network.NonThermalNetwork.set_status}]{\sphinxcrossref{\sphinxcode{\sphinxupquote{set\_status}}}}}(status{[}, excluded\_hubs{]})
&
\sphinxAtStartPar
Define the connection status of all edges except the ones involving a hub from \sphinxtitleref{excluded\_hubs}.
\\
\hline
\end{tabulary}
\par
\sphinxattableend\end{savenotes}
\subsubsection*{Attributes}


\begin{savenotes}\sphinxattablestart
\centering
\begin{tabulary}{\linewidth}[t]{\X{1}{2}\X{1}{2}}
\hline

\sphinxAtStartPar
\sphinxcode{\sphinxupquote{connection}}
&
\sphinxAtStartPar

\\
\hline
\sphinxAtStartPar
{\hyperref[\detokenize{generated/tamos.network.NonThermalNetwork:tamos.network.NonThermalNetwork.eco_count}]{\sphinxcrossref{\sphinxcode{\sphinxupquote{eco\_count}}}}}
&
\sphinxAtStartPar
Whether this instance contributes to the system "Eco" KPI bool
\\
\hline
\sphinxAtStartPar
{\hyperref[\detokenize{generated/tamos.network.NonThermalNetwork:tamos.network.NonThermalNetwork.element}]{\sphinxcrossref{\sphinxcode{\sphinxupquote{element}}}}}
&
\sphinxAtStartPar
Element exchanged between \sphinxtitleref{hubs}.
\\
\hline
\sphinxAtStartPar
{\hyperref[\detokenize{generated/tamos.network.NonThermalNetwork:tamos.network.NonThermalNetwork.hubs}]{\sphinxcrossref{\sphinxcode{\sphinxupquote{hubs}}}}}
&
\sphinxAtStartPar
The hubs involved in the network definition.
\\
\hline
\sphinxAtStartPar
{\hyperref[\detokenize{generated/tamos.network.NonThermalNetwork:tamos.network.NonThermalNetwork.name}]{\sphinxcrossref{\sphinxcode{\sphinxupquote{name}}}}}
&
\sphinxAtStartPar
str.
\\
\hline
\sphinxAtStartPar
\sphinxcode{\sphinxupquote{no\_connection}}
&
\sphinxAtStartPar

\\
\hline
\sphinxAtStartPar
\sphinxcode{\sphinxupquote{optim\_one\_way\_max}}
&
\sphinxAtStartPar

\\
\hline
\sphinxAtStartPar
\sphinxcode{\sphinxupquote{optim\_one\_way\_min}}
&
\sphinxAtStartPar

\\
\hline
\sphinxAtStartPar
\sphinxcode{\sphinxupquote{optim\_two\_ways}}
&
\sphinxAtStartPar

\\
\hline
\sphinxAtStartPar
{\hyperref[\detokenize{generated/tamos.network.NonThermalNetwork:tamos.network.NonThermalNetwork.production_hub}]{\sphinxcrossref{\sphinxcode{\sphinxupquote{production\_hub}}}}}
&
\sphinxAtStartPar
The hub that bears all the distribution losses of the network.
\\
\hline
\sphinxAtStartPar
{\hyperref[\detokenize{generated/tamos.network.NonThermalNetwork:tamos.network.NonThermalNetwork.scale_factor}]{\sphinxcrossref{\sphinxcode{\sphinxupquote{scale\_factor}}}}}
&
\sphinxAtStartPar
Multiplies the two coordinates of each hub in \sphinxtitleref{hubs\_locations}.
\\
\hline
\sphinxAtStartPar
{\hyperref[\detokenize{generated/tamos.network.NonThermalNetwork:tamos.network.NonThermalNetwork.used_elements}]{\sphinxcrossref{\sphinxcode{\sphinxupquote{used\_elements}}}}}
&
\sphinxAtStartPar
Elements used by the component.
\\
\hline
\end{tabulary}
\par
\sphinxattableend\end{savenotes}
\index{compute\_actualized\_cost() (tamos.network.NonThermalNetwork method)@\spxentry{compute\_actualized\_cost()}\spxextra{tamos.network.NonThermalNetwork method}}

\begin{fulllineitems}
\phantomsection\label{\detokenize{generated/tamos.network.NonThermalNetwork:tamos.network.NonThermalNetwork.compute_actualized_cost}}
\pysigstartsignatures
\pysiglinewithargsret{\sphinxbfcode{\sphinxupquote{compute\_actualized\_cost}}}{\emph{\DUrole{n}{CAPEX}}, \emph{\DUrole{n}{OPEX}}, \emph{\DUrole{n}{system\_lifetime}}, \emph{\DUrole{n}{lifetime}\DUrole{o}{=}\DUrole{default_value}{None}}, \emph{\DUrole{n}{discount\_rate}\DUrole{o}{=}\DUrole{default_value}{None}}}{}
\pysigstopsignatures
\sphinxAtStartPar
Computes the cost of a component using its ‘Lifetime’ and ‘Discount rate (\%)’ properties.
\begin{quote}\begin{description}
\sphinxlineitem{Parameters}\begin{itemize}
\item {} 
\sphinxAtStartPar
\sphinxstyleliteralstrong{\sphinxupquote{CAPEX}} (\sphinxstyleliteralemphasis{\sphinxupquote{float}}) \textendash{} Capital Expenditure. Cost in euros paid every \sphinxtitleref{technical\_lifetime} periods.

\item {} 
\sphinxAtStartPar
\sphinxstyleliteralstrong{\sphinxupquote{OPEX}} (\sphinxstyleliteralemphasis{\sphinxupquote{float}}) \textendash{} Operational Expenditure. Cost in euros paid each period.

\item {} 
\sphinxAtStartPar
\sphinxstyleliteralstrong{\sphinxupquote{system\_lifetime}} (\sphinxstyleliteralemphasis{\sphinxupquote{int}}) \textendash{} Number of periods defining the existence of the energy system.

\item {} 
\sphinxAtStartPar
\sphinxstyleliteralstrong{\sphinxupquote{lifetime}} (\sphinxstyleliteralemphasis{\sphinxupquote{int}}\sphinxstyleliteralemphasis{\sphinxupquote{, }}\sphinxstyleliteralemphasis{\sphinxupquote{optional}}) \textendash{} Number of periods defining the existence of the component.
If specified, overwrite the “Lifetime” property.

\item {} 
\sphinxAtStartPar
\sphinxstyleliteralstrong{\sphinxupquote{discount\_rate}} (\sphinxstyleliteralemphasis{\sphinxupquote{float}}) \textendash{} In percent (\%). Describes the importance of the economic amortization process, per period.
If specified, overwrite the “Discount rate (\%)” property.

\end{itemize}

\sphinxlineitem{Returns}
\sphinxAtStartPar
\begin{itemize}
\item {} 
\sphinxAtStartPar
\sphinxstyleemphasis{A 3\sphinxhyphen{}tuple (total\_cost, CAPEX\_share, OPEX\_share) where}

\item {} 
\sphinxAtStartPar
* CAPEX\_share is the share of total cost related to \sphinxtitleref{CAPEX}

\item {} 
\sphinxAtStartPar
* OPEX\_share is the share of total cost related to \sphinxtitleref{OPEX}

\item {} 
\sphinxAtStartPar
\sphinxstyleemphasis{* total\_cost = CAPEX\_share + OPEX\_share}

\end{itemize}


\end{description}\end{quote}
\subsubsection*{Notes}

\sphinxAtStartPar
Takes into account residual value of component in the case \sphinxtitleref{system\_lifetime} is not a multiple of \sphinxtitleref{lifetime}.
In this case, the last replacement occuring at period replacement\_period is paid in proportion of ‘CAPEX’
depending linearly on the number of periods left:
CAPEX * (system\_lifetime \sphinxhyphen{} replacement\_period) / lifetime

\end{fulllineitems}

\index{eco\_count (tamos.network.NonThermalNetwork property)@\spxentry{eco\_count}\spxextra{tamos.network.NonThermalNetwork property}}

\begin{fulllineitems}
\phantomsection\label{\detokenize{generated/tamos.network.NonThermalNetwork:tamos.network.NonThermalNetwork.eco_count}}
\pysigstartsignatures
\pysigline{\sphinxbfcode{\sphinxupquote{property\DUrole{w}{  }}}\sphinxbfcode{\sphinxupquote{eco\_count}}}
\pysigstopsignatures
\sphinxAtStartPar
Whether this instance contributes to the system “Eco” KPI
bool

\end{fulllineitems}

\index{element (tamos.network.NonThermalNetwork property)@\spxentry{element}\spxextra{tamos.network.NonThermalNetwork property}}

\begin{fulllineitems}
\phantomsection\label{\detokenize{generated/tamos.network.NonThermalNetwork:tamos.network.NonThermalNetwork.element}}
\pysigstartsignatures
\pysigline{\sphinxbfcode{\sphinxupquote{property\DUrole{w}{  }}}\sphinxbfcode{\sphinxupquote{element}}}
\pysigstopsignatures
\sphinxAtStartPar
Element exchanged between \sphinxtitleref{hubs}.

\sphinxAtStartPar
Whether \sphinxtitleref{element} is cooled down or warmed up does not define if the network is a heating or cooling network.

\end{fulllineitems}

\index{generate\_MSP() (tamos.network.NonThermalNetwork method)@\spxentry{generate\_MSP()}\spxextra{tamos.network.NonThermalNetwork method}}

\begin{fulllineitems}
\phantomsection\label{\detokenize{generated/tamos.network.NonThermalNetwork:tamos.network.NonThermalNetwork.generate_MSP}}
\pysigstartsignatures
\pysiglinewithargsret{\sphinxbfcode{\sphinxupquote{generate\_MSP}}}{\emph{\DUrole{n}{excluded\_hubs}\DUrole{o}{=}\DUrole{default_value}{None}}}{}
\pysigstopsignatures
\sphinxAtStartPar
Defines a minimum spanning tree (MSP) linking all hubs of \sphinxtitleref{hubs} that are not in \sphinxtitleref{excluded\_hubs}.

\sphinxAtStartPar
The MSP calculation does not account for bidirectionality thus all edges of the MSP graph are given
the status \sphinxtitleref{optim\_two\_ways}.
The status of the other edges (not part of the MSP) are not modified.
\begin{quote}\begin{description}
\sphinxlineitem{Parameters}
\sphinxAtStartPar
\sphinxstyleliteralstrong{\sphinxupquote{excluded\_hubs}} (\sphinxstyleliteralemphasis{\sphinxupquote{list of Hub}}\sphinxstyleliteralemphasis{\sphinxupquote{, }}\sphinxstyleliteralemphasis{\sphinxupquote{optional}}) \textendash{} All hubs from \sphinxtitleref{excluded\_hubs} must be part of \sphinxtitleref{hubs}.

\end{description}\end{quote}

\end{fulllineitems}

\index{get\_connection\_power\_bounds() (tamos.network.NonThermalNetwork method)@\spxentry{get\_connection\_power\_bounds()}\spxextra{tamos.network.NonThermalNetwork method}}

\begin{fulllineitems}
\phantomsection\label{\detokenize{generated/tamos.network.NonThermalNetwork:tamos.network.NonThermalNetwork.get_connection_power_bounds}}
\pysigstartsignatures
\pysiglinewithargsret{\sphinxbfcode{\sphinxupquote{get\_connection\_power\_bounds}}}{\emph{\DUrole{n}{hub\_1}}, \emph{\DUrole{n}{hub\_2}}}{}
\pysigstopsignatures
\sphinxAtStartPar
Returns the power limits of the flow of element from hub \sphinxtitleref{hub\_1} to hub \sphinxtitleref{hub\_2}.
\begin{quote}\begin{description}
\sphinxlineitem{Parameters}\begin{itemize}
\item {} 
\sphinxAtStartPar
\sphinxstyleliteralstrong{\sphinxupquote{hub\_1}} ({\hyperref[\detokenize{generated/tamos.Hub:tamos.Hub}]{\sphinxcrossref{\sphinxstyleliteralemphasis{\sphinxupquote{Hub}}}}}) \textendash{} Must be hubs from \sphinxtitleref{hubs}.

\item {} 
\sphinxAtStartPar
\sphinxstyleliteralstrong{\sphinxupquote{hub\_2}} ({\hyperref[\detokenize{generated/tamos.Hub:tamos.Hub}]{\sphinxcrossref{\sphinxstyleliteralemphasis{\sphinxupquote{Hub}}}}}) \textendash{} Must be hubs from \sphinxtitleref{hubs}.

\end{itemize}

\sphinxlineitem{Return type}
\sphinxAtStartPar
The lower bound and upper bound of the power flow from \sphinxtitleref{hub\_1} to \sphinxtitleref{hub\_2}, in kW.

\end{description}\end{quote}

\end{fulllineitems}

\index{get\_connection\_status() (tamos.network.NonThermalNetwork method)@\spxentry{get\_connection\_status()}\spxextra{tamos.network.NonThermalNetwork method}}

\begin{fulllineitems}
\phantomsection\label{\detokenize{generated/tamos.network.NonThermalNetwork:tamos.network.NonThermalNetwork.get_connection_status}}
\pysigstartsignatures
\pysiglinewithargsret{\sphinxbfcode{\sphinxupquote{get\_connection\_status}}}{\emph{\DUrole{n}{hub\_1}}, \emph{\DUrole{n}{hub\_2}}}{}
\pysigstopsignatures
\sphinxAtStartPar
Returns the status of the directional connection between two hubs.
\begin{quote}\begin{description}
\sphinxlineitem{Parameters}\begin{itemize}
\item {} 
\sphinxAtStartPar
\sphinxstyleliteralstrong{\sphinxupquote{hub\_1}} ({\hyperref[\detokenize{generated/tamos.Hub:tamos.Hub}]{\sphinxcrossref{\sphinxstyleliteralemphasis{\sphinxupquote{Hub}}}}}) \textendash{} Must be hubs from \sphinxtitleref{hubs}.

\item {} 
\sphinxAtStartPar
\sphinxstyleliteralstrong{\sphinxupquote{hub\_2}} ({\hyperref[\detokenize{generated/tamos.Hub:tamos.Hub}]{\sphinxcrossref{\sphinxstyleliteralemphasis{\sphinxupquote{Hub}}}}}) \textendash{} Must be hubs from \sphinxtitleref{hubs}.

\end{itemize}

\sphinxlineitem{Returns}
\sphinxAtStartPar
The status of the edge from hub\_1 to hub\_2.

\sphinxlineitem{Return type}
\sphinxAtStartPar
network status

\end{description}\end{quote}

\end{fulllineitems}

\index{get\_distance() (tamos.network.NonThermalNetwork method)@\spxentry{get\_distance()}\spxextra{tamos.network.NonThermalNetwork method}}

\begin{fulllineitems}
\phantomsection\label{\detokenize{generated/tamos.network.NonThermalNetwork:tamos.network.NonThermalNetwork.get_distance}}
\pysigstartsignatures
\pysiglinewithargsret{\sphinxbfcode{\sphinxupquote{get\_distance}}}{\emph{\DUrole{n}{hub\_1}}, \emph{\DUrole{n}{hub\_2}}}{}
\pysigstopsignatures
\sphinxAtStartPar
Returns the distance between two hubs.
The used distance function can be accessed by the \sphinxtitleref{get\_distance\_function} function
and set using \sphinxtitleref{set\_distance\_function} from tamos.network.
\begin{quote}\begin{description}
\sphinxlineitem{Parameters}\begin{itemize}
\item {} 
\sphinxAtStartPar
\sphinxstyleliteralstrong{\sphinxupquote{hub\_1}} ({\hyperref[\detokenize{generated/tamos.Hub:tamos.Hub}]{\sphinxcrossref{\sphinxstyleliteralemphasis{\sphinxupquote{Hub}}}}}) \textendash{} Must be from \sphinxtitleref{hubs}.

\item {} 
\sphinxAtStartPar
\sphinxstyleliteralstrong{\sphinxupquote{hub\_2}} ({\hyperref[\detokenize{generated/tamos.Hub:tamos.Hub}]{\sphinxcrossref{\sphinxstyleliteralemphasis{\sphinxupquote{Hub}}}}}) \textendash{} Must be from \sphinxtitleref{hubs}.

\end{itemize}

\sphinxlineitem{Return type}
\sphinxAtStartPar
The distance from \sphinxtitleref{hub\_1} to \sphinxtitleref{hub\_2} which is the same than from \sphinxtitleref{hub\_2} to \sphinxtitleref{hub\_1}

\end{description}\end{quote}

\end{fulllineitems}

\index{hubs (tamos.network.NonThermalNetwork property)@\spxentry{hubs}\spxextra{tamos.network.NonThermalNetwork property}}

\begin{fulllineitems}
\phantomsection\label{\detokenize{generated/tamos.network.NonThermalNetwork:tamos.network.NonThermalNetwork.hubs}}
\pysigstartsignatures
\pysigline{\sphinxbfcode{\sphinxupquote{property\DUrole{w}{  }}}\sphinxbfcode{\sphinxupquote{hubs}}}
\pysigstopsignatures
\sphinxAtStartPar
The hubs involved in the network definition.

\sphinxAtStartPar
Some of these hubs might be completely disconnected from the network
(i.e. network.might\_connect(hub) is False) yet they are still considered and associated with MILP decision variables.

\end{fulllineitems}

\index{might\_connect() (tamos.network.NonThermalNetwork method)@\spxentry{might\_connect()}\spxextra{tamos.network.NonThermalNetwork method}}

\begin{fulllineitems}
\phantomsection\label{\detokenize{generated/tamos.network.NonThermalNetwork:tamos.network.NonThermalNetwork.might_connect}}
\pysigstartsignatures
\pysiglinewithargsret{\sphinxbfcode{\sphinxupquote{might\_connect}}}{\emph{\DUrole{n}{hub}}}{}
\pysigstopsignatures
\sphinxAtStartPar
Checks whether a hub is able to connect to the network.
\begin{quote}\begin{description}
\sphinxlineitem{Parameters}
\sphinxAtStartPar
\sphinxstyleliteralstrong{\sphinxupquote{hub}} ({\hyperref[\detokenize{generated/tamos.Hub:tamos.Hub}]{\sphinxcrossref{\sphinxstyleliteralemphasis{\sphinxupquote{Hub}}}}}) \textendash{} 

\sphinxlineitem{Returns}
\sphinxAtStartPar
\begin{itemize}
\item {} 
\sphinxAtStartPar
\sphinxstyleemphasis{True if the following two conditions are met}

\item {} 
\sphinxAtStartPar
* Hub \sphinxtitleref{hub} is one of \sphinxtitleref{hubs}.

\item {} 
\sphinxAtStartPar
* There exists at least one hub \sphinxtitleref{hub\_2} in \sphinxtitleref{hubs} such that connection status from \sphinxtitleref{hub} to \sphinxtitleref{hub\_2} or \sphinxtitleref{hub\_2} to \sphinxtitleref{hub} \textendash{} is different from ‘No connection’.

\end{itemize}


\end{description}\end{quote}

\end{fulllineitems}

\index{name (tamos.network.NonThermalNetwork property)@\spxentry{name}\spxextra{tamos.network.NonThermalNetwork property}}

\begin{fulllineitems}
\phantomsection\label{\detokenize{generated/tamos.network.NonThermalNetwork:tamos.network.NonThermalNetwork.name}}
\pysigstartsignatures
\pysigline{\sphinxbfcode{\sphinxupquote{property\DUrole{w}{  }}}\sphinxbfcode{\sphinxupquote{name}}}
\pysigstopsignatures
\sphinxAtStartPar
str.
This name is used in MILP model description.
names must not exceed 255 characters,
all of which must be alphanumeric (a\sphinxhyphen{}z, A\sphinxhyphen{}Z, 0\sphinxhyphen{}9) or one of these symbols:
! ” \# \$ \% \& , . ; ? @ \_ ‘ ’ \{ \} \textasciitilde{}.
\begin{quote}\begin{description}
\sphinxlineitem{Type}
\sphinxAtStartPar
Name of the instance

\end{description}\end{quote}

\end{fulllineitems}

\index{plot() (tamos.network.NonThermalNetwork method)@\spxentry{plot()}\spxextra{tamos.network.NonThermalNetwork method}}

\begin{fulllineitems}
\phantomsection\label{\detokenize{generated/tamos.network.NonThermalNetwork:tamos.network.NonThermalNetwork.plot}}
\pysigstartsignatures
\pysiglinewithargsret{\sphinxbfcode{\sphinxupquote{plot}}}{}{}
\pysigstopsignatures
\sphinxAtStartPar
Plots a representation of the network hubs and connections in the (x, y) space.

\sphinxAtStartPar
A call to this method gives a visual insight of how the network is parametrized, BEFORE optimization.
The optimization implicitely tranforms every edge status to either ‘No connection’ or ‘Connection’.
\subsubsection*{Notes}

\sphinxAtStartPar
The line linking two hubs is straight for commodity and does not represent
the real distance function used in the MILP model.

\end{fulllineitems}

\index{production\_hub (tamos.network.NonThermalNetwork property)@\spxentry{production\_hub}\spxextra{tamos.network.NonThermalNetwork property}}

\begin{fulllineitems}
\phantomsection\label{\detokenize{generated/tamos.network.NonThermalNetwork:tamos.network.NonThermalNetwork.production_hub}}
\pysigstartsignatures
\pysigline{\sphinxbfcode{\sphinxupquote{property\DUrole{w}{  }}}\sphinxbfcode{\sphinxupquote{production\_hub}}}
\pysigstopsignatures
\sphinxAtStartPar
The hub that bears all the distribution losses of the network.
Must be one of \sphinxtitleref{hubs} and must be able to send \sphinxtitleref{element} to the network.

\end{fulllineitems}

\index{scale\_factor (tamos.network.NonThermalNetwork property)@\spxentry{scale\_factor}\spxextra{tamos.network.NonThermalNetwork property}}

\begin{fulllineitems}
\phantomsection\label{\detokenize{generated/tamos.network.NonThermalNetwork:tamos.network.NonThermalNetwork.scale_factor}}
\pysigstartsignatures
\pysigline{\sphinxbfcode{\sphinxupquote{property\DUrole{w}{  }}}\sphinxbfcode{\sphinxupquote{scale\_factor}}}
\pysigstopsignatures
\sphinxAtStartPar
Multiplies the two coordinates of each hub in \sphinxtitleref{hubs\_locations}.

\sphinxAtStartPar
A scale factor \textgreater{}1 tends to increase the distance between two hubs.
float

\end{fulllineitems}

\index{set\_connection\_power\_bounds() (tamos.network.NonThermalNetwork method)@\spxentry{set\_connection\_power\_bounds()}\spxextra{tamos.network.NonThermalNetwork method}}

\begin{fulllineitems}
\phantomsection\label{\detokenize{generated/tamos.network.NonThermalNetwork:tamos.network.NonThermalNetwork.set_connection_power_bounds}}
\pysigstartsignatures
\pysiglinewithargsret{\sphinxbfcode{\sphinxupquote{set\_connection\_power\_bounds}}}{\emph{\DUrole{n}{hub\_1}}, \emph{\DUrole{n}{hub\_2}}, \emph{\DUrole{n}{power\_lb}\DUrole{o}{=}\DUrole{default_value}{None}}, \emph{\DUrole{n}{power\_ub}\DUrole{o}{=}\DUrole{default_value}{None}}}{}
\pysigstopsignatures
\sphinxAtStartPar
Sets power limits on the flow of element from hub \sphinxtitleref{hub\_1} to hub \sphinxtitleref{hub\_2}.
These limits apply only if the connection from \sphinxtitleref{hub\_1} to \sphinxtitleref{hub\_2} is used.
\begin{quote}\begin{description}
\sphinxlineitem{Parameters}\begin{itemize}
\item {} 
\sphinxAtStartPar
\sphinxstyleliteralstrong{\sphinxupquote{hub\_1}} ({\hyperref[\detokenize{generated/tamos.Hub:tamos.Hub}]{\sphinxcrossref{\sphinxstyleliteralemphasis{\sphinxupquote{Hub}}}}}) \textendash{} Must be hubs from \sphinxtitleref{hubs}.

\item {} 
\sphinxAtStartPar
\sphinxstyleliteralstrong{\sphinxupquote{hub\_2}} ({\hyperref[\detokenize{generated/tamos.Hub:tamos.Hub}]{\sphinxcrossref{\sphinxstyleliteralemphasis{\sphinxupquote{Hub}}}}}) \textendash{} Must be hubs from \sphinxtitleref{hubs}.

\item {} 
\sphinxAtStartPar
\sphinxstyleliteralstrong{\sphinxupquote{power\_lb}} (\sphinxstyleliteralemphasis{\sphinxupquote{int}}\sphinxstyleliteralemphasis{\sphinxupquote{, }}\sphinxstyleliteralemphasis{\sphinxupquote{float}}\sphinxstyleliteralemphasis{\sphinxupquote{ or }}\sphinxstyleliteralemphasis{\sphinxupquote{numpy.ndarray}}) \textendash{} power\_lb \textgreater{}= 0, power\_ub \textgreater{}= 0
In kW.
The lower bound (upper bound) of the flow of \sphinxtitleref{element} from \sphinxtitleref{hub\_1} to \sphinxtitleref{hub\_2}.

\item {} 
\sphinxAtStartPar
\sphinxstyleliteralstrong{\sphinxupquote{power\_ub}} (\sphinxstyleliteralemphasis{\sphinxupquote{int}}\sphinxstyleliteralemphasis{\sphinxupquote{, }}\sphinxstyleliteralemphasis{\sphinxupquote{float}}\sphinxstyleliteralemphasis{\sphinxupquote{ or }}\sphinxstyleliteralemphasis{\sphinxupquote{numpy.ndarray}}) \textendash{} power\_lb \textgreater{}= 0, power\_ub \textgreater{}= 0
In kW.
The lower bound (upper bound) of the flow of \sphinxtitleref{element} from \sphinxtitleref{hub\_1} to \sphinxtitleref{hub\_2}.

\end{itemize}

\end{description}\end{quote}
\subsubsection*{Examples}

\begin{sphinxVerbatim}[commandchars=\\\{\}]
\PYG{g+gp}{\PYGZgt{}\PYGZgt{}\PYGZgt{} }\PYG{n}{network}\PYG{o}{.}\PYG{n}{set\PYGZus{}connection\PYGZus{}power\PYGZus{}bounds}\PYG{p}{(}\PYG{n}{hub\PYGZus{}1}\PYG{p}{,} \PYG{n}{hub\PYGZus{}2}\PYG{p}{,} \PYG{n}{power\PYGZus{}lb}\PYG{o}{=}\PYG{l+m+mi}{400}\PYG{p}{,} \PYG{n}{power\PYGZus{}ub}\PYG{o}{=}\PYG{l+m+mi}{2000}\PYG{p}{)}
\end{sphinxVerbatim}

\sphinxAtStartPar
If the connection \sphinxtitleref{hub\_1} to \sphinxtitleref{hub\_2} exists, the power that flows from \sphinxtitleref{hub\_1} to \sphinxtitleref{hub\_2} must always be in the range
{[}0.4, 2{]} MW.

\begin{sphinxVerbatim}[commandchars=\\\{\}]
\PYG{g+gp}{\PYGZgt{}\PYGZgt{}\PYGZgt{} }\PYG{n}{network}\PYG{o}{.}\PYG{n}{set\PYGZus{}connection\PYGZus{}power\PYGZus{}bounds}\PYG{p}{(}\PYG{n}{hub\PYGZus{}1}\PYG{p}{,} \PYG{n}{hub\PYGZus{}2}\PYG{p}{,} \PYG{n}{power\PYGZus{}lb}\PYG{o}{=}\PYG{l+m+mi}{1000}\PYG{p}{)}
\PYG{g+gp}{\PYGZgt{}\PYGZgt{}\PYGZgt{} }\PYG{n}{network}\PYG{o}{.}\PYG{n}{set\PYGZus{}connection\PYGZus{}power\PYGZus{}bounds}\PYG{p}{(}\PYG{n}{hub\PYGZus{}2}\PYG{p}{,} \PYG{n}{hub\PYGZus{}1}\PYG{p}{,} \PYG{n}{power\PYGZus{}ub}\PYG{o}{=}\PYG{o}{\PYGZhy{}}\PYG{l+m+mi}{1000}\PYG{p}{)}
\end{sphinxVerbatim}

\sphinxAtStartPar
Both calls perform the same operation, but second call is forbidden to make things clearer.

\begin{sphinxVerbatim}[commandchars=\\\{\}]
\PYG{g+gp}{\PYGZgt{}\PYGZgt{}\PYGZgt{} }\PYG{n}{network}\PYG{o}{.}\PYG{n}{set\PYGZus{}connection\PYGZus{}power\PYGZus{}bounds}\PYG{p}{(}\PYG{n}{hub\PYGZus{}1}\PYG{p}{,} \PYG{n}{hub\PYGZus{}2}\PYG{p}{,} \PYG{n}{power\PYGZus{}lb}\PYG{o}{=}\PYG{l+m+mi}{1000}\PYG{p}{)}
\PYG{g+gp}{\PYGZgt{}\PYGZgt{}\PYGZgt{} }\PYG{n}{network}\PYG{o}{.}\PYG{n}{set\PYGZus{}connection\PYGZus{}power\PYGZus{}bounds}\PYG{p}{(}\PYG{n}{hub\PYGZus{}2}\PYG{p}{,} \PYG{n}{hub\PYGZus{}1}\PYG{p}{,} \PYG{n}{power\PYGZus{}lb}\PYG{o}{=}\PYG{l+m+mi}{2000}\PYG{p}{)}
\end{sphinxVerbatim}

\sphinxAtStartPar
The constraints implied by these calls make impossible the existence of both connections
(from \sphinxtitleref{hub\_1} to \sphinxtitleref{hub\_2} and \sphinxtitleref{hub\_2} to \sphinxtitleref{hub\_1}): if 1000 kW of \sphinxtitleref{element} flows from \sphinxtitleref{hub\_1} to \sphinxtitleref{hub\_2}
then \sphinxhyphen{}1000 k\textgreater{} flows from \sphinxtitleref{hub\_2} to \sphinxtitleref{hub\_1} (and vice versa).

\end{fulllineitems}

\index{set\_connection\_status() (tamos.network.NonThermalNetwork method)@\spxentry{set\_connection\_status()}\spxextra{tamos.network.NonThermalNetwork method}}

\begin{fulllineitems}
\phantomsection\label{\detokenize{generated/tamos.network.NonThermalNetwork:tamos.network.NonThermalNetwork.set_connection_status}}
\pysigstartsignatures
\pysiglinewithargsret{\sphinxbfcode{\sphinxupquote{set\_connection\_status}}}{\emph{\DUrole{n}{hub\_1}}, \emph{\DUrole{n}{hub\_2}}, \emph{\DUrole{n}{status}}}{}
\pysigstopsignatures
\sphinxAtStartPar
Defines the connection status of the edge going from \sphinxtitleref{hub\_1} to \sphinxtitleref{hub\_2}.
\begin{quote}\begin{description}
\sphinxlineitem{Parameters}\begin{itemize}
\item {} 
\sphinxAtStartPar
\sphinxstyleliteralstrong{\sphinxupquote{hub\_1}} ({\hyperref[\detokenize{generated/tamos.Hub:tamos.Hub}]{\sphinxcrossref{\sphinxstyleliteralemphasis{\sphinxupquote{Hub}}}}}) \textendash{} Must be hubs from \sphinxtitleref{hubs}.

\item {} 
\sphinxAtStartPar
\sphinxstyleliteralstrong{\sphinxupquote{hub\_2}} ({\hyperref[\detokenize{generated/tamos.Hub:tamos.Hub}]{\sphinxcrossref{\sphinxstyleliteralemphasis{\sphinxupquote{Hub}}}}}) \textendash{} Must be hubs from \sphinxtitleref{hubs}.

\item {} 
\sphinxAtStartPar
\sphinxstyleliteralstrong{\sphinxupquote{status}} (\sphinxtitleref{status} may take 5 values that are attributes of this instance:) \textendash{} \begin{itemize}
\item {} 
\sphinxAtStartPar
no\_connection: flow of \sphinxtitleref{element} is forbidden

\item {} 
\sphinxAtStartPar
connection: flow of \sphinxtitleref{element} is possible

\item {} 
\sphinxAtStartPar
optim\_one\_way\_min: flow of \sphinxtitleref{element} must exist in at least one direction
(from hub\_1 to hub\_2, from hub\_2 to hub\_1 or in both directions)
This status also defines the opposite status, e.g. defining ‘hub\_1 to hub\_2’ status defines the one of ‘hub\_2 to hub\_1’.

\item {} 
\sphinxAtStartPar
optim\_one\_way\_max: flow of \sphinxtitleref{element} may exist in at most one direction
(from hub\_1 to hub\_2 or from hub\_2 to hub\_1)
This status also defines the opposite status, e.g. defining ‘hub\_1 to hub\_2’ status defines the one of ‘hub\_2 to hub\_1’.

\item {} 
\sphinxAtStartPar
optim\_two\_ways: flow of \sphinxtitleref{element} may exist in both directions
(from hub\_1 to hub\_2 and from hub\_2 to hub\_1)
This status also defines the opposite status, e.g. defining ‘hub\_1 to hub\_2’ status defines the one of ‘hub\_2 to hub\_1’.

\end{itemize}


\end{itemize}

\end{description}\end{quote}
\subsubsection*{Notes}

\sphinxAtStartPar
Given two hubs \sphinxtitleref{hub\_1} and \sphinxtitleref{hub\_2}, the order of calls to \sphinxtitleref{set\_connection\_status} matters.

\begin{sphinxVerbatim}[commandchars=\\\{\}]
\PYG{g+gp}{\PYGZgt{}\PYGZgt{}\PYGZgt{} }\PYG{n}{network}\PYG{o}{.}\PYG{n}{set\PYGZus{}connection\PYGZus{}status}\PYG{p}{(}\PYG{n}{hub\PYGZus{}1}\PYG{p}{,} \PYG{n}{hub\PYGZus{}2}\PYG{p}{,} \PYG{n}{network}\PYG{o}{.}\PYG{n}{no\PYGZus{}direction}\PYG{p}{)}
\PYG{g+gp}{\PYGZgt{}\PYGZgt{}\PYGZgt{} }\PYG{n}{network}\PYG{o}{.}\PYG{n}{set\PYGZus{}connection\PYGZus{}status}\PYG{p}{(}\PYG{n}{hub\PYGZus{}2}\PYG{p}{,} \PYG{n}{hub\PYGZus{}1}\PYG{p}{,} \PYG{n}{network}\PYG{o}{.}\PYG{n}{optim\PYGZus{}one\PYGZus{}way\PYGZus{}max}\PYG{p}{)}
\PYG{g+gp}{\PYGZgt{}\PYGZgt{}\PYGZgt{} }\PYG{n}{network}\PYG{o}{.}\PYG{n}{get\PYGZus{}connection\PYGZus{}status}\PYG{p}{(}\PYG{n}{hub\PYGZus{}2}\PYG{p}{,} \PYG{n}{hub\PYGZus{}1}\PYG{p}{)} \PYG{o}{==} \PYG{n}{network}\PYG{o}{.}\PYG{n}{get\PYGZus{}connection\PYGZus{}status}\PYG{p}{(}\PYG{n}{hub\PYGZus{}1}\PYG{p}{,} \PYG{n}{hub\PYGZus{}2}\PYG{p}{)}
\PYG{g+go}{    True}
\end{sphinxVerbatim}

\sphinxAtStartPar
The first line has no effect because the second one defines again the connection from \sphinxtitleref{hub\_1} to \sphinxtitleref{hub\_2}.

\end{fulllineitems}

\index{set\_node\_status() (tamos.network.NonThermalNetwork method)@\spxentry{set\_node\_status()}\spxextra{tamos.network.NonThermalNetwork method}}

\begin{fulllineitems}
\phantomsection\label{\detokenize{generated/tamos.network.NonThermalNetwork:tamos.network.NonThermalNetwork.set_node_status}}
\pysigstartsignatures
\pysiglinewithargsret{\sphinxbfcode{\sphinxupquote{set\_node\_status}}}{\emph{\DUrole{n}{hub}}, \emph{\DUrole{n}{status}}}{}
\pysigstopsignatures
\sphinxAtStartPar
Defines the connection status of every incoming and outcoming edge of a hub.
\begin{quote}\begin{description}
\sphinxlineitem{Parameters}\begin{itemize}
\item {} 
\sphinxAtStartPar
\sphinxstyleliteralstrong{\sphinxupquote{hub}} ({\hyperref[\detokenize{generated/tamos.Hub:tamos.Hub}]{\sphinxcrossref{\sphinxstyleliteralemphasis{\sphinxupquote{Hub}}}}}) \textendash{} Must be from \sphinxtitleref{hubs}.

\item {} 
\sphinxAtStartPar
\sphinxstyleliteralstrong{\sphinxupquote{status}} (\sphinxtitleref{status} may take 5 values that are attributes of this instance:) \textendash{} \begin{itemize}
\item {} 
\sphinxAtStartPar
no\_connection: flow of \sphinxtitleref{element} is forbidden

\item {} 
\sphinxAtStartPar
connection: flow of \sphinxtitleref{element} is possible

\item {} 
\sphinxAtStartPar
optim\_one\_way\_min: flow of \sphinxtitleref{element} must exist in at least one direction
(from hub\_1 to hub\_2, from hub\_2 to hub\_1 or in both directions)
This status also defines the opposite status, e.g. defining ‘hub\_1 to hub\_2’ status defines the one of ‘hub\_2 to hub\_1’.

\item {} 
\sphinxAtStartPar
optim\_one\_way\_max: flow of \sphinxtitleref{element} may exist in at most one direction
(from hub\_1 to hub\_2 or from hub\_2 to hub\_1)
This status also defines the opposite status, e.g. defining ‘hub\_1 to hub\_2’ status defines the one of ‘hub\_2 to hub\_1’.

\item {} 
\sphinxAtStartPar
optim\_two\_ways: flow of \sphinxtitleref{element} may exist in both directions
(from hub\_1 to hub\_2 and from hub\_2 to hub\_1)
This status also defines the opposite status, e.g. defining ‘hub\_1 to hub\_2’ status defines the one of ‘hub\_2 to hub\_1’.

\end{itemize}


\end{itemize}

\end{description}\end{quote}

\end{fulllineitems}

\index{set\_status() (tamos.network.NonThermalNetwork method)@\spxentry{set\_status()}\spxextra{tamos.network.NonThermalNetwork method}}

\begin{fulllineitems}
\phantomsection\label{\detokenize{generated/tamos.network.NonThermalNetwork:tamos.network.NonThermalNetwork.set_status}}
\pysigstartsignatures
\pysiglinewithargsret{\sphinxbfcode{\sphinxupquote{set\_status}}}{\emph{\DUrole{n}{status}}, \emph{\DUrole{n}{excluded\_hubs}\DUrole{o}{=}\DUrole{default_value}{None}}}{}
\pysigstopsignatures
\sphinxAtStartPar
Define the connection status of all edges except the ones involving a hub from \sphinxtitleref{excluded\_hubs}.
\begin{quote}\begin{description}
\sphinxlineitem{Parameters}\begin{itemize}
\item {} 
\sphinxAtStartPar
\sphinxstyleliteralstrong{\sphinxupquote{status}} (\sphinxtitleref{status} may take 5 values that are attributes of this instance:) \textendash{} \begin{itemize}
\item {} 
\sphinxAtStartPar
no\_connection: flow of \sphinxtitleref{element} is forbidden

\item {} 
\sphinxAtStartPar
connection: flow of \sphinxtitleref{element} is possible

\item {} 
\sphinxAtStartPar
optim\_one\_way\_min: flow of \sphinxtitleref{element} must exist in at least one direction
(from hub\_1 to hub\_2, from hub\_2 to hub\_1 or in both directions)
This status also defines the opposite status, e.g. defining ‘hub\_1 to hub\_2’ status defines the one of ‘hub\_2 to hub\_1’.

\item {} 
\sphinxAtStartPar
optim\_one\_way\_max: flow of \sphinxtitleref{element} may exist in at most one direction
(from hub\_1 to hub\_2 or from hub\_2 to hub\_1)
This status also defines the opposite status, e.g. defining ‘hub\_1 to hub\_2’ status defines the one of ‘hub\_2 to hub\_1’.

\item {} 
\sphinxAtStartPar
optim\_two\_ways: flow of \sphinxtitleref{element} may exist in both directions
(from hub\_1 to hub\_2 and from hub\_2 to hub\_1)
This status also defines the opposite status, e.g. defining ‘hub\_1 to hub\_2’ status defines the one of ‘hub\_2 to hub\_1’.

\end{itemize}


\item {} 
\sphinxAtStartPar
\sphinxstyleliteralstrong{\sphinxupquote{excluded\_hubs}} (\sphinxstyleliteralemphasis{\sphinxupquote{list of Hub}}\sphinxstyleliteralemphasis{\sphinxupquote{, }}\sphinxstyleliteralemphasis{\sphinxupquote{optional}}) \textendash{} All hubs from \sphinxtitleref{excluded\_hubs} must be part of \sphinxtitleref{hubs}.

\end{itemize}

\end{description}\end{quote}

\end{fulllineitems}

\index{used\_elements (tamos.network.NonThermalNetwork property)@\spxentry{used\_elements}\spxextra{tamos.network.NonThermalNetwork property}}

\begin{fulllineitems}
\phantomsection\label{\detokenize{generated/tamos.network.NonThermalNetwork:tamos.network.NonThermalNetwork.used_elements}}
\pysigstartsignatures
\pysigline{\sphinxbfcode{\sphinxupquote{property\DUrole{w}{  }}}\sphinxbfcode{\sphinxupquote{used\_elements}}}
\pysigstopsignatures
\sphinxAtStartPar
Elements used by the component.

\end{fulllineitems}


\end{fulllineitems}


\sphinxstepscope


\subsection{tamos.network.ThermalNetwork}
\label{\detokenize{generated/tamos.network.ThermalNetwork:tamos-network-thermalnetwork}}\label{\detokenize{generated/tamos.network.ThermalNetwork::doc}}\index{ThermalNetwork (class in tamos.network)@\spxentry{ThermalNetwork}\spxextra{class in tamos.network}}

\begin{fulllineitems}
\phantomsection\label{\detokenize{generated/tamos.network.ThermalNetwork:tamos.network.ThermalNetwork}}
\pysigstartsignatures
\pysiglinewithargsret{\sphinxbfcode{\sphinxupquote{class\DUrole{w}{  }}}\sphinxcode{\sphinxupquote{tamos.network.}}\sphinxbfcode{\sphinxupquote{ThermalNetwork}}}{\emph{\DUrole{n}{hubs\_locations}}, \emph{\DUrole{n}{element}}, \emph{\DUrole{n}{properties}}, \emph{\DUrole{n}{production\_hub}}, \emph{\DUrole{n}{production\_mode}\DUrole{o}{=}\DUrole{default_value}{\textquotesingle{}heat\&cold\textquotesingle{}}}, \emph{\DUrole{n}{scale\_factor}\DUrole{o}{=}\DUrole{default_value}{1}}, \emph{\DUrole{n}{eco\_count}\DUrole{o}{=}\DUrole{default_value}{True}}, \emph{\DUrole{n}{name}\DUrole{o}{=}\DUrole{default_value}{None}}}{}
\pysigstopsignatures\index{\_\_init\_\_() (tamos.network.ThermalNetwork method)@\spxentry{\_\_init\_\_()}\spxextra{tamos.network.ThermalNetwork method}}

\begin{fulllineitems}
\phantomsection\label{\detokenize{generated/tamos.network.ThermalNetwork:tamos.network.ThermalNetwork.__init__}}
\pysigstartsignatures
\pysiglinewithargsret{\sphinxbfcode{\sphinxupquote{\_\_init\_\_}}}{\emph{\DUrole{n}{hubs\_locations}}, \emph{\DUrole{n}{element}}, \emph{\DUrole{n}{properties}}, \emph{\DUrole{n}{production\_hub}}, \emph{\DUrole{n}{production\_mode}\DUrole{o}{=}\DUrole{default_value}{\textquotesingle{}heat\&cold\textquotesingle{}}}, \emph{\DUrole{n}{scale\_factor}\DUrole{o}{=}\DUrole{default_value}{1}}, \emph{\DUrole{n}{eco\_count}\DUrole{o}{=}\DUrole{default_value}{True}}, \emph{\DUrole{n}{name}\DUrole{o}{=}\DUrole{default_value}{None}}}{}
\pysigstopsignatures
\sphinxAtStartPar
ThermalNetwork instances makes possible to share ThermalVectorPair elements between hubs.

\sphinxAtStartPar
Power is exchanged between two hubs, given a distribution losses proportional to the difference between
network temperature and soil temperature.
All distribution losses must be compensated for by an additional power in \sphinxtitleref{production\_hub}.
The investment cost is proportional to the network length.

\sphinxAtStartPar
NonThermalNetwork components are associated with the following exported decision variables:
\begin{itemize}
\item {} 
\sphinxAtStartPar
X\_N(hub\_1, hub\_2), binary.
Whether a connection from hub \sphinxtitleref{hub\_1} to hub \sphinxtitleref{hub\_2} exists and allows a flow of \sphinxtitleref{element}.
Note that X\_N(hub\_1, hub\_2) is different from X\_N(hub\_2, hub\_1).

\item {} 
\sphinxAtStartPar
Y\_N(hub\_1, hub\_2), binary.
Whether a connection between hubs \sphinxtitleref{hub\_1} and \sphinxtitleref{hub\_2} exists and allows a flow of \sphinxtitleref{element}.

\item {} 
\sphinxAtStartPar
X\_SYS(hub), binary.
Whether the hub \sphinxtitleref{hub} is connected to at least one other hub, no matter the direction of the connection.

\item {} 
\sphinxAtStartPar
For all t, F\_SYS(hub, t), continuous, in kW.
The power related to \sphinxtitleref{element} going from hub \sphinxtitleref{hub} to the network.

\item {} 
\sphinxAtStartPar
For all t, F\_N(hub\_1, hub\_2, t), continuous, in kW.
The power related to \sphinxtitleref{element} going from hub \sphinxtitleref{hub\_1} to hub \sphinxtitleref{hub\_2} through the network.
Note that F\_N(hub\_1, hub\_2, t) is the opposite of F\_N(hub\_2, hub\_1, t).

\end{itemize}

\sphinxAtStartPar
NonThermalNetwork components declare the following KPIs:
\begin{itemize}
\item {} 
\sphinxAtStartPar
\sphinxtitleref{COST\_network}
In euros.
Contributes to the “Eco” objective function.

\end{itemize}
\begin{quote}\begin{description}
\sphinxlineitem{Parameters}\begin{itemize}
\item {} 
\sphinxAtStartPar
\sphinxstyleliteralstrong{\sphinxupquote{hubs\_locations}} (\sphinxstyleliteralemphasis{\sphinxupquote{dict \{Hub:}}\sphinxstyleliteralemphasis{\sphinxupquote{ (}}\sphinxstyleliteralemphasis{\sphinxupquote{float}}\sphinxstyleliteralemphasis{\sphinxupquote{, }}\sphinxstyleliteralemphasis{\sphinxupquote{float}}\sphinxstyleliteralemphasis{\sphinxupquote{)}}\sphinxstyleliteralemphasis{\sphinxupquote{\}}}) \textendash{} \begin{itemize}
\item {} 
\sphinxAtStartPar
Keys of \sphinxtitleref{hubs\_locations} are the hubs possibly connected by the network.
They define the \sphinxtitleref{hubs} attribute.

\item {} 
\sphinxAtStartPar
Values of \sphinxtitleref{hubs\_locations} define x and y coordinates in space.
In km.
They describe the position of the hub given the absolute reference (0, 0).
Used to calculate distance between two hubs. The used distance function can be accessed by the
\sphinxtitleref{get\_distance\_function} function and set using \sphinxtitleref{set\_distance\_function} from tamos.network.

\end{itemize}


\item {} 
\sphinxAtStartPar
\sphinxstyleliteralstrong{\sphinxupquote{element}} (\sphinxstyleliteralemphasis{\sphinxupquote{ThermalVectorPair}}) \textendash{} Element exchanged between \sphinxtitleref{hubs}.
Whether \sphinxtitleref{element} is cooled down or warmed up does not define if the network is a heating or cooling network.

\item {} 
\sphinxAtStartPar
\sphinxstyleliteralstrong{\sphinxupquote{properties}} (\sphinxstyleliteralemphasis{\sphinxupquote{dict \{str: int}}\sphinxstyleliteralemphasis{\sphinxupquote{ | }}\sphinxstyleliteralemphasis{\sphinxupquote{float\}}}) \textendash{} 
\sphinxAtStartPar
Techno\sphinxhyphen{}economic properties of the network.
The \sphinxtitleref{properties} attribute must include the following keys:
\begin{itemize}
\item {} 
\sphinxAtStartPar
”Losses (\%/km)”

\item {} 
\sphinxAtStartPar
”CAPEX (EUR/km)”

\item {} 
\sphinxAtStartPar
”OPEX (\%CAPEX)”

\item {} 
\sphinxAtStartPar
”Variable OPEX (EUR/MWh)”

\end{itemize}


\item {} 
\sphinxAtStartPar
\sphinxstyleliteralstrong{\sphinxupquote{production\_hub}} ({\hyperref[\detokenize{generated/tamos.Hub:tamos.Hub}]{\sphinxcrossref{\sphinxstyleliteralemphasis{\sphinxupquote{Hub}}}}}) \textendash{} 
\sphinxAtStartPar
The hub that bears:
\begin{itemize}
\item {} 
\sphinxAtStartPar
all the distribution losses of the network.

\item {} 
\sphinxAtStartPar
the costs associated with the “Variable OPEX (EUR/MWh)” property.

\end{itemize}

\sphinxAtStartPar
Must be one of \sphinxtitleref{hubs} and must be able to exchange \sphinxtitleref{element} with the network.


\item {} 
\sphinxAtStartPar
\sphinxstyleliteralstrong{\sphinxupquote{production\_mode}} (\sphinxstyleliteralemphasis{\sphinxupquote{\{"heat\&cold"}}\sphinxstyleliteralemphasis{\sphinxupquote{, }}\sphinxstyleliteralemphasis{\sphinxupquote{"heat"}}\sphinxstyleliteralemphasis{\sphinxupquote{, }}\sphinxstyleliteralemphasis{\sphinxupquote{"cold"\}}}\sphinxstyleliteralemphasis{\sphinxupquote{, }}\sphinxstyleliteralemphasis{\sphinxupquote{optional}}\sphinxstyleliteralemphasis{\sphinxupquote{, }}\sphinxstyleliteralemphasis{\sphinxupquote{default "heat\&cold"}}) \textendash{} 
\sphinxAtStartPar
Related to \sphinxtitleref{production\_hub}.
Used to speed up the KPI declaration regarding the \sphinxtitleref{Variable OPEX (EUR/MWh)} property.
Provides insight on the sign of the flow going from \sphinxtitleref{production\_hub} to the network.
Whether \sphinxtitleref{element} is cooled down or warmed up is independent from \sphinxtitleref{production\_mode}.
\begin{itemize}
\item {} 
\sphinxAtStartPar
”heat” (“cold”): only heat (cold) is send to the network by \sphinxtitleref{production\_hub}.
The flow is always of the same sign, thus the KPI constraint is like:
energy = sum(a\_given\_sign * power(t) * dt)

\item {} 
\sphinxAtStartPar
”heat” (“cold”): only heat (cold) is send to the network by \sphinxtitleref{production\_hub}.
The sign of the flow may change during the operation period, thus the KPI constraint is like:
energy = sum(abs(power(t)) * dt)

\end{itemize}

\sphinxAtStartPar
Specifying “heat” or “cold” will speed up the KPI declaration but describes a particular state of energy flows.
Specifying “heat\&cold” makes the KPI declaration long (but has no impact on resolution) but works in all cases.


\item {} 
\sphinxAtStartPar
\sphinxstyleliteralstrong{\sphinxupquote{scale\_factor}} (\sphinxstyleliteralemphasis{\sphinxupquote{float}}\sphinxstyleliteralemphasis{\sphinxupquote{, }}\sphinxstyleliteralemphasis{\sphinxupquote{optional}}\sphinxstyleliteralemphasis{\sphinxupquote{, }}\sphinxstyleliteralemphasis{\sphinxupquote{default 1}}) \textendash{} Multiplies the two coordinates of each hub in \sphinxtitleref{hubs\_locations}.
A scale factor \textgreater{}1 tends to increase the distance between two hubs.

\item {} 
\sphinxAtStartPar
\sphinxstyleliteralstrong{\sphinxupquote{eco\_count}} (\sphinxstyleliteralemphasis{\sphinxupquote{bool}}\sphinxstyleliteralemphasis{\sphinxupquote{, }}\sphinxstyleliteralemphasis{\sphinxupquote{optional}}\sphinxstyleliteralemphasis{\sphinxupquote{, }}\sphinxstyleliteralemphasis{\sphinxupquote{default True}}) \textendash{} Whether this instance contributes to the system “Eco” KPI.

\item {} 
\sphinxAtStartPar
\sphinxstyleliteralstrong{\sphinxupquote{name}} (\sphinxstyleliteralemphasis{\sphinxupquote{str}}\sphinxstyleliteralemphasis{\sphinxupquote{, }}\sphinxstyleliteralemphasis{\sphinxupquote{optional}}) \textendash{} 

\end{itemize}

\end{description}\end{quote}
\subsubsection*{Notes}
\begin{enumerate}
\sphinxsetlistlabels{\arabic}{enumi}{enumii}{}{.}%
\item {} 
\sphinxAtStartPar
A network component describe an oriented graph where each node is a hub and each edge is a connection between two hubs.

\item {} 
\sphinxAtStartPar
Connection status between two hubs can be defined using the \sphinxtitleref{set\_connection\_status} and \sphinxtitleref{set\_node\_status} methods.
Per default, all pairs of hubs are connected according to the status \sphinxtitleref{no\_connection}.

\item {} 
\sphinxAtStartPar
The connection of \sphinxtitleref{production\_hub} to every other hub using the network is not mandatory,
i.e. nothing constrains the network graph to be connected.

\item {} 
\sphinxAtStartPar
The variable OPEX applies only on the thermal production of \sphinxtitleref{production\_hub}.
Another MILP implementation could have taken into account all hubs sending power to the network, at the cost of
additional continuous decision variables and constraints.

\end{enumerate}

\end{fulllineitems}

\subsubsection*{Methods}


\begin{savenotes}\sphinxattablestart
\centering
\begin{tabulary}{\linewidth}[t]{\X{1}{2}\X{1}{2}}
\hline

\sphinxAtStartPar
{\hyperref[\detokenize{generated/tamos.network.ThermalNetwork:tamos.network.ThermalNetwork.__init__}]{\sphinxcrossref{\sphinxcode{\sphinxupquote{\_\_init\_\_}}}}}(hubs\_locations, element, ...{[}, ...{]})
&
\sphinxAtStartPar
ThermalNetwork instances makes possible to share ThermalVectorPair elements between hubs.
\\
\hline
\sphinxAtStartPar
{\hyperref[\detokenize{generated/tamos.network.ThermalNetwork:tamos.network.ThermalNetwork.compute_actualized_cost}]{\sphinxcrossref{\sphinxcode{\sphinxupquote{compute\_actualized\_cost}}}}}(CAPEX, OPEX, ...{[}, ...{]})
&
\sphinxAtStartPar
Computes the cost of a component using its \textquotesingle{}Lifetime\textquotesingle{} and \textquotesingle{}Discount rate (\%)\textquotesingle{} properties.
\\
\hline
\sphinxAtStartPar
{\hyperref[\detokenize{generated/tamos.network.ThermalNetwork:tamos.network.ThermalNetwork.generate_MSP}]{\sphinxcrossref{\sphinxcode{\sphinxupquote{generate\_MSP}}}}}({[}excluded\_hubs{]})
&
\sphinxAtStartPar
Defines a minimum spanning tree (MSP) linking all hubs of \sphinxtitleref{hubs} that are not in \sphinxtitleref{excluded\_hubs}.
\\
\hline
\sphinxAtStartPar
{\hyperref[\detokenize{generated/tamos.network.ThermalNetwork:tamos.network.ThermalNetwork.get_connection_power_bounds}]{\sphinxcrossref{\sphinxcode{\sphinxupquote{get\_connection\_power\_bounds}}}}}(hub\_1, hub\_2)
&
\sphinxAtStartPar
Returns the power limits of the flow of element from hub \sphinxtitleref{hub\_1} to hub \sphinxtitleref{hub\_2}.
\\
\hline
\sphinxAtStartPar
{\hyperref[\detokenize{generated/tamos.network.ThermalNetwork:tamos.network.ThermalNetwork.get_connection_status}]{\sphinxcrossref{\sphinxcode{\sphinxupquote{get\_connection\_status}}}}}(hub\_1, hub\_2)
&
\sphinxAtStartPar
Returns the status of the directional connection between two hubs.
\\
\hline
\sphinxAtStartPar
{\hyperref[\detokenize{generated/tamos.network.ThermalNetwork:tamos.network.ThermalNetwork.get_distance}]{\sphinxcrossref{\sphinxcode{\sphinxupquote{get\_distance}}}}}(hub\_1, hub\_2)
&
\sphinxAtStartPar
Returns the distance between two hubs.
\\
\hline
\sphinxAtStartPar
{\hyperref[\detokenize{generated/tamos.network.ThermalNetwork:tamos.network.ThermalNetwork.might_connect}]{\sphinxcrossref{\sphinxcode{\sphinxupquote{might\_connect}}}}}(hub)
&
\sphinxAtStartPar
Checks whether a hub is able to connect to the network.
\\
\hline
\sphinxAtStartPar
{\hyperref[\detokenize{generated/tamos.network.ThermalNetwork:tamos.network.ThermalNetwork.plot}]{\sphinxcrossref{\sphinxcode{\sphinxupquote{plot}}}}}()
&
\sphinxAtStartPar
Plots a representation of the network hubs and connections in the (x, y) space.
\\
\hline
\sphinxAtStartPar
{\hyperref[\detokenize{generated/tamos.network.ThermalNetwork:tamos.network.ThermalNetwork.set_connection_power_bounds}]{\sphinxcrossref{\sphinxcode{\sphinxupquote{set\_connection\_power\_bounds}}}}}(hub\_1, hub\_2{[}, ...{]})
&
\sphinxAtStartPar
Sets power limits on the flow of element from hub \sphinxtitleref{hub\_1} to hub \sphinxtitleref{hub\_2}.
\\
\hline
\sphinxAtStartPar
{\hyperref[\detokenize{generated/tamos.network.ThermalNetwork:tamos.network.ThermalNetwork.set_connection_status}]{\sphinxcrossref{\sphinxcode{\sphinxupquote{set\_connection\_status}}}}}(hub\_1, hub\_2, status)
&
\sphinxAtStartPar
Defines the connection status of the edge going from \sphinxtitleref{hub\_1} to \sphinxtitleref{hub\_2}.
\\
\hline
\sphinxAtStartPar
{\hyperref[\detokenize{generated/tamos.network.ThermalNetwork:tamos.network.ThermalNetwork.set_node_status}]{\sphinxcrossref{\sphinxcode{\sphinxupquote{set\_node\_status}}}}}(hub, status)
&
\sphinxAtStartPar
Defines the connection status of every incoming and outcoming edge of a hub.
\\
\hline
\sphinxAtStartPar
{\hyperref[\detokenize{generated/tamos.network.ThermalNetwork:tamos.network.ThermalNetwork.set_soil_properties}]{\sphinxcrossref{\sphinxcode{\sphinxupquote{set\_soil\_properties}}}}}(soil\_temperature, U{[}, ...{]})
&
\sphinxAtStartPar
Sets the physical properties that define thermal losses.
\\
\hline
\sphinxAtStartPar
{\hyperref[\detokenize{generated/tamos.network.ThermalNetwork:tamos.network.ThermalNetwork.set_status}]{\sphinxcrossref{\sphinxcode{\sphinxupquote{set\_status}}}}}(status{[}, excluded\_hubs{]})
&
\sphinxAtStartPar
Define the connection status of all edges except the ones involving a hub from \sphinxtitleref{excluded\_hubs}.
\\
\hline
\end{tabulary}
\par
\sphinxattableend\end{savenotes}
\subsubsection*{Attributes}


\begin{savenotes}\sphinxattablestart
\centering
\begin{tabulary}{\linewidth}[t]{\X{1}{2}\X{1}{2}}
\hline

\sphinxAtStartPar
\sphinxcode{\sphinxupquote{connection}}
&
\sphinxAtStartPar

\\
\hline
\sphinxAtStartPar
{\hyperref[\detokenize{generated/tamos.network.ThermalNetwork:tamos.network.ThermalNetwork.eco_count}]{\sphinxcrossref{\sphinxcode{\sphinxupquote{eco\_count}}}}}
&
\sphinxAtStartPar
Whether this instance contributes to the system "Eco" KPI bool
\\
\hline
\sphinxAtStartPar
{\hyperref[\detokenize{generated/tamos.network.ThermalNetwork:tamos.network.ThermalNetwork.element}]{\sphinxcrossref{\sphinxcode{\sphinxupquote{element}}}}}
&
\sphinxAtStartPar
Element exchanged between \sphinxtitleref{hubs}.
\\
\hline
\sphinxAtStartPar
{\hyperref[\detokenize{generated/tamos.network.ThermalNetwork:tamos.network.ThermalNetwork.hubs}]{\sphinxcrossref{\sphinxcode{\sphinxupquote{hubs}}}}}
&
\sphinxAtStartPar
The hubs involved in the network definition.
\\
\hline
\sphinxAtStartPar
{\hyperref[\detokenize{generated/tamos.network.ThermalNetwork:tamos.network.ThermalNetwork.name}]{\sphinxcrossref{\sphinxcode{\sphinxupquote{name}}}}}
&
\sphinxAtStartPar
str.
\\
\hline
\sphinxAtStartPar
\sphinxcode{\sphinxupquote{no\_connection}}
&
\sphinxAtStartPar

\\
\hline
\sphinxAtStartPar
\sphinxcode{\sphinxupquote{optim\_one\_way\_max}}
&
\sphinxAtStartPar

\\
\hline
\sphinxAtStartPar
\sphinxcode{\sphinxupquote{optim\_one\_way\_min}}
&
\sphinxAtStartPar

\\
\hline
\sphinxAtStartPar
\sphinxcode{\sphinxupquote{optim\_two\_ways}}
&
\sphinxAtStartPar

\\
\hline
\sphinxAtStartPar
{\hyperref[\detokenize{generated/tamos.network.ThermalNetwork:tamos.network.ThermalNetwork.production_hub}]{\sphinxcrossref{\sphinxcode{\sphinxupquote{production\_hub}}}}}
&
\sphinxAtStartPar
The hub that bears:
\\
\hline
\sphinxAtStartPar
{\hyperref[\detokenize{generated/tamos.network.ThermalNetwork:tamos.network.ThermalNetwork.production_mode}]{\sphinxcrossref{\sphinxcode{\sphinxupquote{production\_mode}}}}}
&
\sphinxAtStartPar
Related to \sphinxtitleref{production\_hub}.
\\
\hline
\sphinxAtStartPar
{\hyperref[\detokenize{generated/tamos.network.ThermalNetwork:tamos.network.ThermalNetwork.scale_factor}]{\sphinxcrossref{\sphinxcode{\sphinxupquote{scale\_factor}}}}}
&
\sphinxAtStartPar
Multiplies the two coordinates of each hub in \sphinxtitleref{hubs\_locations}.
\\
\hline
\sphinxAtStartPar
{\hyperref[\detokenize{generated/tamos.network.ThermalNetwork:tamos.network.ThermalNetwork.used_elements}]{\sphinxcrossref{\sphinxcode{\sphinxupquote{used\_elements}}}}}
&
\sphinxAtStartPar
Elements used by the component.
\\
\hline
\end{tabulary}
\par
\sphinxattableend\end{savenotes}
\index{compute\_actualized\_cost() (tamos.network.ThermalNetwork method)@\spxentry{compute\_actualized\_cost()}\spxextra{tamos.network.ThermalNetwork method}}

\begin{fulllineitems}
\phantomsection\label{\detokenize{generated/tamos.network.ThermalNetwork:tamos.network.ThermalNetwork.compute_actualized_cost}}
\pysigstartsignatures
\pysiglinewithargsret{\sphinxbfcode{\sphinxupquote{compute\_actualized\_cost}}}{\emph{\DUrole{n}{CAPEX}}, \emph{\DUrole{n}{OPEX}}, \emph{\DUrole{n}{system\_lifetime}}, \emph{\DUrole{n}{lifetime}\DUrole{o}{=}\DUrole{default_value}{None}}, \emph{\DUrole{n}{discount\_rate}\DUrole{o}{=}\DUrole{default_value}{None}}}{}
\pysigstopsignatures
\sphinxAtStartPar
Computes the cost of a component using its ‘Lifetime’ and ‘Discount rate (\%)’ properties.
\begin{quote}\begin{description}
\sphinxlineitem{Parameters}\begin{itemize}
\item {} 
\sphinxAtStartPar
\sphinxstyleliteralstrong{\sphinxupquote{CAPEX}} (\sphinxstyleliteralemphasis{\sphinxupquote{float}}) \textendash{} Capital Expenditure. Cost in euros paid every \sphinxtitleref{technical\_lifetime} periods.

\item {} 
\sphinxAtStartPar
\sphinxstyleliteralstrong{\sphinxupquote{OPEX}} (\sphinxstyleliteralemphasis{\sphinxupquote{float}}) \textendash{} Operational Expenditure. Cost in euros paid each period.

\item {} 
\sphinxAtStartPar
\sphinxstyleliteralstrong{\sphinxupquote{system\_lifetime}} (\sphinxstyleliteralemphasis{\sphinxupquote{int}}) \textendash{} Number of periods defining the existence of the energy system.

\item {} 
\sphinxAtStartPar
\sphinxstyleliteralstrong{\sphinxupquote{lifetime}} (\sphinxstyleliteralemphasis{\sphinxupquote{int}}\sphinxstyleliteralemphasis{\sphinxupquote{, }}\sphinxstyleliteralemphasis{\sphinxupquote{optional}}) \textendash{} Number of periods defining the existence of the component.
If specified, overwrite the “Lifetime” property.

\item {} 
\sphinxAtStartPar
\sphinxstyleliteralstrong{\sphinxupquote{discount\_rate}} (\sphinxstyleliteralemphasis{\sphinxupquote{float}}) \textendash{} In percent (\%). Describes the importance of the economic amortization process, per period.
If specified, overwrite the “Discount rate (\%)” property.

\end{itemize}

\sphinxlineitem{Returns}
\sphinxAtStartPar
\begin{itemize}
\item {} 
\sphinxAtStartPar
\sphinxstyleemphasis{A 3\sphinxhyphen{}tuple (total\_cost, CAPEX\_share, OPEX\_share) where}

\item {} 
\sphinxAtStartPar
* CAPEX\_share is the share of total cost related to \sphinxtitleref{CAPEX}

\item {} 
\sphinxAtStartPar
* OPEX\_share is the share of total cost related to \sphinxtitleref{OPEX}

\item {} 
\sphinxAtStartPar
\sphinxstyleemphasis{* total\_cost = CAPEX\_share + OPEX\_share}

\end{itemize}


\end{description}\end{quote}
\subsubsection*{Notes}

\sphinxAtStartPar
Takes into account residual value of component in the case \sphinxtitleref{system\_lifetime} is not a multiple of \sphinxtitleref{lifetime}.
In this case, the last replacement occuring at period replacement\_period is paid in proportion of ‘CAPEX’
depending linearly on the number of periods left:
CAPEX * (system\_lifetime \sphinxhyphen{} replacement\_period) / lifetime

\end{fulllineitems}

\index{eco\_count (tamos.network.ThermalNetwork property)@\spxentry{eco\_count}\spxextra{tamos.network.ThermalNetwork property}}

\begin{fulllineitems}
\phantomsection\label{\detokenize{generated/tamos.network.ThermalNetwork:tamos.network.ThermalNetwork.eco_count}}
\pysigstartsignatures
\pysigline{\sphinxbfcode{\sphinxupquote{property\DUrole{w}{  }}}\sphinxbfcode{\sphinxupquote{eco\_count}}}
\pysigstopsignatures
\sphinxAtStartPar
Whether this instance contributes to the system “Eco” KPI
bool

\end{fulllineitems}

\index{element (tamos.network.ThermalNetwork property)@\spxentry{element}\spxextra{tamos.network.ThermalNetwork property}}

\begin{fulllineitems}
\phantomsection\label{\detokenize{generated/tamos.network.ThermalNetwork:tamos.network.ThermalNetwork.element}}
\pysigstartsignatures
\pysigline{\sphinxbfcode{\sphinxupquote{property\DUrole{w}{  }}}\sphinxbfcode{\sphinxupquote{element}}}
\pysigstopsignatures
\sphinxAtStartPar
Element exchanged between \sphinxtitleref{hubs}.

\sphinxAtStartPar
Whether \sphinxtitleref{element} is cooled down or warmed up does not define if the network is a heating or cooling network.

\end{fulllineitems}

\index{generate\_MSP() (tamos.network.ThermalNetwork method)@\spxentry{generate\_MSP()}\spxextra{tamos.network.ThermalNetwork method}}

\begin{fulllineitems}
\phantomsection\label{\detokenize{generated/tamos.network.ThermalNetwork:tamos.network.ThermalNetwork.generate_MSP}}
\pysigstartsignatures
\pysiglinewithargsret{\sphinxbfcode{\sphinxupquote{generate\_MSP}}}{\emph{\DUrole{n}{excluded\_hubs}\DUrole{o}{=}\DUrole{default_value}{None}}}{}
\pysigstopsignatures
\sphinxAtStartPar
Defines a minimum spanning tree (MSP) linking all hubs of \sphinxtitleref{hubs} that are not in \sphinxtitleref{excluded\_hubs}.

\sphinxAtStartPar
The MSP calculation does not account for bidirectionality thus all edges of the MSP graph are given
the status \sphinxtitleref{optim\_two\_ways}.
The status of the other edges (not part of the MSP) are not modified.
\begin{quote}\begin{description}
\sphinxlineitem{Parameters}
\sphinxAtStartPar
\sphinxstyleliteralstrong{\sphinxupquote{excluded\_hubs}} (\sphinxstyleliteralemphasis{\sphinxupquote{list of Hub}}\sphinxstyleliteralemphasis{\sphinxupquote{, }}\sphinxstyleliteralemphasis{\sphinxupquote{optional}}) \textendash{} All hubs from \sphinxtitleref{excluded\_hubs} must be part of \sphinxtitleref{hubs}.

\end{description}\end{quote}

\end{fulllineitems}

\index{get\_connection\_power\_bounds() (tamos.network.ThermalNetwork method)@\spxentry{get\_connection\_power\_bounds()}\spxextra{tamos.network.ThermalNetwork method}}

\begin{fulllineitems}
\phantomsection\label{\detokenize{generated/tamos.network.ThermalNetwork:tamos.network.ThermalNetwork.get_connection_power_bounds}}
\pysigstartsignatures
\pysiglinewithargsret{\sphinxbfcode{\sphinxupquote{get\_connection\_power\_bounds}}}{\emph{\DUrole{n}{hub\_1}}, \emph{\DUrole{n}{hub\_2}}}{}
\pysigstopsignatures
\sphinxAtStartPar
Returns the power limits of the flow of element from hub \sphinxtitleref{hub\_1} to hub \sphinxtitleref{hub\_2}.
\begin{quote}\begin{description}
\sphinxlineitem{Parameters}\begin{itemize}
\item {} 
\sphinxAtStartPar
\sphinxstyleliteralstrong{\sphinxupquote{hub\_1}} ({\hyperref[\detokenize{generated/tamos.Hub:tamos.Hub}]{\sphinxcrossref{\sphinxstyleliteralemphasis{\sphinxupquote{Hub}}}}}) \textendash{} Must be hubs from \sphinxtitleref{hubs}.

\item {} 
\sphinxAtStartPar
\sphinxstyleliteralstrong{\sphinxupquote{hub\_2}} ({\hyperref[\detokenize{generated/tamos.Hub:tamos.Hub}]{\sphinxcrossref{\sphinxstyleliteralemphasis{\sphinxupquote{Hub}}}}}) \textendash{} Must be hubs from \sphinxtitleref{hubs}.

\end{itemize}

\sphinxlineitem{Return type}
\sphinxAtStartPar
The lower bound and upper bound of the power flow from \sphinxtitleref{hub\_1} to \sphinxtitleref{hub\_2}, in kW.

\end{description}\end{quote}

\end{fulllineitems}

\index{get\_connection\_status() (tamos.network.ThermalNetwork method)@\spxentry{get\_connection\_status()}\spxextra{tamos.network.ThermalNetwork method}}

\begin{fulllineitems}
\phantomsection\label{\detokenize{generated/tamos.network.ThermalNetwork:tamos.network.ThermalNetwork.get_connection_status}}
\pysigstartsignatures
\pysiglinewithargsret{\sphinxbfcode{\sphinxupquote{get\_connection\_status}}}{\emph{\DUrole{n}{hub\_1}}, \emph{\DUrole{n}{hub\_2}}}{}
\pysigstopsignatures
\sphinxAtStartPar
Returns the status of the directional connection between two hubs.
\begin{quote}\begin{description}
\sphinxlineitem{Parameters}\begin{itemize}
\item {} 
\sphinxAtStartPar
\sphinxstyleliteralstrong{\sphinxupquote{hub\_1}} ({\hyperref[\detokenize{generated/tamos.Hub:tamos.Hub}]{\sphinxcrossref{\sphinxstyleliteralemphasis{\sphinxupquote{Hub}}}}}) \textendash{} Must be hubs from \sphinxtitleref{hubs}.

\item {} 
\sphinxAtStartPar
\sphinxstyleliteralstrong{\sphinxupquote{hub\_2}} ({\hyperref[\detokenize{generated/tamos.Hub:tamos.Hub}]{\sphinxcrossref{\sphinxstyleliteralemphasis{\sphinxupquote{Hub}}}}}) \textendash{} Must be hubs from \sphinxtitleref{hubs}.

\end{itemize}

\sphinxlineitem{Returns}
\sphinxAtStartPar
The status of the edge from hub\_1 to hub\_2.

\sphinxlineitem{Return type}
\sphinxAtStartPar
network status

\end{description}\end{quote}

\end{fulllineitems}

\index{get\_distance() (tamos.network.ThermalNetwork method)@\spxentry{get\_distance()}\spxextra{tamos.network.ThermalNetwork method}}

\begin{fulllineitems}
\phantomsection\label{\detokenize{generated/tamos.network.ThermalNetwork:tamos.network.ThermalNetwork.get_distance}}
\pysigstartsignatures
\pysiglinewithargsret{\sphinxbfcode{\sphinxupquote{get\_distance}}}{\emph{\DUrole{n}{hub\_1}}, \emph{\DUrole{n}{hub\_2}}}{}
\pysigstopsignatures
\sphinxAtStartPar
Returns the distance between two hubs.
The used distance function can be accessed by the \sphinxtitleref{get\_distance\_function} function
and set using \sphinxtitleref{set\_distance\_function} from tamos.network.
\begin{quote}\begin{description}
\sphinxlineitem{Parameters}\begin{itemize}
\item {} 
\sphinxAtStartPar
\sphinxstyleliteralstrong{\sphinxupquote{hub\_1}} ({\hyperref[\detokenize{generated/tamos.Hub:tamos.Hub}]{\sphinxcrossref{\sphinxstyleliteralemphasis{\sphinxupquote{Hub}}}}}) \textendash{} Must be from \sphinxtitleref{hubs}.

\item {} 
\sphinxAtStartPar
\sphinxstyleliteralstrong{\sphinxupquote{hub\_2}} ({\hyperref[\detokenize{generated/tamos.Hub:tamos.Hub}]{\sphinxcrossref{\sphinxstyleliteralemphasis{\sphinxupquote{Hub}}}}}) \textendash{} Must be from \sphinxtitleref{hubs}.

\end{itemize}

\sphinxlineitem{Return type}
\sphinxAtStartPar
The distance from \sphinxtitleref{hub\_1} to \sphinxtitleref{hub\_2} which is the same than from \sphinxtitleref{hub\_2} to \sphinxtitleref{hub\_1}

\end{description}\end{quote}

\end{fulllineitems}

\index{hubs (tamos.network.ThermalNetwork property)@\spxentry{hubs}\spxextra{tamos.network.ThermalNetwork property}}

\begin{fulllineitems}
\phantomsection\label{\detokenize{generated/tamos.network.ThermalNetwork:tamos.network.ThermalNetwork.hubs}}
\pysigstartsignatures
\pysigline{\sphinxbfcode{\sphinxupquote{property\DUrole{w}{  }}}\sphinxbfcode{\sphinxupquote{hubs}}}
\pysigstopsignatures
\sphinxAtStartPar
The hubs involved in the network definition.

\sphinxAtStartPar
Some of these hubs might be completely disconnected from the network
(i.e. network.might\_connect(hub) is False) yet they are still considered and associated with MILP decision variables.

\end{fulllineitems}

\index{might\_connect() (tamos.network.ThermalNetwork method)@\spxentry{might\_connect()}\spxextra{tamos.network.ThermalNetwork method}}

\begin{fulllineitems}
\phantomsection\label{\detokenize{generated/tamos.network.ThermalNetwork:tamos.network.ThermalNetwork.might_connect}}
\pysigstartsignatures
\pysiglinewithargsret{\sphinxbfcode{\sphinxupquote{might\_connect}}}{\emph{\DUrole{n}{hub}}}{}
\pysigstopsignatures
\sphinxAtStartPar
Checks whether a hub is able to connect to the network.
\begin{quote}\begin{description}
\sphinxlineitem{Parameters}
\sphinxAtStartPar
\sphinxstyleliteralstrong{\sphinxupquote{hub}} ({\hyperref[\detokenize{generated/tamos.Hub:tamos.Hub}]{\sphinxcrossref{\sphinxstyleliteralemphasis{\sphinxupquote{Hub}}}}}) \textendash{} 

\sphinxlineitem{Returns}
\sphinxAtStartPar
\begin{itemize}
\item {} 
\sphinxAtStartPar
\sphinxstyleemphasis{True if the following two conditions are met}

\item {} 
\sphinxAtStartPar
* Hub \sphinxtitleref{hub} is one of \sphinxtitleref{hubs}.

\item {} 
\sphinxAtStartPar
* There exists at least one hub \sphinxtitleref{hub\_2} in \sphinxtitleref{hubs} such that connection status from \sphinxtitleref{hub} to \sphinxtitleref{hub\_2} or \sphinxtitleref{hub\_2} to \sphinxtitleref{hub} \textendash{} is different from ‘No connection’.

\end{itemize}


\end{description}\end{quote}

\end{fulllineitems}

\index{name (tamos.network.ThermalNetwork property)@\spxentry{name}\spxextra{tamos.network.ThermalNetwork property}}

\begin{fulllineitems}
\phantomsection\label{\detokenize{generated/tamos.network.ThermalNetwork:tamos.network.ThermalNetwork.name}}
\pysigstartsignatures
\pysigline{\sphinxbfcode{\sphinxupquote{property\DUrole{w}{  }}}\sphinxbfcode{\sphinxupquote{name}}}
\pysigstopsignatures
\sphinxAtStartPar
str.
This name is used in MILP model description.
names must not exceed 255 characters,
all of which must be alphanumeric (a\sphinxhyphen{}z, A\sphinxhyphen{}Z, 0\sphinxhyphen{}9) or one of these symbols:
! ” \# \$ \% \& , . ; ? @ \_ ‘ ’ \{ \} \textasciitilde{}.
\begin{quote}\begin{description}
\sphinxlineitem{Type}
\sphinxAtStartPar
Name of the instance

\end{description}\end{quote}

\end{fulllineitems}

\index{plot() (tamos.network.ThermalNetwork method)@\spxentry{plot()}\spxextra{tamos.network.ThermalNetwork method}}

\begin{fulllineitems}
\phantomsection\label{\detokenize{generated/tamos.network.ThermalNetwork:tamos.network.ThermalNetwork.plot}}
\pysigstartsignatures
\pysiglinewithargsret{\sphinxbfcode{\sphinxupquote{plot}}}{}{}
\pysigstopsignatures
\sphinxAtStartPar
Plots a representation of the network hubs and connections in the (x, y) space.

\sphinxAtStartPar
A call to this method gives a visual insight of how the network is parametrized, BEFORE optimization.
The optimization implicitely tranforms every edge status to either ‘No connection’ or ‘Connection’.
\subsubsection*{Notes}

\sphinxAtStartPar
The line linking two hubs is straight for commodity and does not represent
the real distance function used in the MILP model.

\end{fulllineitems}

\index{production\_hub (tamos.network.ThermalNetwork property)@\spxentry{production\_hub}\spxextra{tamos.network.ThermalNetwork property}}

\begin{fulllineitems}
\phantomsection\label{\detokenize{generated/tamos.network.ThermalNetwork:tamos.network.ThermalNetwork.production_hub}}
\pysigstartsignatures
\pysigline{\sphinxbfcode{\sphinxupquote{property\DUrole{w}{  }}}\sphinxbfcode{\sphinxupquote{production\_hub}}}
\pysigstopsignatures
\sphinxAtStartPar
The hub that bears:
\begin{itemize}
\item {} 
\sphinxAtStartPar
all the distribution losses of the network.

\item {} 
\sphinxAtStartPar
the costs associated with the “Variable OPEX (EUR/MWh)” property.

\end{itemize}

\sphinxAtStartPar
Must be one of \sphinxtitleref{hubs} and must be able to exchange \sphinxtitleref{element} with the network.

\end{fulllineitems}

\index{production\_mode (tamos.network.ThermalNetwork property)@\spxentry{production\_mode}\spxextra{tamos.network.ThermalNetwork property}}

\begin{fulllineitems}
\phantomsection\label{\detokenize{generated/tamos.network.ThermalNetwork:tamos.network.ThermalNetwork.production_mode}}
\pysigstartsignatures
\pysigline{\sphinxbfcode{\sphinxupquote{property\DUrole{w}{  }}}\sphinxbfcode{\sphinxupquote{production\_mode}}}
\pysigstopsignatures
\sphinxAtStartPar
Related to \sphinxtitleref{production\_hub}.

\sphinxAtStartPar
\{“heat\&cold”, “heat”, “cold”\}, optional, default “heat\&cold”
Used to speed up the KPI declaration regarding the \sphinxtitleref{Variable OPEX (EUR/MWh)} property.
Provides insight on the sign of the flow going from \sphinxtitleref{production\_hub} to the network.
Whether \sphinxtitleref{element} is cooled down or warmed up is independent from \sphinxtitleref{production\_mode}.
\begin{itemize}
\item {} 
\sphinxAtStartPar
“heat” (“cold”): only heat (cold) is send to the network by \sphinxtitleref{production\_hub}.
The flow is always of the same sign, thus the KPI constraint is like:
energy = sum(a\_given\_sign * power(t) * dt)

\item {} 
\sphinxAtStartPar
“heat” (“cold”): only heat (cold) is send to the network by \sphinxtitleref{production\_hub}.
The sign of the flow may change during the operation period, thus the KPI constraint is like:
energy = sum(abs(power(t)) * dt)

\end{itemize}

\sphinxAtStartPar
Specifying “heat” or “cold” will speed up the KPI declaration but describes a particular state of energy flows.
Specifying “heat\&cold” makes the KPI declaration long (but has no impact on resolution) but works in all cases.

\end{fulllineitems}

\index{scale\_factor (tamos.network.ThermalNetwork property)@\spxentry{scale\_factor}\spxextra{tamos.network.ThermalNetwork property}}

\begin{fulllineitems}
\phantomsection\label{\detokenize{generated/tamos.network.ThermalNetwork:tamos.network.ThermalNetwork.scale_factor}}
\pysigstartsignatures
\pysigline{\sphinxbfcode{\sphinxupquote{property\DUrole{w}{  }}}\sphinxbfcode{\sphinxupquote{scale\_factor}}}
\pysigstopsignatures
\sphinxAtStartPar
Multiplies the two coordinates of each hub in \sphinxtitleref{hubs\_locations}.

\sphinxAtStartPar
A scale factor \textgreater{}1 tends to increase the distance between two hubs.
float

\end{fulllineitems}

\index{set\_connection\_power\_bounds() (tamos.network.ThermalNetwork method)@\spxentry{set\_connection\_power\_bounds()}\spxextra{tamos.network.ThermalNetwork method}}

\begin{fulllineitems}
\phantomsection\label{\detokenize{generated/tamos.network.ThermalNetwork:tamos.network.ThermalNetwork.set_connection_power_bounds}}
\pysigstartsignatures
\pysiglinewithargsret{\sphinxbfcode{\sphinxupquote{set\_connection\_power\_bounds}}}{\emph{\DUrole{n}{hub\_1}}, \emph{\DUrole{n}{hub\_2}}, \emph{\DUrole{n}{power\_lb}\DUrole{o}{=}\DUrole{default_value}{None}}, \emph{\DUrole{n}{power\_ub}\DUrole{o}{=}\DUrole{default_value}{None}}}{}
\pysigstopsignatures
\sphinxAtStartPar
Sets power limits on the flow of element from hub \sphinxtitleref{hub\_1} to hub \sphinxtitleref{hub\_2}.
These limits apply only if the connection from \sphinxtitleref{hub\_1} to \sphinxtitleref{hub\_2} is used.
\begin{quote}\begin{description}
\sphinxlineitem{Parameters}\begin{itemize}
\item {} 
\sphinxAtStartPar
\sphinxstyleliteralstrong{\sphinxupquote{hub\_1}} ({\hyperref[\detokenize{generated/tamos.Hub:tamos.Hub}]{\sphinxcrossref{\sphinxstyleliteralemphasis{\sphinxupquote{Hub}}}}}) \textendash{} Must be hubs from \sphinxtitleref{hubs}.

\item {} 
\sphinxAtStartPar
\sphinxstyleliteralstrong{\sphinxupquote{hub\_2}} ({\hyperref[\detokenize{generated/tamos.Hub:tamos.Hub}]{\sphinxcrossref{\sphinxstyleliteralemphasis{\sphinxupquote{Hub}}}}}) \textendash{} Must be hubs from \sphinxtitleref{hubs}.

\item {} 
\sphinxAtStartPar
\sphinxstyleliteralstrong{\sphinxupquote{power\_lb}} (\sphinxstyleliteralemphasis{\sphinxupquote{int}}\sphinxstyleliteralemphasis{\sphinxupquote{, }}\sphinxstyleliteralemphasis{\sphinxupquote{float}}\sphinxstyleliteralemphasis{\sphinxupquote{ or }}\sphinxstyleliteralemphasis{\sphinxupquote{numpy.ndarray}}) \textendash{} power\_lb \textgreater{}= 0, power\_ub \textgreater{}= 0
In kW.
The lower bound (upper bound) of the flow of \sphinxtitleref{element} from \sphinxtitleref{hub\_1} to \sphinxtitleref{hub\_2}.

\item {} 
\sphinxAtStartPar
\sphinxstyleliteralstrong{\sphinxupquote{power\_ub}} (\sphinxstyleliteralemphasis{\sphinxupquote{int}}\sphinxstyleliteralemphasis{\sphinxupquote{, }}\sphinxstyleliteralemphasis{\sphinxupquote{float}}\sphinxstyleliteralemphasis{\sphinxupquote{ or }}\sphinxstyleliteralemphasis{\sphinxupquote{numpy.ndarray}}) \textendash{} power\_lb \textgreater{}= 0, power\_ub \textgreater{}= 0
In kW.
The lower bound (upper bound) of the flow of \sphinxtitleref{element} from \sphinxtitleref{hub\_1} to \sphinxtitleref{hub\_2}.

\end{itemize}

\end{description}\end{quote}
\subsubsection*{Examples}

\begin{sphinxVerbatim}[commandchars=\\\{\}]
\PYG{g+gp}{\PYGZgt{}\PYGZgt{}\PYGZgt{} }\PYG{n}{network}\PYG{o}{.}\PYG{n}{set\PYGZus{}connection\PYGZus{}power\PYGZus{}bounds}\PYG{p}{(}\PYG{n}{hub\PYGZus{}1}\PYG{p}{,} \PYG{n}{hub\PYGZus{}2}\PYG{p}{,} \PYG{n}{power\PYGZus{}lb}\PYG{o}{=}\PYG{l+m+mi}{400}\PYG{p}{,} \PYG{n}{power\PYGZus{}ub}\PYG{o}{=}\PYG{l+m+mi}{2000}\PYG{p}{)}
\end{sphinxVerbatim}

\sphinxAtStartPar
If the connection \sphinxtitleref{hub\_1} to \sphinxtitleref{hub\_2} exists, the power that flows from \sphinxtitleref{hub\_1} to \sphinxtitleref{hub\_2} must always be in the range
{[}0.4, 2{]} MW.

\begin{sphinxVerbatim}[commandchars=\\\{\}]
\PYG{g+gp}{\PYGZgt{}\PYGZgt{}\PYGZgt{} }\PYG{n}{network}\PYG{o}{.}\PYG{n}{set\PYGZus{}connection\PYGZus{}power\PYGZus{}bounds}\PYG{p}{(}\PYG{n}{hub\PYGZus{}1}\PYG{p}{,} \PYG{n}{hub\PYGZus{}2}\PYG{p}{,} \PYG{n}{power\PYGZus{}lb}\PYG{o}{=}\PYG{l+m+mi}{1000}\PYG{p}{)}
\PYG{g+gp}{\PYGZgt{}\PYGZgt{}\PYGZgt{} }\PYG{n}{network}\PYG{o}{.}\PYG{n}{set\PYGZus{}connection\PYGZus{}power\PYGZus{}bounds}\PYG{p}{(}\PYG{n}{hub\PYGZus{}2}\PYG{p}{,} \PYG{n}{hub\PYGZus{}1}\PYG{p}{,} \PYG{n}{power\PYGZus{}ub}\PYG{o}{=}\PYG{o}{\PYGZhy{}}\PYG{l+m+mi}{1000}\PYG{p}{)}
\end{sphinxVerbatim}

\sphinxAtStartPar
Both calls perform the same operation, but second call is forbidden to make things clearer.

\begin{sphinxVerbatim}[commandchars=\\\{\}]
\PYG{g+gp}{\PYGZgt{}\PYGZgt{}\PYGZgt{} }\PYG{n}{network}\PYG{o}{.}\PYG{n}{set\PYGZus{}connection\PYGZus{}power\PYGZus{}bounds}\PYG{p}{(}\PYG{n}{hub\PYGZus{}1}\PYG{p}{,} \PYG{n}{hub\PYGZus{}2}\PYG{p}{,} \PYG{n}{power\PYGZus{}lb}\PYG{o}{=}\PYG{l+m+mi}{1000}\PYG{p}{)}
\PYG{g+gp}{\PYGZgt{}\PYGZgt{}\PYGZgt{} }\PYG{n}{network}\PYG{o}{.}\PYG{n}{set\PYGZus{}connection\PYGZus{}power\PYGZus{}bounds}\PYG{p}{(}\PYG{n}{hub\PYGZus{}2}\PYG{p}{,} \PYG{n}{hub\PYGZus{}1}\PYG{p}{,} \PYG{n}{power\PYGZus{}lb}\PYG{o}{=}\PYG{l+m+mi}{2000}\PYG{p}{)}
\end{sphinxVerbatim}

\sphinxAtStartPar
The constraints implied by these calls make impossible the existence of both connections
(from \sphinxtitleref{hub\_1} to \sphinxtitleref{hub\_2} and \sphinxtitleref{hub\_2} to \sphinxtitleref{hub\_1}): if 1000 kW of \sphinxtitleref{element} flows from \sphinxtitleref{hub\_1} to \sphinxtitleref{hub\_2}
then \sphinxhyphen{}1000 k\textgreater{} flows from \sphinxtitleref{hub\_2} to \sphinxtitleref{hub\_1} (and vice versa).

\end{fulllineitems}

\index{set\_connection\_status() (tamos.network.ThermalNetwork method)@\spxentry{set\_connection\_status()}\spxextra{tamos.network.ThermalNetwork method}}

\begin{fulllineitems}
\phantomsection\label{\detokenize{generated/tamos.network.ThermalNetwork:tamos.network.ThermalNetwork.set_connection_status}}
\pysigstartsignatures
\pysiglinewithargsret{\sphinxbfcode{\sphinxupquote{set\_connection\_status}}}{\emph{\DUrole{n}{hub\_1}}, \emph{\DUrole{n}{hub\_2}}, \emph{\DUrole{n}{status}}}{}
\pysigstopsignatures
\sphinxAtStartPar
Defines the connection status of the edge going from \sphinxtitleref{hub\_1} to \sphinxtitleref{hub\_2}.
\begin{quote}\begin{description}
\sphinxlineitem{Parameters}\begin{itemize}
\item {} 
\sphinxAtStartPar
\sphinxstyleliteralstrong{\sphinxupquote{hub\_1}} ({\hyperref[\detokenize{generated/tamos.Hub:tamos.Hub}]{\sphinxcrossref{\sphinxstyleliteralemphasis{\sphinxupquote{Hub}}}}}) \textendash{} Must be hubs from \sphinxtitleref{hubs}.

\item {} 
\sphinxAtStartPar
\sphinxstyleliteralstrong{\sphinxupquote{hub\_2}} ({\hyperref[\detokenize{generated/tamos.Hub:tamos.Hub}]{\sphinxcrossref{\sphinxstyleliteralemphasis{\sphinxupquote{Hub}}}}}) \textendash{} Must be hubs from \sphinxtitleref{hubs}.

\item {} 
\sphinxAtStartPar
\sphinxstyleliteralstrong{\sphinxupquote{status}} (\sphinxtitleref{status} may take 5 values that are attributes of this instance:) \textendash{} \begin{itemize}
\item {} 
\sphinxAtStartPar
no\_connection: flow of \sphinxtitleref{element} is forbidden

\item {} 
\sphinxAtStartPar
connection: flow of \sphinxtitleref{element} is possible

\item {} 
\sphinxAtStartPar
optim\_one\_way\_min: flow of \sphinxtitleref{element} must exist in at least one direction
(from hub\_1 to hub\_2, from hub\_2 to hub\_1 or in both directions)
This status also defines the opposite status, e.g. defining ‘hub\_1 to hub\_2’ status defines the one of ‘hub\_2 to hub\_1’.

\item {} 
\sphinxAtStartPar
optim\_one\_way\_max: flow of \sphinxtitleref{element} may exist in at most one direction
(from hub\_1 to hub\_2 or from hub\_2 to hub\_1)
This status also defines the opposite status, e.g. defining ‘hub\_1 to hub\_2’ status defines the one of ‘hub\_2 to hub\_1’.

\item {} 
\sphinxAtStartPar
optim\_two\_ways: flow of \sphinxtitleref{element} may exist in both directions
(from hub\_1 to hub\_2 and from hub\_2 to hub\_1)
This status also defines the opposite status, e.g. defining ‘hub\_1 to hub\_2’ status defines the one of ‘hub\_2 to hub\_1’.

\end{itemize}


\end{itemize}

\end{description}\end{quote}
\subsubsection*{Notes}

\sphinxAtStartPar
Given two hubs \sphinxtitleref{hub\_1} and \sphinxtitleref{hub\_2}, the order of calls to \sphinxtitleref{set\_connection\_status} matters.

\begin{sphinxVerbatim}[commandchars=\\\{\}]
\PYG{g+gp}{\PYGZgt{}\PYGZgt{}\PYGZgt{} }\PYG{n}{network}\PYG{o}{.}\PYG{n}{set\PYGZus{}connection\PYGZus{}status}\PYG{p}{(}\PYG{n}{hub\PYGZus{}1}\PYG{p}{,} \PYG{n}{hub\PYGZus{}2}\PYG{p}{,} \PYG{n}{network}\PYG{o}{.}\PYG{n}{no\PYGZus{}direction}\PYG{p}{)}
\PYG{g+gp}{\PYGZgt{}\PYGZgt{}\PYGZgt{} }\PYG{n}{network}\PYG{o}{.}\PYG{n}{set\PYGZus{}connection\PYGZus{}status}\PYG{p}{(}\PYG{n}{hub\PYGZus{}2}\PYG{p}{,} \PYG{n}{hub\PYGZus{}1}\PYG{p}{,} \PYG{n}{network}\PYG{o}{.}\PYG{n}{optim\PYGZus{}one\PYGZus{}way\PYGZus{}max}\PYG{p}{)}
\PYG{g+gp}{\PYGZgt{}\PYGZgt{}\PYGZgt{} }\PYG{n}{network}\PYG{o}{.}\PYG{n}{get\PYGZus{}connection\PYGZus{}status}\PYG{p}{(}\PYG{n}{hub\PYGZus{}2}\PYG{p}{,} \PYG{n}{hub\PYGZus{}1}\PYG{p}{)} \PYG{o}{==} \PYG{n}{network}\PYG{o}{.}\PYG{n}{get\PYGZus{}connection\PYGZus{}status}\PYG{p}{(}\PYG{n}{hub\PYGZus{}1}\PYG{p}{,} \PYG{n}{hub\PYGZus{}2}\PYG{p}{)}
\PYG{g+go}{    True}
\end{sphinxVerbatim}

\sphinxAtStartPar
The first line has no effect because the second one defines again the connection from \sphinxtitleref{hub\_1} to \sphinxtitleref{hub\_2}.

\end{fulllineitems}

\index{set\_node\_status() (tamos.network.ThermalNetwork method)@\spxentry{set\_node\_status()}\spxextra{tamos.network.ThermalNetwork method}}

\begin{fulllineitems}
\phantomsection\label{\detokenize{generated/tamos.network.ThermalNetwork:tamos.network.ThermalNetwork.set_node_status}}
\pysigstartsignatures
\pysiglinewithargsret{\sphinxbfcode{\sphinxupquote{set\_node\_status}}}{\emph{\DUrole{n}{hub}}, \emph{\DUrole{n}{status}}}{}
\pysigstopsignatures
\sphinxAtStartPar
Defines the connection status of every incoming and outcoming edge of a hub.
\begin{quote}\begin{description}
\sphinxlineitem{Parameters}\begin{itemize}
\item {} 
\sphinxAtStartPar
\sphinxstyleliteralstrong{\sphinxupquote{hub}} ({\hyperref[\detokenize{generated/tamos.Hub:tamos.Hub}]{\sphinxcrossref{\sphinxstyleliteralemphasis{\sphinxupquote{Hub}}}}}) \textendash{} Must be from \sphinxtitleref{hubs}.

\item {} 
\sphinxAtStartPar
\sphinxstyleliteralstrong{\sphinxupquote{status}} (\sphinxtitleref{status} may take 5 values that are attributes of this instance:) \textendash{} \begin{itemize}
\item {} 
\sphinxAtStartPar
no\_connection: flow of \sphinxtitleref{element} is forbidden

\item {} 
\sphinxAtStartPar
connection: flow of \sphinxtitleref{element} is possible

\item {} 
\sphinxAtStartPar
optim\_one\_way\_min: flow of \sphinxtitleref{element} must exist in at least one direction
(from hub\_1 to hub\_2, from hub\_2 to hub\_1 or in both directions)
This status also defines the opposite status, e.g. defining ‘hub\_1 to hub\_2’ status defines the one of ‘hub\_2 to hub\_1’.

\item {} 
\sphinxAtStartPar
optim\_one\_way\_max: flow of \sphinxtitleref{element} may exist in at most one direction
(from hub\_1 to hub\_2 or from hub\_2 to hub\_1)
This status also defines the opposite status, e.g. defining ‘hub\_1 to hub\_2’ status defines the one of ‘hub\_2 to hub\_1’.

\item {} 
\sphinxAtStartPar
optim\_two\_ways: flow of \sphinxtitleref{element} may exist in both directions
(from hub\_1 to hub\_2 and from hub\_2 to hub\_1)
This status also defines the opposite status, e.g. defining ‘hub\_1 to hub\_2’ status defines the one of ‘hub\_2 to hub\_1’.

\end{itemize}


\end{itemize}

\end{description}\end{quote}

\end{fulllineitems}

\index{set\_soil\_properties() (tamos.network.ThermalNetwork method)@\spxentry{set\_soil\_properties()}\spxextra{tamos.network.ThermalNetwork method}}

\begin{fulllineitems}
\phantomsection\label{\detokenize{generated/tamos.network.ThermalNetwork:tamos.network.ThermalNetwork.set_soil_properties}}
\pysigstartsignatures
\pysiglinewithargsret{\sphinxbfcode{\sphinxupquote{set\_soil\_properties}}}{\emph{\DUrole{n}{soil\_temperature}}, \emph{\DUrole{n}{U}}, \emph{\DUrole{n}{losses\_direction: Both\textquotesingle{}}}, \emph{\DUrole{n}{\textquotesingle{}Heat losses\textquotesingle{}}}, \emph{\DUrole{n}{\textquotesingle{}Heat gains = \textquotesingle{}Both\textquotesingle{}}}}{}
\pysigstopsignatures
\sphinxAtStartPar
Sets the physical properties that define thermal losses.
\begin{quote}\begin{description}
\sphinxlineitem{Parameters}\begin{itemize}
\item {} 
\sphinxAtStartPar
\sphinxstyleliteralstrong{\sphinxupquote{soil\_temperature}} (\sphinxstyleliteralemphasis{\sphinxupquote{int}}\sphinxstyleliteralemphasis{\sphinxupquote{, }}\sphinxstyleliteralemphasis{\sphinxupquote{float}}\sphinxstyleliteralemphasis{\sphinxupquote{ or }}\sphinxstyleliteralemphasis{\sphinxupquote{numpy.ndarray}}) \textendash{} In Kelvins (K).
The temperature of the soil at the buried depth of the network pipes.

\item {} 
\sphinxAtStartPar
\sphinxstyleliteralstrong{\sphinxupquote{U}} (\sphinxstyleliteralemphasis{\sphinxupquote{int}}\sphinxstyleliteralemphasis{\sphinxupquote{, }}\sphinxstyleliteralemphasis{\sphinxupquote{float}}\sphinxstyleliteralemphasis{\sphinxupquote{ or }}\sphinxstyleliteralemphasis{\sphinxupquote{numpy.ndarray}}) \textendash{} In W/(m.K).
Specific heat loss per routed meter. Includes both supply  and return pipes.

\item {} 
\sphinxAtStartPar
\sphinxstyleliteralstrong{\sphinxupquote{losses\_direction}} (\sphinxstyleliteralemphasis{\sphinxupquote{\{\textquotesingle{}Both\textquotesingle{}}}\sphinxstyleliteralemphasis{\sphinxupquote{, }}\sphinxstyleliteralemphasis{\sphinxupquote{\textquotesingle{}Heat losses\textquotesingle{}}}\sphinxstyleliteralemphasis{\sphinxupquote{, }}\sphinxstyleliteralemphasis{\sphinxupquote{\textquotesingle{}Heat gains\textquotesingle{}\}}}\sphinxstyleliteralemphasis{\sphinxupquote{, }}\sphinxstyleliteralemphasis{\sphinxupquote{optional}}\sphinxstyleliteralemphasis{\sphinxupquote{, }}\sphinxstyleliteralemphasis{\sphinxupquote{default \textquotesingle{}Both\textquotesingle{}}}) \textendash{} 
\sphinxAtStartPar
Direction of the heat exchanges between the network infrastructure and the soil.
Specifying ‘Heat losses’ (‘Heat gains’) allows to prevent from happening the case where
cold (heat) must be produced in \sphinxtitleref{production\_hub} to compensate thermal gains (losses) in
a district heating (district cooling) network, when thermal demand is lower than soil thermal exchanges.
\begin{itemize}
\item {} 
\sphinxAtStartPar
’Heat losses’: only thermal energy exchanges from the network to the soil are taken into account,
others are set to 0.

\item {} 
\sphinxAtStartPar
’Heat gains’: only thermal energy exchanges from the soil to the network are taken into account,
others are set to 0.

\item {} 
\sphinxAtStartPar
’Both’: all thermal energy exchanges are taken into account.

\end{itemize}


\end{itemize}

\end{description}\end{quote}
\subsubsection*{Notes}
\begin{enumerate}
\sphinxsetlistlabels{\arabic}{enumi}{enumii}{}{.}%
\item {} 
\sphinxAtStartPar
The thermal energy exchanged between the network infrastructure and the soil is like:
thermal\_exchanges(t) = sign * U * network\_length * ((T\_warm(t)+T\_cold(t)) / 2 \sphinxhyphen{} T\_soil(t))
With:
\begin{itemize}
\item {} 
\sphinxAtStartPar
network\_length: the length of all the edges in the network.
If connections are bidirectional, length is accounted for only once.

\item {} 
\sphinxAtStartPar
T\_warm(t): temperature of the warm vector of \sphinxtitleref{element}

\item {} 
\sphinxAtStartPar
T\_cold(t): temperature of the cold vector of \sphinxtitleref{element}

\item {} 
\sphinxAtStartPar
T\_soil(t): temperature of the soil

\end{itemize}

\item {} 
\sphinxAtStartPar
By default, set\_soil\_properties is called with \sphinxtitleref{soil\_temperature} =273+10, \sphinxtitleref{U} =0.7, \sphinxtitleref{losses\_direction} =’Both’.

\item {} 
\sphinxAtStartPar
This method does not assume pipes are laid down underground: setting a \sphinxtitleref{U} value according to the \sphinxtitleref{soil\_temperature}
value is enough to describe any surrounding environment of the pipes.

\end{enumerate}

\end{fulllineitems}

\index{set\_status() (tamos.network.ThermalNetwork method)@\spxentry{set\_status()}\spxextra{tamos.network.ThermalNetwork method}}

\begin{fulllineitems}
\phantomsection\label{\detokenize{generated/tamos.network.ThermalNetwork:tamos.network.ThermalNetwork.set_status}}
\pysigstartsignatures
\pysiglinewithargsret{\sphinxbfcode{\sphinxupquote{set\_status}}}{\emph{\DUrole{n}{status}}, \emph{\DUrole{n}{excluded\_hubs}\DUrole{o}{=}\DUrole{default_value}{None}}}{}
\pysigstopsignatures
\sphinxAtStartPar
Define the connection status of all edges except the ones involving a hub from \sphinxtitleref{excluded\_hubs}.
\begin{quote}\begin{description}
\sphinxlineitem{Parameters}\begin{itemize}
\item {} 
\sphinxAtStartPar
\sphinxstyleliteralstrong{\sphinxupquote{status}} (\sphinxtitleref{status} may take 5 values that are attributes of this instance:) \textendash{} \begin{itemize}
\item {} 
\sphinxAtStartPar
no\_connection: flow of \sphinxtitleref{element} is forbidden

\item {} 
\sphinxAtStartPar
connection: flow of \sphinxtitleref{element} is possible

\item {} 
\sphinxAtStartPar
optim\_one\_way\_min: flow of \sphinxtitleref{element} must exist in at least one direction
(from hub\_1 to hub\_2, from hub\_2 to hub\_1 or in both directions)
This status also defines the opposite status, e.g. defining ‘hub\_1 to hub\_2’ status defines the one of ‘hub\_2 to hub\_1’.

\item {} 
\sphinxAtStartPar
optim\_one\_way\_max: flow of \sphinxtitleref{element} may exist in at most one direction
(from hub\_1 to hub\_2 or from hub\_2 to hub\_1)
This status also defines the opposite status, e.g. defining ‘hub\_1 to hub\_2’ status defines the one of ‘hub\_2 to hub\_1’.

\item {} 
\sphinxAtStartPar
optim\_two\_ways: flow of \sphinxtitleref{element} may exist in both directions
(from hub\_1 to hub\_2 and from hub\_2 to hub\_1)
This status also defines the opposite status, e.g. defining ‘hub\_1 to hub\_2’ status defines the one of ‘hub\_2 to hub\_1’.

\end{itemize}


\item {} 
\sphinxAtStartPar
\sphinxstyleliteralstrong{\sphinxupquote{excluded\_hubs}} (\sphinxstyleliteralemphasis{\sphinxupquote{list of Hub}}\sphinxstyleliteralemphasis{\sphinxupquote{, }}\sphinxstyleliteralemphasis{\sphinxupquote{optional}}) \textendash{} All hubs from \sphinxtitleref{excluded\_hubs} must be part of \sphinxtitleref{hubs}.

\end{itemize}

\end{description}\end{quote}

\end{fulllineitems}

\index{used\_elements (tamos.network.ThermalNetwork property)@\spxentry{used\_elements}\spxextra{tamos.network.ThermalNetwork property}}

\begin{fulllineitems}
\phantomsection\label{\detokenize{generated/tamos.network.ThermalNetwork:tamos.network.ThermalNetwork.used_elements}}
\pysigstartsignatures
\pysigline{\sphinxbfcode{\sphinxupquote{property\DUrole{w}{  }}}\sphinxbfcode{\sphinxupquote{used\_elements}}}
\pysigstopsignatures
\sphinxAtStartPar
Elements used by the component.

\end{fulllineitems}


\end{fulllineitems}


\sphinxstepscope


\subsection{tamos.network.HREThermalNetwork}
\label{\detokenize{generated/tamos.network.HREThermalNetwork:tamos-network-hrethermalnetwork}}\label{\detokenize{generated/tamos.network.HREThermalNetwork::doc}}\index{HREThermalNetwork (class in tamos.network)@\spxentry{HREThermalNetwork}\spxextra{class in tamos.network}}

\begin{fulllineitems}
\phantomsection\label{\detokenize{generated/tamos.network.HREThermalNetwork:tamos.network.HREThermalNetwork}}
\pysigstartsignatures
\pysiglinewithargsret{\sphinxbfcode{\sphinxupquote{class\DUrole{w}{  }}}\sphinxcode{\sphinxupquote{tamos.network.}}\sphinxbfcode{\sphinxupquote{HREThermalNetwork}}}{\emph{\DUrole{n}{hubs\_locations}}, \emph{\DUrole{n}{element}}, \emph{\DUrole{n}{properties}}, \emph{\DUrole{n}{production\_hub}}, \emph{\DUrole{n}{production\_mode}\DUrole{o}{=}\DUrole{default_value}{\textquotesingle{}heat\&cold\textquotesingle{}}}, \emph{\DUrole{n}{scale\_factor}\DUrole{o}{=}\DUrole{default_value}{1}}, \emph{\DUrole{n}{eco\_count}\DUrole{o}{=}\DUrole{default_value}{True}}, \emph{\DUrole{n}{name}\DUrole{o}{=}\DUrole{default_value}{None}}}{}
\pysigstopsignatures\index{\_\_init\_\_() (tamos.network.HREThermalNetwork method)@\spxentry{\_\_init\_\_()}\spxextra{tamos.network.HREThermalNetwork method}}

\begin{fulllineitems}
\phantomsection\label{\detokenize{generated/tamos.network.HREThermalNetwork:tamos.network.HREThermalNetwork.__init__}}
\pysigstartsignatures
\pysiglinewithargsret{\sphinxbfcode{\sphinxupquote{\_\_init\_\_}}}{\emph{\DUrole{n}{hubs\_locations}}, \emph{\DUrole{n}{element}}, \emph{\DUrole{n}{properties}}, \emph{\DUrole{n}{production\_hub}}, \emph{\DUrole{n}{production\_mode}\DUrole{o}{=}\DUrole{default_value}{\textquotesingle{}heat\&cold\textquotesingle{}}}, \emph{\DUrole{n}{scale\_factor}\DUrole{o}{=}\DUrole{default_value}{1}}, \emph{\DUrole{n}{eco\_count}\DUrole{o}{=}\DUrole{default_value}{True}}, \emph{\DUrole{n}{name}\DUrole{o}{=}\DUrole{default_value}{None}}}{}
\pysigstopsignatures
\sphinxAtStartPar
HREThermalNetwork is a thermal network model that differs from \sphinxtitleref{ThermalNetwork} by an investment cost
being dependant from the annual network linear energy density. It is an adaptation of the model described in
(Persson et al., 2019) (%
\begin{footnote}[1]\sphinxAtStartFootnote
Persson U, Wiechers E, Möller B, Werner S. Heat Roadmap Europe: Heat distribution costs.
Energy 2019;176:604\textendash{}22. \sphinxurl{https://doi.org/10.1016/j.energy.2019.03.189}.
%
\end{footnote}, %
\begin{footnote}[2]\sphinxAtStartFootnote
Persson U, Werner S. Heat distribution and the future competitiveness of district heating.
Applied Energy 2011;88:568\textendash{}76. \sphinxurl{https://doi.org/10.1016/j.apenergy.2010.09.020}.
%
\end{footnote}).

\sphinxAtStartPar
Power is exchanged between two hubs, given a distribution losses proportional to the difference between
network temperature and soil temperature.
All distribution losses must be compensated for by an additional power in \sphinxtitleref{production\_hub}.

\sphinxAtStartPar
NonThermalNetwork components are associated with the following exported decision variables:
\begin{itemize}
\item {} 
\sphinxAtStartPar
X\_N(hub\_1, hub\_2), binary.
Whether a connection from hub \sphinxtitleref{hub\_1} to hub \sphinxtitleref{hub\_2} exists and allows a flow of \sphinxtitleref{element}.
Note that X\_N(hub\_1, hub\_2) is different from X\_N(hub\_2, hub\_1).

\item {} 
\sphinxAtStartPar
Y\_N(hub\_1, hub\_2), binary.
Whether a connection between hubs \sphinxtitleref{hub\_1} and \sphinxtitleref{hub\_2} exists and allows a flow of \sphinxtitleref{element}.

\item {} 
\sphinxAtStartPar
X\_SYS(hub), binary.
Whether the hub \sphinxtitleref{hub} is connected to at least one other hub, no matter the direction of the connection.

\item {} 
\sphinxAtStartPar
For all t, F\_SYS(hub, t), continuous, in kW.
The power related to \sphinxtitleref{element} going from hub \sphinxtitleref{hub} to the network.

\item {} 
\sphinxAtStartPar
For all t, F\_N(hub\_1, hub\_2, t), continuous, in kW.
The power related to \sphinxtitleref{element} going from hub \sphinxtitleref{hub\_1} to hub \sphinxtitleref{hub\_2} through the network.
Note that F\_N(hub\_1, hub\_2, t) is the opposite of F\_N(hub\_2, hub\_1, t).

\end{itemize}

\sphinxAtStartPar
NonThermalNetwork components declare the following KPIs:
\begin{itemize}
\item {} 
\sphinxAtStartPar
\sphinxtitleref{COST\_network}
In euros.
Contributes to the “Eco” objective function.

\end{itemize}
\begin{quote}\begin{description}
\sphinxlineitem{Parameters}\begin{itemize}
\item {} 
\sphinxAtStartPar
\sphinxstyleliteralstrong{\sphinxupquote{hubs\_locations}} (\sphinxstyleliteralemphasis{\sphinxupquote{dict \{Hub:}}\sphinxstyleliteralemphasis{\sphinxupquote{ (}}\sphinxstyleliteralemphasis{\sphinxupquote{float}}\sphinxstyleliteralemphasis{\sphinxupquote{, }}\sphinxstyleliteralemphasis{\sphinxupquote{float}}\sphinxstyleliteralemphasis{\sphinxupquote{)}}\sphinxstyleliteralemphasis{\sphinxupquote{\}}}) \textendash{} \begin{itemize}
\item {} 
\sphinxAtStartPar
Keys of \sphinxtitleref{hubs\_locations} are the hubs possibly connected by the network.
They define the \sphinxtitleref{hubs} attribute.

\item {} 
\sphinxAtStartPar
Values of \sphinxtitleref{hubs\_locations} define x and y coordinates in space.
In km.
They describe the position of the hub given the absolute reference (0, 0).
Used to calculate distance between two hubs. The used distance function can be accessed by the
\sphinxtitleref{get\_distance\_function} function and set using \sphinxtitleref{set\_distance\_function} from tamos.network.

\end{itemize}


\item {} 
\sphinxAtStartPar
\sphinxstyleliteralstrong{\sphinxupquote{element}} (\sphinxstyleliteralemphasis{\sphinxupquote{ThermalVectorPair}}) \textendash{} Element exchanged between \sphinxtitleref{hubs}.
Whether \sphinxtitleref{element} is cooled down or warmed up does not define if the network is a heating or cooling network.

\item {} 
\sphinxAtStartPar
\sphinxstyleliteralstrong{\sphinxupquote{properties}} (\sphinxstyleliteralemphasis{\sphinxupquote{dict \{str: int}}\sphinxstyleliteralemphasis{\sphinxupquote{ | }}\sphinxstyleliteralemphasis{\sphinxupquote{float\}}}) \textendash{} 
\sphinxAtStartPar
Techno\sphinxhyphen{}economic properties of the network.
The \sphinxtitleref{properties} attribute must include the following keys:
\begin{itemize}
\item {} 
\sphinxAtStartPar
”Losses (\%/km)”

\item {} 
\sphinxAtStartPar
”OPEX (\%CAPEX)”

\item {} 
\sphinxAtStartPar
”Variable OPEX (EUR/MWh)”

\item {} 
\sphinxAtStartPar
”Used network length (m)”

\item {} 
\sphinxAtStartPar
”Capex subsidy (\%)”

\end{itemize}


\item {} 
\sphinxAtStartPar
\sphinxstyleliteralstrong{\sphinxupquote{production\_hub}} ({\hyperref[\detokenize{generated/tamos.Hub:tamos.Hub}]{\sphinxcrossref{\sphinxstyleliteralemphasis{\sphinxupquote{Hub}}}}}) \textendash{} 
\sphinxAtStartPar
The hub that bears:
\begin{itemize}
\item {} 
\sphinxAtStartPar
all the distribution losses of the network.

\item {} 
\sphinxAtStartPar
the costs associated with the “Variable OPEX (EUR/MWh)” property.

\item {} 
\sphinxAtStartPar
the investment cost, which is associated with the properties
“OPEX (\%CAPEX)”, “Used network length (m)” and “Capex subsidy (\%)”

\end{itemize}

\sphinxAtStartPar
Must be one of \sphinxtitleref{hubs} and must be able to exchange \sphinxtitleref{element} with the network.


\item {} 
\sphinxAtStartPar
\sphinxstyleliteralstrong{\sphinxupquote{production\_mode}} (\sphinxstyleliteralemphasis{\sphinxupquote{\{"heat\&cold"}}\sphinxstyleliteralemphasis{\sphinxupquote{, }}\sphinxstyleliteralemphasis{\sphinxupquote{"heat"}}\sphinxstyleliteralemphasis{\sphinxupquote{, }}\sphinxstyleliteralemphasis{\sphinxupquote{"cold"\}}}\sphinxstyleliteralemphasis{\sphinxupquote{, }}\sphinxstyleliteralemphasis{\sphinxupquote{optional}}\sphinxstyleliteralemphasis{\sphinxupquote{, }}\sphinxstyleliteralemphasis{\sphinxupquote{default "heat\&cold"}}) \textendash{} 
\sphinxAtStartPar
Related to \sphinxtitleref{production\_hub}.
Used to speed up the KPI declaration regarding the \sphinxtitleref{Variable OPEX (EUR/MWh)} property.
Provides insight on the sign of the flow going from \sphinxtitleref{production\_hub} to the network.
Whether \sphinxtitleref{element} is cooled down or warmed up is independent from \sphinxtitleref{production\_mode}.
\begin{itemize}
\item {} 
\sphinxAtStartPar
”heat” (“cold”): only heat (cold) is send to the network by \sphinxtitleref{production\_hub}.
The flow is always of the same sign, thus the KPI constraint is like:
energy = sum(a\_given\_sign * power(t) * dt)

\item {} 
\sphinxAtStartPar
”heat” (“cold”): only heat (cold) is send to the network by \sphinxtitleref{production\_hub}.
The sign of the flow may change during the operation period, thus the KPI constraint is like:
energy = sum(abs(power(t)) * dt)

\end{itemize}

\sphinxAtStartPar
Specifying “heat” or “cold” will speed up the KPI declaration but describes a particular state of energy flows.
Specifying “heat\&cold” makes the KPI declaration long (but has no impact on resolution) but works in all cases.


\item {} 
\sphinxAtStartPar
\sphinxstyleliteralstrong{\sphinxupquote{scale\_factor}} (\sphinxstyleliteralemphasis{\sphinxupquote{float}}\sphinxstyleliteralemphasis{\sphinxupquote{, }}\sphinxstyleliteralemphasis{\sphinxupquote{optional}}\sphinxstyleliteralemphasis{\sphinxupquote{, }}\sphinxstyleliteralemphasis{\sphinxupquote{default 1}}) \textendash{} Multiplies the two coordinates of each hub in \sphinxtitleref{hubs\_locations}.
A scale factor \textgreater{}1 tends to increase the distance between two hubs.

\item {} 
\sphinxAtStartPar
\sphinxstyleliteralstrong{\sphinxupquote{eco\_count}} (\sphinxstyleliteralemphasis{\sphinxupquote{bool}}\sphinxstyleliteralemphasis{\sphinxupquote{, }}\sphinxstyleliteralemphasis{\sphinxupquote{optional}}\sphinxstyleliteralemphasis{\sphinxupquote{, }}\sphinxstyleliteralemphasis{\sphinxupquote{default True}}) \textendash{} Whether this instance contributes to the system “Eco” KPI.

\item {} 
\sphinxAtStartPar
\sphinxstyleliteralstrong{\sphinxupquote{name}} (\sphinxstyleliteralemphasis{\sphinxupquote{str}}\sphinxstyleliteralemphasis{\sphinxupquote{, }}\sphinxstyleliteralemphasis{\sphinxupquote{optional}}) \textendash{} 

\end{itemize}

\end{description}\end{quote}
\subsubsection*{Notes}
\begin{enumerate}
\sphinxsetlistlabels{\arabic}{enumi}{enumii}{}{.}%
\item {} 
\sphinxAtStartPar
A network component describe an oriented graph where each node is a hub and each edge is a connection between two hubs.

\item {} 
\sphinxAtStartPar
Connection status between two hubs can be defined using the \sphinxtitleref{set\_connection\_status} and \sphinxtitleref{set\_node\_status} methods.
Per default, all pairs of hubs are connected according to the status \sphinxtitleref{no\_connection}.

\item {} 
\sphinxAtStartPar
The connection of \sphinxtitleref{production\_hub} to every other hub using the network is not mandatory,
i.e. nothing constrains the network graph to be connected.

\item {} 
\sphinxAtStartPar
The variable OPEX applies only on the thermal production of \sphinxtitleref{production\_hub}.
Another MILP implementation could have taken into account all hubs sending power to the network, at the cost of
additional continuous decision variables and constraints.

\item {} 
\sphinxAtStartPar
The linear heat density calculation implies a non\sphinxhyphen{}linear expression involving the network length L.
This model relies on the network length being a parameter, “Used network length (m)”.

\item {} 
\sphinxAtStartPar
The property “Capex subsidy (\%)” decreases the investment cost but does not impact the variable cost.

\end{enumerate}
\subsubsection*{References}

\end{fulllineitems}

\subsubsection*{Methods}


\begin{savenotes}\sphinxattablestart
\centering
\begin{tabulary}{\linewidth}[t]{\X{1}{2}\X{1}{2}}
\hline

\sphinxAtStartPar
{\hyperref[\detokenize{generated/tamos.network.HREThermalNetwork:tamos.network.HREThermalNetwork.__init__}]{\sphinxcrossref{\sphinxcode{\sphinxupquote{\_\_init\_\_}}}}}(hubs\_locations, element, ...{[}, ...{]})
&
\sphinxAtStartPar
HREThermalNetwork is a thermal network model that differs from \sphinxtitleref{ThermalNetwork} by an investment cost being dependant from the annual network linear energy density.
\\
\hline
\sphinxAtStartPar
{\hyperref[\detokenize{generated/tamos.network.HREThermalNetwork:tamos.network.HREThermalNetwork.compute_actualized_cost}]{\sphinxcrossref{\sphinxcode{\sphinxupquote{compute\_actualized\_cost}}}}}(CAPEX, OPEX, ...{[}, ...{]})
&
\sphinxAtStartPar
Computes the cost of a component using its \textquotesingle{}Lifetime\textquotesingle{} and \textquotesingle{}Discount rate (\%)\textquotesingle{} properties.
\\
\hline
\sphinxAtStartPar
{\hyperref[\detokenize{generated/tamos.network.HREThermalNetwork:tamos.network.HREThermalNetwork.generate_MSP}]{\sphinxcrossref{\sphinxcode{\sphinxupquote{generate\_MSP}}}}}({[}excluded\_hubs{]})
&
\sphinxAtStartPar
Defines a minimum spanning tree (MSP) linking all hubs of \sphinxtitleref{hubs} that are not in \sphinxtitleref{excluded\_hubs}.
\\
\hline
\sphinxAtStartPar
{\hyperref[\detokenize{generated/tamos.network.HREThermalNetwork:tamos.network.HREThermalNetwork.get_connection_power_bounds}]{\sphinxcrossref{\sphinxcode{\sphinxupquote{get\_connection\_power\_bounds}}}}}(hub\_1, hub\_2)
&
\sphinxAtStartPar
Returns the power limits of the flow of element from hub \sphinxtitleref{hub\_1} to hub \sphinxtitleref{hub\_2}.
\\
\hline
\sphinxAtStartPar
{\hyperref[\detokenize{generated/tamos.network.HREThermalNetwork:tamos.network.HREThermalNetwork.get_connection_status}]{\sphinxcrossref{\sphinxcode{\sphinxupquote{get\_connection\_status}}}}}(hub\_1, hub\_2)
&
\sphinxAtStartPar
Returns the status of the directional connection between two hubs.
\\
\hline
\sphinxAtStartPar
{\hyperref[\detokenize{generated/tamos.network.HREThermalNetwork:tamos.network.HREThermalNetwork.get_distance}]{\sphinxcrossref{\sphinxcode{\sphinxupquote{get\_distance}}}}}(hub\_1, hub\_2)
&
\sphinxAtStartPar
Returns the distance between two hubs.
\\
\hline
\sphinxAtStartPar
{\hyperref[\detokenize{generated/tamos.network.HREThermalNetwork:tamos.network.HREThermalNetwork.might_connect}]{\sphinxcrossref{\sphinxcode{\sphinxupquote{might\_connect}}}}}(hub)
&
\sphinxAtStartPar
Checks whether a hub is able to connect to the network.
\\
\hline
\sphinxAtStartPar
{\hyperref[\detokenize{generated/tamos.network.HREThermalNetwork:tamos.network.HREThermalNetwork.plot}]{\sphinxcrossref{\sphinxcode{\sphinxupquote{plot}}}}}()
&
\sphinxAtStartPar
Plots a representation of the network hubs and connections in the (x, y) space.
\\
\hline
\sphinxAtStartPar
{\hyperref[\detokenize{generated/tamos.network.HREThermalNetwork:tamos.network.HREThermalNetwork.set_connection_power_bounds}]{\sphinxcrossref{\sphinxcode{\sphinxupquote{set\_connection\_power\_bounds}}}}}(hub\_1, hub\_2{[}, ...{]})
&
\sphinxAtStartPar
Sets power limits on the flow of element from hub \sphinxtitleref{hub\_1} to hub \sphinxtitleref{hub\_2}.
\\
\hline
\sphinxAtStartPar
{\hyperref[\detokenize{generated/tamos.network.HREThermalNetwork:tamos.network.HREThermalNetwork.set_connection_status}]{\sphinxcrossref{\sphinxcode{\sphinxupquote{set\_connection\_status}}}}}(hub\_1, hub\_2, status)
&
\sphinxAtStartPar
Defines the connection status of the edge going from \sphinxtitleref{hub\_1} to \sphinxtitleref{hub\_2}.
\\
\hline
\sphinxAtStartPar
{\hyperref[\detokenize{generated/tamos.network.HREThermalNetwork:tamos.network.HREThermalNetwork.set_node_status}]{\sphinxcrossref{\sphinxcode{\sphinxupquote{set\_node\_status}}}}}(hub, status)
&
\sphinxAtStartPar
Defines the connection status of every incoming and outcoming edge of a hub.
\\
\hline
\sphinxAtStartPar
{\hyperref[\detokenize{generated/tamos.network.HREThermalNetwork:tamos.network.HREThermalNetwork.set_soil_properties}]{\sphinxcrossref{\sphinxcode{\sphinxupquote{set\_soil\_properties}}}}}(soil\_temperature, U{[}, ...{]})
&
\sphinxAtStartPar
Sets the physical properties that define thermal losses.
\\
\hline
\sphinxAtStartPar
{\hyperref[\detokenize{generated/tamos.network.HREThermalNetwork:tamos.network.HREThermalNetwork.set_status}]{\sphinxcrossref{\sphinxcode{\sphinxupquote{set\_status}}}}}(status{[}, excluded\_hubs{]})
&
\sphinxAtStartPar
Define the connection status of all edges except the ones involving a hub from \sphinxtitleref{excluded\_hubs}.
\\
\hline
\end{tabulary}
\par
\sphinxattableend\end{savenotes}
\subsubsection*{Attributes}


\begin{savenotes}\sphinxattablestart
\centering
\begin{tabulary}{\linewidth}[t]{\X{1}{2}\X{1}{2}}
\hline

\sphinxAtStartPar
\sphinxcode{\sphinxupquote{connection}}
&
\sphinxAtStartPar

\\
\hline
\sphinxAtStartPar
{\hyperref[\detokenize{generated/tamos.network.HREThermalNetwork:tamos.network.HREThermalNetwork.eco_count}]{\sphinxcrossref{\sphinxcode{\sphinxupquote{eco\_count}}}}}
&
\sphinxAtStartPar
Whether this instance contributes to the system "Eco" KPI bool
\\
\hline
\sphinxAtStartPar
{\hyperref[\detokenize{generated/tamos.network.HREThermalNetwork:tamos.network.HREThermalNetwork.element}]{\sphinxcrossref{\sphinxcode{\sphinxupquote{element}}}}}
&
\sphinxAtStartPar
Element exchanged between \sphinxtitleref{hubs}.
\\
\hline
\sphinxAtStartPar
{\hyperref[\detokenize{generated/tamos.network.HREThermalNetwork:tamos.network.HREThermalNetwork.hubs}]{\sphinxcrossref{\sphinxcode{\sphinxupquote{hubs}}}}}
&
\sphinxAtStartPar
The hubs involved in the network definition.
\\
\hline
\sphinxAtStartPar
{\hyperref[\detokenize{generated/tamos.network.HREThermalNetwork:tamos.network.HREThermalNetwork.name}]{\sphinxcrossref{\sphinxcode{\sphinxupquote{name}}}}}
&
\sphinxAtStartPar
str.
\\
\hline
\sphinxAtStartPar
\sphinxcode{\sphinxupquote{no\_connection}}
&
\sphinxAtStartPar

\\
\hline
\sphinxAtStartPar
\sphinxcode{\sphinxupquote{optim\_one\_way\_max}}
&
\sphinxAtStartPar

\\
\hline
\sphinxAtStartPar
\sphinxcode{\sphinxupquote{optim\_one\_way\_min}}
&
\sphinxAtStartPar

\\
\hline
\sphinxAtStartPar
\sphinxcode{\sphinxupquote{optim\_two\_ways}}
&
\sphinxAtStartPar

\\
\hline
\sphinxAtStartPar
{\hyperref[\detokenize{generated/tamos.network.HREThermalNetwork:tamos.network.HREThermalNetwork.production_hub}]{\sphinxcrossref{\sphinxcode{\sphinxupquote{production\_hub}}}}}
&
\sphinxAtStartPar
The hub that bears:
\\
\hline
\sphinxAtStartPar
{\hyperref[\detokenize{generated/tamos.network.HREThermalNetwork:tamos.network.HREThermalNetwork.production_mode}]{\sphinxcrossref{\sphinxcode{\sphinxupquote{production\_mode}}}}}
&
\sphinxAtStartPar
Related to \sphinxtitleref{production\_hub}.
\\
\hline
\sphinxAtStartPar
{\hyperref[\detokenize{generated/tamos.network.HREThermalNetwork:tamos.network.HREThermalNetwork.scale_factor}]{\sphinxcrossref{\sphinxcode{\sphinxupquote{scale\_factor}}}}}
&
\sphinxAtStartPar
Multiplies the two coordinates of each hub in \sphinxtitleref{hubs\_locations}.
\\
\hline
\sphinxAtStartPar
{\hyperref[\detokenize{generated/tamos.network.HREThermalNetwork:tamos.network.HREThermalNetwork.used_elements}]{\sphinxcrossref{\sphinxcode{\sphinxupquote{used\_elements}}}}}
&
\sphinxAtStartPar
Elements used by the component.
\\
\hline
\end{tabulary}
\par
\sphinxattableend\end{savenotes}
\index{compute\_actualized\_cost() (tamos.network.HREThermalNetwork method)@\spxentry{compute\_actualized\_cost()}\spxextra{tamos.network.HREThermalNetwork method}}

\begin{fulllineitems}
\phantomsection\label{\detokenize{generated/tamos.network.HREThermalNetwork:tamos.network.HREThermalNetwork.compute_actualized_cost}}
\pysigstartsignatures
\pysiglinewithargsret{\sphinxbfcode{\sphinxupquote{compute\_actualized\_cost}}}{\emph{\DUrole{n}{CAPEX}}, \emph{\DUrole{n}{OPEX}}, \emph{\DUrole{n}{system\_lifetime}}, \emph{\DUrole{n}{lifetime}\DUrole{o}{=}\DUrole{default_value}{None}}, \emph{\DUrole{n}{discount\_rate}\DUrole{o}{=}\DUrole{default_value}{None}}}{}
\pysigstopsignatures
\sphinxAtStartPar
Computes the cost of a component using its ‘Lifetime’ and ‘Discount rate (\%)’ properties.
\begin{quote}\begin{description}
\sphinxlineitem{Parameters}\begin{itemize}
\item {} 
\sphinxAtStartPar
\sphinxstyleliteralstrong{\sphinxupquote{CAPEX}} (\sphinxstyleliteralemphasis{\sphinxupquote{float}}) \textendash{} Capital Expenditure. Cost in euros paid every \sphinxtitleref{technical\_lifetime} periods.

\item {} 
\sphinxAtStartPar
\sphinxstyleliteralstrong{\sphinxupquote{OPEX}} (\sphinxstyleliteralemphasis{\sphinxupquote{float}}) \textendash{} Operational Expenditure. Cost in euros paid each period.

\item {} 
\sphinxAtStartPar
\sphinxstyleliteralstrong{\sphinxupquote{system\_lifetime}} (\sphinxstyleliteralemphasis{\sphinxupquote{int}}) \textendash{} Number of periods defining the existence of the energy system.

\item {} 
\sphinxAtStartPar
\sphinxstyleliteralstrong{\sphinxupquote{lifetime}} (\sphinxstyleliteralemphasis{\sphinxupquote{int}}\sphinxstyleliteralemphasis{\sphinxupquote{, }}\sphinxstyleliteralemphasis{\sphinxupquote{optional}}) \textendash{} Number of periods defining the existence of the component.
If specified, overwrite the “Lifetime” property.

\item {} 
\sphinxAtStartPar
\sphinxstyleliteralstrong{\sphinxupquote{discount\_rate}} (\sphinxstyleliteralemphasis{\sphinxupquote{float}}) \textendash{} In percent (\%). Describes the importance of the economic amortization process, per period.
If specified, overwrite the “Discount rate (\%)” property.

\end{itemize}

\sphinxlineitem{Returns}
\sphinxAtStartPar
\begin{itemize}
\item {} 
\sphinxAtStartPar
\sphinxstyleemphasis{A 3\sphinxhyphen{}tuple (total\_cost, CAPEX\_share, OPEX\_share) where}

\item {} 
\sphinxAtStartPar
* CAPEX\_share is the share of total cost related to \sphinxtitleref{CAPEX}

\item {} 
\sphinxAtStartPar
* OPEX\_share is the share of total cost related to \sphinxtitleref{OPEX}

\item {} 
\sphinxAtStartPar
\sphinxstyleemphasis{* total\_cost = CAPEX\_share + OPEX\_share}

\end{itemize}


\end{description}\end{quote}
\subsubsection*{Notes}

\sphinxAtStartPar
Takes into account residual value of component in the case \sphinxtitleref{system\_lifetime} is not a multiple of \sphinxtitleref{lifetime}.
In this case, the last replacement occuring at period replacement\_period is paid in proportion of ‘CAPEX’
depending linearly on the number of periods left:
CAPEX * (system\_lifetime \sphinxhyphen{} replacement\_period) / lifetime

\end{fulllineitems}

\index{eco\_count (tamos.network.HREThermalNetwork property)@\spxentry{eco\_count}\spxextra{tamos.network.HREThermalNetwork property}}

\begin{fulllineitems}
\phantomsection\label{\detokenize{generated/tamos.network.HREThermalNetwork:tamos.network.HREThermalNetwork.eco_count}}
\pysigstartsignatures
\pysigline{\sphinxbfcode{\sphinxupquote{property\DUrole{w}{  }}}\sphinxbfcode{\sphinxupquote{eco\_count}}}
\pysigstopsignatures
\sphinxAtStartPar
Whether this instance contributes to the system “Eco” KPI
bool

\end{fulllineitems}

\index{element (tamos.network.HREThermalNetwork property)@\spxentry{element}\spxextra{tamos.network.HREThermalNetwork property}}

\begin{fulllineitems}
\phantomsection\label{\detokenize{generated/tamos.network.HREThermalNetwork:tamos.network.HREThermalNetwork.element}}
\pysigstartsignatures
\pysigline{\sphinxbfcode{\sphinxupquote{property\DUrole{w}{  }}}\sphinxbfcode{\sphinxupquote{element}}}
\pysigstopsignatures
\sphinxAtStartPar
Element exchanged between \sphinxtitleref{hubs}.

\sphinxAtStartPar
Whether \sphinxtitleref{element} is cooled down or warmed up does not define if the network is a heating or cooling network.

\end{fulllineitems}

\index{generate\_MSP() (tamos.network.HREThermalNetwork method)@\spxentry{generate\_MSP()}\spxextra{tamos.network.HREThermalNetwork method}}

\begin{fulllineitems}
\phantomsection\label{\detokenize{generated/tamos.network.HREThermalNetwork:tamos.network.HREThermalNetwork.generate_MSP}}
\pysigstartsignatures
\pysiglinewithargsret{\sphinxbfcode{\sphinxupquote{generate\_MSP}}}{\emph{\DUrole{n}{excluded\_hubs}\DUrole{o}{=}\DUrole{default_value}{None}}}{}
\pysigstopsignatures
\sphinxAtStartPar
Defines a minimum spanning tree (MSP) linking all hubs of \sphinxtitleref{hubs} that are not in \sphinxtitleref{excluded\_hubs}.

\sphinxAtStartPar
The MSP calculation does not account for bidirectionality thus all edges of the MSP graph are given
the status \sphinxtitleref{optim\_two\_ways}.
The status of the other edges (not part of the MSP) are not modified.
\begin{quote}\begin{description}
\sphinxlineitem{Parameters}
\sphinxAtStartPar
\sphinxstyleliteralstrong{\sphinxupquote{excluded\_hubs}} (\sphinxstyleliteralemphasis{\sphinxupquote{list of Hub}}\sphinxstyleliteralemphasis{\sphinxupquote{, }}\sphinxstyleliteralemphasis{\sphinxupquote{optional}}) \textendash{} All hubs from \sphinxtitleref{excluded\_hubs} must be part of \sphinxtitleref{hubs}.

\end{description}\end{quote}

\end{fulllineitems}

\index{get\_connection\_power\_bounds() (tamos.network.HREThermalNetwork method)@\spxentry{get\_connection\_power\_bounds()}\spxextra{tamos.network.HREThermalNetwork method}}

\begin{fulllineitems}
\phantomsection\label{\detokenize{generated/tamos.network.HREThermalNetwork:tamos.network.HREThermalNetwork.get_connection_power_bounds}}
\pysigstartsignatures
\pysiglinewithargsret{\sphinxbfcode{\sphinxupquote{get\_connection\_power\_bounds}}}{\emph{\DUrole{n}{hub\_1}}, \emph{\DUrole{n}{hub\_2}}}{}
\pysigstopsignatures
\sphinxAtStartPar
Returns the power limits of the flow of element from hub \sphinxtitleref{hub\_1} to hub \sphinxtitleref{hub\_2}.
\begin{quote}\begin{description}
\sphinxlineitem{Parameters}\begin{itemize}
\item {} 
\sphinxAtStartPar
\sphinxstyleliteralstrong{\sphinxupquote{hub\_1}} ({\hyperref[\detokenize{generated/tamos.Hub:tamos.Hub}]{\sphinxcrossref{\sphinxstyleliteralemphasis{\sphinxupquote{Hub}}}}}) \textendash{} Must be hubs from \sphinxtitleref{hubs}.

\item {} 
\sphinxAtStartPar
\sphinxstyleliteralstrong{\sphinxupquote{hub\_2}} ({\hyperref[\detokenize{generated/tamos.Hub:tamos.Hub}]{\sphinxcrossref{\sphinxstyleliteralemphasis{\sphinxupquote{Hub}}}}}) \textendash{} Must be hubs from \sphinxtitleref{hubs}.

\end{itemize}

\sphinxlineitem{Return type}
\sphinxAtStartPar
The lower bound and upper bound of the power flow from \sphinxtitleref{hub\_1} to \sphinxtitleref{hub\_2}, in kW.

\end{description}\end{quote}

\end{fulllineitems}

\index{get\_connection\_status() (tamos.network.HREThermalNetwork method)@\spxentry{get\_connection\_status()}\spxextra{tamos.network.HREThermalNetwork method}}

\begin{fulllineitems}
\phantomsection\label{\detokenize{generated/tamos.network.HREThermalNetwork:tamos.network.HREThermalNetwork.get_connection_status}}
\pysigstartsignatures
\pysiglinewithargsret{\sphinxbfcode{\sphinxupquote{get\_connection\_status}}}{\emph{\DUrole{n}{hub\_1}}, \emph{\DUrole{n}{hub\_2}}}{}
\pysigstopsignatures
\sphinxAtStartPar
Returns the status of the directional connection between two hubs.
\begin{quote}\begin{description}
\sphinxlineitem{Parameters}\begin{itemize}
\item {} 
\sphinxAtStartPar
\sphinxstyleliteralstrong{\sphinxupquote{hub\_1}} ({\hyperref[\detokenize{generated/tamos.Hub:tamos.Hub}]{\sphinxcrossref{\sphinxstyleliteralemphasis{\sphinxupquote{Hub}}}}}) \textendash{} Must be hubs from \sphinxtitleref{hubs}.

\item {} 
\sphinxAtStartPar
\sphinxstyleliteralstrong{\sphinxupquote{hub\_2}} ({\hyperref[\detokenize{generated/tamos.Hub:tamos.Hub}]{\sphinxcrossref{\sphinxstyleliteralemphasis{\sphinxupquote{Hub}}}}}) \textendash{} Must be hubs from \sphinxtitleref{hubs}.

\end{itemize}

\sphinxlineitem{Returns}
\sphinxAtStartPar
The status of the edge from hub\_1 to hub\_2.

\sphinxlineitem{Return type}
\sphinxAtStartPar
network status

\end{description}\end{quote}

\end{fulllineitems}

\index{get\_distance() (tamos.network.HREThermalNetwork method)@\spxentry{get\_distance()}\spxextra{tamos.network.HREThermalNetwork method}}

\begin{fulllineitems}
\phantomsection\label{\detokenize{generated/tamos.network.HREThermalNetwork:tamos.network.HREThermalNetwork.get_distance}}
\pysigstartsignatures
\pysiglinewithargsret{\sphinxbfcode{\sphinxupquote{get\_distance}}}{\emph{\DUrole{n}{hub\_1}}, \emph{\DUrole{n}{hub\_2}}}{}
\pysigstopsignatures
\sphinxAtStartPar
Returns the distance between two hubs.
The used distance function can be accessed by the \sphinxtitleref{get\_distance\_function} function
and set using \sphinxtitleref{set\_distance\_function} from tamos.network.
\begin{quote}\begin{description}
\sphinxlineitem{Parameters}\begin{itemize}
\item {} 
\sphinxAtStartPar
\sphinxstyleliteralstrong{\sphinxupquote{hub\_1}} ({\hyperref[\detokenize{generated/tamos.Hub:tamos.Hub}]{\sphinxcrossref{\sphinxstyleliteralemphasis{\sphinxupquote{Hub}}}}}) \textendash{} Must be from \sphinxtitleref{hubs}.

\item {} 
\sphinxAtStartPar
\sphinxstyleliteralstrong{\sphinxupquote{hub\_2}} ({\hyperref[\detokenize{generated/tamos.Hub:tamos.Hub}]{\sphinxcrossref{\sphinxstyleliteralemphasis{\sphinxupquote{Hub}}}}}) \textendash{} Must be from \sphinxtitleref{hubs}.

\end{itemize}

\sphinxlineitem{Return type}
\sphinxAtStartPar
The distance from \sphinxtitleref{hub\_1} to \sphinxtitleref{hub\_2} which is the same than from \sphinxtitleref{hub\_2} to \sphinxtitleref{hub\_1}

\end{description}\end{quote}

\end{fulllineitems}

\index{hubs (tamos.network.HREThermalNetwork property)@\spxentry{hubs}\spxextra{tamos.network.HREThermalNetwork property}}

\begin{fulllineitems}
\phantomsection\label{\detokenize{generated/tamos.network.HREThermalNetwork:tamos.network.HREThermalNetwork.hubs}}
\pysigstartsignatures
\pysigline{\sphinxbfcode{\sphinxupquote{property\DUrole{w}{  }}}\sphinxbfcode{\sphinxupquote{hubs}}}
\pysigstopsignatures
\sphinxAtStartPar
The hubs involved in the network definition.

\sphinxAtStartPar
Some of these hubs might be completely disconnected from the network
(i.e. network.might\_connect(hub) is False) yet they are still considered and associated with MILP decision variables.

\end{fulllineitems}

\index{might\_connect() (tamos.network.HREThermalNetwork method)@\spxentry{might\_connect()}\spxextra{tamos.network.HREThermalNetwork method}}

\begin{fulllineitems}
\phantomsection\label{\detokenize{generated/tamos.network.HREThermalNetwork:tamos.network.HREThermalNetwork.might_connect}}
\pysigstartsignatures
\pysiglinewithargsret{\sphinxbfcode{\sphinxupquote{might\_connect}}}{\emph{\DUrole{n}{hub}}}{}
\pysigstopsignatures
\sphinxAtStartPar
Checks whether a hub is able to connect to the network.
\begin{quote}\begin{description}
\sphinxlineitem{Parameters}
\sphinxAtStartPar
\sphinxstyleliteralstrong{\sphinxupquote{hub}} ({\hyperref[\detokenize{generated/tamos.Hub:tamos.Hub}]{\sphinxcrossref{\sphinxstyleliteralemphasis{\sphinxupquote{Hub}}}}}) \textendash{} 

\sphinxlineitem{Returns}
\sphinxAtStartPar
\begin{itemize}
\item {} 
\sphinxAtStartPar
\sphinxstyleemphasis{True if the following two conditions are met}

\item {} 
\sphinxAtStartPar
* Hub \sphinxtitleref{hub} is one of \sphinxtitleref{hubs}.

\item {} 
\sphinxAtStartPar
* There exists at least one hub \sphinxtitleref{hub\_2} in \sphinxtitleref{hubs} such that connection status from \sphinxtitleref{hub} to \sphinxtitleref{hub\_2} or \sphinxtitleref{hub\_2} to \sphinxtitleref{hub} \textendash{} is different from ‘No connection’.

\end{itemize}


\end{description}\end{quote}

\end{fulllineitems}

\index{name (tamos.network.HREThermalNetwork property)@\spxentry{name}\spxextra{tamos.network.HREThermalNetwork property}}

\begin{fulllineitems}
\phantomsection\label{\detokenize{generated/tamos.network.HREThermalNetwork:tamos.network.HREThermalNetwork.name}}
\pysigstartsignatures
\pysigline{\sphinxbfcode{\sphinxupquote{property\DUrole{w}{  }}}\sphinxbfcode{\sphinxupquote{name}}}
\pysigstopsignatures
\sphinxAtStartPar
str.
This name is used in MILP model description.
names must not exceed 255 characters,
all of which must be alphanumeric (a\sphinxhyphen{}z, A\sphinxhyphen{}Z, 0\sphinxhyphen{}9) or one of these symbols:
! ” \# \$ \% \& , . ; ? @ \_ ‘ ’ \{ \} \textasciitilde{}.
\begin{quote}\begin{description}
\sphinxlineitem{Type}
\sphinxAtStartPar
Name of the instance

\end{description}\end{quote}

\end{fulllineitems}

\index{plot() (tamos.network.HREThermalNetwork method)@\spxentry{plot()}\spxextra{tamos.network.HREThermalNetwork method}}

\begin{fulllineitems}
\phantomsection\label{\detokenize{generated/tamos.network.HREThermalNetwork:tamos.network.HREThermalNetwork.plot}}
\pysigstartsignatures
\pysiglinewithargsret{\sphinxbfcode{\sphinxupquote{plot}}}{}{}
\pysigstopsignatures
\sphinxAtStartPar
Plots a representation of the network hubs and connections in the (x, y) space.

\sphinxAtStartPar
A call to this method gives a visual insight of how the network is parametrized, BEFORE optimization.
The optimization implicitely tranforms every edge status to either ‘No connection’ or ‘Connection’.
\subsubsection*{Notes}

\sphinxAtStartPar
The line linking two hubs is straight for commodity and does not represent
the real distance function used in the MILP model.

\end{fulllineitems}

\index{production\_hub (tamos.network.HREThermalNetwork property)@\spxentry{production\_hub}\spxextra{tamos.network.HREThermalNetwork property}}

\begin{fulllineitems}
\phantomsection\label{\detokenize{generated/tamos.network.HREThermalNetwork:tamos.network.HREThermalNetwork.production_hub}}
\pysigstartsignatures
\pysigline{\sphinxbfcode{\sphinxupquote{property\DUrole{w}{  }}}\sphinxbfcode{\sphinxupquote{production\_hub}}}
\pysigstopsignatures
\sphinxAtStartPar
The hub that bears:
\begin{itemize}
\item {} 
\sphinxAtStartPar
all the distribution losses of the network.

\item {} 
\sphinxAtStartPar
the costs associated with the “Variable OPEX (EUR/MWh)” property.

\end{itemize}

\sphinxAtStartPar
Must be one of \sphinxtitleref{hubs} and must be able to exchange \sphinxtitleref{element} with the network.

\end{fulllineitems}

\index{production\_mode (tamos.network.HREThermalNetwork property)@\spxentry{production\_mode}\spxextra{tamos.network.HREThermalNetwork property}}

\begin{fulllineitems}
\phantomsection\label{\detokenize{generated/tamos.network.HREThermalNetwork:tamos.network.HREThermalNetwork.production_mode}}
\pysigstartsignatures
\pysigline{\sphinxbfcode{\sphinxupquote{property\DUrole{w}{  }}}\sphinxbfcode{\sphinxupquote{production\_mode}}}
\pysigstopsignatures
\sphinxAtStartPar
Related to \sphinxtitleref{production\_hub}.

\sphinxAtStartPar
\{“heat\&cold”, “heat”, “cold”\}, optional, default “heat\&cold”
Used to speed up the KPI declaration regarding the \sphinxtitleref{Variable OPEX (EUR/MWh)} property.
Provides insight on the sign of the flow going from \sphinxtitleref{production\_hub} to the network.
Whether \sphinxtitleref{element} is cooled down or warmed up is independent from \sphinxtitleref{production\_mode}.
\begin{itemize}
\item {} 
\sphinxAtStartPar
“heat” (“cold”): only heat (cold) is send to the network by \sphinxtitleref{production\_hub}.
The flow is always of the same sign, thus the KPI constraint is like:
energy = sum(a\_given\_sign * power(t) * dt)

\item {} 
\sphinxAtStartPar
“heat” (“cold”): only heat (cold) is send to the network by \sphinxtitleref{production\_hub}.
The sign of the flow may change during the operation period, thus the KPI constraint is like:
energy = sum(abs(power(t)) * dt)

\end{itemize}

\sphinxAtStartPar
Specifying “heat” or “cold” will speed up the KPI declaration but describes a particular state of energy flows.
Specifying “heat\&cold” makes the KPI declaration long (but has no impact on resolution) but works in all cases.

\end{fulllineitems}

\index{scale\_factor (tamos.network.HREThermalNetwork property)@\spxentry{scale\_factor}\spxextra{tamos.network.HREThermalNetwork property}}

\begin{fulllineitems}
\phantomsection\label{\detokenize{generated/tamos.network.HREThermalNetwork:tamos.network.HREThermalNetwork.scale_factor}}
\pysigstartsignatures
\pysigline{\sphinxbfcode{\sphinxupquote{property\DUrole{w}{  }}}\sphinxbfcode{\sphinxupquote{scale\_factor}}}
\pysigstopsignatures
\sphinxAtStartPar
Multiplies the two coordinates of each hub in \sphinxtitleref{hubs\_locations}.

\sphinxAtStartPar
A scale factor \textgreater{}1 tends to increase the distance between two hubs.
float

\end{fulllineitems}

\index{set\_connection\_power\_bounds() (tamos.network.HREThermalNetwork method)@\spxentry{set\_connection\_power\_bounds()}\spxextra{tamos.network.HREThermalNetwork method}}

\begin{fulllineitems}
\phantomsection\label{\detokenize{generated/tamos.network.HREThermalNetwork:tamos.network.HREThermalNetwork.set_connection_power_bounds}}
\pysigstartsignatures
\pysiglinewithargsret{\sphinxbfcode{\sphinxupquote{set\_connection\_power\_bounds}}}{\emph{\DUrole{n}{hub\_1}}, \emph{\DUrole{n}{hub\_2}}, \emph{\DUrole{n}{power\_lb}\DUrole{o}{=}\DUrole{default_value}{None}}, \emph{\DUrole{n}{power\_ub}\DUrole{o}{=}\DUrole{default_value}{None}}}{}
\pysigstopsignatures
\sphinxAtStartPar
Sets power limits on the flow of element from hub \sphinxtitleref{hub\_1} to hub \sphinxtitleref{hub\_2}.
These limits apply only if the connection from \sphinxtitleref{hub\_1} to \sphinxtitleref{hub\_2} is used.
\begin{quote}\begin{description}
\sphinxlineitem{Parameters}\begin{itemize}
\item {} 
\sphinxAtStartPar
\sphinxstyleliteralstrong{\sphinxupquote{hub\_1}} ({\hyperref[\detokenize{generated/tamos.Hub:tamos.Hub}]{\sphinxcrossref{\sphinxstyleliteralemphasis{\sphinxupquote{Hub}}}}}) \textendash{} Must be hubs from \sphinxtitleref{hubs}.

\item {} 
\sphinxAtStartPar
\sphinxstyleliteralstrong{\sphinxupquote{hub\_2}} ({\hyperref[\detokenize{generated/tamos.Hub:tamos.Hub}]{\sphinxcrossref{\sphinxstyleliteralemphasis{\sphinxupquote{Hub}}}}}) \textendash{} Must be hubs from \sphinxtitleref{hubs}.

\item {} 
\sphinxAtStartPar
\sphinxstyleliteralstrong{\sphinxupquote{power\_lb}} (\sphinxstyleliteralemphasis{\sphinxupquote{int}}\sphinxstyleliteralemphasis{\sphinxupquote{, }}\sphinxstyleliteralemphasis{\sphinxupquote{float}}\sphinxstyleliteralemphasis{\sphinxupquote{ or }}\sphinxstyleliteralemphasis{\sphinxupquote{numpy.ndarray}}) \textendash{} power\_lb \textgreater{}= 0, power\_ub \textgreater{}= 0
In kW.
The lower bound (upper bound) of the flow of \sphinxtitleref{element} from \sphinxtitleref{hub\_1} to \sphinxtitleref{hub\_2}.

\item {} 
\sphinxAtStartPar
\sphinxstyleliteralstrong{\sphinxupquote{power\_ub}} (\sphinxstyleliteralemphasis{\sphinxupquote{int}}\sphinxstyleliteralemphasis{\sphinxupquote{, }}\sphinxstyleliteralemphasis{\sphinxupquote{float}}\sphinxstyleliteralemphasis{\sphinxupquote{ or }}\sphinxstyleliteralemphasis{\sphinxupquote{numpy.ndarray}}) \textendash{} power\_lb \textgreater{}= 0, power\_ub \textgreater{}= 0
In kW.
The lower bound (upper bound) of the flow of \sphinxtitleref{element} from \sphinxtitleref{hub\_1} to \sphinxtitleref{hub\_2}.

\end{itemize}

\end{description}\end{quote}
\subsubsection*{Examples}

\begin{sphinxVerbatim}[commandchars=\\\{\}]
\PYG{g+gp}{\PYGZgt{}\PYGZgt{}\PYGZgt{} }\PYG{n}{network}\PYG{o}{.}\PYG{n}{set\PYGZus{}connection\PYGZus{}power\PYGZus{}bounds}\PYG{p}{(}\PYG{n}{hub\PYGZus{}1}\PYG{p}{,} \PYG{n}{hub\PYGZus{}2}\PYG{p}{,} \PYG{n}{power\PYGZus{}lb}\PYG{o}{=}\PYG{l+m+mi}{400}\PYG{p}{,} \PYG{n}{power\PYGZus{}ub}\PYG{o}{=}\PYG{l+m+mi}{2000}\PYG{p}{)}
\end{sphinxVerbatim}

\sphinxAtStartPar
If the connection \sphinxtitleref{hub\_1} to \sphinxtitleref{hub\_2} exists, the power that flows from \sphinxtitleref{hub\_1} to \sphinxtitleref{hub\_2} must always be in the range
{[}0.4, 2{]} MW.

\begin{sphinxVerbatim}[commandchars=\\\{\}]
\PYG{g+gp}{\PYGZgt{}\PYGZgt{}\PYGZgt{} }\PYG{n}{network}\PYG{o}{.}\PYG{n}{set\PYGZus{}connection\PYGZus{}power\PYGZus{}bounds}\PYG{p}{(}\PYG{n}{hub\PYGZus{}1}\PYG{p}{,} \PYG{n}{hub\PYGZus{}2}\PYG{p}{,} \PYG{n}{power\PYGZus{}lb}\PYG{o}{=}\PYG{l+m+mi}{1000}\PYG{p}{)}
\PYG{g+gp}{\PYGZgt{}\PYGZgt{}\PYGZgt{} }\PYG{n}{network}\PYG{o}{.}\PYG{n}{set\PYGZus{}connection\PYGZus{}power\PYGZus{}bounds}\PYG{p}{(}\PYG{n}{hub\PYGZus{}2}\PYG{p}{,} \PYG{n}{hub\PYGZus{}1}\PYG{p}{,} \PYG{n}{power\PYGZus{}ub}\PYG{o}{=}\PYG{o}{\PYGZhy{}}\PYG{l+m+mi}{1000}\PYG{p}{)}
\end{sphinxVerbatim}

\sphinxAtStartPar
Both calls perform the same operation, but second call is forbidden to make things clearer.

\begin{sphinxVerbatim}[commandchars=\\\{\}]
\PYG{g+gp}{\PYGZgt{}\PYGZgt{}\PYGZgt{} }\PYG{n}{network}\PYG{o}{.}\PYG{n}{set\PYGZus{}connection\PYGZus{}power\PYGZus{}bounds}\PYG{p}{(}\PYG{n}{hub\PYGZus{}1}\PYG{p}{,} \PYG{n}{hub\PYGZus{}2}\PYG{p}{,} \PYG{n}{power\PYGZus{}lb}\PYG{o}{=}\PYG{l+m+mi}{1000}\PYG{p}{)}
\PYG{g+gp}{\PYGZgt{}\PYGZgt{}\PYGZgt{} }\PYG{n}{network}\PYG{o}{.}\PYG{n}{set\PYGZus{}connection\PYGZus{}power\PYGZus{}bounds}\PYG{p}{(}\PYG{n}{hub\PYGZus{}2}\PYG{p}{,} \PYG{n}{hub\PYGZus{}1}\PYG{p}{,} \PYG{n}{power\PYGZus{}lb}\PYG{o}{=}\PYG{l+m+mi}{2000}\PYG{p}{)}
\end{sphinxVerbatim}

\sphinxAtStartPar
The constraints implied by these calls make impossible the existence of both connections
(from \sphinxtitleref{hub\_1} to \sphinxtitleref{hub\_2} and \sphinxtitleref{hub\_2} to \sphinxtitleref{hub\_1}): if 1000 kW of \sphinxtitleref{element} flows from \sphinxtitleref{hub\_1} to \sphinxtitleref{hub\_2}
then \sphinxhyphen{}1000 k\textgreater{} flows from \sphinxtitleref{hub\_2} to \sphinxtitleref{hub\_1} (and vice versa).

\end{fulllineitems}

\index{set\_connection\_status() (tamos.network.HREThermalNetwork method)@\spxentry{set\_connection\_status()}\spxextra{tamos.network.HREThermalNetwork method}}

\begin{fulllineitems}
\phantomsection\label{\detokenize{generated/tamos.network.HREThermalNetwork:tamos.network.HREThermalNetwork.set_connection_status}}
\pysigstartsignatures
\pysiglinewithargsret{\sphinxbfcode{\sphinxupquote{set\_connection\_status}}}{\emph{\DUrole{n}{hub\_1}}, \emph{\DUrole{n}{hub\_2}}, \emph{\DUrole{n}{status}}}{}
\pysigstopsignatures
\sphinxAtStartPar
Defines the connection status of the edge going from \sphinxtitleref{hub\_1} to \sphinxtitleref{hub\_2}.
\begin{quote}\begin{description}
\sphinxlineitem{Parameters}\begin{itemize}
\item {} 
\sphinxAtStartPar
\sphinxstyleliteralstrong{\sphinxupquote{hub\_1}} ({\hyperref[\detokenize{generated/tamos.Hub:tamos.Hub}]{\sphinxcrossref{\sphinxstyleliteralemphasis{\sphinxupquote{Hub}}}}}) \textendash{} Must be hubs from \sphinxtitleref{hubs}.

\item {} 
\sphinxAtStartPar
\sphinxstyleliteralstrong{\sphinxupquote{hub\_2}} ({\hyperref[\detokenize{generated/tamos.Hub:tamos.Hub}]{\sphinxcrossref{\sphinxstyleliteralemphasis{\sphinxupquote{Hub}}}}}) \textendash{} Must be hubs from \sphinxtitleref{hubs}.

\item {} 
\sphinxAtStartPar
\sphinxstyleliteralstrong{\sphinxupquote{status}} (\sphinxtitleref{status} may take 5 values that are attributes of this instance:) \textendash{} \begin{itemize}
\item {} 
\sphinxAtStartPar
no\_connection: flow of \sphinxtitleref{element} is forbidden

\item {} 
\sphinxAtStartPar
connection: flow of \sphinxtitleref{element} is possible

\item {} 
\sphinxAtStartPar
optim\_one\_way\_min: flow of \sphinxtitleref{element} must exist in at least one direction
(from hub\_1 to hub\_2, from hub\_2 to hub\_1 or in both directions)
This status also defines the opposite status, e.g. defining ‘hub\_1 to hub\_2’ status defines the one of ‘hub\_2 to hub\_1’.

\item {} 
\sphinxAtStartPar
optim\_one\_way\_max: flow of \sphinxtitleref{element} may exist in at most one direction
(from hub\_1 to hub\_2 or from hub\_2 to hub\_1)
This status also defines the opposite status, e.g. defining ‘hub\_1 to hub\_2’ status defines the one of ‘hub\_2 to hub\_1’.

\item {} 
\sphinxAtStartPar
optim\_two\_ways: flow of \sphinxtitleref{element} may exist in both directions
(from hub\_1 to hub\_2 and from hub\_2 to hub\_1)
This status also defines the opposite status, e.g. defining ‘hub\_1 to hub\_2’ status defines the one of ‘hub\_2 to hub\_1’.

\end{itemize}


\end{itemize}

\end{description}\end{quote}
\subsubsection*{Notes}

\sphinxAtStartPar
Given two hubs \sphinxtitleref{hub\_1} and \sphinxtitleref{hub\_2}, the order of calls to \sphinxtitleref{set\_connection\_status} matters.

\begin{sphinxVerbatim}[commandchars=\\\{\}]
\PYG{g+gp}{\PYGZgt{}\PYGZgt{}\PYGZgt{} }\PYG{n}{network}\PYG{o}{.}\PYG{n}{set\PYGZus{}connection\PYGZus{}status}\PYG{p}{(}\PYG{n}{hub\PYGZus{}1}\PYG{p}{,} \PYG{n}{hub\PYGZus{}2}\PYG{p}{,} \PYG{n}{network}\PYG{o}{.}\PYG{n}{no\PYGZus{}direction}\PYG{p}{)}
\PYG{g+gp}{\PYGZgt{}\PYGZgt{}\PYGZgt{} }\PYG{n}{network}\PYG{o}{.}\PYG{n}{set\PYGZus{}connection\PYGZus{}status}\PYG{p}{(}\PYG{n}{hub\PYGZus{}2}\PYG{p}{,} \PYG{n}{hub\PYGZus{}1}\PYG{p}{,} \PYG{n}{network}\PYG{o}{.}\PYG{n}{optim\PYGZus{}one\PYGZus{}way\PYGZus{}max}\PYG{p}{)}
\PYG{g+gp}{\PYGZgt{}\PYGZgt{}\PYGZgt{} }\PYG{n}{network}\PYG{o}{.}\PYG{n}{get\PYGZus{}connection\PYGZus{}status}\PYG{p}{(}\PYG{n}{hub\PYGZus{}2}\PYG{p}{,} \PYG{n}{hub\PYGZus{}1}\PYG{p}{)} \PYG{o}{==} \PYG{n}{network}\PYG{o}{.}\PYG{n}{get\PYGZus{}connection\PYGZus{}status}\PYG{p}{(}\PYG{n}{hub\PYGZus{}1}\PYG{p}{,} \PYG{n}{hub\PYGZus{}2}\PYG{p}{)}
\PYG{g+go}{    True}
\end{sphinxVerbatim}

\sphinxAtStartPar
The first line has no effect because the second one defines again the connection from \sphinxtitleref{hub\_1} to \sphinxtitleref{hub\_2}.

\end{fulllineitems}

\index{set\_node\_status() (tamos.network.HREThermalNetwork method)@\spxentry{set\_node\_status()}\spxextra{tamos.network.HREThermalNetwork method}}

\begin{fulllineitems}
\phantomsection\label{\detokenize{generated/tamos.network.HREThermalNetwork:tamos.network.HREThermalNetwork.set_node_status}}
\pysigstartsignatures
\pysiglinewithargsret{\sphinxbfcode{\sphinxupquote{set\_node\_status}}}{\emph{\DUrole{n}{hub}}, \emph{\DUrole{n}{status}}}{}
\pysigstopsignatures
\sphinxAtStartPar
Defines the connection status of every incoming and outcoming edge of a hub.
\begin{quote}\begin{description}
\sphinxlineitem{Parameters}\begin{itemize}
\item {} 
\sphinxAtStartPar
\sphinxstyleliteralstrong{\sphinxupquote{hub}} ({\hyperref[\detokenize{generated/tamos.Hub:tamos.Hub}]{\sphinxcrossref{\sphinxstyleliteralemphasis{\sphinxupquote{Hub}}}}}) \textendash{} Must be from \sphinxtitleref{hubs}.

\item {} 
\sphinxAtStartPar
\sphinxstyleliteralstrong{\sphinxupquote{status}} (\sphinxtitleref{status} may take 5 values that are attributes of this instance:) \textendash{} \begin{itemize}
\item {} 
\sphinxAtStartPar
no\_connection: flow of \sphinxtitleref{element} is forbidden

\item {} 
\sphinxAtStartPar
connection: flow of \sphinxtitleref{element} is possible

\item {} 
\sphinxAtStartPar
optim\_one\_way\_min: flow of \sphinxtitleref{element} must exist in at least one direction
(from hub\_1 to hub\_2, from hub\_2 to hub\_1 or in both directions)
This status also defines the opposite status, e.g. defining ‘hub\_1 to hub\_2’ status defines the one of ‘hub\_2 to hub\_1’.

\item {} 
\sphinxAtStartPar
optim\_one\_way\_max: flow of \sphinxtitleref{element} may exist in at most one direction
(from hub\_1 to hub\_2 or from hub\_2 to hub\_1)
This status also defines the opposite status, e.g. defining ‘hub\_1 to hub\_2’ status defines the one of ‘hub\_2 to hub\_1’.

\item {} 
\sphinxAtStartPar
optim\_two\_ways: flow of \sphinxtitleref{element} may exist in both directions
(from hub\_1 to hub\_2 and from hub\_2 to hub\_1)
This status also defines the opposite status, e.g. defining ‘hub\_1 to hub\_2’ status defines the one of ‘hub\_2 to hub\_1’.

\end{itemize}


\end{itemize}

\end{description}\end{quote}

\end{fulllineitems}

\index{set\_soil\_properties() (tamos.network.HREThermalNetwork method)@\spxentry{set\_soil\_properties()}\spxextra{tamos.network.HREThermalNetwork method}}

\begin{fulllineitems}
\phantomsection\label{\detokenize{generated/tamos.network.HREThermalNetwork:tamos.network.HREThermalNetwork.set_soil_properties}}
\pysigstartsignatures
\pysiglinewithargsret{\sphinxbfcode{\sphinxupquote{set\_soil\_properties}}}{\emph{\DUrole{n}{soil\_temperature}}, \emph{\DUrole{n}{U}}, \emph{\DUrole{n}{losses\_direction: Both\textquotesingle{}}}, \emph{\DUrole{n}{\textquotesingle{}Heat losses\textquotesingle{}}}, \emph{\DUrole{n}{\textquotesingle{}Heat gains = \textquotesingle{}Both\textquotesingle{}}}}{}
\pysigstopsignatures
\sphinxAtStartPar
Sets the physical properties that define thermal losses.
\begin{quote}\begin{description}
\sphinxlineitem{Parameters}\begin{itemize}
\item {} 
\sphinxAtStartPar
\sphinxstyleliteralstrong{\sphinxupquote{soil\_temperature}} (\sphinxstyleliteralemphasis{\sphinxupquote{int}}\sphinxstyleliteralemphasis{\sphinxupquote{, }}\sphinxstyleliteralemphasis{\sphinxupquote{float}}\sphinxstyleliteralemphasis{\sphinxupquote{ or }}\sphinxstyleliteralemphasis{\sphinxupquote{numpy.ndarray}}) \textendash{} In Kelvins (K).
The temperature of the soil at the buried depth of the network pipes.

\item {} 
\sphinxAtStartPar
\sphinxstyleliteralstrong{\sphinxupquote{U}} (\sphinxstyleliteralemphasis{\sphinxupquote{int}}\sphinxstyleliteralemphasis{\sphinxupquote{, }}\sphinxstyleliteralemphasis{\sphinxupquote{float}}\sphinxstyleliteralemphasis{\sphinxupquote{ or }}\sphinxstyleliteralemphasis{\sphinxupquote{numpy.ndarray}}) \textendash{} In W/(m.K).
Specific heat loss per routed meter. Includes both supply  and return pipes.

\item {} 
\sphinxAtStartPar
\sphinxstyleliteralstrong{\sphinxupquote{losses\_direction}} (\sphinxstyleliteralemphasis{\sphinxupquote{\{\textquotesingle{}Both\textquotesingle{}}}\sphinxstyleliteralemphasis{\sphinxupquote{, }}\sphinxstyleliteralemphasis{\sphinxupquote{\textquotesingle{}Heat losses\textquotesingle{}}}\sphinxstyleliteralemphasis{\sphinxupquote{, }}\sphinxstyleliteralemphasis{\sphinxupquote{\textquotesingle{}Heat gains\textquotesingle{}\}}}\sphinxstyleliteralemphasis{\sphinxupquote{, }}\sphinxstyleliteralemphasis{\sphinxupquote{optional}}\sphinxstyleliteralemphasis{\sphinxupquote{, }}\sphinxstyleliteralemphasis{\sphinxupquote{default \textquotesingle{}Both\textquotesingle{}}}) \textendash{} 
\sphinxAtStartPar
Direction of the heat exchanges between the network infrastructure and the soil.
Specifying ‘Heat losses’ (‘Heat gains’) allows to prevent from happening the case where
cold (heat) must be produced in \sphinxtitleref{production\_hub} to compensate thermal gains (losses) in
a district heating (district cooling) network, when thermal demand is lower than soil thermal exchanges.
\begin{itemize}
\item {} 
\sphinxAtStartPar
’Heat losses’: only thermal energy exchanges from the network to the soil are taken into account,
others are set to 0.

\item {} 
\sphinxAtStartPar
’Heat gains’: only thermal energy exchanges from the soil to the network are taken into account,
others are set to 0.

\item {} 
\sphinxAtStartPar
’Both’: all thermal energy exchanges are taken into account.

\end{itemize}


\end{itemize}

\end{description}\end{quote}
\subsubsection*{Notes}
\begin{enumerate}
\sphinxsetlistlabels{\arabic}{enumi}{enumii}{}{.}%
\item {} 
\sphinxAtStartPar
The thermal energy exchanged between the network infrastructure and the soil is like:
thermal\_exchanges(t) = sign * U * network\_length * ((T\_warm(t)+T\_cold(t)) / 2 \sphinxhyphen{} T\_soil(t))
With:
\begin{itemize}
\item {} 
\sphinxAtStartPar
network\_length: the length of all the edges in the network.
If connections are bidirectional, length is accounted for only once.

\item {} 
\sphinxAtStartPar
T\_warm(t): temperature of the warm vector of \sphinxtitleref{element}

\item {} 
\sphinxAtStartPar
T\_cold(t): temperature of the cold vector of \sphinxtitleref{element}

\item {} 
\sphinxAtStartPar
T\_soil(t): temperature of the soil

\end{itemize}

\item {} 
\sphinxAtStartPar
By default, set\_soil\_properties is called with \sphinxtitleref{soil\_temperature} =273+10, \sphinxtitleref{U} =0.7, \sphinxtitleref{losses\_direction} =’Both’.

\item {} 
\sphinxAtStartPar
This method does not assume pipes are laid down underground: setting a \sphinxtitleref{U} value according to the \sphinxtitleref{soil\_temperature}
value is enough to describe any surrounding environment of the pipes.

\end{enumerate}

\end{fulllineitems}

\index{set\_status() (tamos.network.HREThermalNetwork method)@\spxentry{set\_status()}\spxextra{tamos.network.HREThermalNetwork method}}

\begin{fulllineitems}
\phantomsection\label{\detokenize{generated/tamos.network.HREThermalNetwork:tamos.network.HREThermalNetwork.set_status}}
\pysigstartsignatures
\pysiglinewithargsret{\sphinxbfcode{\sphinxupquote{set\_status}}}{\emph{\DUrole{n}{status}}, \emph{\DUrole{n}{excluded\_hubs}\DUrole{o}{=}\DUrole{default_value}{None}}}{}
\pysigstopsignatures
\sphinxAtStartPar
Define the connection status of all edges except the ones involving a hub from \sphinxtitleref{excluded\_hubs}.
\begin{quote}\begin{description}
\sphinxlineitem{Parameters}\begin{itemize}
\item {} 
\sphinxAtStartPar
\sphinxstyleliteralstrong{\sphinxupquote{status}} (\sphinxtitleref{status} may take 5 values that are attributes of this instance:) \textendash{} \begin{itemize}
\item {} 
\sphinxAtStartPar
no\_connection: flow of \sphinxtitleref{element} is forbidden

\item {} 
\sphinxAtStartPar
connection: flow of \sphinxtitleref{element} is possible

\item {} 
\sphinxAtStartPar
optim\_one\_way\_min: flow of \sphinxtitleref{element} must exist in at least one direction
(from hub\_1 to hub\_2, from hub\_2 to hub\_1 or in both directions)
This status also defines the opposite status, e.g. defining ‘hub\_1 to hub\_2’ status defines the one of ‘hub\_2 to hub\_1’.

\item {} 
\sphinxAtStartPar
optim\_one\_way\_max: flow of \sphinxtitleref{element} may exist in at most one direction
(from hub\_1 to hub\_2 or from hub\_2 to hub\_1)
This status also defines the opposite status, e.g. defining ‘hub\_1 to hub\_2’ status defines the one of ‘hub\_2 to hub\_1’.

\item {} 
\sphinxAtStartPar
optim\_two\_ways: flow of \sphinxtitleref{element} may exist in both directions
(from hub\_1 to hub\_2 and from hub\_2 to hub\_1)
This status also defines the opposite status, e.g. defining ‘hub\_1 to hub\_2’ status defines the one of ‘hub\_2 to hub\_1’.

\end{itemize}


\item {} 
\sphinxAtStartPar
\sphinxstyleliteralstrong{\sphinxupquote{excluded\_hubs}} (\sphinxstyleliteralemphasis{\sphinxupquote{list of Hub}}\sphinxstyleliteralemphasis{\sphinxupquote{, }}\sphinxstyleliteralemphasis{\sphinxupquote{optional}}) \textendash{} All hubs from \sphinxtitleref{excluded\_hubs} must be part of \sphinxtitleref{hubs}.

\end{itemize}

\end{description}\end{quote}

\end{fulllineitems}

\index{used\_elements (tamos.network.HREThermalNetwork property)@\spxentry{used\_elements}\spxextra{tamos.network.HREThermalNetwork property}}

\begin{fulllineitems}
\phantomsection\label{\detokenize{generated/tamos.network.HREThermalNetwork:tamos.network.HREThermalNetwork.used_elements}}
\pysigstartsignatures
\pysigline{\sphinxbfcode{\sphinxupquote{property\DUrole{w}{  }}}\sphinxbfcode{\sphinxupquote{used\_elements}}}
\pysigstopsignatures
\sphinxAtStartPar
Elements used by the component.

\end{fulllineitems}


\end{fulllineitems}



\section{Utilities}
\label{\detokenize{network_components:utilities}}

\begin{savenotes}\sphinxattablestart
\centering
\begin{tabulary}{\linewidth}[t]{\X{1}{2}\X{1}{2}}
\hline

\sphinxAtStartPar
{\hyperref[\detokenize{generated/tamos.network.get_distance_function:tamos.network.get_distance_function}]{\sphinxcrossref{\sphinxcode{\sphinxupquote{tamos.network.get\_distance\_function}}}}}()
&
\sphinxAtStartPar
Returns the function that gives the distance in km between two hubs.
\\
\hline
\sphinxAtStartPar
{\hyperref[\detokenize{generated/tamos.network.set_distance_function:tamos.network.set_distance_function}]{\sphinxcrossref{\sphinxcode{\sphinxupquote{tamos.network.set\_distance\_function}}}}}(...)
&
\sphinxAtStartPar
Defines the function that gives the distance in km between two hubs.
\\
\hline
\end{tabulary}
\par
\sphinxattableend\end{savenotes}

\sphinxstepscope


\subsection{tamos.network.get\_distance\_function}
\label{\detokenize{generated/tamos.network.get_distance_function:tamos-network-get-distance-function}}\label{\detokenize{generated/tamos.network.get_distance_function::doc}}\index{get\_distance\_function() (in module tamos.network)@\spxentry{get\_distance\_function()}\spxextra{in module tamos.network}}

\begin{fulllineitems}
\phantomsection\label{\detokenize{generated/tamos.network.get_distance_function:tamos.network.get_distance_function}}
\pysigstartsignatures
\pysiglinewithargsret{\sphinxcode{\sphinxupquote{tamos.network.}}\sphinxbfcode{\sphinxupquote{get\_distance\_function}}}{}{}
\pysigstopsignatures
\sphinxAtStartPar
Returns the function that gives the distance in km between two hubs.

\end{fulllineitems}


\sphinxstepscope


\subsection{tamos.network.set\_distance\_function}
\label{\detokenize{generated/tamos.network.set_distance_function:tamos-network-set-distance-function}}\label{\detokenize{generated/tamos.network.set_distance_function::doc}}\index{set\_distance\_function() (in module tamos.network)@\spxentry{set\_distance\_function()}\spxextra{in module tamos.network}}

\begin{fulllineitems}
\phantomsection\label{\detokenize{generated/tamos.network.set_distance_function:tamos.network.set_distance_function}}
\pysigstartsignatures
\pysiglinewithargsret{\sphinxcode{\sphinxupquote{tamos.network.}}\sphinxbfcode{\sphinxupquote{set\_distance\_function}}}{\emph{\DUrole{n}{distance\_function}}}{}
\pysigstopsignatures
\sphinxAtStartPar
Defines the function that gives the distance in km between two hubs.
\begin{quote}\begin{description}
\sphinxlineitem{Parameters}\begin{itemize}
\item {} 
\sphinxAtStartPar
\sphinxstyleliteralstrong{\sphinxupquote{distance\_function}} (\sphinxstyleliteralemphasis{\sphinxupquote{callable}}\sphinxstyleliteralemphasis{\sphinxupquote{, }}\sphinxstyleliteralemphasis{\sphinxupquote{f}}\sphinxstyleliteralemphasis{\sphinxupquote{(}}\sphinxstyleliteralemphasis{\sphinxupquote{u}}\sphinxstyleliteralemphasis{\sphinxupquote{, }}\sphinxstyleliteralemphasis{\sphinxupquote{v}}\sphinxstyleliteralemphasis{\sphinxupquote{)}}\sphinxstyleliteralemphasis{\sphinxupquote{, }}\sphinxstyleliteralemphasis{\sphinxupquote{that meets the following definition:}}) \textendash{} 

\item {} 
\sphinxAtStartPar
\sphinxstyleliteralstrong{\sphinxupquote{{[}}}\sphinxstyleliteralstrong{\sphinxupquote{x}} (\sphinxstyleliteralemphasis{\sphinxupquote{* u is a list}}) \textendash{} 

\item {} 
\sphinxAtStartPar
\sphinxstyleliteralstrong{\sphinxupquote{v}}\sphinxstyleliteralstrong{\sphinxupquote{)}}\sphinxstyleliteralstrong{\sphinxupquote{.}} (\sphinxstyleliteralemphasis{\sphinxupquote{y}}\sphinxstyleliteralemphasis{\sphinxupquote{{]} }}\sphinxstyleliteralemphasis{\sphinxupquote{where x and y are space coordinates of point u}}\sphinxstyleliteralemphasis{\sphinxupquote{ (}}\sphinxstyleliteralemphasis{\sphinxupquote{same for}}) \textendash{} 

\item {} 
\sphinxAtStartPar
\sphinxstyleliteralstrong{\sphinxupquote{f}}\sphinxstyleliteralstrong{\sphinxupquote{(}}\sphinxstyleliteralstrong{\sphinxupquote{u}} (\sphinxstyleliteralemphasis{\sphinxupquote{*}}) \textendash{} 

\item {} 
\sphinxAtStartPar
\sphinxstyleliteralstrong{\sphinxupquote{f}}\sphinxstyleliteralstrong{\sphinxupquote{(}}\sphinxstyleliteralstrong{\sphinxupquote{v}} (\sphinxstyleliteralemphasis{\sphinxupquote{v}}\sphinxstyleliteralemphasis{\sphinxupquote{) }}\sphinxstyleliteralemphasis{\sphinxupquote{=}}) \textendash{} 

\item {} 
\sphinxAtStartPar
\sphinxstyleliteralstrong{\sphinxupquote{u}}\sphinxstyleliteralstrong{\sphinxupquote{)}} \textendash{} 

\item {} 
\sphinxAtStartPar
\sphinxstyleliteralstrong{\sphinxupquote{f}}\sphinxstyleliteralstrong{\sphinxupquote{(}}\sphinxstyleliteralstrong{\sphinxupquote{u}} \textendash{} 

\item {} 
\sphinxAtStartPar
\sphinxstyleliteralstrong{\sphinxupquote{km}} (\sphinxstyleliteralemphasis{\sphinxupquote{v}}\sphinxstyleliteralemphasis{\sphinxupquote{) }}\sphinxstyleliteralemphasis{\sphinxupquote{returns a distance in}}) \textendash{} 

\end{itemize}

\end{description}\end{quote}
\subsubsection*{Notes}

\sphinxAtStartPar
Default distance function is \sphinxtitleref{scipy.spatial.distance.cityblock}.

\end{fulllineitems}


\sphinxstepscope


\chapter{Constraining components and the use of components}
\label{\detokenize{components_constrained_use:constraining-components-and-the-use-of-components}}\label{\detokenize{components_constrained_use::doc}}

\begin{savenotes}\sphinxattablestart
\centering
\begin{tabulary}{\linewidth}[t]{\X{1}{2}\X{1}{2}}
\hline

\sphinxAtStartPar
{\hyperref[\detokenize{generated/tamos.InterfaceMask:tamos.InterfaceMask}]{\sphinxcrossref{\sphinxcode{\sphinxupquote{tamos.InterfaceMask}}}}}(component{[}, element, ...{]})
&
\sphinxAtStartPar

\\
\hline
\sphinxAtStartPar
{\hyperref[\detokenize{generated/tamos.Hub.components_assemblies:tamos.Hub.components_assemblies}]{\sphinxcrossref{\sphinxcode{\sphinxupquote{tamos.Hub.components\_assemblies}}}}}
&
\sphinxAtStartPar
Components assemblies of the hub.
\\
\hline
\sphinxAtStartPar
{\hyperref[\detokenize{generated/tamos.MILPModel.components_assemblies:tamos.MILPModel.components_assemblies}]{\sphinxcrossref{\sphinxcode{\sphinxupquote{tamos.MILPModel.components\_assemblies}}}}}
&
\sphinxAtStartPar
Components assemblies of the model.
\\
\hline
\end{tabulary}
\par
\sphinxattableend\end{savenotes}

\sphinxstepscope


\section{tamos.InterfaceMask}
\label{\detokenize{generated/tamos.InterfaceMask:tamos-interfacemask}}\label{\detokenize{generated/tamos.InterfaceMask::doc}}\index{InterfaceMask (class in tamos)@\spxentry{InterfaceMask}\spxextra{class in tamos}}

\begin{fulllineitems}
\phantomsection\label{\detokenize{generated/tamos.InterfaceMask:tamos.InterfaceMask}}
\pysigstartsignatures
\pysiglinewithargsret{\sphinxbfcode{\sphinxupquote{class\DUrole{w}{  }}}\sphinxcode{\sphinxupquote{tamos.}}\sphinxbfcode{\sphinxupquote{InterfaceMask}}}{\emph{\DUrole{n}{component}}, \emph{\DUrole{n}{element}\DUrole{o}{=}\DUrole{default_value}{None}}, \emph{\DUrole{n}{name}\DUrole{o}{=}\DUrole{default_value}{None}}, \emph{\DUrole{n}{power\_lb}\DUrole{o}{=}\DUrole{default_value}{None}}, \emph{\DUrole{n}{power\_ub}\DUrole{o}{=}\DUrole{default_value}{None}}, \emph{\DUrole{n}{energy\_lb}\DUrole{o}{=}\DUrole{default_value}{None}}, \emph{\DUrole{n}{energy\_ub}\DUrole{o}{=}\DUrole{default_value}{None}}}{}
\pysigstopsignatures\index{\_\_init\_\_() (tamos.InterfaceMask method)@\spxentry{\_\_init\_\_()}\spxextra{tamos.InterfaceMask method}}

\begin{fulllineitems}
\phantomsection\label{\detokenize{generated/tamos.InterfaceMask:tamos.InterfaceMask.__init__}}
\pysigstartsignatures
\pysiglinewithargsret{\sphinxbfcode{\sphinxupquote{\_\_init\_\_}}}{\emph{\DUrole{n}{component}}, \emph{\DUrole{n}{element}\DUrole{o}{=}\DUrole{default_value}{None}}, \emph{\DUrole{n}{name}\DUrole{o}{=}\DUrole{default_value}{None}}, \emph{\DUrole{n}{power\_lb}\DUrole{o}{=}\DUrole{default_value}{None}}, \emph{\DUrole{n}{power\_ub}\DUrole{o}{=}\DUrole{default_value}{None}}, \emph{\DUrole{n}{energy\_lb}\DUrole{o}{=}\DUrole{default_value}{None}}, \emph{\DUrole{n}{energy\_ub}\DUrole{o}{=}\DUrole{default_value}{None}}}{}
\pysigstopsignatures
\sphinxAtStartPar
Defines power and energy constraints regarding exchanges between a component and the hub it is affected to.

\sphinxAtStartPar
Must be passed to a hub to be effective.
An InterfaceMask instance can be affected to different hubs. Constraints are applied in an independent manner
no matter the hubs it is affected to.
\begin{quote}\begin{description}
\sphinxlineitem{Parameters}\begin{itemize}
\item {} 
\sphinxAtStartPar
\sphinxstyleliteralstrong{\sphinxupquote{component}} (\sphinxstyleliteralemphasis{\sphinxupquote{production}}\sphinxstyleliteralemphasis{\sphinxupquote{, }}\sphinxstyleliteralemphasis{\sphinxupquote{storage}}\sphinxstyleliteralemphasis{\sphinxupquote{, }}\sphinxstyleliteralemphasis{\sphinxupquote{element\_IO}}\sphinxstyleliteralemphasis{\sphinxupquote{ or }}\sphinxstyleliteralemphasis{\sphinxupquote{network instance}}) \textendash{} 

\item {} 
\sphinxAtStartPar
\sphinxstyleliteralstrong{\sphinxupquote{element}} (\sphinxstyleliteralemphasis{\sphinxupquote{element instance}}\sphinxstyleliteralemphasis{\sphinxupquote{, }}\sphinxstyleliteralemphasis{\sphinxupquote{optional}}) \textendash{} Must be provided only if \sphinxtitleref{component} is a production or storage instance.

\item {} 
\sphinxAtStartPar
\sphinxstyleliteralstrong{\sphinxupquote{name}} (\sphinxstyleliteralemphasis{\sphinxupquote{str}}\sphinxstyleliteralemphasis{\sphinxupquote{, }}\sphinxstyleliteralemphasis{\sphinxupquote{optional}}) \textendash{} 

\item {} 
\sphinxAtStartPar
\sphinxstyleliteralstrong{\sphinxupquote{power\_lb}} (\sphinxstyleliteralemphasis{\sphinxupquote{numpy.ndarray}}\sphinxstyleliteralemphasis{\sphinxupquote{ or }}\sphinxstyleliteralemphasis{\sphinxupquote{float}}\sphinxstyleliteralemphasis{\sphinxupquote{, }}\sphinxstyleliteralemphasis{\sphinxupquote{optional}}) \textendash{} The lower bound (upper bound) of the power related to the element \sphinxtitleref{element} (if relevant) of component \sphinxtitleref{component}.

\item {} 
\sphinxAtStartPar
\sphinxstyleliteralstrong{\sphinxupquote{power\_ub}} (\sphinxstyleliteralemphasis{\sphinxupquote{numpy.ndarray}}\sphinxstyleliteralemphasis{\sphinxupquote{ or }}\sphinxstyleliteralemphasis{\sphinxupquote{float}}\sphinxstyleliteralemphasis{\sphinxupquote{, }}\sphinxstyleliteralemphasis{\sphinxupquote{optional}}) \textendash{} The lower bound (upper bound) of the power related to the element \sphinxtitleref{element} (if relevant) of component \sphinxtitleref{component}.

\item {} 
\sphinxAtStartPar
\sphinxstyleliteralstrong{\sphinxupquote{energy\_lb}} (\sphinxstyleliteralemphasis{\sphinxupquote{float}}\sphinxstyleliteralemphasis{\sphinxupquote{, }}\sphinxstyleliteralemphasis{\sphinxupquote{optional}}) \textendash{} The lower bound (upper bound) of the energy related to the element \sphinxtitleref{element} (if relevant) of component \sphinxtitleref{component}.
Energy is calculated as sum(power(t) * dt(t)) on the entire operation period.

\item {} 
\sphinxAtStartPar
\sphinxstyleliteralstrong{\sphinxupquote{energy\_ub}} (\sphinxstyleliteralemphasis{\sphinxupquote{float}}\sphinxstyleliteralemphasis{\sphinxupquote{, }}\sphinxstyleliteralemphasis{\sphinxupquote{optional}}) \textendash{} The lower bound (upper bound) of the energy related to the element \sphinxtitleref{element} (if relevant) of component \sphinxtitleref{component}.
Energy is calculated as sum(power(t) * dt(t)) on the entire operation period.

\end{itemize}

\end{description}\end{quote}
\subsubsection*{Examples}

\begin{sphinxVerbatim}[commandchars=\\\{\}]
\PYG{g+gp}{\PYGZgt{}\PYGZgt{}\PYGZgt{} }\PYG{n}{tms}\PYG{o}{.}\PYG{n}{InterfaceMask}\PYG{p}{(}\PYG{n}{component}\PYG{o}{=}\PYG{n}{electric\PYGZus{}heater}\PYG{p}{,} \PYG{n}{element}\PYG{o}{=}\PYG{n}{electricity}\PYG{p}{,} \PYG{n}{energy\PYGZus{}ub}\PYG{o}{=}\PYG{l+m+mi}{1000}\PYG{p}{)}
\end{sphinxVerbatim}

\sphinxAtStartPar
The amount of energy related to element \sphinxtitleref{electricity} during the entire operation period must not exceed 1000 kWh.

\begin{sphinxVerbatim}[commandchars=\\\{\}]
\PYG{g+gp}{\PYGZgt{}\PYGZgt{}\PYGZgt{} }\PYG{n}{tms}\PYG{o}{.}\PYG{n}{InterfaceMask}\PYG{p}{(}\PYG{n}{component}\PYG{o}{=}\PYG{n}{electric\PYGZus{}heater}\PYG{p}{,} \PYG{n}{element}\PYG{o}{=}\PYG{n}{heat}\PYG{p}{,} \PYG{n}{power\PYGZus{}lb}\PYG{o}{=}\PYG{l+m+mi}{5}\PYG{p}{)}
\end{sphinxVerbatim}

\sphinxAtStartPar
The amount of energy related to element \sphinxtitleref{heat} must be always greater than 5 kW.

\begin{sphinxVerbatim}[commandchars=\\\{\}]
\PYG{g+gp}{\PYGZgt{}\PYGZgt{}\PYGZgt{} }\PYG{n}{tms}\PYG{o}{.}\PYG{n}{InterfaceMask}\PYG{p}{(}\PYG{n}{component}\PYG{o}{=}\PYG{n}{electric\PYGZus{}heater}\PYG{p}{,} \PYG{n}{element}\PYG{o}{=}\PYG{n}{heat}\PYG{p}{,} \PYG{n}{power\PYGZus{}lb}\PYG{o}{=}\PYG{n}{numpy}\PYG{o}{.}\PYG{n}{linspace}\PYG{p}{(}\PYG{l+m+mi}{1}\PYG{p}{,} \PYG{l+m+mi}{10}\PYG{p}{,} \PYG{l+m+mi}{8760}\PYG{p}{)}\PYG{p}{,}
\PYG{g+gp}{... }                                                           \PYG{n}{power\PYGZus{}ub}\PYG{o}{=}\PYG{n}{numpy}\PYG{o}{.}\PYG{n}{linspace}\PYG{p}{(}\PYG{l+m+mi}{1}\PYG{p}{,} \PYG{l+m+mi}{10}\PYG{p}{,} \PYG{l+m+mi}{8760}\PYG{p}{)}\PYG{p}{)}
\end{sphinxVerbatim}

\sphinxAtStartPar
The amount of energy related to element \sphinxtitleref{heat} must increase linearly with time, from 1 to 10 kW.

\end{fulllineitems}

\subsubsection*{Methods}


\begin{savenotes}\sphinxattablestart
\centering
\begin{tabulary}{\linewidth}[t]{\X{1}{2}\X{1}{2}}
\hline

\sphinxAtStartPar
{\hyperref[\detokenize{generated/tamos.InterfaceMask:tamos.InterfaceMask.__init__}]{\sphinxcrossref{\sphinxcode{\sphinxupquote{\_\_init\_\_}}}}}(component{[}, element, name, ...{]})
&
\sphinxAtStartPar
Defines power and energy constraints regarding exchanges between a component and the hub it is affected to.
\\
\hline
\end{tabulary}
\par
\sphinxattableend\end{savenotes}
\subsubsection*{Attributes}


\begin{savenotes}\sphinxattablestart
\centering
\begin{tabulary}{\linewidth}[t]{\X{1}{2}\X{1}{2}}
\hline

\sphinxAtStartPar
{\hyperref[\detokenize{generated/tamos.InterfaceMask:tamos.InterfaceMask.component}]{\sphinxcrossref{\sphinxcode{\sphinxupquote{component}}}}}
&
\sphinxAtStartPar
The component whose element exchanges with the hub interface are constrained.
\\
\hline
\sphinxAtStartPar
{\hyperref[\detokenize{generated/tamos.InterfaceMask:tamos.InterfaceMask.element}]{\sphinxcrossref{\sphinxcode{\sphinxupquote{element}}}}}
&
\sphinxAtStartPar
If \sphinxtitleref{component} is a storage or production instance, the element whose power flows and energy balances are constrained.
\\
\hline
\sphinxAtStartPar
{\hyperref[\detokenize{generated/tamos.InterfaceMask:tamos.InterfaceMask.energy_lb}]{\sphinxcrossref{\sphinxcode{\sphinxupquote{energy\_lb}}}}}
&
\sphinxAtStartPar
The lower bound of the energy related to the element \sphinxtitleref{element} (if relevant) of component \sphinxtitleref{component}.
\\
\hline
\sphinxAtStartPar
{\hyperref[\detokenize{generated/tamos.InterfaceMask:tamos.InterfaceMask.energy_ub}]{\sphinxcrossref{\sphinxcode{\sphinxupquote{energy\_ub}}}}}
&
\sphinxAtStartPar
The upper bound of the energy related to the element \sphinxtitleref{element} (if relevant) of component \sphinxtitleref{component}.
\\
\hline
\sphinxAtStartPar
{\hyperref[\detokenize{generated/tamos.InterfaceMask:tamos.InterfaceMask.name}]{\sphinxcrossref{\sphinxcode{\sphinxupquote{name}}}}}
&
\sphinxAtStartPar
str.
\\
\hline
\sphinxAtStartPar
{\hyperref[\detokenize{generated/tamos.InterfaceMask:tamos.InterfaceMask.power_lb}]{\sphinxcrossref{\sphinxcode{\sphinxupquote{power\_lb}}}}}
&
\sphinxAtStartPar
The lower bound of the power related to the element \sphinxtitleref{element} (if relevant) of component \sphinxtitleref{component}.
\\
\hline
\sphinxAtStartPar
{\hyperref[\detokenize{generated/tamos.InterfaceMask:tamos.InterfaceMask.power_ub}]{\sphinxcrossref{\sphinxcode{\sphinxupquote{power\_ub}}}}}
&
\sphinxAtStartPar
The upper bound of the power related to the element \sphinxtitleref{element} (if relevant) of component \sphinxtitleref{component}.
\\
\hline
\end{tabulary}
\par
\sphinxattableend\end{savenotes}
\index{component (tamos.InterfaceMask property)@\spxentry{component}\spxextra{tamos.InterfaceMask property}}

\begin{fulllineitems}
\phantomsection\label{\detokenize{generated/tamos.InterfaceMask:tamos.InterfaceMask.component}}
\pysigstartsignatures
\pysigline{\sphinxbfcode{\sphinxupquote{property\DUrole{w}{  }}}\sphinxbfcode{\sphinxupquote{component}}}
\pysigstopsignatures
\sphinxAtStartPar
The component whose element exchanges with the hub interface are constrained.

\end{fulllineitems}

\index{element (tamos.InterfaceMask property)@\spxentry{element}\spxextra{tamos.InterfaceMask property}}

\begin{fulllineitems}
\phantomsection\label{\detokenize{generated/tamos.InterfaceMask:tamos.InterfaceMask.element}}
\pysigstartsignatures
\pysigline{\sphinxbfcode{\sphinxupquote{property\DUrole{w}{  }}}\sphinxbfcode{\sphinxupquote{element}}}
\pysigstopsignatures
\sphinxAtStartPar
If \sphinxtitleref{component} is a storage or production instance, the element whose power flows and energy balances are constrained.

\end{fulllineitems}

\index{energy\_lb (tamos.InterfaceMask property)@\spxentry{energy\_lb}\spxextra{tamos.InterfaceMask property}}

\begin{fulllineitems}
\phantomsection\label{\detokenize{generated/tamos.InterfaceMask:tamos.InterfaceMask.energy_lb}}
\pysigstartsignatures
\pysigline{\sphinxbfcode{\sphinxupquote{property\DUrole{w}{  }}}\sphinxbfcode{\sphinxupquote{energy\_lb}}}
\pysigstopsignatures
\sphinxAtStartPar
The lower bound of the energy related to the element \sphinxtitleref{element} (if relevant) of component \sphinxtitleref{component}.
Energy is calculated as sum(power(t) * dt(t)) on the entire optimization period.
None or float.

\end{fulllineitems}

\index{energy\_ub (tamos.InterfaceMask property)@\spxentry{energy\_ub}\spxextra{tamos.InterfaceMask property}}

\begin{fulllineitems}
\phantomsection\label{\detokenize{generated/tamos.InterfaceMask:tamos.InterfaceMask.energy_ub}}
\pysigstartsignatures
\pysigline{\sphinxbfcode{\sphinxupquote{property\DUrole{w}{  }}}\sphinxbfcode{\sphinxupquote{energy\_ub}}}
\pysigstopsignatures
\sphinxAtStartPar
The upper bound of the energy related to the element \sphinxtitleref{element} (if relevant) of component \sphinxtitleref{component}.
Energy is calculated as sum(power(t) * dt(t)) on the entire optimization period.
None or float.

\end{fulllineitems}

\index{name (tamos.InterfaceMask property)@\spxentry{name}\spxextra{tamos.InterfaceMask property}}

\begin{fulllineitems}
\phantomsection\label{\detokenize{generated/tamos.InterfaceMask:tamos.InterfaceMask.name}}
\pysigstartsignatures
\pysigline{\sphinxbfcode{\sphinxupquote{property\DUrole{w}{  }}}\sphinxbfcode{\sphinxupquote{name}}}
\pysigstopsignatures
\sphinxAtStartPar
str.
This name is used in MILP model description.
names must not exceed 255 characters,
all of which must be alphanumeric (a\sphinxhyphen{}z, A\sphinxhyphen{}Z, 0\sphinxhyphen{}9) or one of these symbols:
! ” \# \$ \% \& , . ; ? @ \_ ‘ ’ \{ \} \textasciitilde{}.
\begin{quote}\begin{description}
\sphinxlineitem{Type}
\sphinxAtStartPar
Name of the instance

\end{description}\end{quote}

\end{fulllineitems}

\index{power\_lb (tamos.InterfaceMask property)@\spxentry{power\_lb}\spxextra{tamos.InterfaceMask property}}

\begin{fulllineitems}
\phantomsection\label{\detokenize{generated/tamos.InterfaceMask:tamos.InterfaceMask.power_lb}}
\pysigstartsignatures
\pysigline{\sphinxbfcode{\sphinxupquote{property\DUrole{w}{  }}}\sphinxbfcode{\sphinxupquote{power\_lb}}}
\pysigstopsignatures
\sphinxAtStartPar
The lower bound of the power related to the element \sphinxtitleref{element} (if relevant) of component \sphinxtitleref{component}.
None, numpy.ndarray or float.

\end{fulllineitems}

\index{power\_ub (tamos.InterfaceMask property)@\spxentry{power\_ub}\spxextra{tamos.InterfaceMask property}}

\begin{fulllineitems}
\phantomsection\label{\detokenize{generated/tamos.InterfaceMask:tamos.InterfaceMask.power_ub}}
\pysigstartsignatures
\pysigline{\sphinxbfcode{\sphinxupquote{property\DUrole{w}{  }}}\sphinxbfcode{\sphinxupquote{power\_ub}}}
\pysigstopsignatures
\sphinxAtStartPar
The upper bound of the power related to the element \sphinxtitleref{element} (if relevant) of component \sphinxtitleref{component}.
None, numpy.ndarray or float.

\end{fulllineitems}


\end{fulllineitems}


\sphinxstepscope


\section{tamos.Hub.components\_assemblies}
\label{\detokenize{generated/tamos.Hub.components_assemblies:tamos-hub-components-assemblies}}\label{\detokenize{generated/tamos.Hub.components_assemblies::doc}}\index{components\_assemblies (tamos.Hub property)@\spxentry{components\_assemblies}\spxextra{tamos.Hub property}}

\begin{fulllineitems}
\phantomsection\label{\detokenize{generated/tamos.Hub.components_assemblies:tamos.Hub.components_assemblies}}
\pysigstartsignatures
\pysigline{\sphinxbfcode{\sphinxupquote{property\DUrole{w}{  }}}\sphinxcode{\sphinxupquote{Hub.}}\sphinxbfcode{\sphinxupquote{components\_assemblies}}}
\pysigstopsignatures
\sphinxAtStartPar
Components assemblies of the hub.

\sphinxAtStartPar
Must be provided as a list of 3\sphinxhyphen{}tuple objects (n\_min, n\_max, components) where:
\begin{itemize}
\item {} 
\sphinxAtStartPar
\sphinxtitleref{n\_min} (\sphinxtitleref{n\_max}) is the minimum (maximum) number of components from \sphinxtitleref{components} that must be installed in the hub.

\item {} 
\sphinxAtStartPar
\sphinxtitleref{components} is a component or list of production, storage or element\_IO components. If one component of \sphinxtitleref{components} is not one of hub components,
the 3\sphinxhyphen{}tuple (n\_min, n\_max, components) is ignored during constraints declaration.

\end{itemize}
\subsubsection*{Examples}

\begin{sphinxVerbatim}[commandchars=\\\{\}]
\PYG{g+gp}{\PYGZgt{}\PYGZgt{}\PYGZgt{} }\PYG{n}{hub}\PYG{o}{.}\PYG{n}{components\PYGZus{}assemblies} \PYG{o}{=} \PYG{p}{[}\PYG{p}{(}\PYG{l+m+mi}{1}\PYG{p}{,} \PYG{l+m+mi}{2}\PYG{p}{,} \PYG{p}{[}\PYG{n}{heat\PYGZus{}load\PYGZus{}1}\PYG{p}{,} \PYG{n}{heat\PYGZus{}load\PYGZus{}2}\PYG{p}{]}\PYG{p}{)}\PYG{p}{,} \PYG{p}{(}\PYG{l+m+mi}{1}\PYG{p}{,} \PYG{l+m+mi}{1}\PYG{p}{,} \PYG{n}{heat\PYGZus{}load\PYGZus{}3}\PYG{p}{)}\PYG{p}{,} \PYG{p}{(}\PYG{l+m+mi}{0}\PYG{p}{,} \PYG{l+m+mi}{1}\PYG{p}{,} \PYG{p}{[}\PYG{n}{electric\PYGZus{}heater\PYGZus{}1}\PYG{p}{,} \PYG{n}{electric\PYGZus{}heater\PYGZus{}2}\PYG{p}{]}\PYG{p}{)}\PYG{p}{]}\PYG{p}{)}
\end{sphinxVerbatim}

\sphinxAtStartPar
At least one of the first and second loads must be used.
The third load must be used. At most one of the two heaters must be used.

\begin{sphinxVerbatim}[commandchars=\\\{\}]
\PYG{g+gp}{\PYGZgt{}\PYGZgt{}\PYGZgt{} }\PYG{n}{hub}\PYG{o}{.}\PYG{n}{components\PYGZus{}assemblies} \PYG{o}{=} \PYG{p}{[}\PYG{p}{(}\PYG{l+m+mi}{0}\PYG{p}{,} \PYG{l+m+mi}{1}\PYG{p}{,} \PYG{p}{[}\PYG{n}{electric\PYGZus{}heater}\PYG{p}{,} \PYG{n}{heat\PYGZus{}grid}\PYG{p}{]}\PYG{p}{)}\PYG{p}{,} \PYG{p}{(}\PYG{l+m+mi}{1}\PYG{p}{,} \PYG{l+m+mi}{1}\PYG{p}{,} \PYG{n}{heat\PYGZus{}load}\PYG{p}{)}\PYG{p}{]}\PYG{p}{)}
\end{sphinxVerbatim}

\sphinxAtStartPar
The load must be used. Either electric\_heater or heat\_grid can be used.

\end{fulllineitems}


\sphinxstepscope


\section{tamos.MILPModel.components\_assemblies}
\label{\detokenize{generated/tamos.MILPModel.components_assemblies:tamos-milpmodel-components-assemblies}}\label{\detokenize{generated/tamos.MILPModel.components_assemblies::doc}}\index{components\_assemblies (tamos.MILPModel property)@\spxentry{components\_assemblies}\spxextra{tamos.MILPModel property}}

\begin{fulllineitems}
\phantomsection\label{\detokenize{generated/tamos.MILPModel.components_assemblies:tamos.MILPModel.components_assemblies}}
\pysigstartsignatures
\pysigline{\sphinxbfcode{\sphinxupquote{property\DUrole{w}{  }}}\sphinxcode{\sphinxupquote{MILPModel.}}\sphinxbfcode{\sphinxupquote{components\_assemblies}}}
\pysigstopsignatures
\sphinxAtStartPar
Components assemblies of the model.

\sphinxAtStartPar
Must be provided as a list of 3\sphinxhyphen{}tuple objects (n\_min, n\_max, components) where:
\begin{itemize}
\item {} 
\sphinxAtStartPar
\sphinxtitleref{n\_min} (\sphinxtitleref{n\_max}) is the minimum (maximum) number of components from \sphinxtitleref{components} that must be installed, all hubs of \sphinxtitleref{hubs} included.

\item {} 
\sphinxAtStartPar
\sphinxtitleref{components} is a component or list of production, storage or element\_IO components.
For any component of \sphinxtitleref{components}, if this component is not in at least one hub of this MILPModel instance,
the 3\sphinxhyphen{}tuple (n\_min, n\_max, components) is ignored during constraints declaration.

\end{itemize}
\subsubsection*{Examples}

\begin{sphinxVerbatim}[commandchars=\\\{\}]
\PYG{g+gp}{\PYGZgt{}\PYGZgt{}\PYGZgt{} }\PYG{n}{hub\PYGZus{}1} \PYG{o}{=} \PYG{n}{Hub}\PYG{p}{(}\PYG{n}{components}\PYG{o}{=}\PYG{p}{[}\PYG{n}{heat\PYGZus{}load\PYGZus{}1}\PYG{p}{,} \PYG{n}{heat\PYGZus{}load\PYGZus{}2}\PYG{p}{]}\PYG{p}{)}
\PYG{g+gp}{\PYGZgt{}\PYGZgt{}\PYGZgt{} }\PYG{n}{hub\PYGZus{}2} \PYG{o}{=} \PYG{n}{Hub}\PYG{p}{(}\PYG{n}{components}\PYG{o}{=}\PYG{p}{[}\PYG{n}{heat\PYGZus{}load\PYGZus{}1}\PYG{p}{]}\PYG{p}{)}
\PYG{g+gp}{\PYGZgt{}\PYGZgt{}\PYGZgt{} }\PYG{n}{hub\PYGZus{}3} \PYG{o}{=} \PYG{n}{Hub}\PYG{p}{(}\PYG{n}{components}\PYG{o}{=}\PYG{p}{[}\PYG{n}{heat\PYGZus{}load\PYGZus{}2}\PYG{p}{]}\PYG{p}{)}
\PYG{g+gp}{\PYGZgt{}\PYGZgt{}\PYGZgt{} }\PYG{n}{MILPModel} \PYG{o}{=} \PYG{n}{MILPModel}\PYG{p}{(}\PYG{n}{hubs}\PYG{o}{=}\PYG{p}{[}\PYG{n}{hub\PYGZus{}1}\PYG{p}{,} \PYG{n}{hub\PYGZus{}2}\PYG{p}{,} \PYG{n}{hub\PYGZus{}3}\PYG{p}{]}\PYG{p}{)}
\PYG{g+gp}{\PYGZgt{}\PYGZgt{}\PYGZgt{} }\PYG{n}{MILPModel}\PYG{o}{.}\PYG{n}{components\PYGZus{}assemblies} \PYG{o}{=} \PYG{p}{[}\PYG{p}{(}\PYG{l+m+mi}{0}\PYG{p}{,} \PYG{l+m+mi}{1}\PYG{p}{,} \PYG{n}{heat\PYGZus{}load\PYGZus{}1}\PYG{p}{)}\PYG{p}{,} \PYG{p}{(}\PYG{l+m+mi}{2}\PYG{p}{,} \PYG{l+m+mi}{3}\PYG{p}{,} \PYG{p}{[}\PYG{n}{heat\PYGZus{}load\PYGZus{}1}\PYG{p}{,} \PYG{n}{heat\PYGZus{}load\PYGZus{}2}\PYG{p}{]}\PYG{p}{)}\PYG{p}{]}
\end{sphinxVerbatim}

\sphinxAtStartPar
heat\_load\_1 might be used at most one time, all hubs {[}hub\_1, hub\_2, hub\_3{]} included.
the number of times heat\_load\_1 or heat\_load\_2 are used in all hubs {[}hub\_1, hub\_2, hub\_3{]}
is greater than 2 but smaller than 3.

\end{fulllineitems}


\sphinxstepscope


\chapter{Data input and output}
\label{\detokenize{data_IO:data-input-and-output}}\label{\detokenize{data_IO::doc}}

\section{Export and aggregate results}
\label{\detokenize{data_IO:export-and-aggregate-results}}

\begin{savenotes}\sphinxattablestart
\centering
\begin{tabulary}{\linewidth}[t]{\X{1}{2}\X{1}{2}}
\hline

\sphinxAtStartPar
{\hyperref[\detokenize{generated/tamos.data_IO.ResultsExport:tamos.data_IO.ResultsExport}]{\sphinxcrossref{\sphinxcode{\sphinxupquote{tamos.data\_IO.ResultsExport}}}}}(MILPModel{[}, ...{]})
&
\sphinxAtStartPar

\\
\hline
\sphinxAtStartPar
{\hyperref[\detokenize{generated/tamos.data_IO.ResultsBatch:tamos.data_IO.ResultsBatch}]{\sphinxcrossref{\sphinxcode{\sphinxupquote{tamos.data\_IO.ResultsBatch}}}}}(working\_dir{[}, name{]})
&
\sphinxAtStartPar

\\
\hline
\end{tabulary}
\par
\sphinxattableend\end{savenotes}

\sphinxstepscope


\subsection{tamos.data\_IO.ResultsExport}
\label{\detokenize{generated/tamos.data_IO.ResultsExport:tamos-data-io-resultsexport}}\label{\detokenize{generated/tamos.data_IO.ResultsExport::doc}}\index{ResultsExport (class in tamos.data\_IO)@\spxentry{ResultsExport}\spxextra{class in tamos.data\_IO}}

\begin{fulllineitems}
\phantomsection\label{\detokenize{generated/tamos.data_IO.ResultsExport:tamos.data_IO.ResultsExport}}
\pysigstartsignatures
\pysiglinewithargsret{\sphinxbfcode{\sphinxupquote{class\DUrole{w}{  }}}\sphinxcode{\sphinxupquote{tamos.data\_IO.}}\sphinxbfcode{\sphinxupquote{ResultsExport}}}{\emph{\DUrole{n}{MILPModel}}, \emph{\DUrole{n}{get\_LP}\DUrole{o}{=}\DUrole{default_value}{False}}, \emph{\DUrole{n}{get\_MPS}\DUrole{o}{=}\DUrole{default_value}{False}}, \emph{\DUrole{n}{parent\_working\_dir}\DUrole{o}{=}\DUrole{default_value}{None}}, \emph{\DUrole{n}{csv\_precision}\DUrole{o}{=}\DUrole{default_value}{3}}, \emph{\DUrole{n}{replace\_inverted\_TVP}\DUrole{o}{=}\DUrole{default_value}{False}}}{}
\pysigstopsignatures\index{\_\_init\_\_() (tamos.data\_IO.ResultsExport method)@\spxentry{\_\_init\_\_()}\spxextra{tamos.data\_IO.ResultsExport method}}

\begin{fulllineitems}
\phantomsection\label{\detokenize{generated/tamos.data_IO.ResultsExport:tamos.data_IO.ResultsExport.__init__}}
\pysigstartsignatures
\pysiglinewithargsret{\sphinxbfcode{\sphinxupquote{\_\_init\_\_}}}{\emph{\DUrole{n}{MILPModel}}, \emph{\DUrole{n}{get\_LP}\DUrole{o}{=}\DUrole{default_value}{False}}, \emph{\DUrole{n}{get\_MPS}\DUrole{o}{=}\DUrole{default_value}{False}}, \emph{\DUrole{n}{parent\_working\_dir}\DUrole{o}{=}\DUrole{default_value}{None}}, \emph{\DUrole{n}{csv\_precision}\DUrole{o}{=}\DUrole{default_value}{3}}, \emph{\DUrole{n}{replace\_inverted\_TVP}\DUrole{o}{=}\DUrole{default_value}{False}}}{}
\pysigstopsignatures
\sphinxAtStartPar
ResultsExport instances extract and write to disk the decision variables and KPIs of a solved MILPModel instance.

\sphinxAtStartPar
Variable and KPIs values are collected at the instance creation. These are stored as attributes.
A call to a \sphinxtitleref{write\_} method copies this information on disk as CSV and text files.
A call to \sphinxtitleref{dump\_object} method write a binary form of this instance on disk.
\begin{quote}\begin{description}
\sphinxlineitem{Parameters}\begin{itemize}
\item {} 
\sphinxAtStartPar
\sphinxstyleliteralstrong{\sphinxupquote{MILPModel}} ({\hyperref[\detokenize{generated/tamos.MILPModel:tamos.MILPModel}]{\sphinxcrossref{\sphinxstyleliteralemphasis{\sphinxupquote{MILPModel}}}}}) \textendash{} If the solving of the MILPModel instance did not succeed, only pre solve model information is collected by ResultsExport.
If the solving succeeded, KPIs, decision variables and some non\sphinxhyphen{}essential model results are collected.

\item {} 
\sphinxAtStartPar
\sphinxstyleliteralstrong{\sphinxupquote{get\_LP}} (\sphinxstyleliteralemphasis{\sphinxupquote{bool}}\sphinxstyleliteralemphasis{\sphinxupquote{, }}\sphinxstyleliteralemphasis{\sphinxupquote{optional}}\sphinxstyleliteralemphasis{\sphinxupquote{, }}\sphinxstyleliteralemphasis{\sphinxupquote{default}}\sphinxstyleliteralemphasis{\sphinxupquote{, }}\sphinxstyleliteralemphasis{\sphinxupquote{False}}) \textendash{} Whether to collect the model following the LP standard.

\item {} 
\sphinxAtStartPar
\sphinxstyleliteralstrong{\sphinxupquote{get\_MPS}} (\sphinxstyleliteralemphasis{\sphinxupquote{bool}}\sphinxstyleliteralemphasis{\sphinxupquote{, }}\sphinxstyleliteralemphasis{\sphinxupquote{optional}}\sphinxstyleliteralemphasis{\sphinxupquote{, }}\sphinxstyleliteralemphasis{\sphinxupquote{default False}}) \textendash{} Whether to collect the model following the MPS standard.

\item {} 
\sphinxAtStartPar
\sphinxstyleliteralstrong{\sphinxupquote{parent\_working\_dir}} (\sphinxstyleliteralemphasis{\sphinxupquote{str}}\sphinxstyleliteralemphasis{\sphinxupquote{ or }}\sphinxstyleliteralemphasis{\sphinxupquote{path\sphinxhyphen{}like}}\sphinxstyleliteralemphasis{\sphinxupquote{, }}\sphinxstyleliteralemphasis{\sphinxupquote{optional}}) \textendash{} This defines the \sphinxtitleref{working\_dir} attribute according to working\_dir=parent\_working\_dir/name,
where ‘name’ is the \sphinxtitleref{name} attribute of MILPModel.
The export of results is done in \sphinxtitleref{working\_dir}.
If not provided, \sphinxtitleref{parent\_working\_dir} is a new directory called ‘results’ in the parent working directory.

\item {} 
\sphinxAtStartPar
\sphinxstyleliteralstrong{\sphinxupquote{csv\_precision}} (\sphinxstyleliteralemphasis{\sphinxupquote{int}}\sphinxstyleliteralemphasis{\sphinxupquote{, }}\sphinxstyleliteralemphasis{\sphinxupquote{optional}}\sphinxstyleliteralemphasis{\sphinxupquote{, }}\sphinxstyleliteralemphasis{\sphinxupquote{default 3}}) \textendash{} Number of decimals regarding CSV export of decision variables and KPIs.

\item {} 
\sphinxAtStartPar
\sphinxstyleliteralstrong{\sphinxupquote{replace\_inverted\_TVP}} (\sphinxstyleliteralemphasis{\sphinxupquote{bool}}\sphinxstyleliteralemphasis{\sphinxupquote{, }}\sphinxstyleliteralemphasis{\sphinxupquote{optional}}\sphinxstyleliteralemphasis{\sphinxupquote{, }}\sphinxstyleliteralemphasis{\sphinxupquote{False}}) \textendash{} \begin{itemize}
\item {} 
\sphinxAtStartPar
If True, ThermalVectorPair elements ‘\textasciitilde{}TVP’ are replaced by ‘TVP’ if both of them are used in the model.
This does not change the values of decision variables and KPIs since
a flow associated with ‘TVP’ is the opposite of the same flow associated with ‘\textasciitilde{}TVP’.
Setting {\color{red}\bfseries{}\textasciigrave{}}replace\_inverted\_TVP\textasciigrave{}=True can make the results reading easier.

\item {} 
\sphinxAtStartPar
If False, both ‘\textasciitilde{}TVP’ and ‘TVP’ are kept.

\end{itemize}


\end{itemize}

\end{description}\end{quote}
\subsubsection*{Notes}

\sphinxAtStartPar
A ResultsExport instance is a representation of a MILPModel instance at the time it is created.
The further change in the MILPModel instance are not reflected in the already existing ResultsExport instance.

\end{fulllineitems}

\subsubsection*{Methods}


\begin{savenotes}\sphinxattablestart
\centering
\begin{tabulary}{\linewidth}[t]{\X{1}{2}\X{1}{2}}
\hline

\sphinxAtStartPar
{\hyperref[\detokenize{generated/tamos.data_IO.ResultsExport:tamos.data_IO.ResultsExport.__init__}]{\sphinxcrossref{\sphinxcode{\sphinxupquote{\_\_init\_\_}}}}}(MILPModel{[}, get\_LP, get\_MPS, ...{]})
&
\sphinxAtStartPar
ResultsExport instances extract and write to disk the decision variables and KPIs of a solved MILPModel instance.
\\
\hline
\sphinxAtStartPar
{\hyperref[\detokenize{generated/tamos.data_IO.ResultsExport:tamos.data_IO.ResultsExport.dump_object}]{\sphinxcrossref{\sphinxcode{\sphinxupquote{dump\_object}}}}}()
&
\sphinxAtStartPar
Writes the binary form of this instance.
\\
\hline
\sphinxAtStartPar
{\hyperref[\detokenize{generated/tamos.data_IO.ResultsExport:tamos.data_IO.ResultsExport.load_object}]{\sphinxcrossref{\sphinxcode{\sphinxupquote{load\_object}}}}}(path)
&
\sphinxAtStartPar
Loads a binary ResultsExport or ResultsBatch object.
\\
\hline
\sphinxAtStartPar
{\hyperref[\detokenize{generated/tamos.data_IO.ResultsExport:tamos.data_IO.ResultsExport.reduce_memory}]{\sphinxcrossref{\sphinxcode{\sphinxupquote{reduce\_memory}}}}}(df)
&
\sphinxAtStartPar
Reduces the memory used by a DataFrame by changing the data type of numerical columns.
\\
\hline
\sphinxAtStartPar
{\hyperref[\detokenize{generated/tamos.data_IO.ResultsExport:tamos.data_IO.ResultsExport.remove_small_values}]{\sphinxcrossref{\sphinxcode{\sphinxupquote{remove\_small\_values}}}}}(df, name{[}, threshold, ...{]})
&
\sphinxAtStartPar
Removes small numerical values from a dataframe containing decision variable or KPI values.
\\
\hline
\sphinxAtStartPar
{\hyperref[\detokenize{generated/tamos.data_IO.ResultsExport:tamos.data_IO.ResultsExport.write_KPIs}]{\sphinxcrossref{\sphinxcode{\sphinxupquote{write\_KPIs}}}}}()
&
\sphinxAtStartPar
Writes all KPIs defined at component level in a directory named \textquotesingle{}KPIs\textquotesingle{}.
\\
\hline
\sphinxAtStartPar
{\hyperref[\detokenize{generated/tamos.data_IO.ResultsExport:tamos.data_IO.ResultsExport.write_LP}]{\sphinxcrossref{\sphinxcode{\sphinxupquote{write\_LP}}}}}()
&
\sphinxAtStartPar
Writes the model according to Cplex LP standard.
\\
\hline
\sphinxAtStartPar
{\hyperref[\detokenize{generated/tamos.data_IO.ResultsExport:tamos.data_IO.ResultsExport.write_MPS}]{\sphinxcrossref{\sphinxcode{\sphinxupquote{write\_MPS}}}}}()
&
\sphinxAtStartPar
Writes the model according to the MPS standard.
\\
\hline
\sphinxAtStartPar
{\hyperref[\detokenize{generated/tamos.data_IO.ResultsExport:tamos.data_IO.ResultsExport.write_all}]{\sphinxcrossref{\sphinxcode{\sphinxupquote{write\_all}}}}}()
&
\sphinxAtStartPar
Writes all the content of \sphinxtitleref{MILPModel} collected by this instance.
\\
\hline
\sphinxAtStartPar
{\hyperref[\detokenize{generated/tamos.data_IO.ResultsExport:tamos.data_IO.ResultsExport.write_description}]{\sphinxcrossref{\sphinxcode{\sphinxupquote{write\_description}}}}}()
&
\sphinxAtStartPar
Writes the \sphinxtitleref{description} attribute of \sphinxtitleref{MILPModel}.
\\
\hline
\sphinxAtStartPar
{\hyperref[\detokenize{generated/tamos.data_IO.ResultsExport:tamos.data_IO.ResultsExport.write_exec_time}]{\sphinxcrossref{\sphinxcode{\sphinxupquote{write\_exec\_time}}}}}()
&
\sphinxAtStartPar
Writes the time spent on:
\\
\hline
\sphinxAtStartPar
{\hyperref[\detokenize{generated/tamos.data_IO.ResultsExport:tamos.data_IO.ResultsExport.write_model_complexity}]{\sphinxcrossref{\sphinxcode{\sphinxupquote{write\_model\_complexity}}}}}()
&
\sphinxAtStartPar
Writes the number of decision variables and constraints per type (continuous, discrete, etc...).
\\
\hline
\sphinxAtStartPar
{\hyperref[\detokenize{generated/tamos.data_IO.ResultsExport:tamos.data_IO.ResultsExport.write_solution_summary}]{\sphinxcrossref{\sphinxcode{\sphinxupquote{write\_solution\_summary}}}}}()
&
\sphinxAtStartPar
Writes important information about the solved model, including the values of the 3 main KPIs, "Eco", "CO2" and "Exergy".
\\
\hline
\sphinxAtStartPar
{\hyperref[\detokenize{generated/tamos.data_IO.ResultsExport:tamos.data_IO.ResultsExport.write_unsatisfied_constraints}]{\sphinxcrossref{\sphinxcode{\sphinxupquote{write\_unsatisfied\_constraints}}}}}()
&
\sphinxAtStartPar
Writes every constraint of the model that is unsatisfied after model solve, given a tolerance of 1e\sphinxhyphen{}6.
\\
\hline
\sphinxAtStartPar
{\hyperref[\detokenize{generated/tamos.data_IO.ResultsExport:tamos.data_IO.ResultsExport.write_variables}]{\sphinxcrossref{\sphinxcode{\sphinxupquote{write\_variables}}}}}()
&
\sphinxAtStartPar
Writes all decision variables in a directory named \textquotesingle{}Variables\textquotesingle{}.
\\
\hline
\end{tabulary}
\par
\sphinxattableend\end{savenotes}
\subsubsection*{Attributes}


\begin{savenotes}\sphinxattablestart
\centering
\begin{tabulary}{\linewidth}[t]{\X{1}{2}\X{1}{2}}
\hline

\sphinxAtStartPar
{\hyperref[\detokenize{generated/tamos.data_IO.ResultsExport:tamos.data_IO.ResultsExport.working_dir}]{\sphinxcrossref{\sphinxcode{\sphinxupquote{working\_dir}}}}}
&
\sphinxAtStartPar
Directory the results are read from or exported to.
\\
\hline
\end{tabulary}
\par
\sphinxattableend\end{savenotes}
\index{dump\_object() (tamos.data\_IO.ResultsExport method)@\spxentry{dump\_object()}\spxextra{tamos.data\_IO.ResultsExport method}}

\begin{fulllineitems}
\phantomsection\label{\detokenize{generated/tamos.data_IO.ResultsExport:tamos.data_IO.ResultsExport.dump_object}}
\pysigstartsignatures
\pysiglinewithargsret{\sphinxbfcode{\sphinxupquote{dump\_object}}}{}{}
\pysigstopsignatures
\sphinxAtStartPar
Writes the binary form of this instance.
\subsubsection*{Notes}
\begin{enumerate}
\sphinxsetlistlabels{\arabic}{enumi}{enumii}{}{.}%
\item {} 
\sphinxAtStartPar
Model data (decision variables, constraints, KPIs, etc…) are lost during this operation.

\item {} 
\sphinxAtStartPar
The binary form of this instance is called ‘results \sphinxtitleref{name}’ with \sphinxtitleref{name} the \sphinxtitleref{name} attribute of \sphinxtitleref{MILPModel}.

\item {} 
\sphinxAtStartPar
The \sphinxtitleref{description} attribute of \sphinxtitleref{MILPModel} is also dumped to speed up the call of \sphinxtitleref{ResultsBatch.create\_batch} method.

\end{enumerate}

\end{fulllineitems}

\index{load\_object() (tamos.data\_IO.ResultsExport static method)@\spxentry{load\_object()}\spxextra{tamos.data\_IO.ResultsExport static method}}

\begin{fulllineitems}
\phantomsection\label{\detokenize{generated/tamos.data_IO.ResultsExport:tamos.data_IO.ResultsExport.load_object}}
\pysigstartsignatures
\pysiglinewithargsret{\sphinxbfcode{\sphinxupquote{static\DUrole{w}{  }}}\sphinxbfcode{\sphinxupquote{load\_object}}}{\emph{\DUrole{n}{path}}}{}
\pysigstopsignatures
\sphinxAtStartPar
Loads a binary ResultsExport or ResultsBatch object.
\begin{quote}\begin{description}
\sphinxlineitem{Parameters}
\sphinxAtStartPar
\sphinxstyleliteralstrong{\sphinxupquote{path}} (\sphinxstyleliteralemphasis{\sphinxupquote{str}}\sphinxstyleliteralemphasis{\sphinxupquote{ or }}\sphinxstyleliteralemphasis{\sphinxupquote{path\sphinxhyphen{}like}}) \textendash{} 

\sphinxlineitem{Return type}
\sphinxAtStartPar
The loaded object.

\end{description}\end{quote}

\end{fulllineitems}

\index{reduce\_memory() (tamos.data\_IO.ResultsExport static method)@\spxentry{reduce\_memory()}\spxextra{tamos.data\_IO.ResultsExport static method}}

\begin{fulllineitems}
\phantomsection\label{\detokenize{generated/tamos.data_IO.ResultsExport:tamos.data_IO.ResultsExport.reduce_memory}}
\pysigstartsignatures
\pysiglinewithargsret{\sphinxbfcode{\sphinxupquote{static\DUrole{w}{  }}}\sphinxbfcode{\sphinxupquote{reduce\_memory}}}{\emph{\DUrole{n}{df}}}{}
\pysigstopsignatures
\sphinxAtStartPar
Reduces the memory used by a DataFrame by changing the data type of numerical columns.

\sphinxAtStartPar
Data type of float and integer columns is changed for the type with the smallest memory usage.
\begin{quote}\begin{description}
\sphinxlineitem{Parameters}
\sphinxAtStartPar
\sphinxstyleliteralstrong{\sphinxupquote{df}} (\sphinxstyleliteralemphasis{\sphinxupquote{DataFrame}}) \textendash{} 

\end{description}\end{quote}
\subsubsection*{Notes}

\sphinxAtStartPar
This method works inplace, i.e. \sphinxtitleref{df} is modified.

\end{fulllineitems}

\index{remove\_small\_values() (tamos.data\_IO.ResultsExport static method)@\spxentry{remove\_small\_values()}\spxextra{tamos.data\_IO.ResultsExport static method}}

\begin{fulllineitems}
\phantomsection\label{\detokenize{generated/tamos.data_IO.ResultsExport:tamos.data_IO.ResultsExport.remove_small_values}}
\pysigstartsignatures
\pysiglinewithargsret{\sphinxbfcode{\sphinxupquote{static\DUrole{w}{  }}}\sphinxbfcode{\sphinxupquote{remove\_small\_values}}}{\emph{\DUrole{n}{df}}, \emph{\DUrole{n}{name}}, \emph{\DUrole{n}{threshold}\DUrole{o}{=}\DUrole{default_value}{0.1}}, \emph{\DUrole{n}{drop\_zeros}\DUrole{o}{=}\DUrole{default_value}{None}}}{}
\pysigstopsignatures
\sphinxAtStartPar
Removes small numerical values from a dataframe containing decision variable or KPI values.
\begin{quote}\begin{description}
\sphinxlineitem{Parameters}\begin{itemize}
\item {} 
\sphinxAtStartPar
\sphinxstyleliteralstrong{\sphinxupquote{df}} (\sphinxstyleliteralemphasis{\sphinxupquote{pandas.DataFrame}}) \textendash{} 

\item {} 
\sphinxAtStartPar
\sphinxstyleliteralstrong{\sphinxupquote{name}} (\sphinxstyleliteralemphasis{\sphinxupquote{str}}) \textendash{} Name of the column of \sphinxtitleref{df} that contains the numerical values.

\item {} 
\sphinxAtStartPar
\sphinxstyleliteralstrong{\sphinxupquote{threshold}} (\sphinxstyleliteralemphasis{\sphinxupquote{int}}\sphinxstyleliteralemphasis{\sphinxupquote{ or }}\sphinxstyleliteralemphasis{\sphinxupquote{float}}\sphinxstyleliteralemphasis{\sphinxupquote{, }}\sphinxstyleliteralemphasis{\sphinxupquote{optional}}\sphinxstyleliteralemphasis{\sphinxupquote{, }}\sphinxstyleliteralemphasis{\sphinxupquote{default 0.1}}) \textendash{} All values ‘x’ for which abs(x) \textless{} \sphinxtitleref{threshold} are replaced by 0.

\item {} 
\sphinxAtStartPar
\sphinxstyleliteralstrong{\sphinxupquote{drop\_zeros}} (\sphinxstyleliteralemphasis{\sphinxupquote{\{\textquotesingle{}Series\textquotesingle{}}}\sphinxstyleliteralemphasis{\sphinxupquote{, }}\sphinxstyleliteralemphasis{\sphinxupquote{\textquotesingle{}All\textquotesingle{}\}}}\sphinxstyleliteralemphasis{\sphinxupquote{, }}\sphinxstyleliteralemphasis{\sphinxupquote{optional}}) \textendash{} \begin{itemize}
\item {} 
\sphinxAtStartPar
If not provided, all rows where \sphinxtitleref{name} value is zero are kept.

\item {} 
\sphinxAtStartPar
’Series’: each temporal data series is kept if at least one value in this serie is not zero.
\sphinxtitleref{df} must include a ‘Date’ column for \sphinxtitleref{drop\_zeros\textasciigrave{}=’Series’ to be effective.
A temporal data serie of \textasciigrave{}df} is a subset of df such that only the ‘Date’ and \sphinxtitleref{name} columns
have changing values in this subset.

\item {} 
\sphinxAtStartPar
’All’ : remove all rows where \sphinxtitleref{name} value is zero.

\end{itemize}


\end{itemize}

\end{description}\end{quote}
\subsubsection*{Notes}

\sphinxAtStartPar
This method works inplace, i.e. \sphinxtitleref{df} is modified.
Near\sphinxhyphen{}zero values in column \sphinxtitleref{name} are removed according to \sphinxtitleref{threshold} and \sphinxtitleref{drop\_zeros}.

\end{fulllineitems}

\index{working\_dir (tamos.data\_IO.ResultsExport property)@\spxentry{working\_dir}\spxextra{tamos.data\_IO.ResultsExport property}}

\begin{fulllineitems}
\phantomsection\label{\detokenize{generated/tamos.data_IO.ResultsExport:tamos.data_IO.ResultsExport.working_dir}}
\pysigstartsignatures
\pysigline{\sphinxbfcode{\sphinxupquote{property\DUrole{w}{  }}}\sphinxbfcode{\sphinxupquote{working\_dir}}}
\pysigstopsignatures
\sphinxAtStartPar
Directory the results are read from or exported to.

\end{fulllineitems}

\index{write\_KPIs() (tamos.data\_IO.ResultsExport method)@\spxentry{write\_KPIs()}\spxextra{tamos.data\_IO.ResultsExport method}}

\begin{fulllineitems}
\phantomsection\label{\detokenize{generated/tamos.data_IO.ResultsExport:tamos.data_IO.ResultsExport.write_KPIs}}
\pysigstartsignatures
\pysiglinewithargsret{\sphinxbfcode{\sphinxupquote{write\_KPIs}}}{}{}
\pysigstopsignatures
\sphinxAtStartPar
Writes all KPIs defined at component level in a directory named ‘KPIs’.

\end{fulllineitems}

\index{write\_LP() (tamos.data\_IO.ResultsExport method)@\spxentry{write\_LP()}\spxextra{tamos.data\_IO.ResultsExport method}}

\begin{fulllineitems}
\phantomsection\label{\detokenize{generated/tamos.data_IO.ResultsExport:tamos.data_IO.ResultsExport.write_LP}}
\pysigstartsignatures
\pysiglinewithargsret{\sphinxbfcode{\sphinxupquote{write\_LP}}}{}{}
\pysigstopsignatures
\sphinxAtStartPar
Writes the model according to Cplex LP standard.

\sphinxAtStartPar
The \sphinxtitleref{get\_LP} argument must have been set to ‘True’ at instance creation.
\subsubsection*{Notes}

\sphinxAtStartPar
Cplex LP standard slightly differs from the conventional LP standard.
Prefer the use \sphinxtitleref{write\_MPS} in case a communication with another solver is needed

\end{fulllineitems}

\index{write\_MPS() (tamos.data\_IO.ResultsExport method)@\spxentry{write\_MPS()}\spxextra{tamos.data\_IO.ResultsExport method}}

\begin{fulllineitems}
\phantomsection\label{\detokenize{generated/tamos.data_IO.ResultsExport:tamos.data_IO.ResultsExport.write_MPS}}
\pysigstartsignatures
\pysiglinewithargsret{\sphinxbfcode{\sphinxupquote{write\_MPS}}}{}{}
\pysigstopsignatures
\sphinxAtStartPar
Writes the model according to the MPS standard.

\sphinxAtStartPar
The \sphinxtitleref{get\_MPS} argument must have been set to ‘True’ at instance creation.

\end{fulllineitems}

\index{write\_all() (tamos.data\_IO.ResultsExport method)@\spxentry{write\_all()}\spxextra{tamos.data\_IO.ResultsExport method}}

\begin{fulllineitems}
\phantomsection\label{\detokenize{generated/tamos.data_IO.ResultsExport:tamos.data_IO.ResultsExport.write_all}}
\pysigstartsignatures
\pysiglinewithargsret{\sphinxbfcode{\sphinxupquote{write\_all}}}{}{}
\pysigstopsignatures
\sphinxAtStartPar
Writes all the content of \sphinxtitleref{MILPModel} collected by this instance.

\end{fulllineitems}

\index{write\_description() (tamos.data\_IO.ResultsExport method)@\spxentry{write\_description()}\spxextra{tamos.data\_IO.ResultsExport method}}

\begin{fulllineitems}
\phantomsection\label{\detokenize{generated/tamos.data_IO.ResultsExport:tamos.data_IO.ResultsExport.write_description}}
\pysigstartsignatures
\pysiglinewithargsret{\sphinxbfcode{\sphinxupquote{write\_description}}}{}{}
\pysigstopsignatures
\sphinxAtStartPar
Writes the \sphinxtitleref{description} attribute of \sphinxtitleref{MILPModel}.

\end{fulllineitems}

\index{write\_exec\_time() (tamos.data\_IO.ResultsExport method)@\spxentry{write\_exec\_time()}\spxextra{tamos.data\_IO.ResultsExport method}}

\begin{fulllineitems}
\phantomsection\label{\detokenize{generated/tamos.data_IO.ResultsExport:tamos.data_IO.ResultsExport.write_exec_time}}
\pysigstartsignatures
\pysiglinewithargsret{\sphinxbfcode{\sphinxupquote{write\_exec\_time}}}{}{}
\pysigstopsignatures
\sphinxAtStartPar
Writes the time spent on:
\begin{itemize}
\item {} 
\sphinxAtStartPar
decision variables declaration (‘MILPModel.declare\_variables’)

\item {} 
\sphinxAtStartPar
constraints and KPIs declaration (‘MILPModel.declare\_constraints\_and\_KPIs’)

\item {} 
\sphinxAtStartPar
solving process, in seconds (‘MILPModel.solve’)

\item {} 
\sphinxAtStartPar
solving process, in deterministic Cplex ticks (‘Solving deterministic time’)

\item {} 
\sphinxAtStartPar
total time (variables, constraints, KPIs, solving), in seconds (‘Total time’)

\end{itemize}

\end{fulllineitems}

\index{write\_model\_complexity() (tamos.data\_IO.ResultsExport method)@\spxentry{write\_model\_complexity()}\spxextra{tamos.data\_IO.ResultsExport method}}

\begin{fulllineitems}
\phantomsection\label{\detokenize{generated/tamos.data_IO.ResultsExport:tamos.data_IO.ResultsExport.write_model_complexity}}
\pysigstartsignatures
\pysiglinewithargsret{\sphinxbfcode{\sphinxupquote{write\_model\_complexity}}}{}{}
\pysigstopsignatures
\sphinxAtStartPar
Writes the number of decision variables and constraints per type (continuous, discrete, etc…).

\end{fulllineitems}

\index{write\_solution\_summary() (tamos.data\_IO.ResultsExport method)@\spxentry{write\_solution\_summary()}\spxextra{tamos.data\_IO.ResultsExport method}}

\begin{fulllineitems}
\phantomsection\label{\detokenize{generated/tamos.data_IO.ResultsExport:tamos.data_IO.ResultsExport.write_solution_summary}}
\pysigstartsignatures
\pysiglinewithargsret{\sphinxbfcode{\sphinxupquote{write\_solution\_summary}}}{}{}
\pysigstopsignatures
\sphinxAtStartPar
Writes important information about the solved model, including the values of the 3 main KPIs, “Eco”, “CO2” and “Exergy”.

\end{fulllineitems}

\index{write\_unsatisfied\_constraints() (tamos.data\_IO.ResultsExport method)@\spxentry{write\_unsatisfied\_constraints()}\spxextra{tamos.data\_IO.ResultsExport method}}

\begin{fulllineitems}
\phantomsection\label{\detokenize{generated/tamos.data_IO.ResultsExport:tamos.data_IO.ResultsExport.write_unsatisfied_constraints}}
\pysigstartsignatures
\pysiglinewithargsret{\sphinxbfcode{\sphinxupquote{write\_unsatisfied\_constraints}}}{}{}
\pysigstopsignatures
\sphinxAtStartPar
Writes every constraint of the model that is unsatisfied after model solve, given a tolerance of 1e\sphinxhyphen{}6.

\end{fulllineitems}

\index{write\_variables() (tamos.data\_IO.ResultsExport method)@\spxentry{write\_variables()}\spxextra{tamos.data\_IO.ResultsExport method}}

\begin{fulllineitems}
\phantomsection\label{\detokenize{generated/tamos.data_IO.ResultsExport:tamos.data_IO.ResultsExport.write_variables}}
\pysigstartsignatures
\pysiglinewithargsret{\sphinxbfcode{\sphinxupquote{write\_variables}}}{}{}
\pysigstopsignatures
\sphinxAtStartPar
Writes all decision variables in a directory named ‘Variables’.

\end{fulllineitems}


\end{fulllineitems}


\sphinxstepscope


\subsection{tamos.data\_IO.ResultsBatch}
\label{\detokenize{generated/tamos.data_IO.ResultsBatch:tamos-data-io-resultsbatch}}\label{\detokenize{generated/tamos.data_IO.ResultsBatch::doc}}\index{ResultsBatch (class in tamos.data\_IO)@\spxentry{ResultsBatch}\spxextra{class in tamos.data\_IO}}

\begin{fulllineitems}
\phantomsection\label{\detokenize{generated/tamos.data_IO.ResultsBatch:tamos.data_IO.ResultsBatch}}
\pysigstartsignatures
\pysiglinewithargsret{\sphinxbfcode{\sphinxupquote{class\DUrole{w}{  }}}\sphinxcode{\sphinxupquote{tamos.data\_IO.}}\sphinxbfcode{\sphinxupquote{ResultsBatch}}}{\emph{\DUrole{n}{working\_dir}}, \emph{\DUrole{n}{name}\DUrole{o}{=}\DUrole{default_value}{None}}}{}
\pysigstopsignatures\index{\_\_init\_\_() (tamos.data\_IO.ResultsBatch method)@\spxentry{\_\_init\_\_()}\spxextra{tamos.data\_IO.ResultsBatch method}}

\begin{fulllineitems}
\phantomsection\label{\detokenize{generated/tamos.data_IO.ResultsBatch:tamos.data_IO.ResultsBatch.__init__}}
\pysigstartsignatures
\pysiglinewithargsret{\sphinxbfcode{\sphinxupquote{\_\_init\_\_}}}{\emph{\DUrole{n}{working\_dir}}, \emph{\DUrole{n}{name}\DUrole{o}{=}\DUrole{default_value}{None}}}{}
\pysigstopsignatures
\sphinxAtStartPar
Do not use. See \sphinxtitleref{create\_batch} and \sphinxtitleref{create\_batch\_from\_binaries} class methods.

\end{fulllineitems}

\subsubsection*{Methods}


\begin{savenotes}\sphinxattablestart
\centering
\begin{tabulary}{\linewidth}[t]{\X{1}{2}\X{1}{2}}
\hline

\sphinxAtStartPar
{\hyperref[\detokenize{generated/tamos.data_IO.ResultsBatch:tamos.data_IO.ResultsBatch.__init__}]{\sphinxcrossref{\sphinxcode{\sphinxupquote{\_\_init\_\_}}}}}(working\_dir{[}, name{]})
&
\sphinxAtStartPar
Do not use.
\\
\hline
\sphinxAtStartPar
{\hyperref[\detokenize{generated/tamos.data_IO.ResultsBatch:tamos.data_IO.ResultsBatch.create_batch}]{\sphinxcrossref{\sphinxcode{\sphinxupquote{create\_batch}}}}}(working\_dir, description\_matcher)
&
\sphinxAtStartPar
A ResultsBatch instance binds ResultsExport instances together to ease the process of results analysis.
\\
\hline
\sphinxAtStartPar
{\hyperref[\detokenize{generated/tamos.data_IO.ResultsBatch:tamos.data_IO.ResultsBatch.create_batch_from_binaries}]{\sphinxcrossref{\sphinxcode{\sphinxupquote{create\_batch\_from\_binaries}}}}}(working\_dir, ...)
&
\sphinxAtStartPar
A ResultsBatch instance binds ResultsExport instances together to ease the process of results analysis.
\\
\hline
\sphinxAtStartPar
{\hyperref[\detokenize{generated/tamos.data_IO.ResultsBatch:tamos.data_IO.ResultsBatch.dump_object}]{\sphinxcrossref{\sphinxcode{\sphinxupquote{dump\_object}}}}}()
&
\sphinxAtStartPar
Writes the binary form of this instance in the directory \sphinxtitleref{working\_dir}.
\\
\hline
\sphinxAtStartPar
{\hyperref[\detokenize{generated/tamos.data_IO.ResultsBatch:tamos.data_IO.ResultsBatch.load_object}]{\sphinxcrossref{\sphinxcode{\sphinxupquote{load\_object}}}}}(path)
&
\sphinxAtStartPar
Loads a binary ResultsExport or ResultsBatch object.
\\
\hline
\sphinxAtStartPar
{\hyperref[\detokenize{generated/tamos.data_IO.ResultsBatch:tamos.data_IO.ResultsBatch.reduce_memory}]{\sphinxcrossref{\sphinxcode{\sphinxupquote{reduce\_memory}}}}}(df)
&
\sphinxAtStartPar
Reduces the memory used by a DataFrame by changing the data type of numerical columns.
\\
\hline
\sphinxAtStartPar
{\hyperref[\detokenize{generated/tamos.data_IO.ResultsBatch:tamos.data_IO.ResultsBatch.remove_descriptors}]{\sphinxcrossref{\sphinxcode{\sphinxupquote{remove\_descriptors}}}}}(descriptors{[}, check\_unique{]})
&
\sphinxAtStartPar
Removes some descriptors, i.e. keys of attribute \sphinxtitleref{relevant\_descriptor}.
\\
\hline
\sphinxAtStartPar
{\hyperref[\detokenize{generated/tamos.data_IO.ResultsBatch:tamos.data_IO.ResultsBatch.remove_small_values}]{\sphinxcrossref{\sphinxcode{\sphinxupquote{remove\_small\_values}}}}}(df, name{[}, threshold, ...{]})
&
\sphinxAtStartPar
Removes small numerical values from a dataframe containing decision variable or KPI values.
\\
\hline
\sphinxAtStartPar
{\hyperref[\detokenize{generated/tamos.data_IO.ResultsBatch:tamos.data_IO.ResultsBatch.sum}]{\sphinxcrossref{\sphinxcode{\sphinxupquote{sum}}}}}(RBs)
&
\sphinxAtStartPar
Defines a unique ResultsBatch instance that is the sum of several instances.
\\
\hline
\sphinxAtStartPar
{\hyperref[\detokenize{generated/tamos.data_IO.ResultsBatch:tamos.data_IO.ResultsBatch.write_all}]{\sphinxcrossref{\sphinxcode{\sphinxupquote{write\_all}}}}}()
&
\sphinxAtStartPar
Writes the \sphinxtitleref{vars\_}, \sphinxtitleref{KPIs} and \sphinxtitleref{results} attributes as CSV files, in the directory \sphinxtitleref{working\_dir}.
\\
\hline
\end{tabulary}
\par
\sphinxattableend\end{savenotes}
\subsubsection*{Attributes}


\begin{savenotes}\sphinxattablestart
\centering
\begin{tabulary}{\linewidth}[t]{\X{1}{2}\X{1}{2}}
\hline

\sphinxAtStartPar
{\hyperref[\detokenize{generated/tamos.data_IO.ResultsBatch:tamos.data_IO.ResultsBatch.working_dir}]{\sphinxcrossref{\sphinxcode{\sphinxupquote{working\_dir}}}}}
&
\sphinxAtStartPar
Directory the results are read from or exported to.
\\
\hline
\end{tabulary}
\par
\sphinxattableend\end{savenotes}
\index{create\_batch() (tamos.data\_IO.ResultsBatch class method)@\spxentry{create\_batch()}\spxextra{tamos.data\_IO.ResultsBatch class method}}

\begin{fulllineitems}
\phantomsection\label{\detokenize{generated/tamos.data_IO.ResultsBatch:tamos.data_IO.ResultsBatch.create_batch}}
\pysigstartsignatures
\pysiglinewithargsret{\sphinxbfcode{\sphinxupquote{classmethod\DUrole{w}{  }}}\sphinxbfcode{\sphinxupquote{create\_batch}}}{\emph{\DUrole{n}{working\_dir}}, \emph{\DUrole{n}{description\_matcher}}, \emph{\DUrole{n}{keep\_unknown\_file\_descriptor}\DUrole{o}{=}\DUrole{default_value}{True}}, \emph{\DUrole{n}{keep\_unknown\_matcher\_descriptor}\DUrole{o}{=}\DUrole{default_value}{True}}, \emph{\DUrole{n}{upfront\_processing\_func}\DUrole{o}{=}\DUrole{default_value}{None}}, \emph{\DUrole{n}{keep\_results\_exports}\DUrole{o}{=}\DUrole{default_value}{False}}, \emph{\DUrole{n}{name}\DUrole{o}{=}\DUrole{default_value}{None}}, \emph{\DUrole{o}{**}\DUrole{n}{upfront\_processing\_kwargs}}}{}
\pysigstopsignatures
\sphinxAtStartPar
A ResultsBatch instance binds ResultsExport instances together to ease the process of results analysis.

\sphinxAtStartPar
Important attributes of ResultsBatch instances are:
* \sphinxtitleref{vars\_} : concatenation of the decision variables of the ResultsExport instances
* \sphinxtitleref{KPIs} : concatenation of the KPIs of the ResultsExport instances
* \sphinxtitleref{results} : other information about ResultsExport instances

\sphinxAtStartPar
This method creates a ResultsBatch instance from ResultsExports stored on disk.
It must be used when various optimization results are stored in the same directory and clearly identified by
the \sphinxtitleref{description} attribute of MILPModel instances.

\sphinxAtStartPar
The ResultsExport instances to keep are specified using the \sphinxtitleref{keep\_unknown\_file\_descriptor} and
\sphinxtitleref{keep\_unknown\_matcher\_descriptor} attributes.
\begin{quote}\begin{description}
\sphinxlineitem{Parameters}\begin{itemize}
\item {} 
\sphinxAtStartPar
\sphinxstyleliteralstrong{\sphinxupquote{working\_dir}} (\sphinxstyleliteralemphasis{\sphinxupquote{str}}\sphinxstyleliteralemphasis{\sphinxupquote{ or }}\sphinxstyleliteralemphasis{\sphinxupquote{path\sphinxhyphen{}like}}) \textendash{} The directory containing the ResultsExport results.
This directory is typically the same than the one specified for the \sphinxtitleref{parent\_working\_dir} argument of ResultsExport instances.
This directory is also the one where the aggregated results will be written.

\item {} 
\sphinxAtStartPar
\sphinxstyleliteralstrong{\sphinxupquote{description\_matcher}} (\sphinxstyleliteralemphasis{\sphinxupquote{dict \{str: list}}\sphinxstyleliteralemphasis{\sphinxupquote{(}}\sphinxstyleliteralemphasis{\sphinxupquote{int}}\sphinxstyleliteralemphasis{\sphinxupquote{ | }}\sphinxstyleliteralemphasis{\sphinxupquote{float}}\sphinxstyleliteralemphasis{\sphinxupquote{ | }}\sphinxstyleliteralemphasis{\sphinxupquote{str}}\sphinxstyleliteralemphasis{\sphinxupquote{) }}\sphinxstyleliteralemphasis{\sphinxupquote{| }}\sphinxstyleliteralemphasis{\sphinxupquote{None\}}}) \textendash{} 
\sphinxAtStartPar
Any ResultsExport in \sphinxtitleref{working\_dir} having a \sphinxtitleref{description} attribute matching \sphinxtitleref{description\_matcher} is kept.
Let \sphinxtitleref{key} and \sphinxtitleref{values} such that description\_matcher{[}key{]} = values.
\begin{itemize}
\item {} 
\sphinxAtStartPar
If \sphinxtitleref{values} is None, every ResultsExports having \sphinxtitleref{key} as a key of its \sphinxtitleref{description} attribute is kept.

\item {} 
\sphinxAtStartPar
Else, \sphinxtitleref{values} is a list of int, float and str.
A ResultsExports having \sphinxtitleref{key} as a key of its \sphinxtitleref{description} attribute is kept only if description{[}key{]} is one of \sphinxtitleref{values}.

\end{itemize}


\item {} 
\sphinxAtStartPar
\sphinxstyleliteralstrong{\sphinxupquote{keep\_unknown\_file\_descriptor}} (\sphinxstyleliteralemphasis{\sphinxupquote{bool}}\sphinxstyleliteralemphasis{\sphinxupquote{, }}\sphinxstyleliteralemphasis{\sphinxupquote{optional}}\sphinxstyleliteralemphasis{\sphinxupquote{, }}\sphinxstyleliteralemphasis{\sphinxupquote{default True}}) \textendash{} If False, ResultsExports instances whose \sphinxtitleref{description} attribute contains keys that are not keys of \sphinxtitleref{description\_matcher} are discarded.
If True, these instances are kept only if the the conditions specified by the \sphinxtitleref{description\_matcher} attribute are met.

\item {} 
\sphinxAtStartPar
\sphinxstyleliteralstrong{\sphinxupquote{keep\_unknown\_matcher\_descriptor}} (\sphinxstyleliteralemphasis{\sphinxupquote{bool}}\sphinxstyleliteralemphasis{\sphinxupquote{, }}\sphinxstyleliteralemphasis{\sphinxupquote{optional}}\sphinxstyleliteralemphasis{\sphinxupquote{, }}\sphinxstyleliteralemphasis{\sphinxupquote{default True}}) \textendash{} If False, ResultsExports instances whose \sphinxtitleref{description} attribute lacks keys that are keys of \sphinxtitleref{description\_matcher} are discarded.
If True, these instances are kept only if the the conditions specified by the \sphinxtitleref{description\_matcher} attribute are met.

\item {} 
\sphinxAtStartPar
\sphinxstyleliteralstrong{\sphinxupquote{upfront\_processing\_func}} (\sphinxstyleliteralemphasis{\sphinxupquote{callable f}}\sphinxstyleliteralemphasis{\sphinxupquote{(}}\sphinxstyleliteralemphasis{\sphinxupquote{x}}\sphinxstyleliteralemphasis{\sphinxupquote{)}}\sphinxstyleliteralemphasis{\sphinxupquote{, }}\sphinxstyleliteralemphasis{\sphinxupquote{optional.}}) \textendash{} Useful to process ResultsExport instances before they are aggregated into one single ResultsBatch.
f must accept a ResultsExport instance and performs inplace.
Attribute \sphinxtitleref{description} of \sphinxtitleref{x} must not be modified.

\item {} 
\sphinxAtStartPar
\sphinxstyleliteralstrong{\sphinxupquote{keep\_results\_exports}} (\sphinxstyleliteralemphasis{\sphinxupquote{bool}}\sphinxstyleliteralemphasis{\sphinxupquote{, }}\sphinxstyleliteralemphasis{\sphinxupquote{optional}}\sphinxstyleliteralemphasis{\sphinxupquote{, }}\sphinxstyleliteralemphasis{\sphinxupquote{default False}}) \textendash{} If True, all the ResultsExport instances that match the selection criteria are kept in the \sphinxtitleref{results\_exports} attribute (dict).

\item {} 
\sphinxAtStartPar
\sphinxstyleliteralstrong{\sphinxupquote{name}} (\sphinxstyleliteralemphasis{\sphinxupquote{str}}\sphinxstyleliteralemphasis{\sphinxupquote{, }}\sphinxstyleliteralemphasis{\sphinxupquote{optional}}) \textendash{} Name of this instance.
Setting a name may prevent the deletion of existing files during disk dump of this instance.

\item {} 
\sphinxAtStartPar
\sphinxstyleliteralstrong{\sphinxupquote{upfront\_processing\_kwargs}} (\sphinxstyleliteralemphasis{\sphinxupquote{optional}}) \textendash{} Keyword arguments passed to the \sphinxtitleref{upfront\_processing\_func} callable, like: f(x, {\color{red}\bfseries{}**}kwargs).

\end{itemize}

\sphinxlineitem{Returns}
\sphinxAtStartPar
\sphinxstylestrong{RB}

\sphinxlineitem{Return type}
\sphinxAtStartPar
{\hyperref[\detokenize{generated/tamos.data_IO.ResultsBatch:tamos.data_IO.ResultsBatch}]{\sphinxcrossref{ResultsBatch}}}

\end{description}\end{quote}
\subsubsection*{Notes}
\begin{enumerate}
\sphinxsetlistlabels{\arabic}{enumi}{enumii}{}{.}%
\item {} 
\sphinxAtStartPar
The \sphinxtitleref{relevant\_descriptors} attribute is a dict summing\sphinxhyphen{}up the \sphinxtitleref{description} attribute of the MILPModel instances.

\item {} 
\sphinxAtStartPar
The ‘batch\_analysis’ Jupyter Notebook requires a ResultsBatch instance.

\end{enumerate}

\end{fulllineitems}

\index{create\_batch\_from\_binaries() (tamos.data\_IO.ResultsBatch class method)@\spxentry{create\_batch\_from\_binaries()}\spxextra{tamos.data\_IO.ResultsBatch class method}}

\begin{fulllineitems}
\phantomsection\label{\detokenize{generated/tamos.data_IO.ResultsBatch:tamos.data_IO.ResultsBatch.create_batch_from_binaries}}
\pysigstartsignatures
\pysiglinewithargsret{\sphinxbfcode{\sphinxupquote{classmethod\DUrole{w}{  }}}\sphinxbfcode{\sphinxupquote{create\_batch\_from\_binaries}}}{\emph{\DUrole{n}{working\_dir}}, \emph{\DUrole{n}{results\_exports: List of pp}}, \emph{\DUrole{n}{upfront\_processing\_func=None}}, \emph{\DUrole{n}{name=None}}, \emph{\DUrole{n}{**upfront\_processing\_kwargs}}}{}
\pysigstopsignatures
\sphinxAtStartPar
A ResultsBatch instance binds ResultsExport instances together to ease the process of results analysis.

\sphinxAtStartPar
Important attributes of ResultsBatch instances are:
* \sphinxtitleref{vars\_}: concatenation of the decision variables of the ResultsExport instances
* \sphinxtitleref{KPIs}: concatenation of the KPIs of the ResultsExport instances
* \sphinxtitleref{results}: other information about ResultsExport instances

\sphinxAtStartPar
This method creates a ResultsBatch instance from a list of ResultsExports.
\begin{quote}\begin{description}
\sphinxlineitem{Parameters}\begin{itemize}
\item {} 
\sphinxAtStartPar
\sphinxstyleliteralstrong{\sphinxupquote{working\_dir}} (\sphinxstyleliteralemphasis{\sphinxupquote{str}}\sphinxstyleliteralemphasis{\sphinxupquote{ or }}\sphinxstyleliteralemphasis{\sphinxupquote{path\sphinxhyphen{}like}}) \textendash{} The directory where the aggregated results will be written.

\item {} 
\sphinxAtStartPar
\sphinxstyleliteralstrong{\sphinxupquote{results\_exports}} (\sphinxstyleliteralemphasis{\sphinxupquote{list of ResultsExport}}) \textendash{} ResultsExport instances that define the ResultsBatch instance. Duplicates will be ignored.

\item {} 
\sphinxAtStartPar
\sphinxstyleliteralstrong{\sphinxupquote{upfront\_processing\_func}} (\sphinxstyleliteralemphasis{\sphinxupquote{callable f}}\sphinxstyleliteralemphasis{\sphinxupquote{(}}\sphinxstyleliteralemphasis{\sphinxupquote{x}}\sphinxstyleliteralemphasis{\sphinxupquote{)}}\sphinxstyleliteralemphasis{\sphinxupquote{, }}\sphinxstyleliteralemphasis{\sphinxupquote{optional.}}) \textendash{} Useful to process ResultsExport instances before they are aggregated into one single ResultsBatch.
f must accept a ResultsExport instance and performs inplace.
Attribute \sphinxtitleref{description} of \sphinxtitleref{x} must not be modified.

\item {} 
\sphinxAtStartPar
\sphinxstyleliteralstrong{\sphinxupquote{name}} (\sphinxstyleliteralemphasis{\sphinxupquote{str}}\sphinxstyleliteralemphasis{\sphinxupquote{, }}\sphinxstyleliteralemphasis{\sphinxupquote{optional}}) \textendash{} Name of this instance.
Setting a name may prevent the deletion of existing files during disk dump of this instance.

\item {} 
\sphinxAtStartPar
\sphinxstyleliteralstrong{\sphinxupquote{upfront\_processing\_kwargs}} (\sphinxstyleliteralemphasis{\sphinxupquote{optional}}) \textendash{} Keyword arguments passed to the \sphinxtitleref{upfront\_processing\_func} callable, like: f(x, {\color{red}\bfseries{}**}kwargs).

\end{itemize}

\sphinxlineitem{Returns}
\sphinxAtStartPar
\sphinxstylestrong{RB}

\sphinxlineitem{Return type}
\sphinxAtStartPar
{\hyperref[\detokenize{generated/tamos.data_IO.ResultsBatch:tamos.data_IO.ResultsBatch}]{\sphinxcrossref{ResultsBatch}}}

\end{description}\end{quote}

\end{fulllineitems}

\index{dump\_object() (tamos.data\_IO.ResultsBatch method)@\spxentry{dump\_object()}\spxextra{tamos.data\_IO.ResultsBatch method}}

\begin{fulllineitems}
\phantomsection\label{\detokenize{generated/tamos.data_IO.ResultsBatch:tamos.data_IO.ResultsBatch.dump_object}}
\pysigstartsignatures
\pysiglinewithargsret{\sphinxbfcode{\sphinxupquote{dump\_object}}}{}{}
\pysigstopsignatures
\sphinxAtStartPar
Writes the binary form of this instance in the directory \sphinxtitleref{working\_dir}.

\end{fulllineitems}

\index{load\_object() (tamos.data\_IO.ResultsBatch static method)@\spxentry{load\_object()}\spxextra{tamos.data\_IO.ResultsBatch static method}}

\begin{fulllineitems}
\phantomsection\label{\detokenize{generated/tamos.data_IO.ResultsBatch:tamos.data_IO.ResultsBatch.load_object}}
\pysigstartsignatures
\pysiglinewithargsret{\sphinxbfcode{\sphinxupquote{static\DUrole{w}{  }}}\sphinxbfcode{\sphinxupquote{load\_object}}}{\emph{\DUrole{n}{path}}}{}
\pysigstopsignatures
\sphinxAtStartPar
Loads a binary ResultsExport or ResultsBatch object.
\begin{quote}\begin{description}
\sphinxlineitem{Parameters}
\sphinxAtStartPar
\sphinxstyleliteralstrong{\sphinxupquote{path}} (\sphinxstyleliteralemphasis{\sphinxupquote{str}}\sphinxstyleliteralemphasis{\sphinxupquote{ or }}\sphinxstyleliteralemphasis{\sphinxupquote{path\sphinxhyphen{}like}}) \textendash{} 

\sphinxlineitem{Return type}
\sphinxAtStartPar
The loaded object.

\end{description}\end{quote}

\end{fulllineitems}

\index{reduce\_memory() (tamos.data\_IO.ResultsBatch static method)@\spxentry{reduce\_memory()}\spxextra{tamos.data\_IO.ResultsBatch static method}}

\begin{fulllineitems}
\phantomsection\label{\detokenize{generated/tamos.data_IO.ResultsBatch:tamos.data_IO.ResultsBatch.reduce_memory}}
\pysigstartsignatures
\pysiglinewithargsret{\sphinxbfcode{\sphinxupquote{static\DUrole{w}{  }}}\sphinxbfcode{\sphinxupquote{reduce\_memory}}}{\emph{\DUrole{n}{df}}}{}
\pysigstopsignatures
\sphinxAtStartPar
Reduces the memory used by a DataFrame by changing the data type of numerical columns.

\sphinxAtStartPar
Data type of float and integer columns is changed for the type with the smallest memory usage.
\begin{quote}\begin{description}
\sphinxlineitem{Parameters}
\sphinxAtStartPar
\sphinxstyleliteralstrong{\sphinxupquote{df}} (\sphinxstyleliteralemphasis{\sphinxupquote{DataFrame}}) \textendash{} 

\end{description}\end{quote}
\subsubsection*{Notes}

\sphinxAtStartPar
This method works inplace, i.e. \sphinxtitleref{df} is modified.

\end{fulllineitems}

\index{remove\_descriptors() (tamos.data\_IO.ResultsBatch method)@\spxentry{remove\_descriptors()}\spxextra{tamos.data\_IO.ResultsBatch method}}

\begin{fulllineitems}
\phantomsection\label{\detokenize{generated/tamos.data_IO.ResultsBatch:tamos.data_IO.ResultsBatch.remove_descriptors}}
\pysigstartsignatures
\pysiglinewithargsret{\sphinxbfcode{\sphinxupquote{remove\_descriptors}}}{\emph{\DUrole{n}{descriptors}}, \emph{\DUrole{n}{check\_unique}\DUrole{o}{=}\DUrole{default_value}{False}}}{}
\pysigstopsignatures
\sphinxAtStartPar
Removes some descriptors, i.e. keys of attribute \sphinxtitleref{relevant\_descriptor}.
\begin{quote}\begin{description}
\sphinxlineitem{Parameters}\begin{itemize}
\item {} 
\sphinxAtStartPar
\sphinxstyleliteralstrong{\sphinxupquote{descriptors}} (\sphinxstyleliteralemphasis{\sphinxupquote{list of str}}) \textendash{} Must be a subset of the keys of \sphinxtitleref{relevant\_descriptors}.

\item {} 
\sphinxAtStartPar
\sphinxstyleliteralstrong{\sphinxupquote{check\_unique}} (\sphinxstyleliteralemphasis{\sphinxupquote{bool}}\sphinxstyleliteralemphasis{\sphinxupquote{, }}\sphinxstyleliteralemphasis{\sphinxupquote{optional}}\sphinxstyleliteralemphasis{\sphinxupquote{, }}\sphinxstyleliteralemphasis{\sphinxupquote{default False}}) \textendash{} 
\sphinxAtStartPar
Describes the method behavior when any of the descriptor in \sphinxtitleref{descriptors} is associated with multiple values:
\begin{itemize}
\item {} 
\sphinxAtStartPar
If True, an AssertionError is raised.

\item {} 
\sphinxAtStartPar
If False, descriptor is removed, leading to possible duplicated rows in attributes \sphinxtitleref{vars\_}, \sphinxtitleref{KPIs} and \sphinxtitleref{results}

\end{itemize}


\end{itemize}

\end{description}\end{quote}

\end{fulllineitems}

\index{remove\_small\_values() (tamos.data\_IO.ResultsBatch static method)@\spxentry{remove\_small\_values()}\spxextra{tamos.data\_IO.ResultsBatch static method}}

\begin{fulllineitems}
\phantomsection\label{\detokenize{generated/tamos.data_IO.ResultsBatch:tamos.data_IO.ResultsBatch.remove_small_values}}
\pysigstartsignatures
\pysiglinewithargsret{\sphinxbfcode{\sphinxupquote{static\DUrole{w}{  }}}\sphinxbfcode{\sphinxupquote{remove\_small\_values}}}{\emph{\DUrole{n}{df}}, \emph{\DUrole{n}{name}}, \emph{\DUrole{n}{threshold}\DUrole{o}{=}\DUrole{default_value}{0.1}}, \emph{\DUrole{n}{drop\_zeros}\DUrole{o}{=}\DUrole{default_value}{None}}}{}
\pysigstopsignatures
\sphinxAtStartPar
Removes small numerical values from a dataframe containing decision variable or KPI values.
\begin{quote}\begin{description}
\sphinxlineitem{Parameters}\begin{itemize}
\item {} 
\sphinxAtStartPar
\sphinxstyleliteralstrong{\sphinxupquote{df}} (\sphinxstyleliteralemphasis{\sphinxupquote{pandas.DataFrame}}) \textendash{} 

\item {} 
\sphinxAtStartPar
\sphinxstyleliteralstrong{\sphinxupquote{name}} (\sphinxstyleliteralemphasis{\sphinxupquote{str}}) \textendash{} Name of the column of \sphinxtitleref{df} that contains the numerical values.

\item {} 
\sphinxAtStartPar
\sphinxstyleliteralstrong{\sphinxupquote{threshold}} (\sphinxstyleliteralemphasis{\sphinxupquote{int}}\sphinxstyleliteralemphasis{\sphinxupquote{ or }}\sphinxstyleliteralemphasis{\sphinxupquote{float}}\sphinxstyleliteralemphasis{\sphinxupquote{, }}\sphinxstyleliteralemphasis{\sphinxupquote{optional}}\sphinxstyleliteralemphasis{\sphinxupquote{, }}\sphinxstyleliteralemphasis{\sphinxupquote{default 0.1}}) \textendash{} All values ‘x’ for which abs(x) \textless{} \sphinxtitleref{threshold} are replaced by 0.

\item {} 
\sphinxAtStartPar
\sphinxstyleliteralstrong{\sphinxupquote{drop\_zeros}} (\sphinxstyleliteralemphasis{\sphinxupquote{\{\textquotesingle{}Series\textquotesingle{}}}\sphinxstyleliteralemphasis{\sphinxupquote{, }}\sphinxstyleliteralemphasis{\sphinxupquote{\textquotesingle{}All\textquotesingle{}\}}}\sphinxstyleliteralemphasis{\sphinxupquote{, }}\sphinxstyleliteralemphasis{\sphinxupquote{optional}}) \textendash{} \begin{itemize}
\item {} 
\sphinxAtStartPar
If not provided, all rows where \sphinxtitleref{name} value is zero are kept.

\item {} 
\sphinxAtStartPar
’Series’: each temporal data series is kept if at least one value in this serie is not zero.
\sphinxtitleref{df} must include a ‘Date’ column for \sphinxtitleref{drop\_zeros\textasciigrave{}=’Series’ to be effective.
A temporal data serie of \textasciigrave{}df} is a subset of df such that only the ‘Date’ and \sphinxtitleref{name} columns
have changing values in this subset.

\item {} 
\sphinxAtStartPar
’All’ : remove all rows where \sphinxtitleref{name} value is zero.

\end{itemize}


\end{itemize}

\end{description}\end{quote}
\subsubsection*{Notes}

\sphinxAtStartPar
This method works inplace, i.e. \sphinxtitleref{df} is modified.
Near\sphinxhyphen{}zero values in column \sphinxtitleref{name} are removed according to \sphinxtitleref{threshold} and \sphinxtitleref{drop\_zeros}.

\end{fulllineitems}

\index{sum() (tamos.data\_IO.ResultsBatch class method)@\spxentry{sum()}\spxextra{tamos.data\_IO.ResultsBatch class method}}

\begin{fulllineitems}
\phantomsection\label{\detokenize{generated/tamos.data_IO.ResultsBatch:tamos.data_IO.ResultsBatch.sum}}
\pysigstartsignatures
\pysiglinewithargsret{\sphinxbfcode{\sphinxupquote{classmethod\DUrole{w}{  }}}\sphinxbfcode{\sphinxupquote{sum}}}{\emph{\DUrole{n}{RBs}}}{}
\pysigstopsignatures
\sphinxAtStartPar
Defines a unique ResultsBatch instance that is the sum of several instances.

\sphinxAtStartPar
Attributes \sphinxtitleref{vars\_}, \sphinxtitleref{KPIs} and \sphinxtitleref{results} are concatenated and merged.
Attributes \sphinxtitleref{relevant\_indicators} are merged.
\begin{quote}\begin{description}
\sphinxlineitem{Parameters}
\sphinxAtStartPar
\sphinxstyleliteralstrong{\sphinxupquote{RBs}} (\sphinxstyleliteralemphasis{\sphinxupquote{list of ResultsBatch}}) \textendash{} 

\sphinxlineitem{Returns}
\sphinxAtStartPar
\sphinxstylestrong{RB} \textendash{} A ResultsBatch instance

\sphinxlineitem{Return type}
\sphinxAtStartPar
{\hyperref[\detokenize{generated/tamos.data_IO.ResultsBatch:tamos.data_IO.ResultsBatch}]{\sphinxcrossref{ResultsBatch}}}

\end{description}\end{quote}
\subsubsection*{Notes}
\begin{enumerate}
\sphinxsetlistlabels{\arabic}{enumi}{enumii}{}{.}%
\item {} 
\sphinxAtStartPar
The addition operator ‘+’ may be used instead of a call to \sphinxtitleref{sum(RBs)} for \sphinxtitleref{RBs} of length 2,
i.e. these two lines perform the same operation:

\begin{sphinxVerbatim}[commandchars=\\\{\}]
\PYG{g+gp}{\PYGZgt{}\PYGZgt{}\PYGZgt{} }\PYG{n}{ResultsBatch}\PYG{o}{.}\PYG{n}{sum}\PYG{p}{(}\PYG{p}{[}\PYG{n}{RB1}\PYG{p}{,} \PYG{n}{RB2}\PYG{p}{]}\PYG{p}{)}
\PYG{g+gp}{\PYGZgt{}\PYGZgt{}\PYGZgt{} }\PYG{n}{RB1} \PYG{o}{+} \PYG{n}{RB2}
\end{sphinxVerbatim}

\item {} 
\sphinxAtStartPar
The \sphinxtitleref{working\_dir} attribute of \sphinxtitleref{RB} is the one of the first element of \sphinxtitleref{RBs}.

\end{enumerate}

\end{fulllineitems}

\index{working\_dir (tamos.data\_IO.ResultsBatch property)@\spxentry{working\_dir}\spxextra{tamos.data\_IO.ResultsBatch property}}

\begin{fulllineitems}
\phantomsection\label{\detokenize{generated/tamos.data_IO.ResultsBatch:tamos.data_IO.ResultsBatch.working_dir}}
\pysigstartsignatures
\pysigline{\sphinxbfcode{\sphinxupquote{property\DUrole{w}{  }}}\sphinxbfcode{\sphinxupquote{working\_dir}}}
\pysigstopsignatures
\sphinxAtStartPar
Directory the results are read from or exported to.

\end{fulllineitems}

\index{write\_all() (tamos.data\_IO.ResultsBatch method)@\spxentry{write\_all()}\spxextra{tamos.data\_IO.ResultsBatch method}}

\begin{fulllineitems}
\phantomsection\label{\detokenize{generated/tamos.data_IO.ResultsBatch:tamos.data_IO.ResultsBatch.write_all}}
\pysigstartsignatures
\pysiglinewithargsret{\sphinxbfcode{\sphinxupquote{write\_all}}}{}{}
\pysigstopsignatures
\sphinxAtStartPar
Writes the \sphinxtitleref{vars\_}, \sphinxtitleref{KPIs} and \sphinxtitleref{results} attributes as CSV files, in the directory \sphinxtitleref{working\_dir}.

\end{fulllineitems}


\end{fulllineitems}



\section{Load data from disk}
\label{\detokenize{data_IO:load-data-from-disk}}

\begin{savenotes}\sphinxattablestart
\centering
\begin{tabulary}{\linewidth}[t]{\X{1}{2}\X{1}{2}}
\hline

\sphinxAtStartPar
{\hyperref[\detokenize{generated/tamos.data_IO.read_properties:tamos.data_IO.read_properties}]{\sphinxcrossref{\sphinxcode{\sphinxupquote{tamos.data\_IO.read\_properties}}}}}(path)
&
\sphinxAtStartPar
Read a CSV file describing techno\sphinxhyphen{}economic properties of a component.
\\
\hline
\sphinxAtStartPar
{\hyperref[\detokenize{generated/tamos.data_IO.read_data:tamos.data_IO.read_data}]{\sphinxcrossref{\sphinxcode{\sphinxupquote{tamos.data\_IO.read\_data}}}}}(path, col\_name{[}, ...{]})
&
\sphinxAtStartPar
Read a CSV file describing a serie of values
\\
\hline
\end{tabulary}
\par
\sphinxattableend\end{savenotes}

\sphinxstepscope


\subsection{tamos.data\_IO.read\_properties}
\label{\detokenize{generated/tamos.data_IO.read_properties:tamos-data-io-read-properties}}\label{\detokenize{generated/tamos.data_IO.read_properties::doc}}\index{read\_properties() (in module tamos.data\_IO)@\spxentry{read\_properties()}\spxextra{in module tamos.data\_IO}}

\begin{fulllineitems}
\phantomsection\label{\detokenize{generated/tamos.data_IO.read_properties:tamos.data_IO.read_properties}}
\pysigstartsignatures
\pysiglinewithargsret{\sphinxcode{\sphinxupquote{tamos.data\_IO.}}\sphinxbfcode{\sphinxupquote{read\_properties}}}{\emph{\DUrole{n}{path}}}{}
\pysigstopsignatures
\sphinxAtStartPar
Read a CSV file describing techno\sphinxhyphen{}economic properties of a component.
\begin{quote}\begin{description}
\sphinxlineitem{Parameters}
\sphinxAtStartPar
\sphinxstyleliteralstrong{\sphinxupquote{path}} (\sphinxstyleliteralemphasis{\sphinxupquote{str}}\sphinxstyleliteralemphasis{\sphinxupquote{ or }}\sphinxstyleliteralemphasis{\sphinxupquote{path\sphinxhyphen{}like object}}) \textendash{} The path of the file to load properties from.

\sphinxlineitem{Returns}
\sphinxAtStartPar
\sphinxstylestrong{properties} \textendash{} A dict of the content of the file located at \sphinxtitleref{path}.

\sphinxlineitem{Return type}
\sphinxAtStartPar
dict

\end{description}\end{quote}
\subsubsection*{Notes}
\begin{enumerate}
\sphinxsetlistlabels{\arabic}{enumi}{enumii}{}{.}%
\item {} 
\sphinxAtStartPar
The file must have two columns:
\begin{itemize}
\item {} 
\sphinxAtStartPar
The first column contains properties name

\item {} 
\sphinxAtStartPar
The second column contains properties value

\end{itemize}

\item {} 
\sphinxAtStartPar
The file must not have a header row (i.e. title row).

\item {} 
\sphinxAtStartPar
Lines starting with ‘\#’ are treated as comments.

\end{enumerate}

\end{fulllineitems}


\sphinxstepscope


\subsection{tamos.data\_IO.read\_data}
\label{\detokenize{generated/tamos.data_IO.read_data:tamos-data-io-read-data}}\label{\detokenize{generated/tamos.data_IO.read_data::doc}}\index{read\_data() (in module tamos.data\_IO)@\spxentry{read\_data()}\spxextra{in module tamos.data\_IO}}

\begin{fulllineitems}
\phantomsection\label{\detokenize{generated/tamos.data_IO.read_data:tamos.data_IO.read_data}}
\pysigstartsignatures
\pysiglinewithargsret{\sphinxcode{\sphinxupquote{tamos.data\_IO.}}\sphinxbfcode{\sphinxupquote{read\_data}}}{\emph{\DUrole{n}{path}}, \emph{\DUrole{n}{col\_name}}, \emph{\DUrole{n}{start}\DUrole{o}{=}\DUrole{default_value}{None}}, \emph{\DUrole{n}{end}\DUrole{o}{=}\DUrole{default_value}{None}}}{}
\pysigstopsignatures
\sphinxAtStartPar
Read a CSV file describing a serie of values
\begin{quote}\begin{description}
\sphinxlineitem{Parameters}\begin{itemize}
\item {} 
\sphinxAtStartPar
\sphinxstyleliteralstrong{\sphinxupquote{path}} (\sphinxstyleliteralemphasis{\sphinxupquote{str}}\sphinxstyleliteralemphasis{\sphinxupquote{ or }}\sphinxstyleliteralemphasis{\sphinxupquote{path\sphinxhyphen{}like object}}) \textendash{} The path of the file to load values from.

\item {} 
\sphinxAtStartPar
\sphinxstyleliteralstrong{\sphinxupquote{col\_name}} (\sphinxstyleliteralemphasis{\sphinxupquote{str}}\sphinxstyleliteralemphasis{\sphinxupquote{, }}\sphinxstyleliteralemphasis{\sphinxupquote{optional}}) \textendash{} The name of the column to be loaded from the CSV file.

\item {} 
\sphinxAtStartPar
\sphinxstyleliteralstrong{\sphinxupquote{start}} (\sphinxstyleliteralemphasis{\sphinxupquote{int}}\sphinxstyleliteralemphasis{\sphinxupquote{, }}\sphinxstyleliteralemphasis{\sphinxupquote{optional}}) \textendash{} First row of values being loaded.
If not provided, start=0.

\item {} 
\sphinxAtStartPar
\sphinxstyleliteralstrong{\sphinxupquote{end}} (\sphinxstyleliteralemphasis{\sphinxupquote{int}}\sphinxstyleliteralemphasis{\sphinxupquote{, }}\sphinxstyleliteralemphasis{\sphinxupquote{optional}}) \textendash{} Last row of values being loaded.
If not provided, end is the last row of the file.

\end{itemize}

\sphinxlineitem{Return type}
\sphinxAtStartPar
A numpy.ndarray of the content located at \sphinxtitleref{path} under \sphinxtitleref{col\_name}.

\end{description}\end{quote}
\subsubsection*{Notes}
\begin{enumerate}
\sphinxsetlistlabels{\arabic}{enumi}{enumii}{}{.}%
\item {} 
\sphinxAtStartPar
The file must not have a header row (i.e. title row).

\item {} 
\sphinxAtStartPar
Lines starting with ‘\#’ are treated as comments.

\end{enumerate}

\end{fulllineitems}


\sphinxstepscope


\chapter{Solving tools}
\label{\detokenize{solve_tools:solving-tools}}\label{\detokenize{solve_tools::doc}}

\begin{savenotes}\sphinxattablestart
\centering
\begin{tabulary}{\linewidth}[t]{\X{1}{2}\X{1}{2}}
\hline

\sphinxAtStartPar
{\hyperref[\detokenize{generated/tamos.solve_tools.AdvSolve:tamos.solve_tools.AdvSolve}]{\sphinxcrossref{\sphinxcode{\sphinxupquote{tamos.solve\_tools.AdvSolve}}}}}(MILPModel, ...{[}, ...{]})
&
\sphinxAtStartPar

\\
\hline
\end{tabulary}
\par
\sphinxattableend\end{savenotes}

\sphinxstepscope


\section{tamos.solve\_tools.AdvSolve}
\label{\detokenize{generated/tamos.solve_tools.AdvSolve:tamos-solve-tools-advsolve}}\label{\detokenize{generated/tamos.solve_tools.AdvSolve::doc}}\index{AdvSolve (class in tamos.solve\_tools)@\spxentry{AdvSolve}\spxextra{class in tamos.solve\_tools}}

\begin{fulllineitems}
\phantomsection\label{\detokenize{generated/tamos.solve_tools.AdvSolve:tamos.solve_tools.AdvSolve}}
\pysigstartsignatures
\pysiglinewithargsret{\sphinxbfcode{\sphinxupquote{class\DUrole{w}{  }}}\sphinxcode{\sphinxupquote{tamos.solve\_tools.}}\sphinxbfcode{\sphinxupquote{AdvSolve}}}{\emph{\DUrole{n}{MILPModel}}, \emph{\DUrole{n}{working\_dir}}, \emph{\DUrole{n}{skip\_existing}\DUrole{o}{=}\DUrole{default_value}{True}}, \emph{\DUrole{o}{**}\DUrole{n}{kwargs}}}{}
\pysigstopsignatures\index{\_\_init\_\_() (tamos.solve\_tools.AdvSolve method)@\spxentry{\_\_init\_\_()}\spxextra{tamos.solve\_tools.AdvSolve method}}

\begin{fulllineitems}
\phantomsection\label{\detokenize{generated/tamos.solve_tools.AdvSolve:tamos.solve_tools.AdvSolve.__init__}}
\pysigstartsignatures
\pysiglinewithargsret{\sphinxbfcode{\sphinxupquote{\_\_init\_\_}}}{\emph{\DUrole{n}{MILPModel}}, \emph{\DUrole{n}{working\_dir}}, \emph{\DUrole{n}{skip\_existing}\DUrole{o}{=}\DUrole{default_value}{True}}, \emph{\DUrole{o}{**}\DUrole{n}{kwargs}}}{}
\pysigstopsignatures
\sphinxAtStartPar
AdvSolve is a wrapper that ease the advanced solving of MILP models.

\sphinxAtStartPar
Attribute \sphinxtitleref{last\_binaries} contains the ResultsExport instances of of the latest solved models.
These all are results of \sphinxtitleref{MILPModel} yet following different solving configurations.
\begin{quote}\begin{description}
\sphinxlineitem{Parameters}\begin{itemize}
\item {} 
\sphinxAtStartPar
\sphinxstyleliteralstrong{\sphinxupquote{MILPModel}} ({\hyperref[\detokenize{generated/tamos.MILPModel:tamos.MILPModel}]{\sphinxcrossref{\sphinxstyleliteralemphasis{\sphinxupquote{MILPModel}}}}}) \textendash{} A \sphinxtitleref{MILPModel} instance whose \sphinxtitleref{declare\_variables} and \sphinxtitleref{declare\_constraints\_and\_KPIs}
methods have already been called.
\sphinxtitleref{MILPModel} is used in every call to this instance methods.

\item {} 
\sphinxAtStartPar
\sphinxstyleliteralstrong{\sphinxupquote{working\_dir}} (\sphinxstyleliteralemphasis{\sphinxupquote{str}}\sphinxstyleliteralemphasis{\sphinxupquote{ or }}\sphinxstyleliteralemphasis{\sphinxupquote{path\sphinxhyphen{}like}}) \textendash{} Directory the results are read from (see \sphinxtitleref{skip\_existing}) and written to.

\item {} 
\sphinxAtStartPar
\sphinxstyleliteralstrong{\sphinxupquote{skip\_existing}} (\sphinxstyleliteralemphasis{\sphinxupquote{bool}}\sphinxstyleliteralemphasis{\sphinxupquote{, }}\sphinxstyleliteralemphasis{\sphinxupquote{optional}}\sphinxstyleliteralemphasis{\sphinxupquote{, }}\sphinxstyleliteralemphasis{\sphinxupquote{True}}) \textendash{} \begin{itemize}
\item {} 
\sphinxAtStartPar
If True and if there exists a ResultsExport on disk having the same name than \sphinxtitleref{MILPModel},
existing results are loaded instead of solving again \sphinxtitleref{MILPModel}.

\item {} 
\sphinxAtStartPar
If False, \sphinxtitleref{MILPModel} is solved in all cases.

\end{itemize}


\item {} 
\sphinxAtStartPar
\sphinxstyleliteralstrong{\sphinxupquote{kwargs}} (\sphinxstyleliteralemphasis{\sphinxupquote{optional}}) \textendash{} 
\sphinxAtStartPar
Keywords arguments.
Some are passed to the \sphinxtitleref{solve} method of \sphinxtitleref{MILPModel} instance. These are:
\begin{itemize}
\item {} 
\sphinxAtStartPar
MIP\_gap: float, default 1e\sphinxhyphen{}4

\item {} 
\sphinxAtStartPar
threads: int, default 0

\item {} 
\sphinxAtStartPar
timelimit: int, default 43200

\end{itemize}

\sphinxAtStartPar
Others are passed to \sphinxtitleref{ResultsExport}. These are:
\begin{itemize}
\item {} 
\sphinxAtStartPar
get\_LP: bool, default False

\item {} 
\sphinxAtStartPar
get\_MPS: bool, default False

\item {} 
\sphinxAtStartPar
csv\_precision: int, default 3

\item {} 
\sphinxAtStartPar
replace\_inverted\_TVP: bool, default False

\end{itemize}

\sphinxAtStartPar
\sphinxtitleref{parent\_working\_dir} of ResultsExport will be ignored since \sphinxtitleref{working\_dir} has the same use.


\end{itemize}

\end{description}\end{quote}
\subsubsection*{Notes}
\begin{enumerate}
\sphinxsetlistlabels{\arabic}{enumi}{enumii}{}{.}%
\item {} 
\sphinxAtStartPar
Each MILP solving is performed using the Cplex optimizer.

\item {} 
\sphinxAtStartPar
AdvSolve does not perform model declaration.

\end{enumerate}

\end{fulllineitems}

\subsubsection*{Methods}


\begin{savenotes}\sphinxattablestart
\centering
\begin{tabulary}{\linewidth}[t]{\X{1}{2}\X{1}{2}}
\hline

\sphinxAtStartPar
{\hyperref[\detokenize{generated/tamos.solve_tools.AdvSolve:tamos.solve_tools.AdvSolve.__init__}]{\sphinxcrossref{\sphinxcode{\sphinxupquote{\_\_init\_\_}}}}}(MILPModel, working\_dir{[}, skip\_existing{]})
&
\sphinxAtStartPar
AdvSolve is a wrapper that ease the advanced solving of MILP models.
\\
\hline
\sphinxAtStartPar
\sphinxcode{\sphinxupquote{solve}}(*args, **kwargs)
&
\sphinxAtStartPar

\\
\hline
\sphinxAtStartPar
\sphinxcode{\sphinxupquote{solve\_front}}(*args, **kwargs)
&
\sphinxAtStartPar

\\
\hline
\end{tabulary}
\par
\sphinxattableend\end{savenotes}
\subsubsection*{Attributes}


\begin{savenotes}\sphinxattablestart
\centering
\begin{tabulary}{\linewidth}[t]{\X{1}{2}\X{1}{2}}
\hline

\sphinxAtStartPar
{\hyperref[\detokenize{generated/tamos.solve_tools.AdvSolve:tamos.solve_tools.AdvSolve.MILPModel}]{\sphinxcrossref{\sphinxcode{\sphinxupquote{MILPModel}}}}}
&
\sphinxAtStartPar
Used in every call to this instance methods.
\\
\hline
\sphinxAtStartPar
{\hyperref[\detokenize{generated/tamos.solve_tools.AdvSolve:tamos.solve_tools.AdvSolve.kwargs}]{\sphinxcrossref{\sphinxcode{\sphinxupquote{kwargs}}}}}
&
\sphinxAtStartPar
Keywords arguments.
\\
\hline
\sphinxAtStartPar
{\hyperref[\detokenize{generated/tamos.solve_tools.AdvSolve:tamos.solve_tools.AdvSolve.last_binaries}]{\sphinxcrossref{\sphinxcode{\sphinxupquote{last\_binaries}}}}}
&
\sphinxAtStartPar
The ResultsExport instances of the latest solved models.
\\
\hline
\sphinxAtStartPar
{\hyperref[\detokenize{generated/tamos.solve_tools.AdvSolve:tamos.solve_tools.AdvSolve.skip_existing}]{\sphinxcrossref{\sphinxcode{\sphinxupquote{skip\_existing}}}}}
&
\sphinxAtStartPar
\begin{itemize}
\item {} 
\sphinxAtStartPar
If True and if there exists a ResultsExport on disk having the same name than \sphinxtitleref{MILPModel},

\end{itemize}

\\
\hline
\sphinxAtStartPar
{\hyperref[\detokenize{generated/tamos.solve_tools.AdvSolve:tamos.solve_tools.AdvSolve.working_dir}]{\sphinxcrossref{\sphinxcode{\sphinxupquote{working\_dir}}}}}
&
\sphinxAtStartPar
Directory the results are read from (see \sphinxtitleref{skip\_existing}) and written to.
\\
\hline
\end{tabulary}
\par
\sphinxattableend\end{savenotes}
\index{MILPModel (tamos.solve\_tools.AdvSolve property)@\spxentry{MILPModel}\spxextra{tamos.solve\_tools.AdvSolve property}}

\begin{fulllineitems}
\phantomsection\label{\detokenize{generated/tamos.solve_tools.AdvSolve:tamos.solve_tools.AdvSolve.MILPModel}}
\pysigstartsignatures
\pysigline{\sphinxbfcode{\sphinxupquote{property\DUrole{w}{  }}}\sphinxbfcode{\sphinxupquote{MILPModel}}}
\pysigstopsignatures
\sphinxAtStartPar
Used in every call to this instance methods.

\end{fulllineitems}

\index{kwargs (tamos.solve\_tools.AdvSolve property)@\spxentry{kwargs}\spxextra{tamos.solve\_tools.AdvSolve property}}

\begin{fulllineitems}
\phantomsection\label{\detokenize{generated/tamos.solve_tools.AdvSolve:tamos.solve_tools.AdvSolve.kwargs}}
\pysigstartsignatures
\pysigline{\sphinxbfcode{\sphinxupquote{property\DUrole{w}{  }}}\sphinxbfcode{\sphinxupquote{kwargs}}}
\pysigstopsignatures
\sphinxAtStartPar
Keywords arguments.
Some are passed to the \sphinxtitleref{solve} method of \sphinxtitleref{MILPModel} instance. These are:
\begin{itemize}
\item {} 
\sphinxAtStartPar
MIP\_gap: float, default 1e\sphinxhyphen{}4

\item {} 
\sphinxAtStartPar
threads: int, default 0

\item {} 
\sphinxAtStartPar
timelimit: int, default 43200

\end{itemize}

\sphinxAtStartPar
Others are passed to \sphinxtitleref{ResultsExport}. These are:
\begin{itemize}
\item {} 
\sphinxAtStartPar
get\_LP: bool, default False

\item {} 
\sphinxAtStartPar
get\_MPS: bool, default False

\item {} 
\sphinxAtStartPar
csv\_precision: int, default 3

\item {} 
\sphinxAtStartPar
replace\_inverted\_TVP: bool, default False

\end{itemize}

\sphinxAtStartPar
\sphinxtitleref{parent\_working\_dir} of ResultsExport will be ignored since \sphinxtitleref{working\_dir} has the same use.

\end{fulllineitems}

\index{last\_binaries (tamos.solve\_tools.AdvSolve property)@\spxentry{last\_binaries}\spxextra{tamos.solve\_tools.AdvSolve property}}

\begin{fulllineitems}
\phantomsection\label{\detokenize{generated/tamos.solve_tools.AdvSolve:tamos.solve_tools.AdvSolve.last_binaries}}
\pysigstartsignatures
\pysigline{\sphinxbfcode{\sphinxupquote{property\DUrole{w}{  }}}\sphinxbfcode{\sphinxupquote{last\_binaries}}}
\pysigstopsignatures
\sphinxAtStartPar
The ResultsExport instances of the latest solved models.
dict \{str: ResultsExport\}

\end{fulllineitems}

\index{skip\_existing (tamos.solve\_tools.AdvSolve property)@\spxentry{skip\_existing}\spxextra{tamos.solve\_tools.AdvSolve property}}

\begin{fulllineitems}
\phantomsection\label{\detokenize{generated/tamos.solve_tools.AdvSolve:tamos.solve_tools.AdvSolve.skip_existing}}
\pysigstartsignatures
\pysigline{\sphinxbfcode{\sphinxupquote{property\DUrole{w}{  }}}\sphinxbfcode{\sphinxupquote{skip\_existing}}}
\pysigstopsignatures\begin{itemize}
\item {} 
\sphinxAtStartPar
If True and if there exists a ResultsExport on disk having the same name than \sphinxtitleref{MILPModel},
existing results are loaded instead of solving again \sphinxtitleref{MILPModel}.

\item {} 
\sphinxAtStartPar
If False, \sphinxtitleref{MILPModel} is solved in all cases.

\end{itemize}

\end{fulllineitems}

\index{working\_dir (tamos.solve\_tools.AdvSolve property)@\spxentry{working\_dir}\spxextra{tamos.solve\_tools.AdvSolve property}}

\begin{fulllineitems}
\phantomsection\label{\detokenize{generated/tamos.solve_tools.AdvSolve:tamos.solve_tools.AdvSolve.working_dir}}
\pysigstartsignatures
\pysigline{\sphinxbfcode{\sphinxupquote{property\DUrole{w}{  }}}\sphinxbfcode{\sphinxupquote{working\_dir}}}
\pysigstopsignatures
\sphinxAtStartPar
Directory the results are read from (see \sphinxtitleref{skip\_existing}) and written to.

\end{fulllineitems}


\end{fulllineitems}


\sphinxstepscope


\chapter{General settings}
\label{\detokenize{general_settings:general-settings}}\label{\detokenize{general_settings::doc}}

\begin{savenotes}\sphinxattablestart
\centering
\begin{tabulary}{\linewidth}[t]{\X{1}{2}\X{1}{2}}
\hline

\sphinxAtStartPar
{\hyperref[\detokenize{generated/tamos.allow_duplicated_names:tamos.allow_duplicated_names}]{\sphinxcrossref{\sphinxcode{\sphinxupquote{tamos.allow\_duplicated\_names}}}}}(allow)
&
\sphinxAtStartPar
Defines the policy regarding components sharing the same name.
\\
\hline
\sphinxAtStartPar
{\hyperref[\detokenize{generated/tamos.use_name_in_MILP:tamos.use_name_in_MILP}]{\sphinxcrossref{\sphinxcode{\sphinxupquote{tamos.use\_name\_in\_MILP}}}}}(bool)
&
\sphinxAtStartPar
Defines whether the name of components is used to give an exhaustive human\sphinxhyphen{}friendly name to decision variables, constraints and KPIs in the MILP problem.
\\
\hline
\sphinxAtStartPar
{\hyperref[\detokenize{generated/tamos.reset_names_list:tamos.reset_names_list}]{\sphinxcrossref{\sphinxcode{\sphinxupquote{tamos.reset\_names\_list}}}}}()
&
\sphinxAtStartPar
Forgets every component name.
\\
\hline
\end{tabulary}
\par
\sphinxattableend\end{savenotes}

\sphinxstepscope


\section{tamos.allow\_duplicated\_names}
\label{\detokenize{generated/tamos.allow_duplicated_names:tamos-allow-duplicated-names}}\label{\detokenize{generated/tamos.allow_duplicated_names::doc}}\index{allow\_duplicated\_names() (in module tamos)@\spxentry{allow\_duplicated\_names()}\spxextra{in module tamos}}

\begin{fulllineitems}
\phantomsection\label{\detokenize{generated/tamos.allow_duplicated_names:tamos.allow_duplicated_names}}
\pysigstartsignatures
\pysiglinewithargsret{\sphinxcode{\sphinxupquote{tamos.}}\sphinxbfcode{\sphinxupquote{allow\_duplicated\_names}}}{\emph{\DUrole{n}{allow}}}{}
\pysigstopsignatures
\sphinxAtStartPar
Defines the policy regarding components sharing the same name.
\begin{quote}\begin{description}
\sphinxlineitem{Parameters}
\sphinxAtStartPar
\sphinxstyleliteralstrong{\sphinxupquote{allow}} (\sphinxstyleliteralemphasis{\sphinxupquote{bool}}) \textendash{} \begin{itemize}
\item {} 
\sphinxAtStartPar
If True, duplicated names might exist.
Leads to problems in model export regarding the identification of different MILP objects.

\item {} 
\sphinxAtStartPar
If False, names are changed during the component declaration so that no duplicate exists.

\end{itemize}


\end{description}\end{quote}

\end{fulllineitems}


\sphinxstepscope


\section{tamos.use\_name\_in\_MILP}
\label{\detokenize{generated/tamos.use_name_in_MILP:tamos-use-name-in-milp}}\label{\detokenize{generated/tamos.use_name_in_MILP::doc}}\index{use\_name\_in\_MILP() (in module tamos)@\spxentry{use\_name\_in\_MILP()}\spxextra{in module tamos}}

\begin{fulllineitems}
\phantomsection\label{\detokenize{generated/tamos.use_name_in_MILP:tamos.use_name_in_MILP}}
\pysigstartsignatures
\pysiglinewithargsret{\sphinxcode{\sphinxupquote{tamos.}}\sphinxbfcode{\sphinxupquote{use\_name\_in\_MILP}}}{\emph{\DUrole{n}{bool}}}{}
\pysigstopsignatures
\sphinxAtStartPar
Defines whether the name of components is used to give an exhaustive human\sphinxhyphen{}friendly name
to decision variables, constraints and KPIs in the MILP problem.

\sphinxAtStartPar
This has no effect on the exported variables and KPIs which will always be named according to the component name.
Consider calling \sphinxtitleref{use\_name\_in\_MILP(True)} only for debugging and exporting LP or MPS files of the model with a
human\sphinxhyphen{}friendly content.
\begin{quote}\begin{description}
\sphinxlineitem{Parameters}
\sphinxAtStartPar
\sphinxstyleliteralstrong{\sphinxupquote{bool}} (\sphinxstyleliteralemphasis{\sphinxupquote{bool}}) \textendash{} \begin{itemize}
\item {} 
\sphinxAtStartPar
If False (default), variables, constraints and KPIs will receive non human\sphinxhyphen{}friendly names.

\item {} 
\sphinxAtStartPar
If True, components names are used.
This has two consequences:
\begin{itemize}
\item {} 
\sphinxAtStartPar
It slightlys lengthen the declaration time of the MILP problem.

\item {} 
\sphinxAtStartPar
It makes LP and MPS files more readable but with a bigger size on disk.

\end{itemize}

\end{itemize}


\end{description}\end{quote}

\end{fulllineitems}


\sphinxstepscope


\section{tamos.reset\_names\_list}
\label{\detokenize{generated/tamos.reset_names_list:tamos-reset-names-list}}\label{\detokenize{generated/tamos.reset_names_list::doc}}\index{reset\_names\_list() (in module tamos)@\spxentry{reset\_names\_list()}\spxextra{in module tamos}}

\begin{fulllineitems}
\phantomsection\label{\detokenize{generated/tamos.reset_names_list:tamos.reset_names_list}}
\pysigstartsignatures
\pysiglinewithargsret{\sphinxcode{\sphinxupquote{tamos.}}\sphinxbfcode{\sphinxupquote{reset\_names\_list}}}{}{}
\pysigstopsignatures
\sphinxAtStartPar
Forgets every component name.

\end{fulllineitems}



\chapter{Indices and tables}
\label{\detokenize{index:indices-and-tables}}\begin{itemize}
\item {} 
\sphinxAtStartPar
\DUrole{xref,std,std-ref}{genindex}

\item {} 
\sphinxAtStartPar
\DUrole{xref,std,std-ref}{modindex}

\item {} 
\sphinxAtStartPar
\DUrole{xref,std,std-ref}{search}

\end{itemize}



\renewcommand{\indexname}{Index}
\printindex
\end{document}